\documentclass{article}
\usepackage{fullpage}
\usepackage[czech]{babel}
\usepackage{amsfonts}

\title{\vspace{-2cm}\vspace{-1.7cm}}
\date{}
\author{}

\begin{document}
\maketitle
\noindent \textbf{Jan Romanovský}

\noindent \textbf{Gymnázium Brno, tř. Kpt. Jaroše}

\noindent \textbf{3.A}

\noindent \textbf{A-\textrm{I}-2}

\textbf{ }

Sčítáme vždy tři číslice vedle sebe, to znamená že pět prostředních číslic bude v součtech započteno třikrát, dvě číslice z každé strany této pětice dvakrát a dvě krajní číslice jen jednou. Když sečteme trojnásobky všech čísel 1-9, vyjde nám 135. Součet všech součtů trojic je 122, resp. 123. Největší možný součet součtů trojic tedy bude 135 bez $2*(1+2)$ -- 1 a 2 dáme na oba konce, a bez $(3+4)$ -- 3 a 4 dáme z obou stran prostřední pětky. Tento bude roven 122, čili v bodu b) rozhodně takové číslo sestavit nelze. V bodě a) jsme již byli omezeni umístěním číslic 1, 2, 3, 4 a po chvíli zkoušení nacházíme číslo, které zadání vyhovuje, a to 249856731.

\end{document}
