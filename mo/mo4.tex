\documentclass{article}
\usepackage{fullpage}
\usepackage[czech]{babel}
\usepackage{amsfonts}

\title{\vspace{-2cm}\vspace{-1.7cm}}
\date{}
\author{}

\begin{document}
\maketitle
\noindent \textbf{Jan Romanovský}

\noindent \textbf{Gymnázium Brno, tř. Kpt. Jaroše}

\noindent \textbf{3.A}

\noindent \textbf{A-\textrm{I}-4}

\textbf{ }

Nechť je prvočíslo $p$ speciální. Potom dělí součet všech provčísel nižších než $p$, označme ho $S$. Tento součet se tedy dá zapsat jako nějaký k-násobek $p$. Označme další prvočíslo po $p$ jako $q$. Aby $q$ bylo také speciální, musí dělit součet všech prvočísel menších než $q$, který se dá zapsat jako $kp + p = (k+1)p$. $q$ rozhodně nedělí $p$, takže dle fundamentální věty aritmetiky musí dělit $k+1$, což znamená, že $q \leq k+1$.

Součet všech přirozených čísel, která jsou menší než $p$, označíme jako $T$. Potom $T=\frac{(p+1)}{2}p$, což je zřejmě větší, než $S$:
\[\frac{(p+1)}{2}p>(k+1)p\]
\[p+1>2(k+1)\]
\[p+1>k+1\]

Rozdíl mezi následujícími prvočísly je ale zřejmě větší, než 1, takže $q>p+1$. Z tohoto všeho můžeme sestavit nerovnici:
\[q\leq k+1 < p+1 < q\]
\[q<q\]

Toto je ale spor, q nemůže být menší než q. Dvě speciální prvočísla po sobě tedy nemohou existovat.
\end{document}
