\documentclass{article}

% Language setting
% Replace `english' with e.g. `spanish' to change the document language
\usepackage[czech]{babel}
\usepackage{amsthm,thmtools,xcolor,amsmath,amssymb, mathtools}

% Set page size and margins
% Replace `letterpaper' with `a4paper' for UK/EU standard size
\usepackage[a4paper,top=2cm,bottom=2cm,left=2cm,right=2cm,marginparwidth=1.75cm]{geometry}

% Useful packages
\usepackage{amsmath}
\usepackage{graphicx}
\usepackage[colorlinks=true, allcolors=blue]{hyperref}
\graphicspath{images/}
\usepackage{enumerate}

%\usepackage{tikzit}
%\input{default.tikzstyles}

\title{DÚ Lineární algebra -- Sada 6}
\author{Jan Romanovský}

\begin{document}
\maketitle

\textbf{(6.1)} Hledáme předpis pro prvky $P$ množiny $V$.
$$P=ax^4+bx^3+cx^2+dx+e$$
$$a+b+c+d+e=0$$
$$P(-1)=P(2):a-b+c-d+e=16a+8b+4c+2d+e$$
Nyní řešíme soustavu dvou rovnic o pěti neznámých.
$$a+b+c+d+e=0$$
$$a-b+c-d+e=16a+8b+4c+2d+e$$

$$0=15a+9b+3c+3d$$
$$a = k, b = l, c = m$$
$$d = -5k-3l-m$$
$$e = 4k+2l$$

$$P = kx^4+lx^3+mx^2+(-5k-3l-m)x+(4k+2l)$$
Dále zapíšeme obecný polynom $P$ jako lineární kombinaci tří konkrétních polynomů $P_1, P_2, P_3$, které najdeme vhodnou volbou parametrů.
$$k=1, l = 0, m=0: P_1 = x^4-5x+4$$
$$k = 0, l = 1, m = 0: P_2 = x^3-3x+2$$
$$k = 0, l = 0, m = 1: P_3 = x^2-x$$
S touto volbou parametrů potom přímo $V = \left\{kP_1+lP_2+mP_3;k, l, m \in \mathbb R\right\}$, vidíme lineární kombinaci, tedy přímo $V = \text{span}(P_1,  P_2, P_3)$ což je zjevně podprostor $\mathbb R[x]_{\leq4}$.
\end{document}
