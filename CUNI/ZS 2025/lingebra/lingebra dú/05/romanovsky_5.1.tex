\documentclass{article}

% Language setting
% Replace `english' with e.g. `spanish' to change the document language
\usepackage[czech]{babel}
\usepackage{amsthm,thmtools,xcolor,amsmath,amssymb, mathtools}

% Set page size and margins
% Replace `letterpaper' with `a4paper' for UK/EU standard size
\usepackage[a4paper,top=2cm,bottom=2cm,left=2cm,right=2cm,marginparwidth=1.75cm]{geometry}

% Useful packages
\usepackage{amsmath}
\usepackage{graphicx}
\usepackage[colorlinks=true, allcolors=blue]{hyperref}
\graphicspath{images/}
\usepackage{enumerate}

%\usepackage{tikzit}
%\input{default.tikzstyles}

\title{DÚ Lineární algebra -- Sada 5}
\author{Jan Romanovský}

\begin{document}
\maketitle

\textbf{(5.1)} Nalezneme matici inverzní jako matici inverzní zprava (protože A je zřejmě regulární) pomocí řádkových elementárních úprav. Mějme rozšířenou matici.
$$ \left(\begin{array}{c c c c c c c | c c c c c c c}
    0 & 0 & 0 & \dots & 0 & 0 & 1 & 1 & 0 & 0 & \dots & 0 & 0 & 0\\
    0 & 0 & 0 & \dots & 0 & 2 & 2 & 0 & 1 & 0 & \dots & 0 & 0 & 0\\
    0 & 0 & 0 & \dots & 3 & 3 & 3 & 0 & 0 & 1 & \dots & 0 & 0 & 0\\
    &&&\vdots&&&&&&&\vdots\\
    0 & 0 & n-2 & \dots & n-2 & n-2 & n-2 & 0 & 0 & 0 & \dots & 1 & 0 & 0\\
    0 & n-1 & n-1 & \dots & n-1 & n-1 & n-1 & 0 & 0 & 0 & \dots & 0 & 1 & 0\\
    n & n & n & \dots & n & n & n & 0 & 0 & 0 & \dots & 0 & 0 & 1
\end{array}\right)\ $$
    Přehodíme řádky tak, aby byl poslední na prvním místě, předposlední na druhém místě, ..., první na posledním místě.
$$ \left(\begin{array}{c c c c c c c | c c c c c c c}
    n & n & n & \dots & n & n & n & 0 & 0 & 0 & \dots & 0 & 0 & 1\\
    0 & n-1 & n-1 & \dots & n-1 & n-1 & n-1 & 0 & 0 & 0 & \dots & 0 & 1 & 0\\
    0 & 0 & n-2 & \dots & n-2 & n-2 & n-2 & 0 & 0 & 0 & \dots & 1 & 0 & 0\\
    &&&\vdots&&&&&&&\vdots\\
    0 & 0 & 0 & \dots & 3 & 3 & 3 & 0 & 0 & 1 & \dots & 0 & 0 & 0\\
    0 & 0 & 0 & \dots & 0 & 2 & 2 & 0 & 1 & 0 & \dots & 0 & 0 & 0\\
    0 & 0 & 0 & \dots & 0 & 0 & 1 & 1 & 0 & 0 & \dots & 0 & 0 & 0
\end{array}\right) $$
    Dále řádky přenásobíme tak, abychom měli vlevo všude jedničky.
$$ \left(\begin{array}{c c c c c c c | c c c c c c c}
    1 & 1 & 1 & \dots & 1 & 1 & 1 & 0 & 0 & 0 & \dots & 0 & 0 & \frac{1}{n}\\
    0 & 1 & 1 & \dots & 1 & 1 & 1 & 0 & 0 & 0 & \dots & 0 & \frac{1}{n-1} & 0\\
    0 & 0 & 1 & \dots & 1 & 1 & 1 & 0 & 0 & 0 & \dots & \frac{1}{n-2} & 0 & 0\\
    &&&\vdots&&&&&&&\vdots\\
    0 & 0 & 0 & \dots & 1 & 1 & 1 & 0 & 0 & \frac{1}{3} & \dots & 0 & 0 & 0\\
    0 & 0 & 0 & \dots & 0 & 1 & 1 & 0 & \frac{1}{2} & 0 & \dots & 0 & 0 & 0\\
    0 & 0 & 0 & \dots & 0 & 0 & 1 & 1 & 0 & 0 & \dots & 0 & 0 & 0
\end{array}\right) $$
    A nakonec od $k$-tého řádku odečteme $k+1$., ten poslední už je ve tvaru, který chceme.
$$ \left(\begin{array}{c c c c c c c | c c c c c c c}
    1 & 0 & 0 & \dots & 0 & 0 & 0 & 0 & 0 & 0 & \dots & 0 & -\frac{1}{n-1} & \frac{1}{n}\\
    0 & 1 & 0 & \dots & 0 & 0 & 0 & 0 & 0 & 0 & \dots & -\frac{1}{n-2} & \frac{1}{n-1} & 0\\
    0 & 0 & 1 & \dots & 0 & 0 & 0 & 0 & 0 & 0 & \dots & \frac{1}{n-2} & 0 & 0\\
    &&&\vdots&&&&&&&\vdots\\
    0 & 0 & 0 & \dots & 1 & 0 & 0 & 0 & -\frac{1}{2} & \frac{1}{3} & \dots & 0 & 0 & 0\\
    0 & 0 & 0 & \dots & 0 & 1 & 0 & -1 & \frac{1}{2} & 0 & \dots & 0 & 0 & 0\\
    0 & 0 & 0 & \dots & 0 & 0 & 1 & 1 & 0 & 0 & \dots & 0 & 0 & 0
\end{array}\right) $$
Inverzní matici k matici $A$ pak nalezneme napravo od svislé čáry.
$$A^{-1}= \left(\begin{array}{c c c c c c c}
    0 & 0 & 0 & \dots & 0 & -\frac{1}{n-1} & \frac{1}{n}\\
    0 & 0 & 0 & \dots & -\frac{1}{n-2} & \frac{1}{n-1} & 0\\
    0 & 0 & 0 & \dots & \frac{1}{n-2} & 0 & 0\\
    &&&\vdots\\
    0 & -\frac{1}{2} & \frac{1}{3} & \dots & 0 & 0 & 0\\
    -1 & \frac{1}{2} & 0 & \dots & 0 & 0 & 0\\
    1 & 0 & 0 & \dots & 0 & 0 & 0
\end{array}\right).$$
\end{document}
