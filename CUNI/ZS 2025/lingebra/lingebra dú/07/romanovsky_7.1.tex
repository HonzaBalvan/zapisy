\documentclass{article}

% Language setting
% Replace `english' with e.g. `spanish' to change the document language
\usepackage[czech]{babel}
\usepackage{amsthm,thmtools,xcolor,amsmath,amssymb, mathtools}

% Set page size and margins
% Replace `letterpaper' with `a4paper' for UK/EU standard size
\usepackage[a4paper,top=2cm,bottom=2cm,left=2cm,right=2cm,marginparwidth=1.75cm]{geometry}

% Useful packages
\usepackage{amsmath}
\usepackage{graphicx}
\usepackage[colorlinks=true, allcolors=blue]{hyperref}
\graphicspath{images/}
\usepackage{enumerate}

%\usepackage{tikzit}
%\input{default.tikzstyles}

\title{DÚ Lineární algebra -- Sada 7}
\author{Jan Romanovský}

\begin{document}
\maketitle

\textbf{(7.1)} Když budu vektory vybírat po jednom:
\begin{itemize}
    \item budu mít $(7^3-1)$ možností pro první -- beru všechny vektory $\mathbb Z_7^3$ mimo nulový vektor -- posloupnost s nulovým vektorem je vždy lin. záv., všechny jiné posloupnosti s jedním vektorem jsou lin. nezáv.,
    \item pro druhý vektor mám $(7^3-7)$ možností -- všechny vektory $\mathbb Z_7^3$ mimo násobků prvního -- násobky prvního vektoru jsou zřejmě lin. komb. prvního, tedy lin. záv., to mi vyřadí i nulu jako nula-násobek,
    \item a nakonec pro třetí vektor mám $(7^3-7^2)$ možností -- všechny vektory $\mathbb Z_7^3$ mimo násobky prvního nebo druhého -- tedy takových, které by šly zapsat jako lin. komb. prvních dvou, tedy by byly lin. záv., nulový vektor znovu vyřazen.
\end{itemize}
Tedy celkový počet je $(7^3-7)(7^3-1)(7^3-7^2)=33\,784\,128$.

\end{document}
