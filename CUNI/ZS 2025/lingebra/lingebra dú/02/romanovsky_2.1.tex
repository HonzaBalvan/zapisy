\documentclass{article}

% Language setting
% Replace `english' with e.g. `spanish' to change the document language
\usepackage[czech]{babel}
\usepackage{amsthm,thmtools,xcolor,amsmath,amssymb, mathtools}

% Set page size and margins
% Replace `letterpaper' with `a4paper' for UK/EU standard size
\usepackage[a4paper,top=2cm,bottom=2cm,left=3cm,right=3cm,marginparwidth=1.75cm]{geometry}

% Useful packages
\usepackage{amsmath}
\usepackage{graphicx}
\usepackage[colorlinks=true, allcolors=blue]{hyperref}
\graphicspath{images/}
\usepackage{enumerate}

%\usepackage{tikzit}
%\input{default.tikzstyles}

\title{DÚ Lineární algebra -- Sada 2}
\author{Jan Romanovský}

\begin{document}
\maketitle

\textbf{(2.1)} Hledáme uspoř. trojice $\left(x,y,z\right)$. Převedeme matici na odstupňovaný tvar a dále řešíme zpětnou substitucí:
$$\left(\begin{array}{c c c | c }
    -i & a & 1+i & 0\\
    1 & 3i & b & 0\\
    i & -3 & 1 & 1
\end{array}\right) \sim
\left(\begin{array}{c c c | c }
    1 & 3i & b & 0\\
    -i & a & 1+i & 0\\
    i & -3 & 1 & 1
\end{array}\right) \sim
\left(\begin{array}{c c c | c }
    1 & 3i & b & 0\\
    -i & a & 1+i & 0\\
    0 & a-3 & 2+i & 1
\end{array}\right) \sim
\left(\begin{array}{c c c | c }
    1 & 3i & b & 0\\
    1 & ia & i-1 & 0\\
    0 & a-3 & 2+i & 1
\end{array}\right) \sim$$
$$\sim \left(\begin{array}{c c c | c }
    1 & 3i & b & 0\\
    0 & ia-3i & i-1-b & 0\\
    0 & a-3 & 2+i & 1
\end{array}\right) \sim
\left(\begin{array}{c c c | c }
    1 & 3i & b & 0\\
    0 & a-3 & 1+i+ib & 0\\
    0 & a-3 & 2+i & 1
\end{array}\right) \sim
\left(\begin{array}{c c c | c }
    1 & 3i & b & 0\\
    0 & a-3 & 1+i+ib & 0\\
    0 & 0 & 1-ib & 1
\end{array}\right)$$

$$\left(1-ib\right)z=1$$
\begin{enumerate}[i.]
    \item $b = -i: 0z = 1$ -- nelze
    \item $b \neq -i:$

$$z = \frac{1}{1-ib}$$
$$(a-3)y+\left(1+i+ib\right)z = 1$$
\begin{enumerate}[a.]
    \item $a=3$:
    $$\left(1+i+ib\right)z = 1$$
    $$\frac{\left(1+i+ib\right)}{\left(1-ib\right)}=1$$
    $$1+i+ib = 1-ib$$
    $$2ib = -i$$
    $$b = \frac{-1}{2}$$
    $$z = \frac{1}{1+\frac{i}{2}}= \frac{2}{2+i}=\frac{2-i}{5}$$
    $$y = t$$
    $$x =- 3iy +\frac{z}{2} = \frac{2-i-30it}{10}$$
    \item $a\neq 3$:
    $$y = \frac{-1-i-ib}{\left(1-ib\right)(a-3)}$$
    $$x + 3iy + bz = 0$$
    $$x = -3i\frac{-1-i-ib}{\left(1-ib\right)(a-3)}-b\frac{1}{1-ib}$$
    $$x = \frac{3i-3-3b}{\left(1-ib\right)(a-3)}-\frac{b}{1-ib}$$
    $$x = \frac{3i - 3- 3b-ab+3b}{(1-ib)\left(a-3\right)}$$
    $$x = \frac{3i - 3 -ab}{(1-ib)\left(a-3\right)}$$
\end{enumerate}
\end{enumerate}
Když dáme vše dohromady: $P = \left\{(\frac{3i - 3 -ab}{(1-ib)\left(a-3\right)}, \frac{-1-i-ib}{a-3}, \frac{1}{1-ib}); a \in \mathbb C \smallsetminus \{3\}, b \in \mathbb C \smallsetminus \{-i\}\right\}\cup\left\{(\frac{2-i-30it}{10},t,\frac{2-i}{5});t \in\mathbb C\right\}$
\end{document}
