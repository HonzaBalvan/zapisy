\documentclass{article}

% Language setting
% Replace `english' with e.g. `spanish' to change the document language
\usepackage[czech]{babel}
\usepackage{amsthm,thmtools,xcolor,amsmath,amssymb, mathtools}

% Set page size and margins
% Replace `letterpaper' with `a4paper' for UK/EU standard size
\usepackage[a4paper,top=2cm,bottom=2cm,left=2cm,right=2cm,marginparwidth=1.75cm]{geometry}

% Useful packages
\usepackage{amsmath}
\usepackage{graphicx}
\usepackage[colorlinks=true, allcolors=blue]{hyperref}
\graphicspath{images/}
\usepackage{enumerate}
\usepackage{tabularx}

\DeclareMathOperator{\Ker}{Ker}

%\usepackage{tikzit}
%\input{default.tikzstyles}

\title{DÚ Lineární algebra -- Sada 9}
\author{Jan Romanovský}

\begin{document}
\maketitle

\textbf{(9.1)} Provedeme Gaussovu eliminaci na matici $A_{a, b}$.
$$\begin{pmatrix}
  a + 2 & b & 1 & 2\\
  2a + 1 & 2b & a + 3 & 1\\
  3a + b + 3 & 3b & a + 4 & b + 3
\end{pmatrix}\sim
\begin{pmatrix}
  a + 2 & b & 1 & 2\\
  -3 & 0 & a + 1 & -3\\
  b - 3 & 0 & a + 1 & b - 3
\end{pmatrix}\sim
\begin{pmatrix}
  a + 2 & b & 1 & 2\\
  -3 & 0 & a + 1 & -3\\
  b & 0 & 0 & b
\end{pmatrix}$$
Vidíme, že můžeme vynulovat pouze třetí řádek, a to volbou parametru $b = 0$. V tom případě bude $\dim \operatorname{Im}A_{a, b}= 2$, jindy vždy $\dim \operatorname{Im}A_{a, b}=3$. Z tvrzení o dimenzi jádra a sloupcového prostoru můžeme rovnou určit dimenze ostatní prostorů.

\begin{center}
\begin{tabular}{ c | c | c | c | c }
  \, & $\dim \operatorname{Im}A_{a, b}$ & $\dim \operatorname{Im}A_{a, b}^T$ & $\dim \Ker A_{a, b}$ & $\dim \Ker A_{a, b}^T$ \\
  \hline
  $b = 0$ & 2 & 2 & 2 & 2\\
  $b \neq 0$ & 3 & 3 & 1 & 1\\
\end{tabular}
\end{center}
\end{document}
