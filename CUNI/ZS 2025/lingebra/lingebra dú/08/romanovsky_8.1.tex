\documentclass{article}

% Language setting
% Replace `english' with e.g. `spanish' to change the document language
\usepackage[czech]{babel}
\usepackage{amsthm,thmtools,xcolor,amsmath,amssymb, mathtools}

% Set page size and margins
% Replace `letterpaper' with `a4paper' for UK/EU standard size
\usepackage[a4paper,top=2cm,bottom=2cm,left=2cm,right=2cm,marginparwidth=1.75cm]{geometry}

% Useful packages
\usepackage{amsmath}
\usepackage{graphicx}
\usepackage[colorlinks=true, allcolors=blue]{hyperref}
\graphicspath{images/}
\usepackage{enumerate}

%\usepackage{tikzit}
%\input{default.tikzstyles}

\title{DÚ Lineární algebra -- Sada 8}
\author{Jan Romanovský}

\begin{document}
\maketitle

\textbf{(8.1)} Zapíšeme si úplně obecně jak by mohly vypadat prvky $A$ vektorového prostoru \textbf{W}.
$$A = \left( \begin{array}{c c c}
  a & b & c\\
  d & e & f\\
  g & h & i
\end{array} \right)$$
Když nějaký vektor patří do jádra matice znamená to, že řeší homogenní soustavu zadanou maticí. To nás vede na soustavu tří rovnic o devíti neznámých.

\begin{gather*}
  a + 2b + 3c = 0\\
  d + 2e + 3f = 0\\
  g + 2h + 3i = 0\\
  \,\\
  \underline{b = k}, \underline{c = l} \implies \underline{a = -2k - 3l}\\
  \underline{e = m}, \underline{f = n} \implies \undeline{d = -2m - 3n}\\
  \underline{h = o}, \underline{i = p} \implies \underline{g = -2o - 3p}
\end{gather*}

\noindent Bázové matice nyní získáme vhodnou volbou parametrů, pokaždé za jeden zvolíme 1 a za ostatní 0, to nám dá 6 matic.
$$\left\{\left( \begin{array}{c c c}
  -2 & 1 & 0\\
  0 & 0 & 0\\
  0 & 0 & 0
\end{array} \right)
\left( \begin{array}{c c c}
  -3 & 0 & 1\\
  0 & 0 & 0\\
  0 & 0 & 0
\end{array} \right)
\left( \begin{array}{c c c}
  0 & 0 & 0\\
  -2 & 1 & 0\\
  0 & 0 & 0
\end{array} \right)
\left( \begin{array}{c c c}
  0 & 0 & 0\\
  -3 & 0 & 1\\
  0 & 0 & 0
\end{array} \right)
\left( \begin{array}{c c c}
  0 & 0 & 0\\
  0 & 0 & 0\\
  -2 & 1 & 0
\end{array} \right)
\left( \begin{array}{c c c}
  0 & 0 & 0\\
  0 & 0 & 0\\
  -3 & 0 & 1
\end{array} \right)\right\}$$
\end{document}
