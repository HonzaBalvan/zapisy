\documentclass[11pt]{template/cauchy}
\usepackage[czech]{babel}

% IMPORTY
\usepackage{enumerate}
\usepackage{mlmodern}
\usepackage{template/mathsphystools}
\usepackage{amsthm,thmtools,xcolor,amsmath,amssymb, mathtools}
\usepackage[scr]{rsfso}
% \usepackage{thmstyles}
\usepackage{tabularx}
\usepackage{graphicx}
\usepackage{multicol}
\usepackage{appendix}
\usepackage{tabularx}
\usepackage{subcaption}
\usepackage{hyperref}
\usepackage{comment}
\usepackage{lscape}
\usepackage{xcolor}
\usepackage{wrapfig}
\usepackage{centernot}
\usepackage{pgfplots}
\usepackage{caption}
\usepackage{cancel}
\graphicspath{{images/}}
\pgfplotsset{width=10cm,compat=1.9}

% \usepackage{tikzit}
% \input{default.tikzstyles}
% je potřeba přiložit soubor default.tikzstyles


% STYLY
% definice, příklad, cvičení, věta, lemma, řešení, poznámka, důsledek, axiom, motivace

\declaretheoremstyle[
  headfont=\color{red}\normalfont\bfseries,
  bodyfont=\color{black}\normalfont,
]{def}

\declaretheoremstyle[
  headfont=\color{blue}\normalfont\bfseries,
  bodyfont=\color{black}\normalfont,
]{pr}

\declaretheoremstyle[
  headfont=\color{teal}\normalfont\bfseries,
  bodyfont=\color{black}\normalfont\itshape,
]{cv}

\declaretheoremstyle[
  headfont=\color{green}\normalfont\bfseries,
  bodyfont=\color{black}\normalfont\itshape,
]{veta}

\declaretheoremstyle[
  headfont=\color{lime}\normalfont\bfseries,
  bodyfont=\color{black}\normalfont\itshape,
]{lemma}

\declaretheoremstyle[
  headfont=\color{black}\normalfont\itshape,
  bodyfont=\color{black}\normalfont,
  numbered=no
]{res}

\declaretheoremstyle[
  headfont=\color{brown}\normalfont\bfseries,
  bodyfont=\color{black}\normalfont,
]{pozn}

\declaretheoremstyle[
  headfont=\color{brown}\normalfont\bfseries,
  bodyfont=\color{black}\normalfont,
]{dusl}

\declaretheoremstyle[
  headfont=\color{cyan}\normalfont\bfseries,
  bodyfont=\color{black}\normalfont,
]{axiom}

\declaretheoremstyle[
  headfont=\color{black}\normalfont\bfseries,
  bodyfont=\color{black}\normalfont,
]{motivace}

\declaretheorem[
  style=def,
  name=Definice,
]{definice}

\declaretheorem[
  style=pr,
  name=Příklad,
]{priklad}

\declaretheorem[
  style=cv,
  name=Cvičení,
]{cviceni}

\declaretheorem[
  style=res,
  name=Řešení,
]{reseni}

\declaretheorem[
  style=veta,
  name=Věta,
]{veta}

\declaretheorem[
  style=lemma,
  name=Lemma,
]{lemma}

\declaretheorem[
  style=pozn,
  name=Poznámka,
]{poznamka}

\declaretheorem[
  style=dusl,
  name=Důsledek,
]{dusledek}

\declaretheorem[
  style=axiom,
  name=Axiom,
]{axiom}

\declaretheorem[
  style=motivace,
  name=Motivace,
]{motivace}

\DeclareMathOperator{\tg}{tg}
\DeclareMathOperator{\cotg}{cotg}
\DeclareMathOperator{\arctg}{arctg}
\DeclareMathOperator{\arccotg}{arccotg}
\DeclareMathOperator{\sgn}{sgn}

\title{Úvod do sem patří název \\ {\large Zápisky z přednášky Jméno Příjmení učitele}}
\author{Jméno Příjmení}
\date{}

\begin{document}

\maketitle

% ÚVODNÍ INFORMACE
% dobré pro poznačení informačních věcí od učitele,
% např. jeho email, jména skript apod.
%!TEX root = ../main.tex
%
\phantomsection
\addcontentsline{toc}{section}{Úvodní informace}
\begin{center}
{\bfseries\Large Úvodní informace}\par\vspace{1em}
\end{center}

% SEM PATŘÍ TEXT


% ZNAČENÍ
% pokud nepotřebujete, zakomentujte řádek níže
% všechny změny provádějte v souboru additional/conventions.tex
%!TEX root = ../main.tex
%
\phantomsection
\addcontentsline{toc}{section}{Značení}
\vspace{2em}
\begin{center}
{\bfseries\Large Značení}\par\vspace{1em}
\end{center}

% SEM PATŘÍ TEXT
látka -- skládá se z částic s klidovou hmotností, nějaký objekt
pole -- neskládá se z částic, ale zprostředkuje silové působení mezi částicemi (grav., el., mag., ...)
popis pole -- klasicky pomocí fyzikální veličiny nebo kvantovš jako výměnu zprostředkujících (intermediálních) polních částic
typy polí
\begin{itemize}
\item skalární pole -- popsáno skalární veličinou v prostoru
\item vektorové pole -- popsáno vektorovou veličinou v prostoru
\item homogenní x heterogenní
\item isotropní x anisotropní
\end{itemize}

Metoda:
Hypotéza
\begin{itemize}
\item předpoklad možného stavu, domněnka, jejíž platnost není ještě plně prokázána
\item je pdoložená řadou faktů vytyčujících salší směr výzkumu
\item hyp. je testovatelná, lze potvrdit nebo vyvrátit
\end{itemize}
Teorie
\begin{itemize}
\item popisuje zákonitosti a souvislosti celé skupiny jevů v určitém oboru
\item co nejlepší přiblížení realitě, vystavěno na objektivních důkazech
\item je vnitřně konsistentní, každá teorie ale má své limity
\item na základě teorie a hypotéz vytvoříme model
\item Model -- kvalitativní, kvantitativní (matem.), nějaké zjednodušení problému
\end{itemize}
Zákon
\begin{itemize}
\item popisuje pozorování v přírodě (zatímco teorie se ho snaží vysvětlit)
\item je vždy pravdivý -- vychází z mnoha pozorování
\end{itemize}
Postulát -- Axiom
\begin{itemize}
\item je výchozí předpoklad (tvrzení), který je v dané teorii obecně přijímán jako pravdivý (a ta teorie na tom stojí)
\end{itemize}

Fyzikální veličiny
\begin{itemize}
\item extenzivní (kvantita -- hmota, náboj), itenzity (kvality -- teplota, napětí)
\item určeny velikostí a rozměrem (jednotkou): veličina = číselný údaj krát jednotka
\item soustavy jednotek jsou věc dohody
\item nejběžnější SI, dále např. absolutní -- Gaussova (cgs -- převodové konstanty (v grav., elstat. síle) položíme rovny 1, pak se dají věci navzájem odvodit)
\end{itemize}

Fyzikální veličiny -- přehled
\begin{itemize}
\item skaláry -- vyjádřeny jedním údajem (velikostí); invariantní vůči volbě souřadnicové soustavy
\item vektory -- vyjádřeny obecně $n$ složkami v $n$-D prostoru, ve 3D si můžeme představit jako orientovanou úsečku; závisí na volbě souřadnicové soustavy
\item ve fyzice se zpravidla použivá pravotočivá pravoúhlá kartézská souřadnicová soustava
\item můžeme využít také např. polární souřadnice, sférické souřadnice
\end{itemize}

Základy vektorového počtu
\begin{itemize}
\item skalární součin -- vektor $\times$ vektor $\rightarrow$ skalár; používá se pro velikosti, úhly, kolmost, ...; značíme $\textbf{a}\cdot\textbf{b}$ nebo $(\textbf{b})$
\item vektorový součin -- vektor $\times$ vektor $\rightarrow$ vektor, funguje jen ve 3D, pozor na správný směr -- pravotočivo; používá se pro určení kolmice k rovině, počítání obsahu, ...
\end{itemize}

Přehled vztahů TODO zkopírovat z prezentace; zjistit co je tenzor


% KAPITOLY
% \input{chapters/01_nazev_kapitoly}

%2025/09/29

má webové stránky -- tam je i něco ke zkoušce

1. Úvod, reálná čísla, funkce

Značení: $P, Q$ ... výroky, $x, y, z$ ... čísla (prvky), $A, B, M$ ... množiny
         $P & Q$ ... $P$ zároveň s $Q$
         $P \lor Q$ ... $P$ nebo $Q$
         $P \implies Q$ ... $P$ implikuje $Q$
         $P \iff Q$ ... $P$ je ekvivalentní s $Q$
         $\neg P$ ... negace $P$
         $\forall x$ ... pro všechna $x$
         $\exists x$ ... existuje $x$
         $\exists! x$ ... existuje právě 1 $x$
         $x \in A$ ... $x$ je prvkem $A$
         $A \subset B$ ... $A$ je podmnožina $B$
         $\{a_1, a_2, ...\}$ ... množina definována výčtem prvků $a_1, a_2, ...$
         ${x \in M, \varphi(x)}$ ... množina definována vlastností $\varphi(x)$
         $\emptyset, \{\}$ ... prázdná množina
         $A \cup B$ ... sjednocení množin $A, B$
         $A \cap B$ ... průnik množin $A$, $B$
         $A \setminus B$ ... rozdíl množin (prvky z $A$, které nejsou v $B$)
Věta A1. (Algebraické vlastnosti $\mathbb{R}$) Existuje množina reálných čísel $\mathbb{R}$, která obsahuje prvky 0 a 1, a jsou na ní definovány operace "$\cdot$" a "$+$" tak, že $\forall x, y, z \in \mathbb{R}$ platí:
\item $x+y = y+x, x \cdot y = y \cdot x$
\item $x+(y+z) = (x+y)+z, x\cdot (y\cdot z) = (x \cdot y) \cdot z$
\item $x \cdot (y+z) = x \cdot y + x \cdot z$
\item $0 + x = x$, $1 \cdot x = x$
\item

Věta A2. (Uspořádání $\mathbb{R}$) Na množině $\mathbb{R}$ je definována relace "$<$" tak, že $\forall x, y, z \in \mathbb{R}$ platí:
\item $x = y$, nebo $x < y$, nebo $y < x$
\item $x < y & y < z \implies x < z$
\item $x < y \implies x + z < y + z$
\item $0 < x & 0 < y \implies 0 < x \cdot y$

Poznámka. $x \leq y$ je zkratka za $(x < y) \lor (x = y)$. Z věty A2 opět lze vyvodit další známé poučky, např. $x^2 \geq 0 \forall x \in \mathbb R$, apod.

Poznámka. Význačné podmnožiny $\mathbb R$
\item přirozená čísla $\mathbb N = \{1, 2, 3, ...\}$
\item celá čísla $\mathbb Z = \mathbb N \cup \{0, -1, -2, ---\}$
\item racionální čísla $\mathbb Q = \{\frac{p}{q}; p \in \mathbb Z, q \in \mathbb N\}$
\item intervaly s krajními body $(a, b)$ = $\{x \in \mathbb R; a < x < b\}$, resp. ..., resp. neomezené pozor hranaté závorky! a ne zobáčky -- uvidíme

Definice. (Absolutní hodnota) Nechť $x \in \mathbb R$. Potom $x \def \begin{cases}{x, x \geq 0 \\ -x, x < 0}\end{cases}$

Lemma 1.1. Nechť $a \geq 0, b \in \mathbb R$ lib. Potom:
$$ |b| \leq a \iff -a \geq b \geq a.$$
Dúkaz. (rozborem případů)
\item $b \geq 0$
\item $b < 0$

Věta 1.1. (Trojúhelníková nerovnost) $\forall x, y \in \mathbb R:$
\item $|x \pm y| \leq |x| + |y|$
\item $|x \pm y| \geq ||x|-|y||$
Důkaz.
\item $|x| \leq |x|, |\pm y| \leq |y| \implies (L.1.1.) -|x| \leq x \leq |x|, -|y| \leq \pm y \leq |y|$, když sečtu $-|x|-|y| \leq x \pm y \leq |x|+|y| \implies (L.1.1.) |x\pm y| \leq |x| + |y|$
\item TRIK $\mp y = x - ( x \pm y)$

Věta B. (Odmocnina v $\mathbb R$)
\item Nechť  $n \in \mathbb N$ je sudé. Pak $\forall a \in [0, \infty) \exists! b \in [0, \infty)$ takové, že $b^n = a$
\item Nechť $n \in \mathbb N$ je liché. Pak $\forall a \in \mathbb R \exists! b \in \mathbb R$ takové, že $b^n = a (b = \root{n}{a})$.

Poznámka. \item \sqrt{1} = 1, \root{3}{-1} = -1, \sqrt{-1} není definována
\item \root{n}{x^n} = x \forall x \in \mathbb R, n liché
\item \sqrt{x^2} = |x|

Věta 1.2. Existují iraconální čísla

Věta A3 (Axiomy \mathbb N) \item \forall x \in \mathbb R \exists n \in \mathbb N: n > x (Archimédova vlastnost)
\item Nechť M \subset \mathbb N splňuje: \item 1 \in M \item \forall n \in \mathbb N: n \in M? n \in M \implies n+1 \in M \end{enumerate} potom M = \mathbb N. (princip indukce

Poznámka. alternativní ekvivalentní formulace: \item \forall \varepsilon > 0 \exists n \in \mathbb N: n\varepsilon > 1
\item Nechť \phi (n) je formule s proměnnou n \in \mathbb N, nechť n_0 \in \mathbb N. Nechť platí: \item \phi(n) \item \forall n \geq n_0: \phi (n) \implies \phi (n+1) \end{enumerate} potom platí \phi (n) \forall n \geq n_0.

Poznámka. indukce ještě jinak: \forall M \subset \mathbb N, M \neq \emptyset: M má nejmenší prvek.


01/10/2025

TODO k větě 1.2. doplnit důkaz o iracionálnosti odmocniny ze 3

Věta 1.3. Každý otevřený interval obsahuje nekonečně mnoho racionálních a iracionálních čísel.
Důkaz. (pro iracionální)
BÚNO: I = (a,b), 0 \leq a < b
položme x_n = \frac{n\sqrt{3}}{m}, n \in \mathbb N, m \in \mathbb N velké tak, že m > \frac{2\sqrt{3}}{b-a}\iff \frac{2\sqrt{3}}{m} < b-a
M = \{n \in \mathbb N:x_n \geq b\}, nechť n_0 je nejmenší prvek M (viz A3)
tvrdíme: x_{n_0-1}, x_{n_0-2}\in (a,b), tj. a<x_{n_0-2}<x_{n_0-1}<b
zřejmě x_n \nin \mathbb Q, \forall n \neq 0
a<\frac{n_0-2}{m} \sqrt{3}=\frac{n_0}{m} \sqrt{3} - \frac{2\sqrt{3}}{m}, první zlomek \geq b, druhý < b-a \implies >a

Definice. Nechť M \subset \mathbb R. Potom
\begin{itemize}
\item x \in M nazveme maximum (největší prvek) M, pokud \forall y \in M: y \leq x
\item x \in M nazveme minimum (nejmenší prvek) M, pokud \forall y \in M: y \geq x
\item K \in \mathbb R nazveme horní odhad M, jestliže \forall x \in M: x \leq K
\item K \in \mathbb R nazveme dolní odhad M, jestliže \forall x \in M: x \geq K
\end{itemize}
Množina M se nazve
\begin{itemize}
\item shora omezená, má-li nějaký horní odhad
\item zdola omezená, má-li nějaký dolní odhad
\item omezená, má-li horní i dolní odhad.
\end{itemize}

Příklad. \item M = [0,1) \item \mathbb N = \{1, 2, 3, ...\}

Definice. Nechť M \in \mathbb R, Číslo S \in \mathbb R nazveme supremum M, značíme S = sup M, jestliže:
\begin{enumerate}[i.]
\item \forall x \in M: x \leq S
\item \forall S^\prime <S: \exists y \in M: y > S^\prime
\end{enumerate}
čili S je nejmenší horní odhad M.

Poznámka. \begin{itemize}
\item Supremum užitečně zobecňuje pojem maximum.
\item Existuje nejvýše jedno sup M.

Věta A4. (Úplnost \mathbb R) Nechť M \subset \mathbb R je neprázdná a shora omezená. Pak \exists! S \in \mathbb R takové, že S = sup M.

Definice. Nechť M \in \mathbb R, Číslo s \in \mathbb R nazveme infimum M, značíme s = inf M, jestliže:
\begin{enumerate}[i.]
\item \forall x \in M: x \geq s
\item \forall s^\prime >s: \exists y \in M: y < s^\prime
\end{enumerate}
čili s je největší dolní odhad M.

Věta A4^\prime. Nechť M \subset \mathbb R je neprázdná a zdola omezená. Pak \exists! s \in \mathbb R takové, že s = inf M.

Definice. (Rozšířená reálná čísla) \mathbb R^* = \mathbb R + \{- \infty, + \infty\}. Uspořádání a aritmetika v \mathbb R^*:
\begin{itemize}
\item \forall x \in \mathbb R: -\infty < x < +\infty, -\infty < +\infty
\item \forall x \in \mathbb R: x+\infty=+\infty, x-\infty=-\infty, +\infty+\infty=+\infty, -\infty-\infty=-\infty
\item \forall x > 0 x \cdot (\pm \infty) = \pm \infty, \pm \infty \cdot (+\infty) = \pm \infty, \pm \infty \cdot (-\infty) = \mp \infty
\item \forall x < 0 x \cdot (\pm \infty) = \mp \infty
\item \forall x \in \mathbb R: \frac{x}{\pm \infty} = 0
\end{itemize}
Nedefinováno zůstává + \infty -\infty, -\infty+\infty, 0 \cdot( \pm \infty), \frac{x}{0}, \frac{\pm \infty}{\pm \infty}

Definice. Nechť X, Y množiny. Funkce f: X \rightarrow Y je libovolný předpis, který každému x \in \mathbb X přiřadí jednozačně určený prvek z Y. Dále definujme
\begin{itemize}
\item obraz množiny M \subset X: f(M) = \{f(x); x \in M\}
\item vzor množiny N \subset Y: f^{-1}(N) = \{x, f(x) \in N\} (tadyto -1 není inverzní funkce, toto můžu říct i pro nezinvertovatelnou funkci)
\end{itemize}
Funkce je
\begin{itemize}
\item prostá, pokud x \neq y \implies f(x) \neq f(y)
\item "na", pokud f(X)=Y, tj. \forall y \in Y \exists x \in X: f(x)=y
\item je vzájemně jednoznačná (1-1), je-li prostá i "na" (tady lze dej. inv. fci)

TODO složené zobrazení def. obor?, inverzní a invertovatelná funkce

\section{2. Reálné funkce -- limita a spojitost}

Úmluva. reálné fce $f(x): \mathbb R \rightarrow \mathbb R$ (tj. nikoliv $\mathbb R^*$)
        $\exists \delta > 0$ ... $\exists \delta \in (0, + \infty)$, tj. $\delta \neq \pm \infty$

Příklad. $f(x) = \frac{1}{x}$ ... $D_f = \mathbb R \smallsetminus \{0\}$, nikoliv $\frac{1}{\pm \infty} = 0

Terminologie. $f(x)$ se nazve na množině M:
\begin{enumerate}[i.]
\item rostoucí, pokud $\forall x, y \in M: x<y \implies f(x) < f(y)$
\item klesající, pokud $\forall x, y \in M: x<y \implies f(x) > f(y)$
\item neklesající, pokud $\forall x, y \in M: x<y \implies f(x) \leq f(y)$
\item nerostoucí, pokud $\forall x, y \in M: x<y \implies f(x) \geq f(y)$
\end{enumerate}
Tyto funkce nazveme monotónní, dále fce z bodů i. a ii. nazveme ryze monotonní na množině M.

Funkce jsou dále sudé, liché, periodické, ...

Definice. f(x) se nazve na množině M
\begin{enumerate}[i.]
\item shora omezená, pokud $\exists K \in \mathbb R \forall x \in M: f(x) \leq K$
\item zdola omezená, pokud $\exists L \in \mathbb R \forall x \in M: f(x) \geq L$
\item omezená, je-li shora i zdola omezená, tj. $\exists K, L \in \mathbb R \forall x \in M: L \leq f(x) \leq K$. (tady je ekviv. tvrzení s abs. hodnotou a tohoto komicky dlouhý důkaz)
\end{enumerate}

Definice. (Okolí bodu) Nechť $\delta > 0, x \in \mathbb R$. Pak
\begin{enumerate}[i.]
\item $\mathscr U (x_0, \delta) = (x_0 - \delta, x_0 + \delta)$ nazveme kruhové $\delta$-okolí bodu $x_0$
\item $\mathscr P (x_0, \delta) = (x_0 - \delta, x_0 + \delta) \smallsetminus \{x_0\}$ nazveme prstencové $\delta$-okolí bodu $x_0$
\item $\mathscr U_+ (x_0, \delta) = [x_0, x_0 + \delta)$ nazveme pravé kruhové $\delta$-okolí bodu $x_0$
\item $\mathscr U_- (x_0, \delta) = (x_0 - \delta, x_0]$ nazveme levé kruhové $\delta$-okolí bodu $x_0$
\item $\mathscr U_+ (x_0, \delta) = (x_0, x_0 + \delta)$ nazveme pravé prstencové $\delta$-okolí bodu $x_0$
\item $\mathscr U_- (x_0, \delta) = (x_0 - \delta, x_0)$ nazveme levé prstencové $\delta$-okolí bodu $x_0$
\end{enumerate}
Dále definujme
\begin{enumerate}[i.]
\item $\mathscr U (+\infty, \delta) = (\frac{1}{\delta}, +\infty]$
\item $\mathscr P (+\infty, \delta) = (\frac{1}{\delta}, +\infty)$
\item ... -\infty, levé pravé
\end{enumerate}

Poznámka. \begin{itemize}
\item $\delta_1 < \delta_2 \implies \mathscr (x_0, \delta_1) \subset \mathscr U (x_0, \delta_2)$
\item pro $x_0 \in \mathbb R$ platí: $\mathscr U (x_0, \delta) = \{x; |x-x_0| < \delta\}$ a $\mathscr P (x_0, \delta) = \{x; 0 < |x-x_0| < \delta\}$
\item zkrácený zápis $\mathscr U(x_0), \mathscr P(x_0)$

Věta 2.1. (Princip oddělení) Nechť $x_0, x_1 \in \mathbb R^*, x_0 \neq x_1$ t. ž. $\mathscr U (x_0, \delta) \cap \mathscr (x_1, \delta) = \emptyset
Důkaz. BÚNO $x_0<x_1$, rozlišme případy:
\begin{enumerate}
\item $x_0, x_1 \in \mathbb R$: stačí, aby $x_0 + \delta < x_1 - \delta$, tedy $\delta < \frac{x_1 - x_0}{2}$
\item $x_0 = \-infty, x_1 \in \mathbb R$: stačí, aby $\frac{-1}{\delta} < x_1 - \delta$, tedy ... (rozdělím na případy x)
\item $x_0 \in \mathbb R, \x_1 = + \infty$: analogicky k ii.
\item $x_0 = - \infty, x_1 = + \infty$: platí vždy
\end{enumerate}

Poznámka. Dokázali jsme navíc, že $\mathscr U(x_0, \delta)$ je vlevo od $\mathscr U (x_1, \delta)$, tj. $\forall x \in \mathscr U(x_0, \delta), \forall y \in \mathscr U (x_1, \delta): x<y$.

Definice. Nechť $x_0 \in \mathbb R^*, f(x)$ je definována na jistém $\mathscr P(x_0). Číslo $A \in \mathbb R^*$ nazveme limitou $f(x)$ v bodě $x_0$, jestliže:
$$\forall \epsilon >0 \exists \delta >0: x \in \mathscr P(x_0, \delta) \imlplies f(x) \in \mathscr U (A, \epsilon)$$
Značíme $f(x) \rightarrow A, x \rightarrow x_0$ nebo $\lim_{x \rightarrow x_0}{f(x))}=A$. Pokud $A \in \mathbb R$ nazveme limitu vlastní, pokud $A = \pm infty$ nazveme limitu nevlastní.

Poznámka. \begin{itemize}
\item limita nezávisí na $f(x_0)$, nemusí být definováno
\item alternativní zápisy \begin{itemize}
\item $\forall \epsilon > 0 \exists \delta > 0: f(\mathscr P(x_0, \delta)) \subseteq \mathscr U (A, \epsilon)$
\item $\forall \epsilon > 0 \exists \delta > 0: 0 < |x-x_0| < \delta \implies |f(x) - A| < \epsilon$ pokud $x_0, A \in \mathbb R$
\end{itemize}
\item pokud limita existuje, je právě jedna
\end{itemize}
Důkaz. TODO

Příklad. Dirichletova funkce
$$D(x) \def \begin{cases} 1, x \in \mathbb Q \\ 0, x \in \mathbb R \smallsetminus \mathbb Q \end{cases}$$
tvrdíme: $\nexists lim_{x \rightarrow x_0}{D(x)}, x_0 \in \mathbb R^*$ lib.
Důkaz. $x_0, A \in \mathbb R^*$ ... $\neg (D(x) \rightarrow A, x \rightarrow x_0)
       $\exists \epsilon > 0 \forall \delta > 0: f(\mathscr P(x_0, \delta)) \nsubseteq \mathscr U(A, \epsilon)
       stačí zvolit $\epsilon$ tak, že $\mathscr U(A, \epsilon)$ neobsahuje buď 0, nebo 1, protože H(f) = \{0,1\} (lze viz věta 2.1.)

\section{Úvod, reálná čísla, funkce}

\begin{poznamka} Značení:
  \begin{itemize}
    \item $P, Q$ \dots výroky, $x, y, z$ \dots čísla (prvky), $A, B, M$ \dots množiny
    \item $P & Q$ \dots $P$ zároveň s $Q$
    \item $P \lor Q$ \dots $P$ nebo $Q$
    \item $P \implies Q$ \dots $P$ implikuje $Q$
    \item $P \iff Q$ \dots $P$ je ekvivalentní s $Q$
    \item $\neg P$ \dots negace $P$
    \item $\forall x$ \dots pro všechna $x$
    \item $\exists x$ \dots existuje $x$
    \item $\exists! x$ \dots existuje právě 1 $x$
    \item $x \in A$ \dots $x$ je prvkem $A$
    \item $A \subset B$ \dots $A$ je podmnožina $B$
    \item $\{a_1, a_2, \dots\}$ \dots množina definována výčtem prvků $a_1, a_2, \dots$
    \item ${x \in M, \varphi(x)}$ \dots množina definována vlastností $\varphi(x)$
    \item $\emptyset, \{\}$ \dots prázdná množina
    \item $A \cup B$ \dots sjednocení množin $A, B$
    \item $A \cap B$ \dots průnik množin $A$, $B$
    \item $A \setminus B$ \dots rozdíl množin (prvky z $A$, které nejsou v $B$)
  \end{itemize}
\end{poznamka}

\begin{axiom}[Algebraické vlastnosti $\mathbb{R}$]
  Existuje množina reálných čísel $\mathbb{R}$, která obsahuje prvky 0 a 1, a jsou na ní definovány operace "$\cdot$" a "$+$" tak, že $\forall x, y, z \in \mathbb{R}$ platí:
  \begin{enumerate}[i.]
    \item $x+y = y+x, x \cdot y = y \cdot x$
    \item $x+(y+z) = (x+y)+z, x\cdot (y\cdot z) = (x \cdot y) \cdot z$
    \item $x \cdot (y+z) = x \cdot y + x \cdot z$
    \item $0 + x = x$, $1 \cdot x = x$
    \item $0 \cdot x = 0$ a naopak $x \cdot y = 0 \implies x = 0 \lor y = 0$
  \end{enumerate}
\end{axiom}

\begin{axiom}[Uspořádání $\mathbb{R}$]
  Na množině $\mathbb{R}$ je definována relace "$<$" tak, že $\forall x, y, z \in \mathbb{R}$ platí:
  \begin{enumerate}[i.]
    \item $x = y$, nebo $x < y$, nebo $y < x$
    \item $x < y & y < z \implies x < z$
    \item $x < y \implies x + z < y + z$
    \item $0 < x & 0 < y \implies 0 < x \cdot y$
  \end{enumerate}
\end{axiom}

\begin{poznamka}
  $x \leq y$ je zkratka za $(x < y) \lor (x = y)$. Z věty A2 opět lze vyvodit další známé poučky, např. $x^2 \geq 0 \forall x \in \mathbb R$, apod.
\end{poznamka}

\begin{poznamka}[Význačné podmnožiny $\mathbb R$]
  \begin{itemize}
    \item přirozená čísla $\mathbb N = \{1, 2, 3, \dots\}$
    \item celá čísla $\mathbb Z = \mathbb N \cup \{0, -1, -2, ---\}$
    \item racionální čísla $\mathbb Q = \{\frac{p}{q}; p \in \mathbb Z, q \in \mathbb N\}$
    \item intervaly s krajními body $(a, b)$ = $\{x \in \mathbb R; a < x < b\}$, resp. \dots, resp. neomezené pozor hranaté závorky! a ne zobáčky -- uvidíme
  \end{itemize}
\end{poznamka}

\begin{definice}[Absolutní hodnota]
  Nechť $x \in \mathbb R$. Potom $x \def \begin{cases}{x, x \geq 0 \\ -x, x < 0}\end{cases}$
\end{definice}

\begin{lemma}
  Nechť $a \geq 0, b \in \mathbb R$ lib. Potom:
  $$ |b| \leq a \iff -a \geq b \geq a.$$
\end{lemma}

\begin{proof}
  (rozborem případů)
  \item $b \geq 0$
  \item $b < 0$
\end{proof}

\begin{veta}[Trojúhelníková nerovnost]
  $\forall x, y \in \mathbb R:$
  \begin{itemize}
    \item $|x \pm y| \leq |x| + |y|$
    \item $|x \pm y| \geq ||x|-|y||$
  \end{itemize}
\end{veta}

\begin{proof}
  \begin{itemize}
    \item $|x| \leq |x|, |\pm y| \leq |y| \implies (L.1.1.) -|x| \leq x \leq |x|, -|y| \leq \pm y \leq |y|$, když sečtu $-|x|-|y| \leq x \pm y \leq |x|+|y| \implies (L.1.1.) |x\pm y| \leq |x| + |y|$
    \item TRIK $\mp y = x - ( x \pm y)$
  \end{itemize}
\end{proof}

Věta B. (Odmocnina v $\mathbb R$)
\item Nechť  $n \in \mathbb N$ je sudé. Pak $\forall a \in [0, \infty) \exists! b \in [0, \infty)$ takové, že $b^n = a$
\item Nechť $n \in \mathbb N$ je liché. Pak $\forall a \in \mathbb R \exists! b \in \mathbb R$ takové, že $b^n = a (b = \root{n}{a})$.

Poznámka. \item \sqrt{1} = 1, \root{3}{-1} = -1, \sqrt{-1} není definována
\item \root{n}{x^n} = x \forall x \in \mathbb R, n liché
\item \sqrt{x^2} = |x|

Věta 1.2. Existují iraconální čísla

Věta A3 (Axiomy \mathbb N) \item \forall x \in \mathbb R \exists n \in \mathbb N: n > x (Archimédova vlastnost)
\item Nechť M \subset \mathbb N splňuje: \item 1 \in M \item \forall n \in \mathbb N: n \in M? n \in M \implies n+1 \in M \end{enumerate} potom M = \mathbb N. (princip indukce

Poznámka. alternativní ekvivalentní formulace: \item \forall \varepsilon > 0 \exists n \in \mathbb N: n\varepsilon > 1
\item Nechť \phi (n) je formule s proměnnou n \in \mathbb N, nechť n_0 \in \mathbb N. Nechť platí: \item \phi(n) \item \forall n \geq n_0: \phi (n) \implies \phi (n+1) \end{enumerate} potom platí \phi (n) \forall n \geq n_0.

Poznámka. indukce ještě jinak: \forall M \subset \mathbb N, M \neq \emptyset: M má nejmenší prvek.


01/10/2025

TODO k větě 1.2. doplnit důkaz o iracionálnosti odmocniny ze 3

Věta 1.3. Každý otevřený interval obsahuje nekonečně mnoho racionálních a iracionálních čísel.
Důkaz. (pro iracionální)
BÚNO: I = (a,b), 0 \leq a < b
položme x_n = \frac{n\sqrt{3}}{m}, n \in \mathbb N, m \in \mathbb N velké tak, že m > \frac{2\sqrt{3}}{b-a}\iff \frac{2\sqrt{3}}{m} < b-a
M = \{n \in \mathbb N:x_n \geq b\}, nechť n_0 je nejmenší prvek M (viz A3)
tvrdíme: x_{n_0-1}, x_{n_0-2}\in (a,b), tj. a<x_{n_0-2}<x_{n_0-1}<b
zřejmě x_n \nin \mathbb Q, \forall n \neq 0
a<\frac{n_0-2}{m} \sqrt{3}=\frac{n_0}{m} \sqrt{3} - \frac{2\sqrt{3}}{m}, první zlomek \geq b, druhý < b-a \implies >a

Definice. Nechť M \subset \mathbb R. Potom
\begin{itemize}
\item x \in M nazveme maximum (největší prvek) M, pokud \forall y \in M: y \leq x
\item x \in M nazveme minimum (nejmenší prvek) M, pokud \forall y \in M: y \geq x
\item K \in \mathbb R nazveme horní odhad M, jestliže \forall x \in M: x \leq K
\item K \in \mathbb R nazveme dolní odhad M, jestliže \forall x \in M: x \geq K
\end{itemize}
Množina M se nazve
\begin{itemize}
\item shora omezená, má-li nějaký horní odhad
\item zdola omezená, má-li nějaký dolní odhad
\item omezená, má-li horní i dolní odhad.
\end{itemize}

Příklad. \item M = [0,1) \item \mathbb N = \{1, 2, 3, \dots$\}

Definice. Nechť M \in \mathbb R, Číslo S \in \mathbb R nazveme supremum M, značíme S = sup M, jestliže:
\begin{enumerate}[i.]
\item \forall x \in M: x \leq S
\item \forall S^\prime <S: \exists y \in M: y > S^\prime
\end{enumerate}
čili S je nejmenší horní odhad M.

Poznámka. \begin{itemize}
\item Supremum užitečně zobecňuje pojem maximum.
\item Existuje nejvýše jedno sup M.

Věta A4. (Úplnost \mathbb R) Nechť M \subset \mathbb R je neprázdná a shora omezená. Pak \exists! S \in \mathbb R takové, že S = sup M.

Definice. Nechť M \in \mathbb R, Číslo s \in \mathbb R nazveme infimum M, značíme s = inf M, jestliže:
\begin{enumerate}[i.]
\item \forall x \in M: x \geq s
\item \forall s^\prime >s: \exists y \in M: y < s^\prime
\end{enumerate}
čili s je největší dolní odhad M.

Věta A4^\prime. Nechť M \subset \mathbb R je neprázdná a zdola omezená. Pak \exists! s \in \mathbb R takové, že s = inf M.

Definice. (Rozšířená reálná čísla) \mathbb R^* = \mathbb R + \{- \infty, + \infty\}. Uspořádání a aritmetika v \mathbb R^*:
\begin{itemize}
\item \forall x \in \mathbb R: -\infty < x < +\infty, -\infty < +\infty
\item \forall x \in \mathbb R: x+\infty=+\infty, x-\infty=-\infty, +\infty+\infty=+\infty, -\infty-\infty=-\infty
\item \forall x > 0 x \cdot (\pm \infty) = \pm \infty, \pm \infty \cdot (+\infty) = \pm \infty, \pm \infty \cdot (-\infty) = \mp \infty
\item \forall x < 0 x \cdot (\pm \infty) = \mp \infty
\item \forall x \in \mathbb R: \frac{x}{\pm \infty} = 0
\end{itemize}
Nedefinováno zůstává + \infty -\infty, -\infty+\infty, 0 \cdot( \pm \infty), \frac{x}{0}, \frac{\pm \infty}{\pm \infty}

Definice. Nechť X, Y množiny. Funkce f: X \rightarrow Y je libovolný předpis, který každému x \in \mathbb X přiřadí jednozačně určený prvek z Y. Dále definujme
\begin{itemize}
\item obraz množiny M \subset X: f(M) = \{f(x); x \in M\}
\item vzor množiny N \subset Y: f^{-1}(N) = \{x, f(x) \in N\} (tadyto -1 není inverzní funkce, toto můžu říct i pro nezinvertovatelnou funkci)
\end{itemize}
Funkce je
\begin{itemize}
\item prostá, pokud x \neq y \implies f(x) \neq f(y)
\item "na", pokud f(X)=Y, tj. \forall y \in Y \exists x \in X: f(x)=y
\item je vzájemně jednoznačná (1-1), je-li prostá i "na" (tady lze dej. inv. fci)

TODO složené zobrazení def. obor?, inverzní a invertovatelná funkce

%\section{Reálné funkce -- limita a spojitost}

Úmluva. reálné fce $f(x): \mathbb R \rightarrow \mathbb R$ (tj. nikoliv $\mathbb R^*$)
        $\exists \delta > 0$ ... $\exists \delta \in (0, + \infty)$, tj. $\delta \neq \pm \infty$

Příklad. $f(x) = \frac{1}{x}$ ... $D_f = \mathbb R \smallsetminus \{0\}$, nikoliv $\frac{1}{\pm \infty} = 0

Terminologie. $f(x)$ se nazve na množině M:
\begin{enumerate}[i.]
\item rostoucí, pokud $\forall x, y \in M: x<y \implies f(x) < f(y)$
\item klesající, pokud $\forall x, y \in M: x<y \implies f(x) > f(y)$
\item neklesající, pokud $\forall x, y \in M: x<y \implies f(x) \leq f(y)$
\item nerostoucí, pokud $\forall x, y \in M: x<y \implies f(x) \geq f(y)$
\end{enumerate}
Tyto funkce nazveme monotónní, dále fce z bodů i. a ii. nazveme ryze monotonní na množině M.

Funkce jsou dále sudé, liché, periodické, ...

Definice. f(x) se nazve na množině M
\begin{enumerate}[i.]
\item shora omezená, pokud $\exists K \in \mathbb R \forall x \in M: f(x) \leq K$
\item zdola omezená, pokud $\exists L \in \mathbb R \forall x \in M: f(x) \geq L$
\item omezená, je-li shora i zdola omezená, tj. $\exists K, L \in \mathbb R \forall x \in M: L \leq f(x) \leq K$. (tady je ekviv. tvrzení s abs. hodnotou a tohoto komicky dlouhý důkaz)
\end{enumerate}

Definice. (Okolí bodu) Nechť $\delta > 0, x \in \mathbb R$. Pak
\begin{enumerate}[i.]
\item $\mathscr U (x_0, \delta) = (x_0 - \delta, x_0 + \delta)$ nazveme kruhové $\delta$-okolí bodu $x_0$
\item $\mathscr P (x_0, \delta) = (x_0 - \delta, x_0 + \delta) \smallsetminus \{x_0\}$ nazveme prstencové $\delta$-okolí bodu $x_0$
\item $\mathscr U_+ (x_0, \delta) = [x_0, x_0 + \delta)$ nazveme pravé kruhové $\delta$-okolí bodu $x_0$
\item $\mathscr U_- (x_0, \delta) = (x_0 - \delta, x_0]$ nazveme levé kruhové $\delta$-okolí bodu $x_0$
\item $\mathscr U_+ (x_0, \delta) = (x_0, x_0 + \delta)$ nazveme pravé prstencové $\delta$-okolí bodu $x_0$
\item $\mathscr U_- (x_0, \delta) = (x_0 - \delta, x_0)$ nazveme levé prstencové $\delta$-okolí bodu $x_0$
\end{enumerate}
Dále definujme
\begin{enumerate}[i.]
\item $\mathscr U (+\infty, \delta) = (\frac{1}{\delta}, +\infty]$
\item $\mathscr P (+\infty, \delta) = (\frac{1}{\delta}, +\infty)$
\item ... -\infty, levé pravé
\end{enumerate}

Poznámka. \begin{itemize}
\item $\delta_1 < \delta_2 \implies \mathscr (x_0, \delta_1) \subset \mathscr U (x_0, \delta_2)$
\item pro $x_0 \in \mathbb R$ platí: $\mathscr U (x_0, \delta) = \{x; |x-x_0| < \delta\}$ a $\mathscr P (x_0, \delta) = \{x; 0 < |x-x_0| < \delta\}$
\item zkrácený zápis $\mathscr U(x_0), \mathscr P(x_0)$

Věta 2.1. (Princip oddělení) Nechť $x_0, x_1 \in \mathbb R^*, x_0 \neq x_1$ t. ž. $\mathscr U (x_0, \delta) \cap \mathscr (x_1, \delta) = \emptyset
Důkaz. BÚNO $x_0<x_1$, rozlišme případy:
\begin{enumerate}
\item $x_0, x_1 \in \mathbb R$: stačí, aby $x_0 + \delta < x_1 - \delta$, tedy $\delta < \frac{x_1 - x_0}{2}$
\item $x_0 = \-infty, x_1 \in \mathbb R$: stačí, aby $\frac{-1}{\delta} < x_1 - \delta$, tedy ... (rozdělím na případy x)
\item $x_0 \in \mathbb R, \x_1 = + \infty$: analogicky k ii.
\item $x_0 = - \infty, x_1 = + \infty$: platí vždy
\end{enumerate}

Poznámka. Dokázali jsme navíc, že $\mathscr U(x_0, \delta)$ je vlevo od $\mathscr U (x_1, \delta)$, tj. $\forall x \in \mathscr U(x_0, \delta), \forall y \in \mathscr U (x_1, \delta): x<y$.

Definice. Nechť $x_0 \in \mathbb R^*, f(x)$ je definována na jistém $\mathscr P(x_0). Číslo $A \in \mathbb R^*$ nazveme limitou $f(x)$ v bodě $x_0$, jestliže:
$$\forall \epsilon >0 \exists \delta >0: x \in \mathscr P(x_0, \delta) \imlplies f(x) \in \mathscr U (A, \epsilon)$$
Značíme $f(x) \rightarrow A, x \rightarrow x_0$ nebo $\lim_{x \rightarrow x_0}{f(x))}=A$. Pokud $A \in \mathbb R$ nazveme limitu vlastní, pokud $A = \pm infty$ nazveme limitu nevlastní.

Poznámka. \begin{itemize}
\item limita nezávisí na $f(x_0)$, nemusí být definováno
\item alternativní zápisy \begin{itemize}
\item $\forall \epsilon > 0 \exists \delta > 0: f(\mathscr P(x_0, \delta)) \subseteq \mathscr U (A, \epsilon)$
\item $\forall \epsilon > 0 \exists \delta > 0: 0 < |x-x_0| < \delta \implies |f(x) - A| < \epsilon$ pokud $x_0, A \in \mathbb R$
\end{itemize}
\item pokud limita existuje, je právě jedna
\end{itemize}
Důkaz. TODO

Příklad. Dirichletova funkce
$$D(x) \def \begin{cases} 1, x \in \mathbb Q \\ 0, x \in \mathbb R \smallsetminus \mathbb Q \end{cases}$$
tvrdíme: $\nexists lim_{x \rightarrow x_0}{D(x)}, x_0 \in \mathbb R^*$ lib.
Důkaz. $x_0, A \in \mathbb R^*$ ... $\neg (D(x) \rightarrow A, x \rightarrow x_0)
       $\exists \epsilon > 0 \forall \delta > 0: f(\mathscr P(x_0, \delta)) \nsubseteq \mathscr U(A, \epsilon)
       stačí zvolit $\epsilon$ tak, že $\mathscr U(A, \epsilon)$ neobsahuje buď 0, nebo 1, protože H(f) = \{0,1\} (lze viz věta 2.1.)

\input{chapters/03_}
\input{chapters/04_}
\input{chapters/05_}
\input{chapters/06_hlubsi_vlastnosti}
\section{Posloupnosti}

\begin{definice}
  Posloupnost je funkce $a: \mathbb N \to \mathbb R, n \mapsto a_n$, značíme $\{a_n\}_{n\in\mathbb N}$ nebo $\{a_n\}_{n=n_0}^{\infty}$, krátce $\{a_n\}$.
\end{definice}

\begin{priklad}
  \begin{enumerate}[1.]
    \item $a_n = \frac{1}{n}, b_n = (-1)^n$
    \item rekurentně: $a_1 = 0, a_{n+1} = \sqrt{2+a_n}$
  \end{enumerate}
\end{priklad}

\begin{definice}
  Číslo $a\in\mathbb R^*$ se nazve limita posloupnost $\{a_n\}$, pokud
  $$\forall \varepsilon > 0 \,\exists n_0 \in \mathbb N: n \geq n_0 \implies a_n \in \mathscr U(a, \varepsilon).$$
  Značíme $a_n \to a$ nebo $\lim_{n \to+\infty} a_n = a$.
\end{definice}

\begin{poznamka}
  $a \in \mathbb R \dots \{a_n\}$ konverguje

  $a = \pm+\infty \dots \{a_n\}$ diverguje
\end{poznamka}

\begin{poznamka}[Ekvivalentní definice (pokud $a \in \mathbb R$)]
  $$\forall \varepsilon > 0 \,\exists n_0 \in \mathbb N: n \geq n_0 \implies |a_n-a|<\varepsilon$$
  Obecně: $a_n \to a \iff \forall \varepsilon > 0$ platí $a_n \in \mathscr U(a, \varepsilon)$ pro všechna $n$ až na konečně mnoho výjimek
\end{poznamka}

\begin{proof}
  \begin{itemize}
    \item[\uv{$\implies$}] výjimečně jen pro $n < n_0$ (těch je konečně)
    \item[\uv{$\impliedby$}] $V = \left\{n; a_n \notin \mathscr U(a, \varepsilon)\right\} \subset \mathbb N$ konečné

    volme $n_0 > \max V \in \mathbb N$
  \end{itemize}
\end{proof}

\begin{poznamka}
  Pro limity posloupností platí analogie vět pro limity funkcí s analogickými důkazy.
  \begin{enumerate}[i.]
    \item (VoAL, V. 2. 3., V. 2. 7.) $a_n \to a, b_n \to b \implies$ \begin{enumerate}[1.]
      \item $a_n \pm b_n \to a \pm b$
      \item $a_n \cdot b_n \to a \cdot b$
     \item $\frac{a_n}{b_n} \to \frac{a}{b}$, má-li PS smysl
    \end{enumerate}
    \item (V. 2. 9.) je-li $\alpha \leq a_n \leq \beta$ od jistého $n_0, a_n \to a \implies \alpha \leq a \leq \beta$
    \item (V. 2. 10.) je-li $b_n \leq a_n \leq c_n$ od jistého $n_0, b_n \to a, c_n \to a \implies a_n \to a$
    \item nechť $a_n \to 0$, nechť $a_n > (\text{resp. } <)$ $ 0$ od jistého $n_0 \implies \frac{1}{a_n} \to +\infty$ (resp. $-\infty$)
  \end{enumerate}
\end{poznamka}

\begin{proof}
  cíl: $\forall \varepsilon > 0 \exists n_0: n \geq n_0 \implies \frac{1}{a_n} \in \mathscr U(+\infty, \varepsilon)$, tj. $\frac{1}{a_n} > \frac{1}{\varepsilon}$
  \begin{align*}
    \varepsilon >0\text{ dáno:}&\exists n_1 \in \mathbb N: n \geq n_1 \implies a_n \in \mathscr U(0, \varepsilon)\text{, tj. }|a_n|<\varepsilon \implies -\varepsilon < a_n < \varepsilon\\
    &\exists n_2 \in \mathbb N: n \geq n_2 \implies a_n > 0 \implies 0 < a_n < \varepsilon
  \end{align*}
  polož $n_0 \coloneq \max \{n_1, n_2\}$; nechť $n\geq n_0 \implies$ cíl
\end{proof}

\begin{definice}
  Posloupnost se nazve omezená (resp. shora omezená, zdola omezená), pokud $\exists K > 0$ (resp. $K \in \mathbb R$) t. ž. $|a_n| \leq K$ (resp. $a_n \leq K, a_n \geq K$) $\forall n \in \mathbb N$. Posloupnost se nazve rostoucí (resp. neklesající, klesající, nerostoucí) pokud $a_n < a_{n+1}$ (resp. $a_n \leq a_{n+1}, a_n > a_{n+1}, a_n \geq a_{n+1}$) $\forall n \in \mathbb N$. Všechny tyto posloupnosti zároveň nazveme monotónní.
\end{definice}

\begin{veta}
  Nechť $\{a_n\}$ konverguje. Pak $\{a_n\}$ je omezené.
\end{veta}

\begin{proof}
  víme: $\exists a \in \mathbb R$ t. ž. $a_n \to a \dots \varepsilon =1, \exists n_0 \,\forall n \geq n_0: |a_n-a|<1 \iff L = a-1 < a_n < a+1 = K$

  polož $C \coloneq \max\{K, -L\} \implies |a_n|\leq C, \forall n \geq n_0$, to není $\forall n \in \mathbb N$

  položme $\tilde{C} \coloneq \max \{|a_1|, |a_2|, \dots, |a_{n-1}|, C\}$

  nyní zřejmě: $|a_n| \leq \tilde C \forall n \geq 1$ tj. $\forall n \in \mathbb N$.
\end{proof}

\begin{veta}
  Nechť $\{a_n\}$ je monotónní. Pak $\exists a \in \mathbb R^*$ t. ž. $a_n \to a$. Je-li $\{a_n\}$ omezená, pak $a\in \mathbb R$, tj. $\{a_n\}$ konverguje.
\end{veta}

\begin{proof}
  BÚNO $\{a_n\}$ je neklesající, tj. $a_n \leq a_{n+1} \,\forall n \in \mathbb N$, a tedy $a_n \leq a_m \,\forall n \leq m$

  polož $M \coloneq \{a_m; m \in \mathbb R\} \subset \mathbb R$.
  \begin{enumerate}[1.]
    \item nechť $M$ je omezená ($\iff \{ a_n \}$ je omezená); $S \coloneq \sup M$ (A. 4., $M \neq \emptyset $ )
          ukážeme, že $a_n \to S$

          $\varepsilon >0$ dáno: $S^\prime \coloneq S - \varepsilon < S \dots \exists n_0: a_{n_0}>S-\varepsilon$

          $\{a_n\}$ neklesající: $a_n > S - \varepsilon \forall n \geq n_0$, zároveň $a_n \leq S < S + \varepsilon$

          $\implies a_n \in \mathscr U(S, \varepsilon), \forall n \geq n_0$.
          \item nechť $M$ je neomezená ($\iff \{a_n\}$ je neomezená, nutně shora (protože je neklesající))

          ukážeme: $a_n \to +\infty$

          $\varepsilon >0$ dáno: $\exists n_0$ t. ž. $a_{n_0} > \frac{1}{\varepsilon}$ a tedy $a_n > \frac{1}{\varepsilon}, \forall n \geq n_0 \implies a_n \in \mathscr U(+\infty, \varepsilon)$
  \end{enumerate}
\end{proof}

TODO
\begin{veta}
  Následující je ekvivalentní
  \begin{enumerate}
    \item $\{a_n\}$ konverguje
    \item $\{a_n\}$ je cauchyovské
  \end{enumerate}
\end{veta}

\begin{proof}[]
  \begin{itemize}
    \item[(1) $\implies$ (2)] \dots víme $\exists a \in \mathbb R$ t. ž. $a_n \to a$; cíl (B. C.)

    $\varepsilon > 0$ dáno: $\exists n_0 \,\forall n \geq n_0: a_n \in \mathscr U(a, \frac{\varepsilon}{2})$, tj. $|a-a_n|<\frac{\varepsilon}{2}$

     $\implies \forall m, n \geq n_0$ lze psát: $a_m-a_n = (a_m-a)+(a-a_n)$, takže $|a_m-a_n|\leq |a_m-a|+|a_n-a|<\varepsilon$
     \item[(2) $\implies$ (1)] \begin{enumerate}[i.]
       \item $\{a_n\}$ je omezená: užiji (B. C.) pro $\varepsilon=1$: $\exists n_0 \,\forall n, m \geq n_0: |a_m-a_n| < 1$, speciálně $|a_n-a_{n_0}|< 1 \,\forall n \geq n_0$

       $\implies a_{n_0} -1 < a_n < a_{n_0} + 1 \implies \{a_n\}$ omezené
       \item plyne z V. 7. 4.
       \item $\varepsilon >0$ dáno: užiji (B. C.) pro $\frac{\varepsilon}{2} \dots \exists n_0 \,\forall m, n \geq n_0: |a_m-a_n| < \frac{\varepsilon}{2}$

       dále $a_m \in \mathscr U(a, \frac{\varepsilon}{2})$ pro nekonečně $m \implies \exists m \geq n_0$ t. ž. $|a_m-a| < \frac{\varepsilon}{2}$

   \end{enumerate}
  \end{itemize}
  CELKEM $n \geq n_0:|a_n-a_m|+|a_m+a|<\varepsilon$
\end{proof}

\begin{veta}[Heineho věta pro limitu funkce]
  Nechť $f(x)$ je definována na $\mathscr P(x_0), x_0 \in \mathbb R^*$, nechť $A\in \mathbb R^*$. Potom je ekvivalentní
  \begin{enumerate}[1.]
    \item $f(x) \to A, x \to x_0$
    \item $\forall$ posloupnost $\{x_n\}$, která \begin{enumerate}[i.]
      \item $x_n \to x_0$
      \item $x_n \neq x_0 \,\forall n$
    \end{enumerate}  platí $f(x_n) \to A.$
  \end{enumerate}
\end{veta}

\begin{proof}
  \begin{itemize}
    \item[(1) $\implies$ (2)] nechť $\{x_n\}$ splňuje kladené podmínky, cíl: $f(x_n) \to A$

    nechť $\forall \varepsilon > 0 \,\exists n_0: n \geq n_0 \implies f(x_n) \in \mathscr U(A, \varepsilon)$

    $\varepsilon > 0$ dáno: dle (1) $\exists \delta >0$ t. ž. $x\in \mathscr P(x_0, \delta) \implies f(x) \in \mathscr U(A, \varepsilon)$

    dle (i): $\exists n_0$ t. ž. $n \geq n_0 \implies x_n \in \mathscr U(x_0, \delta)$, navíc díky (ii) dokonce $x_n \in \mathscr P (x_0, \delta)$

    CELKEM: $\forall n \geq n_0: f(x_n) \in \mathscr U (A, \varepsilon)$
    \item[(2) $\implies$ (1)] nepřímo: $\neg$(1) $\implies \neg $(2)

    nechť $\neg(f(x) \to A, x \to x_0)$, tedy $\exists \varepsilon > 0 \, \forall \delta >0 \, \exists x \in \mathscr P(x_0, \delta)$ t. ž. $f(x) \notin \mathscr U(A, \varepsilon)$

    fixuji takové $\varepsilon > 0$ a uživám zbytek formule pro $\delta = \frac{1}{n}, n = 1, 2, \dots \implies \exists x_n \in \mathscr P(x_0, \frac{1}{n})$ t. ž. $f(x_n) \notin \mathscr U(A, \varepsilon)$, z těchto $x_n$ získám posloupnost $\{x_n\}$

    vidíme: $|x_n-x_0| <\frac{1}{n}$, avšak $x_n \neq x_0 \, \forall n\implies x_n \to x_0$; nicméně $f(x) \notto A$, tj. $\neg (2)$
  \end{itemize}
\end{proof}

\begin{poznamka}[Užití axiomu výběru (AC)]
  Byl použit při formulaci $\exists x_n \in \mathscr P(x_0, \frac{1}{n})$, tohle bych měl udělat nekonečně mnoho krát, udělám jen pro nějaké jedno ??
\end{poznamka}

\begin{poznamka}[Heineho věta pro limitu zprava]
  Následující tvrzení jsou ekvivalentní
  \begin{enumerate}[1.]
    \item $f(x) \to A, x \to x_0^+$
    \item $\forall$ posloupnosti $\{x_n\}$ t. ž. \begin{enumerate}[i.]
      \item $x_n \to x_0$
      \item $x_n > x_0$
    \end{enumerate}
  \end{enumerate}
\end{poznamka}

\begin{veta}[7.7. Heineho věta pro spojitost funkce v intervalu]
  Nechť $f(x): I \to \mathbb R$. Potom je ekvivalentní
  \begin{enumerate}[1.]
    \item $f(x)$ je spojitá v $I$
    \item $\forall$ posloupnost $\{x_n\}, \forall x_0$ splňující \begin{enumerate}[i.]
      \item $x_n \to x_0$
      \item $x_0 \in I, x_n \in I \,\forall n$
    \end{enumerate}
    platí $f(x_n) \to f(x_0)$.
  \end{enumerate}
\end{veta}

\begin{proof}
  \begin{itemize}
    \item[(1) \implies (2)] \dots víme: $\forall x_0 \in I \,\forall \varepsilon \, \exists \delta >0$ t. ž. $x\in \mathscr U(x_0, \delta) \cap I\implies f(x) \in \mathscr U (f(x_0), \varepsilon)$

    nechť $x_n, x_0$ splňují (i), (ii) \dots cíl: $f(x_n) \to f(x_0)$

    $\epsilon > 0$ dáno $\exists \delta > 0$ t. ž. $x\in \mathscr U(x_0, \delta) \cap I\implies f(x) \in \mathscr U (f(x_0), \varepsilon)$

    dle (i) $\exists n_0 \in \mathbb N: n \geq n_0 \implies x_n \in \mathscr U(x_0, \delta)$

    navíc dle (ii) $x_n, x_0 \in I \implies f(x_n) \in  \mathscr U(f(x_0), \varepsilon), \forall n geq n_0$
    \item[$\neg$(1) $\implies$ $\neg$(2)] \dots nechť neplatí (1), tj.:

    $\exists x_0 \in I \, \exists \varepsilon > 0: \exists x \in \mathscr U(x_0, \delta) \cap I$ t. ž. $f(x) \notin \mathscr U(f(x_0), \varepsilon)$

    fixuji takové $x_0 \in I, \varepsilon > 0$, potom užívám pro $\delta = \frac{1}{n}$ pro $n = 1, 2, \dots$

    $\implies \exists x_n \in \mathscr U(x_0, \frac{1}{n}) \cap I$ t. ž. $f(x_n) \notin \mathscr U(f(x_0), \varepsilon) \, \forall n = 1, 2, \dots$

    zřejmě platí (i), (ii), avšak $f(x_n) \notto f(x_0)$, tj. neplatí (2)
  \end{itemize}
\end{proof}

\begin{priklad}
  $\lim_{x \to + \infty} \left(1+\frac{1}{n}\right)^n = e$ \dots V. 7. 6. $f(x) = (1+x)^{\frac{1}{x}} = \exp \left(\frac{\ln(1+x)}{x}\right)=e^1 \dots x_0 = 0, x_n = \frac{1}{n}, x \to 0, \left( 1+ \frac{1}{x} \right)^x = f(x_n)$, posloupnost splňuje (i), (ii)
\end{priklad}

\begin{veta}[6. 1.]
  Nechť $f(x): [a,b] \to \mathbb R$ je spojité. Pak je zde omezené
\end{veta}

\begin{proof}[pomocí posloupností]
  nepřímo: nechť $f(x)$ je neomezená v $[a,b]$, BÚNO shora

  $M \coloneq f([a, b]) = \left\{f(x), x \in [a, b]\right\}$ není omezené shora

  tj. $\forall K > 0 \,\exists y \ in M$ t. ž. $y > K$ užívám pro $K = n = 1, 2, \dots \implies \exists y_n \in M, y_n > n$, zřejmě $y_n \to + \infty, n \to + \infty$

  $\exists x_n \in [a,b]$ t. ž. $f(x_n) = y_n \to +\infty$

  užiji V. 7. 4., resp. důsl. $\implies \exists$ posloupnost $\{\tilde x_n\} \subseteq \{x_n\}, \exists x_0 \in [a,b]$ t. ž. $\tilde x_n \to x_0$.

  V. 7. 7. $\implies f(\tilde x_n) \to f(x_0) \in \mathbb R$, ale $f(\tilde x_n) = \tilde y_n \to + \infty$ - spor
\end{proof}

\begin{veta}[6.2.]
  Nechť $f(x): [a, b] \to \mathbb R$ spojité. Pak zde má globální maximum a minimum
\end{veta}
\begin{proof}
  polož $S \coloneq \sup\{f(x), x \in [a, b]\} \in \mathbb R$

  cíl: $\exists x_0 \in [a, b]$ t. ž. $f(x_0) = S$

  $n = 1, 2, \dots$: $S-\frac{1}{n}<S \implies \exists x_n \in [a, b]$, t. ž. $S-\frac{1}{n}
\end{proof}

\section{Taylorův polynom}

\begin{priklad}[Motivační]
  Chci aproximaci $f(x) = e^{-2x}$ v bodě $x=0$, hledáme aproximaci $p(x) = a + bc + cx^2$

  Nikoho nepřekvapí, že lineární aproximace bude tečna, tedy $p(x) = f(0) + f^\prime (0) x = 1- 2x$

  Chceme ji ale nějak vylepšit, zpřesnit, a to tím, že přidáme kvadratický člen. Jak ho ale najdeme?

  IDEA: napasujeme vyšší derivace, tj. $f^{\prime\prime}(0) = p^{\prime \prime} (0)$

  $p^{\prime\prime}(0) = 2c, f^{\prime\prime}(0)= 4 \implies c = 2 \implies p(x) = 1 - 2x + 2x^2$

  Jak toto dělat obecně? Jak určit další vyšší členy? Když aproximaci useknu, jak velká je chyba?
\end{priklad}

\begin{definice}
  Nechť $f(x), g(x)$ jsou definovány na $\mathscr P(x_0)$. Řekneme, že
  \begin{itemize}
    \item $f(x)$ je malé ó od $g(x)$ pro $x\to x_0 $, pokud $\frac{f(x)}{g(x)} \to 0, x \to x_0$
    \item $f(x)$ je velké Ó od $g(x)$ pro $x  \to x_0$, pokud $\exists C, \delta > 0$ t. ž. $|f(x)| \leq C|g(x)|\,\forall x \in \mathscr P(x_0, \delta)$
    \item $f(x)$ je řádově rovno $g(x)$ pro $x \to x_0$, pokud $\exists a \in \mathbb R \smallsetminus \{0\}$ t. ž. $\frac{f(x)}{g(x)} \to a, x \to x_0$
  \end{itemize}
  Značíme $f(x) = o(g(x))$ resp. $f(x) = O(g(x))$ resp. $f(x) \sim g(x), x \to x_0$
\end{definice}

\begin{priklad}
  \begin{enumerate}[1.]
    \item $\ln x = o(\sqrt x), x \to + \infty \dots \frac{\ln x }{\sqrt x} \to 0, x \to + \infty$
    \item $\frac{\sin x + \cos x}{x^2 + 1} = O(\frac{1}{x^2}), x \to + \infty \dots |f(x)| \leq \frac{2}{x^2} = 2 \frac{1}{x^2} = C g(x)$
    \item $\sin x \sim x, 1-\cos x \sim x^2, x \to 0 \dots$ (pouze pro cos) $\frac{1-\cos x}{x^2} \to \frac{1}{2} (= a), x \to 0$
  \end{enumerate}
\end{priklad}

\begin{definice}[Derivace vyšších řádů]
  \begin{enumerate}[i.]
    \item $f^{(0)}(x) = f(x)$
    \item $f^{(k+1)}(x) = (f^{(k)}(x))^\prime$, tj. $f^{(1)}(x) = f^\prime (x), f^{(2)}(x) = f^{\prime\prime}(x)$
  \end{enumerate}
  Pro $I \subseteq \mathbb R$ otevřený interval definuji
  $$C^n(I)=\left\{f(x):I \to \mathbb R; f^{(t)}\text{ existuje a je spojitá v }I\, \forall k = 0, 1, \dots n\right\}$$
  Speciálně $C^0(I) = C(I) = \left\{f(x): I \to  \mathbb R \text{ je spojité v } \mathbb R\right\}$
\end{definice}

\begin{definice}
  Pro $x_0 \in \mathbb R, k \geq 0$ celé označme $Q_{k, x_0}(x) = \frac{1}{k!}(x-x_0)^k$, speciálně $Q_{0, x_0}(x) \def_tri_cary 1$, $Q_{1, x_0}(x) = \frac{1}{2}(x-x_0)^2$, atd.
\end{definice}

\begin{lemma}[8.1. Vlastnosti funkcí $Q_{k, x_0}$]
  Platí \begin{enumerate}[i.]
    \item $Q_{k, x_0}$ je polynom stupně $k$
    \item $Q^\prime_{0, x_0} \def_tri_cary 0, Q^\prime_{k, x_0}=Q_{k-1, x_0}\,\forall k \geq 1$
    \item $Q^{(l)}_{k, x_0} = $\begin{cases}
      $1, l = k \\ 0, l \neq k$
  \end{cases} $\forall k, l \geq 0$ celé
  \end{enumerate}
\end{lemma}

\begin{proof}
  \begin{enumerate}[i.]
    \item $(x-x_0)^k =^{\text{(binom. v.)}} x^k + \dots$
    \item $1^prime = 0$; $(Q_{k, x_0})^\prime = \frac{1}{k!} ((x-x_0)^k)^\prime = \frac{1}{k!} k (x-X-0)^{k-1} = \frac{1}{(k-1)!}(x-x_0)^{k-1} = Q_{k-1, x_0}$
    \item $l = k: Q_^{(k)}{k, x_0}(x_0) = Q_{0, x_0}(x_0) = 1; l > k: Q^{(l)}_{k, x_0} \def_tri_cary 0; l < k$ tj. $k = l+s, s \geq 1: Q^{(l)}_{k, x_0}=Q_{s, x_0} = \frac{1}{s!}(x-x_0)^s, x = x_0 \implies Q_{s, x_0}(x_0)=0$
  \end{enumerate}
\end{proof}

\begin{definice}
  Nechť $f(x) \in C^n(\mathscr U(x_0))$. Potom výraz
  $$\sum_{k = 0}^n {\frac{f^{(k)}(x_0)}{k!}(x-x_0)^k}$$
  nazveme Taylorův polynom funkce $f(x)$ v bodě $x_0$ stupně $n$ (též $n$-tý Taylorův polynom). Značíme $T^{f}_{x_0, n}(x)$. Alternativně $T^f_{x_0, n}(x) = \sum_{k=0}^n f^{(k)}(x_0) \cdot Q_{k, x_0}(x)$.
\end{definice}

\begin{priklad}
  \begin{enumerate}[1.]
    \item $f(x) = e^x, x_0 = 0$; $f^{(k)}(x) = e^x \implies f^{(k)}(0) = 1 \, \forall k \geq 0$.
    $$T^{e^x}_{0, n}(x) = \sum^n_{k= 0} \frac{x^k}{k!} = 1 + x + \frac{x^2}{2} + \dots + \frac{x^n}{n!}$$
    \item $f(x) = \sin x, x_0 = 0$; $f^{(k)}(x): \sin x, \cos x, -\sin x, -\cos x, \dots$; $f^{(k)}(0): 1, 0, -1, 0, \dots$
    $$T_{0, 2n+1}^{\sin x} = x- \frac{1}{3!}x^3 + \frac{1}{5!}x^5+ \dots + (-1)^n \frac{x^{2n+1}}{(2n+1)!}$$
    \item $f(x) = (1+x)^a, x_0 = 0, a \in \mathbb R \smallsetminus \{0, 1\}$; $f^{(k)}(x) = a\cdot(a-1)\cdot \dots \cdot (a-k+1)\cdot(1+x)^{a-k}$; $ f^{(k)}(0) = a(a-1)\dots(a-k+1)$.
    $$T^{(1+x)^a}_{0, n}(x) = \sum_{k=0}^n \frac{a(a-1)\dots(a-k+1)}{k!}x^k = 1 + ax + \frac{a(a-1)}{2}x^2 + \frac{a(a-1)(a-2)}{6}x^3 + \dots$$
  \end{enumerate}
\end{priklad}

\begin{veta}[8.1. Aproximační vlastnost Taylorova polynomu]
  Nechť $f(x) \in C^n(\mathscr U(x_0))$. Potom $f(x) = T^{f}_{x_0, n}(x)+o((x-x_0)^n), x\to x_0$. Navíc $T^{f}_{x_0, n}(x)$ je jediný polynom stupně $\leq n$ s touto vlastností.
\end{veta}

\begin{proof}
  BÚNO $x_0 = 0$, označme $p(x) = T^f_{0, n}(x) = \sum_{k=0}^n f^{(k)}(0)\frac{x^k}{k!}$

  klíčové pozorování: $p^{(l)}(0) = f^{(l)}(0)\, \forall l = 0, \dots, n$

  dk. L. 8. 1. $\implies p^{(l)}(0) = \sum_{k=0}^n(f^{(k)}(0)Q_k(x))^{(l)}|_{x=0} = f^{(l)}(0)$

  \begin{enumerate}[1.]
    \item aprox. vlast.: cíl: $\frac{f(x)-p(x)}{(x)^n}\to 0, x \to 0$ -- metoda L´Hospital $\frac{0}{0}$ $n$-krát

    po $n$-tém kroku: $\frac{f^{(n)}(x)-p^{(n)}(x)}{n!}$, když dosadím $x = 0 \implies \frac{0}{n!} = 0$
    \item jedonznačnost: nechť $\exists g(x)$ \dots polynom st $\leq n$ t. ž. $\frac{f(x)-g(x)}{(x)^n}\to 0, x \to 0$ \dots cíl: $g(x) \def_tri_cary p(x) = T^f_{0, n}(x)$

    pomocná úvaha: $\frac{p(x)-g(x)}{x^n} = \frac{f(x)-g(x)}{x^n}-\frac{f(x)-p(x)}{x^n}\to 0-0 = 0, x \to 0$

    pišme $p(x) = \sum_{k=0}^n a_k x^k, g(x) = \sum_{k=0}^n b_k x^k$

    sporem: když se $p(x), g(x)$ nerovnají $\exists s \in \{0, \dots, n\}$ t. ž. $a_s \neq b_s$, BÚNO $s$ nejmenší takové $\implies p(x)-g(x) = \sum_{k = s}^n (a_k -b_k)x^k, a_k - b_k \eqcolon c_k \implies p(x) - g(x) = c_s x^s + \sum_{k=s+1}^n c_k x^k$

    potom $\frac{p(x)-g(x)}{x^s}=$ \begin{cases}
      $c_s + \sum_{k = s+1}^n c_k x^{k-s}$, když $x \to 0$, výraz $\to c_s \neq 0$ \\
      $\frac{p(x)-g(x)}{x^n}\cdot x^{n-s}$ (viz pom. úvaha) $\to 0$ -- SPOR
  \end{cases}
  \end{enumerate}
\end{proof}

\section{Určitý integrál}

\begin{motivace}
  $$\int_a^b f(x)dx = P_1-P_2$$
  \begin{figure}
    \includegraphics{int}
  \end{figure}

  \begin{enumerate}[1.]
    \item $\int_0^1 x^2 dx = [\frac{x^3}{3}]_{0}^{1} = \frac{1}{3}- \frac{0}{3}= \frac{1}{3}$
    \item $\int_0^1 D(x) dx = ???$
  \end{enumerate}
  VIdím, že ne každý integrál zvládne všechno zintegrovat. Postulujme si tedy nějaké obecné požadavky.
\end{motivace}

\begin{poznamka}
  Požadavky na integrál:
  \begin{enumerate}[1.]
    \item (int. konstanty) $\int_a^b c dx = c(b-a)$
    \item (linearita) $\int_a^b (\alpha f(x) + \beta g(x)) dx = \alpha \int_a^b f(x) dx + \beta \int_a^b g(x) dx$
    \item (aditivita intervalů) $\int_a^b f(x) dx + \int_b^c f(x) dx = \int_a^c f(x) dx$
    \item (monotonie) $f(x) \geq g(x) \,\forall x \in (a, b) \implies \int_a^b f(x) dx \geq \int_a^b g(x) dx$
  \end{enumerate}
\end{poznamka}

\begin{poznamka}
  Opakování: $F(x)$ se nazývá primitivní funkce k $f(x)$ v $(a, b)$, jestliže $F^\prime (x) = f(x) \,\forall x \in (a, b)$
\end{poznamka}

\begin{definice}
  Nechť $F(x): (a, b) \to \mathbb R$ je dáno. Výraz (má-li smysl) $$F(b^-) - F(a^+) = \lim_{x \to b^-} F(x) - \lim_{x \to a^+} F(x)$$ se nazývá zobecněný přířůstek $F(x)$ od $a$ do $b$. Značíme $[F(x)]^{x=b}_{x=a}$
\end{definice}

\begin{poznamka}
  \begin{itemize}
    \item $F(x)$ je spojité v $[a, b] \implies [F(x)]_a^b = F(b)-F(a)$
    \item kdy $[F(x)_a^b]$ nemá smysl: buď neexistuje některá z limit, nebo rozdíl nemá smysl v $\mathbb R^*$
  \end{itemize}
\end{poznamka}

\begin{definice}
  Nechť $f(x): (a,b)\to \mathbb R$ je dáno. Potom Newtonův integrál funkce $f(x)$ od $a$ do $b$ definujeme jako $(\mathscr N)\int_a^b f(x) dx \coloneq [F(x)]_a^b$, kde $F(x)$ je libovolná primitvní funkce k $f(x)$ v $(a, b)$.
\end{definice}

\begin{definice}
  Dělení intervalu $[a,b]\dots$ $D: x_0 <x_1<\dots<x_n$, kde $x_0 = a, x_n = b$

  $f(x): [a,b] \to \mathbb R \dots$ omezené funkce

  $m_1 = \inf\{f(x), x \in [x_{i-1}, x_i]\}, M_1 = \sup\{f(x), x \in [x_{i-1}, x_i]\}, i = 1, 2, \dots, n$.

  $s(D, f) = \sum_{i=1}^{n} m_i (x_i - x_{i-1}), S(D, f) = \sum_{i=1}^{n} M_i (x_i - x_{i-1})$ dolní, resp. horní Riemannův součet $f(x)$ příslučný dělení $D$. Dále definujme tzv. dolní, resp. horní Riemannův integrál $f(x)$ of $a$ do $b$.
  $$(\mathscr R)\int_a^b f(x) dx = \sup TODO$$
\end{definice}

\section{Spočetné množiny, číselné obory}

\begin{definice}
  Množina $A$ se nazve spočetná (countable), pokud existuje bijekce $\varphi: \mathbb N \to A$. ($A$ a $\mathbb N$ jsou isomorfní). Názorně prvky $A$ lze srovnat do prosté posloupnosti.
\end{definice}

\begin{priklad}
  \begin{enumerate}
    \item $\mathbb N$
    \item $\mathbb Z = \{0, 1, -1, 2, -2, \dots\}$
    \item $\mathbb Q = \{0, 1, -1, \dots, \frac{1}{2}, \frac{-1}{2}, \frac{3}{2}, \frac{-3}{2}, \dots \frac{1}{k}, \frac{-1}{k}, \frac{2}{k}, \frac{-2}{k}, \dots, \frac{k-1}{k}, \frac{-k+1}{k}, \frac{k+1}{k}, \frac{-k-1}{k}\}$ ??? je to správně TODO
  \end{enumerate}
\end{priklad}

\begin{poznamka}[Značení]
  $A \cross B = \{(a, b); a \in A, b \in B\}$ -- kartézský součin, uspořádaná dvojice

  $\mathbf P(A)$ -- potenční množina, množina všech podmnožin

  $B^A$ -- množina všech funkcí (zobrazení??? TODO) z A do B
\end{poznamka}

\begin{veta}[X.1.]
  \begin{enumerate}[1.]
    \item $A, B$ spočetné $\implies A \cross B$ spočetné
    \item $A_j$ spočetné $\forall j \in \mathbb N \implies$ $\bigcup_{j\in \mathbb N}A_j$ spočetné
  \end{enumerate}
\end{veta}

\begin{proof}
  2. zapíšu do tabulky a vidím, že z toho půjde udělat posloupnost, 2. $\implies$ 1 TODO
\end{proof}

\begin{veta}[X.2.]
  \begin{enumerate}[1.]
    \item $\mathbb R$ je nespočetná
    \item $\{0,1\}^{\mathbb N}$ je nespočetná
    \item $A$ spočetné $\implies \mathbf P(A)$ je nespočetná
  \end{enumerate}
\end{veta}

\begin{proof}
  \begin{enumerate}[1.]
    \item sporem: nechť $\exists \{a_n\}_{n=1}^{+\infty}$ -- posloupnost všech reálných čísel

    definuj podposloupnosti $\{b_n\} < \{r_n\}$ následovně:

    $a_1 = r_1$

    $b_1 = $ nějak jsou omezený a budou mít limitu to je ale ve sporu TODO
    \item plyne z 3., neboť $P(A) \approx \{0,1\}^{\mathbb N}$ -- charakteristická funkce množiny
    \end{itemize}
    \item plyne z následujícího lemmatu
  \end{enumerate}
\end{proof}

\begin{lemma}[X.2 Cantor]
  Žádná množina $X$ není isomorfní s $\mathbf P(X)$.
\end{lemma}

\begin{proof}
  sporem: nechť $\exists \varphi: X \to \mathbf P(X)$, definujme $M \subseteq X$ takto: $M \coloneq \{x; x \neq \varphi(x)\}$

  $\exists a \in X$ t. ž. $\varphi(a) = M$ (bijekce), ale $a \in M \implies a \notin M$ a $a \notin M \implies a \in M$ -- spor
\end{proof}

TODO číselné obory opsat prezentaci


% \listoffigures

\backmatter

\end{document}
