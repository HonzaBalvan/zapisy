\section{Určitý integrál}

\begin{motivace}
  $$\int_a^b f(x)dx = P_1-P_2$$
  \begin{figure}
    \includegraphics{int}
  \end{figure}

  \begin{enumerate}[1.]
    \item $\int_0^1 x^2 dx = [\frac{x^3}{3}]_{0}^{1} = \frac{1}{3}- \frac{0}{3}= \frac{1}{3}$
    \item $\int_0^1 D(x) dx = ???$
  \end{enumerate}
  VIdím, že ne každý integrál zvládne všechno zintegrovat. Postulujme si tedy nějaké obecné požadavky.
\end{motivace}

\begin{poznamka}
  Požadavky na integrál:
  \begin{enumerate}[1.]
    \item (int. konstanty) $\int_a^b c dx = c(b-a)$
    \item (linearita) $\int_a^b (\alpha f(x) + \beta g(x)) dx = \alpha \int_a^b f(x) dx + \beta \int_a^b g(x) dx$
    \item (aditivita intervalů) $\int_a^b f(x) dx + \int_b^c f(x) dx = \int_a^c f(x) dx$
    \item (monotonie) $f(x) \geq g(x) \,\forall x \in (a, b) \implies \int_a^b f(x) dx \geq \int_a^b g(x) dx$
  \end{enumerate}
\end{poznamka}

\begin{poznamka}
  Opakování: $F(x)$ se nazývá primitivní funkce k $f(x)$ v $(a, b)$, jestliže $F^\prime (x) = f(x) \,\forall x \in (a, b)$
\end{poznamka}

\begin{definice}
  Nechť $F(x): (a, b) \to \mathbb R$ je dáno. Výraz (má-li smysl) $$F(b^-) - F(a^+) = \lim_{x \to b^-} F(x) - \lim_{x \to a^+} F(x)$$ se nazývá zobecněný přířůstek $F(x)$ od $a$ do $b$. Značíme $[F(x)]^{x=b}_{x=a}$
\end{definice}

\begin{poznamka}
  \begin{itemize}
    \item $F(x)$ je spojité v $[a, b] \implies [F(x)]_a^b = F(b)-F(a)$
    \item kdy $[F(x)_a^b]$ nemá smysl: buď neexistuje některá z limit, nebo rozdíl nemá smysl v $\mathbb R^*$
  \end{itemize}
\end{poznamka}

\begin{definice}
  Nechť $f(x): (a,b)\to \mathbb R$ je dáno. Potom Newtonův integrál funkce $f(x)$ od $a$ do $b$ definujeme jako $(\mathscr N)\int_a^b f(x) dx \coloneq [F(x)]_a^b$, kde $F(x)$ je libovolná primitvní funkce k $f(x)$ v $(a, b)$.
\end{definice}

\begin{definice}
  Dělení intervalu $[a,b]\dots$ $D: x_0 <x_1<\dots<x_n$, kde $x_0 = a, x_n = b$

  $f(x): [a,b] \to \mathbb R \dots$ omezené funkce

  $m_1 = \inf\{f(x), x \in [x_{i-1}, x_i]\}, M_1 = \sup\{f(x), x \in [x_{i-1}, x_i]\}, i = 1, 2, \dots, n$.

  $s(D, f) = \sum_{i=1}^{n} m_i (x_i - x_{i-1}), S(D, f) = \sum_{i=1}^{n} M_i (x_i - x_{i-1})$ dolní, resp. horní Riemannův součet $f(x)$ příslučný dělení $D$. Dále definujme tzv. dolní, resp. horní Riemannův integrál $f(x)$ of $a$ do $b$.
  $$(\mathscr R)\int_a^b f(x) dx = \sup TODO$$
\end{definice}

\begin{poznamka}
  Nezáleží jakou primitvní funkci si vybereme, protože aditivní konstanty se odečtou. Pokud existuje vlastní Newtonovský integrál, řekneme, že integrál konverguje.
\end{poznamka}

\begin{lemma}
  Nechť $F_a(x), F_b(x), I \to \mathbb R$, kde $I = (a,b)$. Nechť $\forall x \in I: \exists F_a^\prime(x), F_b^\prime(x)$ vlastní a rovnají se. Potom $\exists c \in \mathbb R$ t. ž. $\forall x \in I: F_b(x) = F_a(x) + c$.
\end{lemma}

\begin{proof}
  polož $F(x) \coloneq F_b(x)-F_a(x), x \in I,$ zřejmě platí

  $F^\prime(x) = F^\prime_b(x)-F^\prime_a(x) = 0 \, \forall x \in I$

  $F(x)$ spojité v $I$ (V. 4. 1.)

  polož $c = F(x_0), x_0 \in I$ lib., pevné $\implies F(x) = c \,\forall x \in I$ (Lagrange)
\end{proof}

\begin{dusledek}
  \begin{enumerate}
    \item něco nepřečtu TODO
    \item korektnost definice Newtonova integrálu
  \end{enumerate}
\end{dusledek}

\begin{proof}
  TODO screenshot 17/12 na pc
\end{proof}
 nevim co tady chybi definice riemannova integralu horni dolni soucty deleni

 \begin{poznamka}
   Když se na to člověk dívá geometricky, ten Riemannův integrál dává prostě smysl nevim.
 \end{poznamka}

\begin{definice}
  Dělení $D^\tilde$ nazveme zjemněním dělení $D$, značíme $D \subseteq D^\tilde$, jestliže $D^\tilde$ obsahuje všechny body $D$.
\end{definice}

\begin{lemma}[9.1.]
  Nechť $f(x):[a,b] \to \mathbb R$ je omezená funkce, nechť $D, D^\tilde, D_1, D_2$ jsou dělení $[a,b]$.
  \begin{enumerate}[1.]
    \item Je-li $D \subseteq D^\tilde$ (nicméně pozor, jsou to posloupnosti ne množiny), pak $s(D, f) \leq s(D^\tilde, f)$ a $S(D, f) \geq S(D^\tilde, f)$.
    \item Označíme-li $m = \inf \{f(x), x  \in [a, b]\}, M = \sup \{f(x), x \in [a,b]\}$, pak $m(b-a) \leq s(D_1, f) \leq S(D_2, f) \leq M(b-a)?? (b-a nebo b=a)$
  \end{enumerate}
\end{lemma}

\begin{proof}
  \begin{enumerate}[1.]
    \item BÚNO ukážu: $S(D^\tilde, f) \leq S(D, f)$, kde $D^\tilde = D \cup \{x_i^\tilde\}, x_i^\tilde \in (x_i-1, x_i)$ -- graficky vidím, že tam nemám ten obdélník, TODO
    \item uvažme triviální dělení $D_0: x_0 = a <b = x_1$, uvažme společné zjemnění $D^\tilde = D_1 \cup D_2$

    pozoruji : $s(D_0, f) = m(b-a), S(D_0, f) = M(b-a)$, všimnu si, že $D_0 \subseteq D_1, D_2 \subseteq D^\tilde$

    dle 1. $m(b-a) = s(D_0, f) \leq s(D_1, f) \leq s(D^\tilde, f) \leq S(D^\tilde, f) \leq S(D_2, f) \leq S(D_0, f) = M(b-a)$
  \end{enumerate}
\end{proof}

\begin{dusledek}
  $c_1 \leq f(x) \leq c_2, \forall x \in [a,b] \implies c_1(b-a) \leq (R)\int_{\underline a}^b f \leq (R)\int_{a}^{\underline b} f \leq c_2(b-a)$, speciálně pro $f(x)=c \int = c(b-a$)
\end{dusledek}

\begin{proof}
  TODO
\end{proof}

\begin{lemma}[9.2]
  Nechť omezená fce. Potom je Riem. integrovatelná $\iff \forall \eta > 0 \exists$ dělení $D$ t. ž. $S(D, f) - s(D, f) < \eta$ (P. R.)
\end{lemma}

\begin{proof}
  TODO 05/01
\end{proof}

veta 9.1., dukaz
stejnomerna spojitost
lemma 9.3., dukaz
veta 9.2., dukaz
