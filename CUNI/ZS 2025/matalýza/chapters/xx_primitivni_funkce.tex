
\section{Primitivní funkce????}

\begin{veta}[LHospital]
  Nechť $x_0 \in \mathbb R^*$, nechť $\exists$ vlastní $f^\prime (x), g^\prime (x)$, navíc $ $
\end{veta}
\begin{proof}
  \begin{enumerate}[1.]
    \item $x \to x_0^+, x_0 \in \mathbb R$, případ a); $\epsilon > 0$ dáno:
          $\exists \delta > 0$ t. ž. $\forall c \in \mathscr P_+(x_0, \delta)$ platí:
          $$\frac{f^\prime(x)}{g^\prime(x)} \in \mathscr U(L, \epsilon)$$
          TRIK do/předefiniceinujme $f(x_0)=g(x_0)= 0$ (nemá vliv na $\lim_{x\to x_0}$)
          $\implies f, g$ spojité v $[x_0, x_0 + \delta), \delta > 0$ malé
          \begin{itemize}
            \item v bodě $x_0$ zprava $(f(x), g(x)\to 0 = f(x_0)= g(x_0), x \to x_0)$
            \item v bodech $x \in (x_0, x_0 + \delta) \impliedby $ Věta 4. 1. ($\exists f^\prime, g^\prime$ vlastní)
          \end{itemize}
          Buď $x \in \mathscr P_+(x_0, \delta)$ pevné, libovolné
          Aplikuji větu 6. 8. (C. o stř. h.) na $[x_0, x]$
          $\implies \exists x \in [x_0, x]\subset \mathscr P_+(x_0, \delta)$ t. ž. $\frac{f(x)}{g(x)}=\frac{f(x_0)-f(x)}{g(x_0)-g(x)} = \frac{f^\prime (x)}{g^\prime (x)} \in \mathscr U (L, \epsilon) $
    \item $x \to +\infty$, případ a);
          $$\lim_{x \to +\infty} \frac{f(x)}{g(x)} = \lim_{y \to 0^+} \frac{f(\frac{1}{y})}{g(\frac{1}{y})} = (LP 1.) = \lim_{y \to 0^+} \frac{f^\prime(\frac{1}{y})}{g^\prime(\frac{1}{y})} = \lim_{y \to 0^+}\frac{f^\prime(\frac{1}{y})\cdot\frac{-1}{y^2}}{g^\prime(\frac{1}{y})\cdot \frac{-1}{y^2}} = (L 2. 3.) = \lim_{x \to +\infty} \frac{f^\prime(x)}{g^\prime(x)}  $$
    \item $x \to x_0^+$, případ b), tj. $|g(x)|\to +\infty, x \to x_0$
          Užiji V. 6. 8. na $[x,y]$:
          $$\exists c \in (x, y)\text{ t. ž. } \frac{f(x)-f(y)}{g(x)-g(y)}=\frac{f^\prime(c)}{g^\prime(c)} \implies $$
          pujčím si $\frac{f(x)}{g(x)} = \frac{f(y)}{g(x)}+ \frac{f^\prime (c)}{g^\prime (c)}(1- \frac{g(y)}{g(x)}) = P_1 + P_3 \cdot P_2$, fixuji $y$ blízko $x_0\implies c$ blízko $x_0$, když $x\to x_0\implies P_3 \to L, P_1 \to 0, P_2 \to 1$, a tedy PS blízko $L$.
  \end{enumerate}
\end{proof}

\begin{poznamka}
  Úmluva: $I, J \subset \mathbb R$ jsou intervaly.
\end{poznamka}

\begin{definice}
  Řekneme, že $f(x)$ má v $I$ Darbouxovu vlastnost, pokud $\forall a, b, \in I, \forall \gamma$ mezi $f(a), f(b) \exists c$ mezi $a, b$ t. ž. $f(c) = \gamma$. (Darbouxova věta: $f(x)$ spoj. v $I \implies$ má v $I$ Darboux. vlast.)
\end{definice}

\begin{veta}{*}
  Nechť $I$ je otevřený interval, $f(x)$ je spojitá v $I$, nechť $\exists f^\prime(x)$ vlastní $\forall x \in I$. Pak $f^\prime(x)$ má v $I$ Darbouxovu vlastnost.
\end{veta}

\begin{veta}{Monotonie a znaménko derivace}
  Nechť $I \subset \mathbb R$ lib. interval, $f(x)$ je spojité v $I$, nechť $\exists f^\prime >0$ (resp. $\geq 0$, $\leq 0$, $<0$) $\forall x$ vnitřní bod $I$. Potom $f(x)$ je rostoucí (resp. neklesající, nerostoucí, klesající) v $I$.
\end{veta}
\begin{proof}
  Nechť $x_1 < x_2\in I \implies^? f(x_1)<f(x_2)$
  užiji V. 6. 5. (Lagrange) na $[x_1, x_2]\subset I$ -- spojitost OK, a taky $\exists f^\prime (x) \forall x \in (x_1, x_2) \subset I$
  $\implies \exists c \in (x_1, x_2)$ t. ž. $\frac{f(x_2)-f(x_2)}{x_2-x_1}={f^\prime(c)} \iff {f(x_1)-f(x_2)} = f^\prime(c)\cdot (x_1-x_2) > 0 $
\end{proof}

\begin{priklad}
  $f(x) = x^n,I = [0, +\infty), n \in \mathbb N$
  $f^\prime(x)=nx^{n-1} > 0, x \in (0, +\infty)$ -- otevřený interval je právě vnitřek intervalu $I$
  $\implies (V 6. 10.) f(x)$ rostoucí v $I$.
\end{priklad}

\begin{definice}
  Funkce $f(x)$ se nazve konvexní (resp. ryze konvexní, konkávní, ryze konkávní) v $I$, jestliže $\forall x_1 < x_2 < x_3 \in I: \frac{f(x_2)-f(x_1)}{x_2-x_1}\leq \text{ (resp. $<, \geq, >$) } \frac{f(x_3)-f(x_2)}{x_3-x_2}$.
\end{definice}

\begin{veta}{Konvexita a monotonie derivace}
  Nechť $I$ interval s krajními body $a < b$. Nechť $f(x)$ spojitá v $I$. Nechť $\exists f^\prime (x) \forall x \in (a,b)$ a je rostoucí (resp. neklesající, klesající, nerostoucí). Potom $f(x)$ je ryze konvexní (resp konvexní, ryze konkávní, konkávní) v $I$.
\end{veta}

\begin{proof}
  Nechť $x_1 < x_2 < x_3 \in I \implies^? \frac{f(x_2)-f(x_1)}{x_2-x_1} < \frac{f(x_3)-f(x_2)}{x_3-x_2}$.
  Užiji 2x V. 6. 5. (Lagrange) na $[x_1, x_2]$, pak na $[x_2, x_3]$: $\exists c \in (x_1, x_2)$ t. ž. $\frac{f(x_2)-f(x_1)}{x_2-x_1} = f^\prime(c)$ a $\exists d \in (x_2, x_3)$ t. ž. $\frac{f(x_3)-f(x_2)}{x_3-x_2}= f^\prime(d)$.
  Zřejmě $d>c \implies f^\prime (c) < f^\prime (d) \implies$ to co dk.
\end{proof}

\begin{priklad}
  $f(x) = \frac{1}{1+|x|}$ -- spojité v $\mathbb R$

  $f^\prime (x) = \frac{-1}{(1+|x|)^2} \cdot |x|^\prime = \frac{-\sgn x}{(1+|x|)^2}(x \neq 0)=\begin{cases}
    \frac{-1}{(1+|x|)^2}, x >0 \\ \frac{1}{(1+|x|)^2, x < 0}
  \end{cases}$

  V. 6. 11. $\implies f$ je ryze konvexní v $(-\infty, 0)$ a v $(0, +\infty)$ (ale ne už v $\mathbb R \smallsetminus \{0\}$)
\end{priklad}

\begin{poznamka}{Ekvivalentní definice konvexity}
  $\forall a, b, \in I, \forall \lambda \in (0, 1): f(\lambda a + (1-\lambda) b) \leq \lambda f(a) + (1-\lambda)f(b)$.
\end{poznamka}
