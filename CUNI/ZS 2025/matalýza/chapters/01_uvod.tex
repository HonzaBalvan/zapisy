\section{Úvod, reálná čísla, funkce}

\begin{poznamka} Značení:
  \begin{itemize}
    \item $P, Q$ \dots výroky, $x, y, z$ \dots čísla (prvky), $A, B, M$ \dots množiny
    \item $P & Q$ \dots $P$ zároveň s $Q$
    \item $P \lor Q$ \dots $P$ nebo $Q$
    \item $P \implies Q$ \dots $P$ implikuje $Q$
    \item $P \iff Q$ \dots $P$ je ekvivalentní s $Q$
    \item $\neg P$ \dots negace $P$
    \item $\forall x$ \dots pro všechna $x$
    \item $\exists x$ \dots existuje $x$
    \item $\exists! x$ \dots existuje právě 1 $x$
    \item $x \in A$ \dots $x$ je prvkem $A$
    \item $A \subset B$ \dots $A$ je podmnožina $B$
    \item $\{a_1, a_2, \dots\}$ \dots množina definována výčtem prvků $a_1, a_2, \dots$
    \item ${x \in M, \varphi(x)}$ \dots množina definována vlastností $\varphi(x)$
    \item $\emptyset, \{\}$ \dots prázdná množina
    \item $A \cup B$ \dots sjednocení množin $A, B$
    \item $A \cap B$ \dots průnik množin $A$, $B$
    \item $A \setminus B$ \dots rozdíl množin (prvky z $A$, které nejsou v $B$)
  \end{itemize}
\end{poznamka}

\begin{axiom}[Algebraické vlastnosti $\mathbb{R}$]
  Existuje množina reálných čísel $\mathbb{R}$, která obsahuje prvky 0 a 1, a jsou na ní definovány operace "$\cdot$" a "$+$" tak, že $\forall x, y, z \in \mathbb{R}$ platí:
  \begin{enumerate}[i.]
    \item $x+y = y+x, x \cdot y = y \cdot x$
    \item $x+(y+z) = (x+y)+z, x\cdot (y\cdot z) = (x \cdot y) \cdot z$
    \item $x \cdot (y+z) = x \cdot y + x \cdot z$
    \item $0 + x = x$, $1 \cdot x = x$
    \item $0 \cdot x = 0$ a naopak $x \cdot y = 0 \implies x = 0 \lor y = 0$
  \end{enumerate}
\end{axiom}

\begin{axiom}[Uspořádání $\mathbb{R}$]
  Na množině $\mathbb{R}$ je definována relace "$<$" tak, že $\forall x, y, z \in \mathbb{R}$ platí:
  \begin{enumerate}[i.]
    \item $x = y$, nebo $x < y$, nebo $y < x$
    \item $x < y & y < z \implies x < z$
    \item $x < y \implies x + z < y + z$
    \item $0 < x & 0 < y \implies 0 < x \cdot y$
  \end{enumerate}
\end{axiom}

\begin{poznamka}
  $x \leq y$ je zkratka za $(x < y) \lor (x = y)$. Z věty A2 opět lze vyvodit další známé poučky, např. $x^2 \geq 0 \forall x \in \mathbb R$, apod.
\end{poznamka}

\begin{poznamka}[Význačné podmnožiny $\mathbb R$]
  \begin{itemize}
    \item přirozená čísla $\mathbb N = \{1, 2, 3, \dots\}$
    \item celá čísla $\mathbb Z = \mathbb N \cup \{0, -1, -2, ---\}$
    \item racionální čísla $\mathbb Q = \{\frac{p}{q}; p \in \mathbb Z, q \in \mathbb N\}$
    \item intervaly s krajními body $(a, b)$ = $\{x \in \mathbb R; a < x < b\}$, resp. \dots, resp. neomezené pozor hranaté závorky! a ne zobáčky -- uvidíme
  \end{itemize}
\end{poznamka}

\begin{definice}[Absolutní hodnota]
  Nechť $x \in \mathbb R$. Potom $x \def \begin{cases}{x, x \geq 0 \\ -x, x < 0}\end{cases}$
\end{definice}

\begin{lemma}
  Nechť $a \geq 0, b \in \mathbb R$ lib. Potom:
  $$ |b| \leq a \iff -a \geq b \geq a.$$
\end{lemma}

\begin{proof}
  (rozborem případů)
  \item $b \geq 0$
  \item $b < 0$
\end{proof}

\begin{veta}[Trojúhelníková nerovnost]
  $\forall x, y \in \mathbb R:$
  \begin{itemize}
    \item $|x \pm y| \leq |x| + |y|$
    \item $|x \pm y| \geq ||x|-|y||$
  \end{itemize}
\end{veta}

\begin{proof}
  \begin{itemize}
    \item $|x| \leq |x|, |\pm y| \leq |y| \implies (L.1.1.) -|x| \leq x \leq |x|, -|y| \leq \pm y \leq |y|$, když sečtu $-|x|-|y| \leq x \pm y \leq |x|+|y| \implies (L.1.1.) |x\pm y| \leq |x| + |y|$
    \item TRIK $\mp y = x - ( x \pm y)$
  \end{itemize}
\end{proof}

Věta B. (Odmocnina v $\mathbb R$)
\item Nechť  $n \in \mathbb N$ je sudé. Pak $\forall a \in [0, \infty) \exists! b \in [0, \infty)$ takové, že $b^n = a$
\item Nechť $n \in \mathbb N$ je liché. Pak $\forall a \in \mathbb R \exists! b \in \mathbb R$ takové, že $b^n = a (b = \root{n}{a})$.

Poznámka. \item \sqrt{1} = 1, \root{3}{-1} = -1, \sqrt{-1} není definována
\item \root{n}{x^n} = x \forall x \in \mathbb R, n liché
\item \sqrt{x^2} = |x|

Věta 1.2. Existují iraconální čísla

Věta A3 (Axiomy \mathbb N) \item \forall x \in \mathbb R \exists n \in \mathbb N: n > x (Archimédova vlastnost)
\item Nechť M \subset \mathbb N splňuje: \item 1 \in M \item \forall n \in \mathbb N: n \in M? n \in M \implies n+1 \in M \end{enumerate} potom M = \mathbb N. (princip indukce

Poznámka. alternativní ekvivalentní formulace: \item \forall \varepsilon > 0 \exists n \in \mathbb N: n\varepsilon > 1
\item Nechť \phi (n) je formule s proměnnou n \in \mathbb N, nechť n_0 \in \mathbb N. Nechť platí: \item \phi(n) \item \forall n \geq n_0: \phi (n) \implies \phi (n+1) \end{enumerate} potom platí \phi (n) \forall n \geq n_0.

Poznámka. indukce ještě jinak: \forall M \subset \mathbb N, M \neq \emptyset: M má nejmenší prvek.


01/10/2025

TODO k větě 1.2. doplnit důkaz o iracionálnosti odmocniny ze 3

Věta 1.3. Každý otevřený interval obsahuje nekonečně mnoho racionálních a iracionálních čísel.
Důkaz. (pro iracionální)
BÚNO: I = (a,b), 0 \leq a < b
položme x_n = \frac{n\sqrt{3}}{m}, n \in \mathbb N, m \in \mathbb N velké tak, že m > \frac{2\sqrt{3}}{b-a}\iff \frac{2\sqrt{3}}{m} < b-a
M = \{n \in \mathbb N:x_n \geq b\}, nechť n_0 je nejmenší prvek M (viz A3)
tvrdíme: x_{n_0-1}, x_{n_0-2}\in (a,b), tj. a<x_{n_0-2}<x_{n_0-1}<b
zřejmě x_n \nin \mathbb Q, \forall n \neq 0
a<\frac{n_0-2}{m} \sqrt{3}=\frac{n_0}{m} \sqrt{3} - \frac{2\sqrt{3}}{m}, první zlomek \geq b, druhý < b-a \implies >a

Definice. Nechť M \subset \mathbb R. Potom
\begin{itemize}
\item x \in M nazveme maximum (největší prvek) M, pokud \forall y \in M: y \leq x
\item x \in M nazveme minimum (nejmenší prvek) M, pokud \forall y \in M: y \geq x
\item K \in \mathbb R nazveme horní odhad M, jestliže \forall x \in M: x \leq K
\item K \in \mathbb R nazveme dolní odhad M, jestliže \forall x \in M: x \geq K
\end{itemize}
Množina M se nazve
\begin{itemize}
\item shora omezená, má-li nějaký horní odhad
\item zdola omezená, má-li nějaký dolní odhad
\item omezená, má-li horní i dolní odhad.
\end{itemize}

Příklad. \item M = [0,1) \item \mathbb N = \{1, 2, 3, \dots$\}

Definice. Nechť M \in \mathbb R, Číslo S \in \mathbb R nazveme supremum M, značíme S = sup M, jestliže:
\begin{enumerate}[i.]
\item \forall x \in M: x \leq S
\item \forall S^\prime <S: \exists y \in M: y > S^\prime
\end{enumerate}
čili S je nejmenší horní odhad M.

Poznámka. \begin{itemize}
\item Supremum užitečně zobecňuje pojem maximum.
\item Existuje nejvýše jedno sup M.

Věta A4. (Úplnost \mathbb R) Nechť M \subset \mathbb R je neprázdná a shora omezená. Pak \exists! S \in \mathbb R takové, že S = sup M.

Definice. Nechť M \in \mathbb R, Číslo s \in \mathbb R nazveme infimum M, značíme s = inf M, jestliže:
\begin{enumerate}[i.]
\item \forall x \in M: x \geq s
\item \forall s^\prime >s: \exists y \in M: y < s^\prime
\end{enumerate}
čili s je největší dolní odhad M.

Věta A4^\prime. Nechť M \subset \mathbb R je neprázdná a zdola omezená. Pak \exists! s \in \mathbb R takové, že s = inf M.

Definice. (Rozšířená reálná čísla) \mathbb R^* = \mathbb R + \{- \infty, + \infty\}. Uspořádání a aritmetika v \mathbb R^*:
\begin{itemize}
\item \forall x \in \mathbb R: -\infty < x < +\infty, -\infty < +\infty
\item \forall x \in \mathbb R: x+\infty=+\infty, x-\infty=-\infty, +\infty+\infty=+\infty, -\infty-\infty=-\infty
\item \forall x > 0 x \cdot (\pm \infty) = \pm \infty, \pm \infty \cdot (+\infty) = \pm \infty, \pm \infty \cdot (-\infty) = \mp \infty
\item \forall x < 0 x \cdot (\pm \infty) = \mp \infty
\item \forall x \in \mathbb R: \frac{x}{\pm \infty} = 0
\end{itemize}
Nedefinováno zůstává + \infty -\infty, -\infty+\infty, 0 \cdot( \pm \infty), \frac{x}{0}, \frac{\pm \infty}{\pm \infty}

Definice. Nechť X, Y množiny. Funkce f: X \rightarrow Y je libovolný předpis, který každému x \in \mathbb X přiřadí jednozačně určený prvek z Y. Dále definujme
\begin{itemize}
\item obraz množiny M \subset X: f(M) = \{f(x); x \in M\}
\item vzor množiny N \subset Y: f^{-1}(N) = \{x, f(x) \in N\} (tadyto -1 není inverzní funkce, toto můžu říct i pro nezinvertovatelnou funkci)
\end{itemize}
Funkce je
\begin{itemize}
\item prostá, pokud x \neq y \implies f(x) \neq f(y)
\item "na", pokud f(X)=Y, tj. \forall y \in Y \exists x \in X: f(x)=y
\item je vzájemně jednoznačná (1-1), je-li prostá i "na" (tady lze dej. inv. fci)

TODO složené zobrazení def. obor?, inverzní a invertovatelná funkce
