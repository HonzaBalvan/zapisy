\section{Spočetné množiny, číselné obory}

\begin{definice}
  Množina $A$ se nazve spočetná (countable), pokud existuje bijekce $\varphi: \mathbb N \to A$. ($A$ a $\mathbb N$ jsou isomorfní). Názorně prvky $A$ lze srovnat do prosté posloupnosti.
\end{definice}

\begin{priklad}
  \begin{enumerate}
    \item $\mathbb N$
    \item $\mathbb Z = \{0, 1, -1, 2, -2, \dots\}$
    \item $\mathbb Q = \{0, 1, -1, \dots, \frac{1}{2}, \frac{-1}{2}, \frac{3}{2}, \frac{-3}{2}, \dots \frac{1}{k}, \frac{-1}{k}, \frac{2}{k}, \frac{-2}{k}, \dots, \frac{k-1}{k}, \frac{-k+1}{k}, \frac{k+1}{k}, \frac{-k-1}{k}\}$ ??? je to správně TODO
  \end{enumerate}
\end{priklad}

\begin{poznamka}[Značení]
  $A \cross B = \{(a, b); a \in A, b \in B\}$ -- kartézský součin, uspořádaná dvojice

  $\mathbf P(A)$ -- potenční množina, množina všech podmnožin

  $B^A$ -- množina všech funkcí (zobrazení??? TODO) z A do B
\end{poznamka}

\begin{veta}[X.1.]
  \begin{enumerate}[1.]
    \item $A, B$ spočetné $\implies A \cross B$ spočetné
    \item $A_j$ spočetné $\forall j \in \mathbb N \implies$ $\bigcup_{j\in \mathbb N}A_j$ spočetné
  \end{enumerate}
\end{veta}

\begin{proof}
  2. zapíšu do tabulky a vidím, že z toho půjde udělat posloupnost, 2. $\implies$ 1 TODO
\end{proof}

\begin{veta}[X.2.]
  \begin{enumerate}[1.]
    \item $\mathbb R$ je nespočetná
    \item $\{0,1\}^{\mathbb N}$ je nespočetná
    \item $A$ spočetné $\implies \mathbf P(A)$ je nespočetná
  \end{enumerate}
\end{veta}

\begin{proof}
  \begin{enumerate}[1.]
    \item sporem: nechť $\exists \{a_n\}_{n=1}^{+\infty}$ -- posloupnost všech reálných čísel

    definuj podposloupnosti $\{b_n\} < \{r_n\}$ následovně:

    $a_1 = r_1$

    $b_1 = $ nějak jsou omezený a budou mít limitu to je ale ve sporu TODO
    \item plyne z 3., neboť $P(A) \approx \{0,1\}^{\mathbb N}$ -- charakteristická funkce množiny
    \end{itemize}
    \item plyne z následujícího lemmatu
  \end{enumerate}
\end{proof}

\begin{lemma}[X.2 Cantor]
  Žádná množina $X$ není isomorfní s $\mathbf P(X)$.
\end{lemma}

\begin{proof}
  sporem: nechť $\exists \varphi: X \to \mathbf P(X)$, definujme $M \subseteq X$ takto: $M \coloneq \{x; x \neq \varphi(x)\}$

  $\exists a \in X$ t. ž. $\varphi(a) = M$ (bijekce), ale $a \in M \implies a \notin M$ a $a \notin M \implies a \in M$ -- spor
\end{proof}

TODO číselné obory opsat prezentaci
