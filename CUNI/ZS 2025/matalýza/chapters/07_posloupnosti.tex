\section{Posloupnosti}

\begin{definice}
  Posloupnost je funkce $a: \mathbb N \to \mathbb R, n \mapsto a_n$, značíme $\{a_n\}_{n\in\mathbb N}$ nebo $\{a_n\}_{n=n_0}^{\infty}$, krátce $\{a_n\}$.
\end{definice}

\begin{priklad}
  \begin{enumerate}[1.]
    \item $a_n = \frac{1}{n}, b_n = (-1)^n$
    \item rekurentně: $a_1 = 0, a_{n+1} = \sqrt{2+a_n}$
  \end{enumerate}
\end{priklad}

\begin{definice}
  Číslo $a\in\mathbb R^*$ se nazve limita posloupnost $\{a_n\}$, pokud
  $$\forall \varepsilon > 0 \,\exists n_0 \in \mathbb N: n \geq n_0 \implies a_n \in \mathscr U(a, \varepsilon).$$
  Značíme $a_n \to a$ nebo $\lim_{n \to+\infty} a_n = a$.
\end{definice}

\begin{poznamka}
  $a \in \mathbb R \dots \{a_n\}$ konverguje

  $a = \pm+\infty \dots \{a_n\}$ diverguje
\end{poznamka}

\begin{poznamka}[Ekvivalentní definice (pokud $a \in \mathbb R$)]
  $$\forall \varepsilon > 0 \,\exists n_0 \in \mathbb N: n \geq n_0 \implies |a_n-a|<\varepsilon$$
  Obecně: $a_n \to a \iff \forall \varepsilon > 0$ platí $a_n \in \mathscr U(a, \varepsilon)$ pro všechna $n$ až na konečně mnoho výjimek
\end{poznamka}

\begin{proof}
  \begin{itemize}
    \item[\uv{$\implies$}] výjimečně jen pro $n < n_0$ (těch je konečně)
    \item[\uv{$\impliedby$}] $V = \left\{n; a_n \notin \mathscr U(a, \varepsilon)\right\} \subset \mathbb N$ konečné

    volme $n_0 > \max V \in \mathbb N$
  \end{itemize}
\end{proof}

\begin{poznamka}
  Pro limity posloupností platí analogie vět pro limity funkcí s analogickými důkazy.
  \begin{enumerate}[i.]
    \item (VoAL, V. 2. 3., V. 2. 7.) $a_n \to a, b_n \to b \implies$ \begin{enumerate}[1.]
      \item $a_n \pm b_n \to a \pm b$
      \item $a_n \cdot b_n \to a \cdot b$
     \item $\frac{a_n}{b_n} \to \frac{a}{b}$, má-li PS smysl
    \end{enumerate}
    \item (V. 2. 9.) je-li $\alpha \leq a_n \leq \beta$ od jistého $n_0, a_n \to a \implies \alpha \leq a \leq \beta$
    \item (V. 2. 10.) je-li $b_n \leq a_n \leq c_n$ od jistého $n_0, b_n \to a, c_n \to a \implies a_n \to a$
    \item nechť $a_n \to 0$, nechť $a_n > (\text{resp. } <)$ $ 0$ od jistého $n_0 \implies \frac{1}{a_n} \to +\infty$ (resp. $-\infty$)
  \end{enumerate}
\end{poznamka}

\begin{proof}
  cíl: $\forall \varepsilon > 0 \exists n_0: n \geq n_0 \implies \frac{1}{a_n} \in \mathscr U(+\infty, \varepsilon)$, tj. $\frac{1}{a_n} > \frac{1}{\varepsilon}$
  \begin{align*}
    \varepsilon >0\text{ dáno:}&\exists n_1 \in \mathbb N: n \geq n_1 \implies a_n \in \mathscr U(0, \varepsilon)\text{, tj. }|a_n|<\varepsilon \implies -\varepsilon < a_n < \varepsilon\\
    &\exists n_2 \in \mathbb N: n \geq n_2 \implies a_n > 0 \implies 0 < a_n < \varepsilon
  \end{align*}
  polož $n_0 \coloneq \max \{n_1, n_2\}$; nechť $n\geq n_0 \implies$ cíl
\end{proof}

\begin{definice}
  Posloupnost se nazve omezená (resp. shora omezená, zdola omezená), pokud $\exists K > 0$ (resp. $K \in \mathbb R$) t. ž. $|a_n| \leq K$ (resp. $a_n \leq K, a_n \geq K$) $\forall n \in \mathbb N$. Posloupnost se nazve rostoucí (resp. neklesající, klesající, nerostoucí) pokud $a_n < a_{n+1}$ (resp. $a_n \leq a_{n+1}, a_n > a_{n+1}, a_n \geq a_{n+1}$) $\forall n \in \mathbb N$. Všechny tyto posloupnosti zároveň nazveme monotónní.
\end{definice}

\begin{veta}
  Nechť $\{a_n\}$ konverguje. Pak $\{a_n\}$ je omezené.
\end{veta}

\begin{proof}
  víme: $\exists a \in \mathbb R$ t. ž. $a_n \to a \dots \varepsilon =1, \exists n_0 \,\forall n \geq n_0: |a_n-a|<1 \iff L = a-1 < a_n < a+1 = K$

  polož $C \coloneq \max\{K, -L\} \implies |a_n|\leq C, \forall n \geq n_0$, to není $\forall n \in \mathbb N$

  položme $\tilde{C} \coloneq \max \{|a_1|, |a_2|, \dots, |a_{n-1}|, C\}$

  nyní zřejmě: $|a_n| \leq \tilde C \forall n \geq 1$ tj. $\forall n \in \mathbb N$.
\end{proof}

\begin{veta}
  Nechť $\{a_n\}$ je monotónní. Pak $\exists a \in \mathbb R^*$ t. ž. $a_n \to a$. Je-li $\{a_n\}$ omezená, pak $a\in \mathbb R$, tj. $\{a_n\}$ konverguje.
\end{veta}

\begin{proof}
  BÚNO $\{a_n\}$ je neklesající, tj. $a_n \leq a_{n+1} \,\forall n \in \mathbb N$, a tedy $a_n \leq a_m \,\forall n \leq m$

  polož $M \coloneq \{a_m; m \in \mathbb R\} \subset \mathbb R$.
  \begin{enumerate}[1.]
    \item nechť $M$ je omezená ($\iff \{ a_n \}$ je omezená); $S \coloneq \sup M$ (A. 4., $M \neq \emptyset $ )
          ukážeme, že $a_n \to S$

          $\varepsilon >0$ dáno: $S^\prime \coloneq S - \varepsilon < S \dots \exists n_0: a_{n_0}>S-\varepsilon$

          $\{a_n\}$ neklesající: $a_n > S - \varepsilon \forall n \geq n_0$, zároveň $a_n \leq S < S + \varepsilon$

          $\implies a_n \in \mathscr U(S, \varepsilon), \forall n \geq n_0$.
          \item nechť $M$ je neomezená ($\iff \{a_n\}$ je neomezená, nutně shora (protože je neklesající))

          ukážeme: $a_n \to +\infty$

          $\varepsilon >0$ dáno: $\exists n_0$ t. ž. $a_{n_0} > \frac{1}{\varepsilon}$ a tedy $a_n > \frac{1}{\varepsilon}, \forall n \geq n_0 \implies a_n \in \mathscr U(+\infty, \varepsilon)$
  \end{enumerate}
\end{proof}

TODO
\begin{veta}
  Následující je ekvivalentní
  \begin{enumerate}
    \item $\{a_n\}$ konverguje
    \item $\{a_n\}$ je cauchyovské
  \end{enumerate}
\end{veta}

\begin{proof}[]
  \begin{itemize}
    \item[(1) $\implies$ (2)] \dots víme $\exists a \in \mathbb R$ t. ž. $a_n \to a$; cíl (B. C.)

    $\varepsilon > 0$ dáno: $\exists n_0 \,\forall n \geq n_0: a_n \in \mathscr U(a, \frac{\varepsilon}{2})$, tj. $|a-a_n|<\frac{\varepsilon}{2}$

     $\implies \forall m, n \geq n_0$ lze psát: $a_m-a_n = (a_m-a)+(a-a_n)$, takže $|a_m-a_n|\leq |a_m-a|+|a_n-a|<\varepsilon$
     \item[(2) $\implies$ (1)] \begin{enumerate}[i.]
       \item $\{a_n\}$ je omezená: užiji (B. C.) pro $\varepsilon=1$: $\exists n_0 \,\forall n, m \geq n_0: |a_m-a_n| < 1$, speciálně $|a_n-a_{n_0}|< 1 \,\forall n \geq n_0$

       $\implies a_{n_0} -1 < a_n < a_{n_0} + 1 \implies \{a_n\}$ omezené
       \item plyne z V. 7. 4.
       \item $\varepsilon >0$ dáno: užiji (B. C.) pro $\frac{\varepsilon}{2} \dots \exists n_0 \,\forall m, n \geq n_0: |a_m-a_n| < \frac{\varepsilon}{2}$

       dále $a_m \in \mathscr U(a, \frac{\varepsilon}{2})$ pro nekonečně $m \implies \exists m \geq n_0$ t. ž. $|a_m-a| < \frac{\varepsilon}{2}$

   \end{enumerate}
  \end{itemize}
  CELKEM $n \geq n_0:|a_n-a_m|+|a_m+a|<\varepsilon$
\end{proof}

\begin{veta}[Heineho věta pro limitu funkce]
  Nechť $f(x)$ je definována na $\mathscr P(x_0), x_0 \in \mathbb R^*$, nechť $A\in \mathbb R^*$. Potom je ekvivalentní
  \begin{enumerate}[1.]
    \item $f(x) \to A, x \to x_0$
    \item $\forall$ posloupnost $\{x_n\}$, která \begin{enumerate}[i.]
      \item $x_n \to x_0$
      \item $x_n \neq x_0 \,\forall n$
    \end{enumerate}  platí $f(x_n) \to A.$
  \end{enumerate}
\end{veta}

\begin{proof}
  \begin{itemize}
    \item[(1) $\implies$ (2)] nechť $\{x_n\}$ splňuje kladené podmínky, cíl: $f(x_n) \to A$

    nechť $\forall \varepsilon > 0 \,\exists n_0: n \geq n_0 \implies f(x_n) \in \mathscr U(A, \varepsilon)$

    $\varepsilon > 0$ dáno: dle (1) $\exists \delta >0$ t. ž. $x\in \mathscr P(x_0, \delta) \implies f(x) \in \mathscr U(A, \varepsilon)$

    dle (i): $\exists n_0$ t. ž. $n \geq n_0 \implies x_n \in \mathscr U(x_0, \delta)$, navíc díky (ii) dokonce $x_n \in \mathscr P (x_0, \delta)$

    CELKEM: $\forall n \geq n_0: f(x_n) \in \mathscr U (A, \varepsilon)$
    \item[(2) $\implies$ (1)] nepřímo: $\neg$(1) $\implies \neg $(2)

    nechť $\neg(f(x) \to A, x \to x_0)$, tedy $\exists \varepsilon > 0 \, \forall \delta >0 \, \exists x \in \mathscr P(x_0, \delta)$ t. ž. $f(x) \notin \mathscr U(A, \varepsilon)$

    fixuji takové $\varepsilon > 0$ a uživám zbytek formule pro $\delta = \frac{1}{n}, n = 1, 2, \dots \implies \exists x_n \in \mathscr P(x_0, \frac{1}{n})$ t. ž. $f(x_n) \notin \mathscr U(A, \varepsilon)$, z těchto $x_n$ získám posloupnost $\{x_n\}$

    vidíme: $|x_n-x_0| <\frac{1}{n}$, avšak $x_n \neq x_0 \, \forall n\implies x_n \to x_0$; nicméně $f(x) \notto A$, tj. $\neg (2)$
  \end{itemize}
\end{proof}

\begin{poznamka}[Užití axiomu výběru (AC)]
  Byl použit při formulaci $\exists x_n \in \mathscr P(x_0, \frac{1}{n})$, tohle bych měl udělat nekonečně mnoho krát, udělám jen pro nějaké jedno ??
\end{poznamka}

\begin{poznamka}[Heineho věta pro limitu zprava]
  Následující tvrzení jsou ekvivalentní
  \begin{enumerate}[1.]
    \item $f(x) \to A, x \to x_0^+$
    \item $\forall$ posloupnosti $\{x_n\}$ t. ž. \begin{enumerate}[i.]
      \item $x_n \to x_0$
      \item $x_n > x_0$
    \end{enumerate}
  \end{enumerate}
\end{poznamka}

\begin{veta}[7.7. Heineho věta pro spojitost funkce v intervalu]
  Nechť $f(x): I \to \mathbb R$. Potom je ekvivalentní
  \begin{enumerate}[1.]
    \item $f(x)$ je spojitá v $I$
    \item $\forall$ posloupnost $\{x_n\}, \forall x_0$ splňující \begin{enumerate}[i.]
      \item $x_n \to x_0$
      \item $x_0 \in I, x_n \in I \,\forall n$
    \end{enumerate}
    platí $f(x_n) \to f(x_0)$.
  \end{enumerate}
\end{veta}

\begin{proof}
  \begin{itemize}
    \item[(1) \implies (2)] \dots víme: $\forall x_0 \in I \,\forall \varepsilon \, \exists \delta >0$ t. ž. $x\in \mathscr U(x_0, \delta) \cap I\implies f(x) \in \mathscr U (f(x_0), \varepsilon)$

    nechť $x_n, x_0$ splňují (i), (ii) \dots cíl: $f(x_n) \to f(x_0)$

    $\epsilon > 0$ dáno $\exists \delta > 0$ t. ž. $x\in \mathscr U(x_0, \delta) \cap I\implies f(x) \in \mathscr U (f(x_0), \varepsilon)$

    dle (i) $\exists n_0 \in \mathbb N: n \geq n_0 \implies x_n \in \mathscr U(x_0, \delta)$

    navíc dle (ii) $x_n, x_0 \in I \implies f(x_n) \in  \mathscr U(f(x_0), \varepsilon), \forall n geq n_0$
    \item[$\neg$(1) $\implies$ $\neg$(2)] \dots nechť neplatí (1), tj.:

    $\exists x_0 \in I \, \exists \varepsilon > 0: \exists x \in \mathscr U(x_0, \delta) \cap I$ t. ž. $f(x) \notin \mathscr U(f(x_0), \varepsilon)$

    fixuji takové $x_0 \in I, \varepsilon > 0$, potom užívám pro $\delta = \frac{1}{n}$ pro $n = 1, 2, \dots$

    $\implies \exists x_n \in \mathscr U(x_0, \frac{1}{n}) \cap I$ t. ž. $f(x_n) \notin \mathscr U(f(x_0), \varepsilon) \, \forall n = 1, 2, \dots$

    zřejmě platí (i), (ii), avšak $f(x_n) \notto f(x_0)$, tj. neplatí (2)
  \end{itemize}
\end{proof}

\begin{priklad}
  $\lim_{x \to + \infty} \left(1+\frac{1}{n}\right)^n = e$ \dots V. 7. 6. $f(x) = (1+x)^{\frac{1}{x}} = \exp \left(\frac{\ln(1+x)}{x}\right)=e^1 \dots x_0 = 0, x_n = \frac{1}{n}, x \to 0, \left( 1+ \frac{1}{x} \right)^x = f(x_n)$, posloupnost splňuje (i), (ii)
\end{priklad}

\begin{veta}[6. 1.]
  Nechť $f(x): [a,b] \to \mathbb R$ je spojité. Pak je zde omezené
\end{veta}

\begin{proof}[pomocí posloupností]
  nepřímo: nechť $f(x)$ je neomezená v $[a,b]$, BÚNO shora

  $M \coloneq f([a, b]) = \left\{f(x), x \in [a, b]\right\}$ není omezené shora

  tj. $\forall K > 0 \,\exists y \ in M$ t. ž. $y > K$ užívám pro $K = n = 1, 2, \dots \implies \exists y_n \in M, y_n > n$, zřejmě $y_n \to + \infty, n \to + \infty$

  $\exists x_n \in [a,b]$ t. ž. $f(x_n) = y_n \to +\infty$

  užiji V. 7. 4., resp. důsl. $\implies \exists$ posloupnost $\{\tilde x_n\} \subseteq \{x_n\}, \exists x_0 \in [a,b]$ t. ž. $\tilde x_n \to x_0$.

  V. 7. 7. $\implies f(\tilde x_n) \to f(x_0) \in \mathbb R$, ale $f(\tilde x_n) = \tilde y_n \to + \infty$ - spor
\end{proof}

\begin{veta}[6.2.]
  Nechť $f(x): [a, b] \to \mathbb R$ spojité. Pak zde má globální maximum a minimum
\end{veta}
\begin{proof}
  polož $S \coloneq \sup\{f(x), x \in [a, b]\} \in \mathbb R$

  cíl: $\exists x_0 \in [a, b]$ t. ž. $f(x_0) = S$

  $n = 1, 2, \dots$: $S-\frac{1}{n}<S \implies \exists x_n \in [a, b]$, t. ž. $S-\frac{1}{n}
\end{proof}
