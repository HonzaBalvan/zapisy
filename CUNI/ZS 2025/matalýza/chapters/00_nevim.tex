2025/09/29

má webové stránky -- tam je i něco ke zkoušce

1. Úvod, reálná čísla, funkce

Značení: $P, Q$ ... výroky, $x, y, z$ ... čísla (prvky), $A, B, M$ ... množiny
         $P & Q$ ... $P$ zároveň s $Q$
         $P \lor Q$ ... $P$ nebo $Q$
         $P \implies Q$ ... $P$ implikuje $Q$
         $P \iff Q$ ... $P$ je ekvivalentní s $Q$
         $\neg P$ ... negace $P$
         $\forall x$ ... pro všechna $x$
         $\exists x$ ... existuje $x$
         $\exists! x$ ... existuje právě 1 $x$
         $x \in A$ ... $x$ je prvkem $A$
         $A \subset B$ ... $A$ je podmnožina $B$
         $\{a_1, a_2, ...\}$ ... množina definována výčtem prvků $a_1, a_2, ...$
         ${x \in M, \varphi(x)}$ ... množina definována vlastností $\varphi(x)$
         $\emptyset, \{\}$ ... prázdná množina
         $A \cup B$ ... sjednocení množin $A, B$
         $A \cap B$ ... průnik množin $A$, $B$
         $A \setminus B$ ... rozdíl množin (prvky z $A$, které nejsou v $B$)
Věta A1. (Algebraické vlastnosti $\mathbb{R}$) Existuje množina reálných čísel $\mathbb{R}$, která obsahuje prvky 0 a 1, a jsou na ní definovány operace "$\cdot$" a "$+$" tak, že $\forall x, y, z \in \mathbb{R}$ platí:
\item $x+y = y+x, x \cdot y = y \cdot x$
\item $x+(y+z) = (x+y)+z, x\cdot (y\cdot z) = (x \cdot y) \cdot z$
\item $x \cdot (y+z) = x \cdot y + x \cdot z$
\item $0 + x = x$, $1 \cdot x = x$
\item

Věta A2. (Uspořádání $\mathbb{R}$) Na množině $\mathbb{R}$ je definována relace "$<$" tak, že $\forall x, y, z \in \mathbb{R}$ platí:
\item $x = y$, nebo $x < y$, nebo $y < x$
\item $x < y & y < z \implies x < z$
\item $x < y \implies x + z < y + z$
\item $0 < x & 0 < y \implies 0 < x \cdot y$

Poznámka. $x \leq y$ je zkratka za $(x < y) \lor (x = y)$. Z věty A2 opět lze vyvodit další známé poučky, např. $x^2 \geq 0 \forall x \in \mathbb R$, apod.

Poznámka. Význačné podmnožiny $\mathbb R$
\item přirozená čísla $\mathbb N = \{1, 2, 3, ...\}$
\item celá čísla $\mathbb Z = \mathbb N \cup \{0, -1, -2, ---\}$
\item racionální čísla $\mathbb Q = \{\frac{p}{q}; p \in \mathbb Z, q \in \mathbb N\}$
\item intervaly s krajními body $(a, b)$ = $\{x \in \mathbb R; a < x < b\}$, resp. ..., resp. neomezené pozor hranaté závorky! a ne zobáčky -- uvidíme

Definice. (Absolutní hodnota) Nechť $x \in \mathbb R$. Potom $x \def \begin{cases}{x, x \geq 0 \\ -x, x < 0}\end{cases}$

Lemma 1.1. Nechť $a \geq 0, b \in \mathbb R$ lib. Potom:
$$ |b| \leq a \iff -a \geq b \geq a.$$
Dúkaz. (rozborem případů)
\item $b \geq 0$
\item $b < 0$

Věta 1.1. (Trojúhelníková nerovnost) $\forall x, y \in \mathbb R:$
\item $|x \pm y| \leq |x| + |y|$
\item $|x \pm y| \geq ||x|-|y||$
Důkaz.
\item $|x| \leq |x|, |\pm y| \leq |y| \implies (L.1.1.) -|x| \leq x \leq |x|, -|y| \leq \pm y \leq |y|$, když sečtu $-|x|-|y| \leq x \pm y \leq |x|+|y| \implies (L.1.1.) |x\pm y| \leq |x| + |y|$
\item TRIK $\mp y = x - ( x \pm y)$

Věta B. (Odmocnina v $\mathbb R$)
\item Nechť  $n \in \mathbb N$ je sudé. Pak $\forall a \in [0, \infty) \exists! b \in [0, \infty)$ takové, že $b^n = a$
\item Nechť $n \in \mathbb N$ je liché. Pak $\forall a \in \mathbb R \exists! b \in \mathbb R$ takové, že $b^n = a (b = \root{n}{a})$.

Poznámka. \item \sqrt{1} = 1, \root{3}{-1} = -1, \sqrt{-1} není definována
\item \root{n}{x^n} = x \forall x \in \mathbb R, n liché
\item \sqrt{x^2} = |x|

Věta 1.2. Existují iraconální čísla

Věta A3 (Axiomy \mathbb N) \item \forall x \in \mathbb R \exists n \in \mathbb N: n > x (Archimédova vlastnost)
\item Nechť M \subset \mathbb N splňuje: \item 1 \in M \item \forall n \in \mathbb N: n \in M? n \in M \implies n+1 \in M \end{enumerate} potom M = \mathbb N. (princip indukce

Poznámka. alternativní ekvivalentní formulace: \item \forall \varepsilon > 0 \exists n \in \mathbb N: n\varepsilon > 1
\item Nechť \phi (n) je formule s proměnnou n \in \mathbb N, nechť n_0 \in \mathbb N. Nechť platí: \item \phi(n) \item \forall n \geq n_0: \phi (n) \implies \phi (n+1) \end{enumerate} potom platí \phi (n) \forall n \geq n_0.

Poznámka. indukce ještě jinak: \forall M \subset \mathbb N, M \neq \emptyset: M má nejmenší prvek.


01/10/2025

TODO k větě 1.2. doplnit důkaz o iracionálnosti odmocniny ze 3

Věta 1.3. Každý otevřený interval obsahuje nekonečně mnoho racionálních a iracionálních čísel.
Důkaz. (pro iracionální)
BÚNO: I = (a,b), 0 \leq a < b
položme x_n = \frac{n\sqrt{3}}{m}, n \in \mathbb N, m \in \mathbb N velké tak, že m > \frac{2\sqrt{3}}{b-a}\iff \frac{2\sqrt{3}}{m} < b-a
M = \{n \in \mathbb N:x_n \geq b\}, nechť n_0 je nejmenší prvek M (viz A3)
tvrdíme: x_{n_0-1}, x_{n_0-2}\in (a,b), tj. a<x_{n_0-2}<x_{n_0-1}<b
zřejmě x_n \nin \mathbb Q, \forall n \neq 0
a<\frac{n_0-2}{m} \sqrt{3}=\frac{n_0}{m} \sqrt{3} - \frac{2\sqrt{3}}{m}, první zlomek \geq b, druhý < b-a \implies >a

Definice. Nechť M \subset \mathbb R. Potom
\begin{itemize}
\item x \in M nazveme maximum (největší prvek) M, pokud \forall y \in M: y \leq x
\item x \in M nazveme minimum (nejmenší prvek) M, pokud \forall y \in M: y \geq x
\item K \in \mathbb R nazveme horní odhad M, jestliže \forall x \in M: x \leq K
\item K \in \mathbb R nazveme dolní odhad M, jestliže \forall x \in M: x \geq K
\end{itemize}
Množina M se nazve
\begin{itemize}
\item shora omezená, má-li nějaký horní odhad
\item zdola omezená, má-li nějaký dolní odhad
\item omezená, má-li horní i dolní odhad.
\end{itemize}

Příklad. \item M = [0,1) \item \mathbb N = \{1, 2, 3, ...\}

Definice. Nechť M \in \mathbb R, Číslo S \in \mathbb R nazveme supremum M, značíme S = sup M, jestliže:
\begin{enumerate}[i.]
\item \forall x \in M: x \leq S
\item \forall S^\prime <S: \exists y \in M: y > S^\prime
\end{enumerate}
čili S je nejmenší horní odhad M.

Poznámka. \begin{itemize}
\item Supremum užitečně zobecňuje pojem maximum.
\item Existuje nejvýše jedno sup M.

Věta A4. (Úplnost \mathbb R) Nechť M \subset \mathbb R je neprázdná a shora omezená. Pak \exists! S \in \mathbb R takové, že S = sup M.

Definice. Nechť M \in \mathbb R, Číslo s \in \mathbb R nazveme infimum M, značíme s = inf M, jestliže:
\begin{enumerate}[i.]
\item \forall x \in M: x \geq s
\item \forall s^\prime >s: \exists y \in M: y < s^\prime
\end{enumerate}
čili s je největší dolní odhad M.

Věta A4^\prime. Nechť M \subset \mathbb R je neprázdná a zdola omezená. Pak \exists! s \in \mathbb R takové, že s = inf M.

Definice. (Rozšířená reálná čísla) \mathbb R^* = \mathbb R + \{- \infty, + \infty\}. Uspořádání a aritmetika v \mathbb R^*:
\begin{itemize}
\item \forall x \in \mathbb R: -\infty < x < +\infty, -\infty < +\infty
\item \forall x \in \mathbb R: x+\infty=+\infty, x-\infty=-\infty, +\infty+\infty=+\infty, -\infty-\infty=-\infty
\item \forall x > 0 x \cdot (\pm \infty) = \pm \infty, \pm \infty \cdot (+\infty) = \pm \infty, \pm \infty \cdot (-\infty) = \mp \infty
\item \forall x < 0 x \cdot (\pm \infty) = \mp \infty
\item \forall x \in \mathbb R: \frac{x}{\pm \infty} = 0
\end{itemize}
Nedefinováno zůstává + \infty -\infty, -\infty+\infty, 0 \cdot( \pm \infty), \frac{x}{0}, \frac{\pm \infty}{\pm \infty}

Definice. Nechť X, Y množiny. Funkce f: X \rightarrow Y je libovolný předpis, který každému x \in \mathbb X přiřadí jednozačně určený prvek z Y. Dále definujme
\begin{itemize}
\item obraz množiny M \subset X: f(M) = \{f(x); x \in M\}
\item vzor množiny N \subset Y: f^{-1}(N) = \{x, f(x) \in N\} (tadyto -1 není inverzní funkce, toto můžu říct i pro nezinvertovatelnou funkci)
\end{itemize}
Funkce je
\begin{itemize}
\item prostá, pokud x \neq y \implies f(x) \neq f(y)
\item "na", pokud f(X)=Y, tj. \forall y \in Y \exists x \in X: f(x)=y
\item je vzájemně jednoznačná (1-1), je-li prostá i "na" (tady lze dej. inv. fci)

TODO složené zobrazení def. obor?, inverzní a invertovatelná funkce

\section{2. Reálné funkce -- limita a spojitost}

Úmluva. reálné fce $f(x): \mathbb R \rightarrow \mathbb R$ (tj. nikoliv $\mathbb R^*$)
        $\exists \delta > 0$ ... $\exists \delta \in (0, + \infty)$, tj. $\delta \neq \pm \infty$

Příklad. $f(x) = \frac{1}{x}$ ... $D_f = \mathbb R \smallsetminus \{0\}$, nikoliv $\frac{1}{\pm \infty} = 0

Terminologie. $f(x)$ se nazve na množině M:
\begin{enumerate}[i.]
\item rostoucí, pokud $\forall x, y \in M: x<y \implies f(x) < f(y)$
\item klesající, pokud $\forall x, y \in M: x<y \implies f(x) > f(y)$
\item neklesající, pokud $\forall x, y \in M: x<y \implies f(x) \leq f(y)$
\item nerostoucí, pokud $\forall x, y \in M: x<y \implies f(x) \geq f(y)$
\end{enumerate}
Tyto funkce nazveme monotónní, dále fce z bodů i. a ii. nazveme ryze monotonní na množině M.

Funkce jsou dále sudé, liché, periodické, ...

Definice. f(x) se nazve na množině M
\begin{enumerate}[i.]
\item shora omezená, pokud $\exists K \in \mathbb R \forall x \in M: f(x) \leq K$
\item zdola omezená, pokud $\exists L \in \mathbb R \forall x \in M: f(x) \geq L$
\item omezená, je-li shora i zdola omezená, tj. $\exists K, L \in \mathbb R \forall x \in M: L \leq f(x) \leq K$. (tady je ekviv. tvrzení s abs. hodnotou a tohoto komicky dlouhý důkaz)
\end{enumerate}

Definice. (Okolí bodu) Nechť $\delta > 0, x \in \mathbb R$. Pak
\begin{enumerate}[i.]
\item $\mathscr U (x_0, \delta) = (x_0 - \delta, x_0 + \delta)$ nazveme kruhové $\delta$-okolí bodu $x_0$
\item $\mathscr P (x_0, \delta) = (x_0 - \delta, x_0 + \delta) \smallsetminus \{x_0\}$ nazveme prstencové $\delta$-okolí bodu $x_0$
\item $\mathscr U_+ (x_0, \delta) = [x_0, x_0 + \delta)$ nazveme pravé kruhové $\delta$-okolí bodu $x_0$
\item $\mathscr U_- (x_0, \delta) = (x_0 - \delta, x_0]$ nazveme levé kruhové $\delta$-okolí bodu $x_0$
\item $\mathscr U_+ (x_0, \delta) = (x_0, x_0 + \delta)$ nazveme pravé prstencové $\delta$-okolí bodu $x_0$
\item $\mathscr U_- (x_0, \delta) = (x_0 - \delta, x_0)$ nazveme levé prstencové $\delta$-okolí bodu $x_0$
\end{enumerate}
Dále definujme
\begin{enumerate}[i.]
\item $\mathscr U (+\infty, \delta) = (\frac{1}{\delta}, +\infty]$
\item $\mathscr P (+\infty, \delta) = (\frac{1}{\delta}, +\infty)$
\item ... -\infty, levé pravé
\end{enumerate}

Poznámka. \begin{itemize}
\item $\delta_1 < \delta_2 \implies \mathscr (x_0, \delta_1) \subset \mathscr U (x_0, \delta_2)$
\item pro $x_0 \in \mathbb R$ platí: $\mathscr U (x_0, \delta) = \{x; |x-x_0| < \delta\}$ a $\mathscr P (x_0, \delta) = \{x; 0 < |x-x_0| < \delta\}$
\item zkrácený zápis $\mathscr U(x_0), \mathscr P(x_0)$

Věta 2.1. (Princip oddělení) Nechť $x_0, x_1 \in \mathbb R^*, x_0 \neq x_1$ t. ž. $\mathscr U (x_0, \delta) \cap \mathscr (x_1, \delta) = \emptyset
Důkaz. BÚNO $x_0<x_1$, rozlišme případy:
\begin{enumerate}
\item $x_0, x_1 \in \mathbb R$: stačí, aby $x_0 + \delta < x_1 - \delta$, tedy $\delta < \frac{x_1 - x_0}{2}$
\item $x_0 = \-infty, x_1 \in \mathbb R$: stačí, aby $\frac{-1}{\delta} < x_1 - \delta$, tedy ... (rozdělím na případy x)
\item $x_0 \in \mathbb R, \x_1 = + \infty$: analogicky k ii.
\item $x_0 = - \infty, x_1 = + \infty$: platí vždy
\end{enumerate}

Poznámka. Dokázali jsme navíc, že $\mathscr U(x_0, \delta)$ je vlevo od $\mathscr U (x_1, \delta)$, tj. $\forall x \in \mathscr U(x_0, \delta), \forall y \in \mathscr U (x_1, \delta): x<y$.

Definice. Nechť $x_0 \in \mathbb R^*, f(x)$ je definována na jistém $\mathscr P(x_0). Číslo $A \in \mathbb R^*$ nazveme limitou $f(x)$ v bodě $x_0$, jestliže:
$$\forall \epsilon >0 \exists \delta >0: x \in \mathscr P(x_0, \delta) \imlplies f(x) \in \mathscr U (A, \epsilon)$$
Značíme $f(x) \rightarrow A, x \rightarrow x_0$ nebo $\lim_{x \rightarrow x_0}{f(x))}=A$. Pokud $A \in \mathbb R$ nazveme limitu vlastní, pokud $A = \pm infty$ nazveme limitu nevlastní.

Poznámka. \begin{itemize}
\item limita nezávisí na $f(x_0)$, nemusí být definováno
\item alternativní zápisy \begin{itemize}
\item $\forall \epsilon > 0 \exists \delta > 0: f(\mathscr P(x_0, \delta)) \subseteq \mathscr U (A, \epsilon)$
\item $\forall \epsilon > 0 \exists \delta > 0: 0 < |x-x_0| < \delta \implies |f(x) - A| < \epsilon$ pokud $x_0, A \in \mathbb R$
\end{itemize}
\item pokud limita existuje, je právě jedna
\end{itemize}
Důkaz. TODO

Příklad. Dirichletova funkce
$$D(x) \def \begin{cases} 1, x \in \mathbb Q \\ 0, x \in \mathbb R \smallsetminus \mathbb Q \end{cases}$$
tvrdíme: $\nexists lim_{x \rightarrow x_0}{D(x)}, x_0 \in \mathbb R^*$ lib.
Důkaz. $x_0, A \in \mathbb R^*$ ... $\neg (D(x) \rightarrow A, x \rightarrow x_0)
       $\exists \epsilon > 0 \forall \delta > 0: f(\mathscr P(x_0, \delta)) \nsubseteq \mathscr U(A, \epsilon)
       stačí zvolit $\epsilon$ tak, že $\mathscr U(A, \epsilon)$ neobsahuje buď 0, nebo 1, protože H(f) = \{0,1\} (lze viz věta 2.1.)
