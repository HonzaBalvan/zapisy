\section{Taylorův polynom}

\begin{priklad}[Motivační]
  Chci aproximaci $f(x) = e^{-2x}$ v bodě $x=0$, hledáme aproximaci $p(x) = a + bc + cx^2$

  Nikoho nepřekvapí, že lineární aproximace bude tečna, tedy $p(x) = f(0) + f^\prime (0) x = 1- 2x$

  Chceme ji ale nějak vylepšit, zpřesnit, a to tím, že přidáme kvadratický člen. Jak ho ale najdeme?

  IDEA: napasujeme vyšší derivace, tj. $f^{\prime\prime}(0) = p^{\prime \prime} (0)$

  $p^{\prime\prime}(0) = 2c, f^{\prime\prime}(0)= 4 \implies c = 2 \implies p(x) = 1 - 2x + 2x^2$

  Jak toto dělat obecně? Jak určit další vyšší členy? Když aproximaci useknu, jak velká je chyba?
\end{priklad}

\begin{definice}
  Nechť $f(x), g(x)$ jsou definovány na $\mathscr P(x_0)$. Řekneme, že
  \begin{itemize}
    \item $f(x)$ je malé ó od $g(x)$ pro $x\to x_0 $, pokud $\frac{f(x)}{g(x)} \to 0, x \to x_0$
    \item $f(x)$ je velké Ó od $g(x)$ pro $x  \to x_0$, pokud $\exists C, \delta > 0$ t. ž. $|f(x)| \leq C|g(x)|\,\forall x \in \mathscr P(x_0, \delta)$
    \item $f(x)$ je řádově rovno $g(x)$ pro $x \to x_0$, pokud $\exists a \in \mathbb R \smallsetminus \{0\}$ t. ž. $\frac{f(x)}{g(x)} \to a, x \to x_0$
  \end{itemize}
  Značíme $f(x) = o(g(x))$ resp. $f(x) = O(g(x))$ resp. $f(x) \sim g(x), x \to x_0$
\end{definice}

\begin{priklad}
  \begin{enumerate}[1.]
    \item $\ln x = o(\sqrt x), x \to + \infty \dots \frac{\ln x }{\sqrt x} \to 0, x \to + \infty$
    \item $\frac{\sin x + \cos x}{x^2 + 1} = O(\frac{1}{x^2}), x \to + \infty \dots |f(x)| \leq \frac{2}{x^2} = 2 \frac{1}{x^2} = C g(x)$
    \item $\sin x \sim x, 1-\cos x \sim x^2, x \to 0 \dots$ (pouze pro cos) $\frac{1-\cos x}{x^2} \to \frac{1}{2} (= a), x \to 0$
  \end{enumerate}
\end{priklad}

\begin{definice}[Derivace vyšších řádů]
  \begin{enumerate}[i.]
    \item $f^{(0)}(x) = f(x)$
    \item $f^{(k+1)}(x) = (f^{(k)}(x))^\prime$, tj. $f^{(1)}(x) = f^\prime (x), f^{(2)}(x) = f^{\prime\prime}(x)$
  \end{enumerate}
  Pro $I \subseteq \mathbb R$ otevřený interval definuji
  $$C^n(I)=\left\{f(x):I \to \mathbb R; f^{(t)}\text{ existuje a je spojitá v }I\, \forall k = 0, 1, \dots n\right\}$$
  Speciálně $C^0(I) = C(I) = \left\{f(x): I \to  \mathbb R \text{ je spojité v } \mathbb R\right\}$
\end{definice}

\begin{definice}
  Pro $x_0 \in \mathbb R, k \geq 0$ celé označme $Q_{k, x_0}(x) = \frac{1}{k!}(x-x_0)^k$, speciálně $Q_{0, x_0}(x) \def_tri_cary 1$, $Q_{1, x_0}(x) = \frac{1}{2}(x-x_0)^2$, atd.
\end{definice}

\begin{lemma}[8.1. Vlastnosti funkcí $Q_{k, x_0}$]
  Platí \begin{enumerate}[i.]
    \item $Q_{k, x_0}$ je polynom stupně $k$
    \item $Q^\prime_{0, x_0} \def_tri_cary 0, Q^\prime_{k, x_0}=Q_{k-1, x_0}\,\forall k \geq 1$
    \item $Q^{(l)}_{k, x_0} = $\begin{cases}
      $1, l = k \\ 0, l \neq k$
  \end{cases} $\forall k, l \geq 0$ celé
  \end{enumerate}
\end{lemma}

\begin{proof}
  \begin{enumerate}[i.]
    \item $(x-x_0)^k =^{\text{(binom. v.)}} x^k + \dots$
    \item $1^prime = 0$; $(Q_{k, x_0})^\prime = \frac{1}{k!} ((x-x_0)^k)^\prime = \frac{1}{k!} k (x-X-0)^{k-1} = \frac{1}{(k-1)!}(x-x_0)^{k-1} = Q_{k-1, x_0}$
    \item $l = k: Q_^{(k)}{k, x_0}(x_0) = Q_{0, x_0}(x_0) = 1; l > k: Q^{(l)}_{k, x_0} \def_tri_cary 0; l < k$ tj. $k = l+s, s \geq 1: Q^{(l)}_{k, x_0}=Q_{s, x_0} = \frac{1}{s!}(x-x_0)^s, x = x_0 \implies Q_{s, x_0}(x_0)=0$
  \end{enumerate}
\end{proof}

\begin{definice}
  Nechť $f(x) \in C^n(\mathscr U(x_0))$. Potom výraz
  $$\sum_{k = 0}^n {\frac{f^{(k)}(x_0)}{k!}(x-x_0)^k}$$
  nazveme Taylorův polynom funkce $f(x)$ v bodě $x_0$ stupně $n$ (též $n$-tý Taylorův polynom). Značíme $T^{f}_{x_0, n}(x)$. Alternativně $T^f_{x_0, n}(x) = \sum_{k=0}^n f^{(k)}(x_0) \cdot Q_{k, x_0}(x)$.
\end{definice}

\begin{priklad}
  \begin{enumerate}[1.]
    \item $f(x) = e^x, x_0 = 0$; $f^{(k)}(x) = e^x \implies f^{(k)}(0) = 1 \, \forall k \geq 0$.
    $$T^{e^x}_{0, n}(x) = \sum^n_{k= 0} \frac{x^k}{k!} = 1 + x + \frac{x^2}{2} + \dots + \frac{x^n}{n!}$$
    \item $f(x) = \sin x, x_0 = 0$; $f^{(k)}(x): \sin x, \cos x, -\sin x, -\cos x, \dots$; $f^{(k)}(0): 1, 0, -1, 0, \dots$
    $$T_{0, 2n+1}^{\sin x} = x- \frac{1}{3!}x^3 + \frac{1}{5!}x^5+ \dots + (-1)^n \frac{x^{2n+1}}{(2n+1)!}$$
    \item $f(x) = (1+x)^a, x_0 = 0, a \in \mathbb R \smallsetminus \{0, 1\}$; $f^{(k)}(x) = a\cdot(a-1)\cdot \dots \cdot (a-k+1)\cdot(1+x)^{a-k}$; $ f^{(k)}(0) = a(a-1)\dots(a-k+1)$.
    $$T^{(1+x)^a}_{0, n}(x) = \sum_{k=0}^n \frac{a(a-1)\dots(a-k+1)}{k!}x^k = 1 + ax + \frac{a(a-1)}{2}x^2 + \frac{a(a-1)(a-2)}{6}x^3 + \dots$$
  \end{enumerate}
\end{priklad}

\begin{veta}[8.1. Aproximační vlastnost Taylorova polynomu]
  Nechť $f(x) \in C^n(\mathscr U(x_0))$. Potom $f(x) = T^{f}_{x_0, n}(x)+o((x-x_0)^n), x\to x_0$. Navíc $T^{f}_{x_0, n}(x)$ je jediný polynom stupně $\leq n$ s touto vlastností.
\end{veta}

\begin{proof}
  BÚNO $x_0 = 0$, označme $p(x) = T^f_{0, n}(x) = \sum_{k=0}^n f^{(k)}(0)\frac{x^k}{k!}$

  klíčové pozorování: $p^{(l)}(0) = f^{(l)}(0)\, \forall l = 0, \dots, n$

  dk. L. 8. 1. $\implies p^{(l)}(0) = \sum_{k=0}^n(f^{(k)}(0)Q_k(x))^{(l)}|_{x=0} = f^{(l)}(0)$

  \begin{enumerate}[1.]
    \item aprox. vlast.: cíl: $\frac{f(x)-p(x)}{(x)^n}\to 0, x \to 0$ -- metoda L´Hospital $\frac{0}{0}$ $n$-krát

    po $n$-tém kroku: $\frac{f^{(n)}(x)-p^{(n)}(x)}{n!}$, když dosadím $x = 0 \implies \frac{0}{n!} = 0$
    \item jedonznačnost: nechť $\exists g(x)$ \dots polynom st $\leq n$ t. ž. $\frac{f(x)-g(x)}{(x)^n}\to 0, x \to 0$ \dots cíl: $g(x) \def_tri_cary p(x) = T^f_{0, n}(x)$

    pomocná úvaha: $\frac{p(x)-g(x)}{x^n} = \frac{f(x)-g(x)}{x^n}-\frac{f(x)-p(x)}{x^n}\to 0-0 = 0, x \to 0$

    pišme $p(x) = \sum_{k=0}^n a_k x^k, g(x) = \sum_{k=0}^n b_k x^k$

    sporem: když se $p(x), g(x)$ nerovnají $\exists s \in \{0, \dots, n\}$ t. ž. $a_s \neq b_s$, BÚNO $s$ nejmenší takové $\implies p(x)-g(x) = \sum_{k = s}^n (a_k -b_k)x^k, a_k - b_k \eqcolon c_k \implies p(x) - g(x) = c_s x^s + \sum_{k=s+1}^n c_k x^k$

    potom $\frac{p(x)-g(x)}{x^s}=$ \begin{cases}
      $c_s + \sum_{k = s+1}^n c_k x^{k-s}$, když $x \to 0$, výraz $\to c_s \neq 0$ \\
      $\frac{p(x)-g(x)}{x^n}\cdot x^{n-s}$ (viz pom. úvaha) $\to 0$ -- SPOR
  \end{cases}
  \end{enumerate}
\end{proof}
