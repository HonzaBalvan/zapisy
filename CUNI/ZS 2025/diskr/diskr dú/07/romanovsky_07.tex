\documentclass{article}

% Language setting
% Replace `english' with e.g. `spanish' to change the document language
\usepackage[czech]{babel}
\usepackage{amsthm,thmtools,xcolor,amsmath,amssymb, mathtools}

% Set page size and margins
% Replace `letterpaper' with `a4paper' for UK/EU standard size
\usepackage[a4paper,top=2cm,bottom=2cm,left=3cm,right=3cm,marginparwidth=1.75cm]{geometry}

% Useful packages
\usepackage{amsmath}
\usepackage{enumerate}
\usepackage{graphicx}
\usepackage[colorlinks=true, allcolors=blue]{hyperref}
\graphicspath{images/}

%\usepackage{tikzit}
%\input{default.tikzstyles}

\title{DÚ Diskrétní matematika -- Sada 7}
\author{Jan Romanovský}

\begin{document}
\maketitle

\noindent\textbf{Příklad 1. Prolomení hotelového trezoru.}\\

\noindent$\Omega$ \dots všechny možné kombinace

\noindent$A$ \dots kód obsahuje alespoň jednu 4 a alespoň jednu 7

\noindent$K$ \dots kód obsahuje právě jednu 4 a právě jednu 7

\noindent$L$ \dots kód obsahuje dohromady třikrát 4 a 7, tedy dvakrát 4 a jednou 7 nebo dvakrát 7 a jednou 4

\noindent$M$ \dots kód obsahuje dohromady čtyřikrát 4 a 7, tedy třikrát 4 a jednou 7, dvakrát 4 a dvakrát 7 nebo jednou 4 a třikrát 7\\

\noindent$|\Omega|= 10^4 = 10\,000$

\noindent$|K| = {4 \choose 2}\cdot 1\cdot2\cdot8^2$ -- prvním činitelem vybírám místa, kde budou 4 a 7, druhým, jaké je jejich vnitřní pořadí (v tomto případě neřeším -- mám 1 od každého), třetím prohazuju 4 a 7 a posledním vybírám čísla mimo 4 a 7

\noindent$|L| = {4 \choose 3}\cdot \frac{3!}{2!}\cdot 2\cdot 8$ -- prvním činitelem vybírám místa, kde budou 4 a 7, druhým, jaké je jejich vnitřní pořadí, třetím prohazuju 4 a 7 a posledním vybírám čísla mimo 4 a 7

\noindent$|M| = {4 \choose 4}\cdot\frac{4!}{2!2!}\cdot 1 \cdot 1 + {4\choose4} \cdot \frac{4!}{3!1!}\cdot 2 \cdot 1$ -- první sčítanec je možnost dvakrát 4 a dvakrát 7, druhý je třikrát 4 a jednou 7 nebo třikrát 7 a jednou 4; prvním činitelem vybírám místa, kde budou 4 a 7, druhým, jaké je jejich vnitřní pořadí, třetím prohazuju 4 a 7 a posledním vybírám čísla mimo 4 a 7\\

\noindent$|A|= |K| + |L| + |M| = 974$

\noindent$P(A) = \frac{|A|}{|\Omega|}=\frac{974}{10000} = 0,0974 = 9,74 \,\%$\\

\noindent\textbf{Příklad 2. Popletené písemky.}

\noindent Ptáme se vlastně na střední hodnotu počtu pevných bodů náhodné permutace, $\mathbb E(A)$, počet pevných bodů bude naše náhodná veličina. Pokud si zavedeme indikátor $A_i$, který přiřadí $i$-tému studentu $1$, když dostane vlastní písemku a $0$ jindy, pak $P(A_i) = \frac{1}{n}$ -- právě 1 z n písemek je ta správná. Potom navíc díky linearitě střední hodnoty $\mathbb E(A) = \sum_{i=1}^n \mathbb E(A_i) = n \cdot \frac{1}{n}=1$.
\end{document}
