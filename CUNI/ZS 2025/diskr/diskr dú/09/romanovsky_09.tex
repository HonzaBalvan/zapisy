\documentclass{article}

% Language setting
% Replace `english' with e.g. `spanish' to change the document language
\usepackage[czech]{babel}
\usepackage{amsthm,thmtools,xcolor,amsmath,amssymb, mathtools}

% Set page size and margins
% Replace `letterpaper' with `a4paper' for UK/EU standard size
\usepackage[a4paper,top=2cm,bottom=2cm,left=3cm,right=3cm,marginparwidth=1.75cm]{geometry}

% Useful packages
\usepackage{amsmath}
\usepackage{enumerate}
\usepackage{graphicx}
\usepackage[colorlinks=true, allcolors=blue]{hyperref}
\graphicspath{images/}

%\usepackage{tikzit}
%\input{default.tikzstyles}

\title{DÚ Diskrétní matematika -- Sada 9}
\author{Jan Romanovský}

\begin{document}
\maketitle

\noindent\textbf{Příklad 1. Automorfismy stromu.}\\

\noindent Když každý automorfismus stromu zachovává vzdálenosti vrcholů, mužeme si vzít všechny vrcholy, které mají od všech ostatních tu nejmenší maximální vzdálenost a o nich víme, že se zobrazí na takové vrcholy, které mají od všech ostatních zase tu nejmenší maximální vzdálenost -- tzv. střed grafu. Střed stromu má vždy jeden, nebo dva vrcholy, což vyplývá z algoritmu pro hledání středu stromu -- osekáme v jednom kroku všechny listy, tím nám vznikne nový strom zase s nějakými listy, takto osekáváme dokud máme co osekávat a buď nám zůstane jeden vrchol -- jednoprvkový střed, nebo žádný vrchol -- dvouprvkový střed kde středem jsou dva poslední osekané vrcholy, které budou spojené hranou. Když bude mít střed jeden vrchol, zobrazí se sám na sebe -- bude fixovaným vrcholem. Když bude mít dva, každý se může zobrazit sám na sebe -- 2 fixované vrcholy, nebo se můžou zobrazit na sebe navzájem -- fixovaná hrana. $\qed$

\noindent\textbf{Příklad 2. Podstromy.}\\

\noindent Postupujme sporem. Mějme tři podstromy $A, B, C$ takové, že každé dva mají neprázdný průnik, ale průnik všech třech je prázdný. Pokud podstromy $A$ a $B$ mají průnik v bodě $X$, průnik podstromů $B$ a $C$ nesmí být v bodě $X$, jinak by byl průnik všech třech neprázdný, nechť mají průnik v bodě $Y$. Potom průnik $C$ a $A$ musí být zase v bodě jiném od $X, Y$, nebo průnik všech tří není prázdný. Takto nám ale vznikne cyklus obsahující body $X, Y, Z$ a procházející přes podstromy $A, B, C$ což je spor s tím, že máme podstromy stromu. $\qed$
\end{document}
