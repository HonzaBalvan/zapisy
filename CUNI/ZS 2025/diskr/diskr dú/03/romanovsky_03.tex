\documentclass{article}

% Language setting
% Replace `english' with e.g. `spanish' to change the document language
\usepackage[czech]{babel}
\usepackage{amsthm,thmtools,xcolor,amsmath,amssymb, mathtools}

% Set page size and margins
% Replace `letterpaper' with `a4paper' for UK/EU standard size
\usepackage[a4paper,top=2cm,bottom=2cm,left=3cm,right=3cm,marginparwidth=1.75cm]{geometry}

% Useful packages
\usepackage{amsmath}
\usepackage{enumerate}
\usepackage{graphicx}
\usepackage[colorlinks=true, allcolors=blue]{hyperref}
\graphicspath{images/}

\usepackage{tikzit}
\input{default.tikzstyles}

\title{DÚ Diskrétní matematika -- Sada 3}
\author{Jan Romanovský}

\begin{document}
\maketitle

\section*{Příklad 1. Zmrzlinová laboratoř}
\begin{enumerate}[a)]
    \item variace bez opakování: $N = {n \choose k}\cdot k! = n^{\underline{k}} = 11 \cdot 10 \cdot 9 = 990$
    \item kombinace s opakováním: $N = {n+k-1\choose n-1} = {11+4-1\choose11-1} = 1001$
    \item \begin{enumerate}[i.]
        \item variace s opakováním: $N = n^k = 11^5 = 161051$
        \item variace s opakováním, pravidlo součinu: $N = N_Z \cdot N_C \cdot N_P = n^k \cdot m^l \cdot {k + l \choose k} = 11^3 \cdot 5^2 \cdot {5 \choose 2} = 332750$ -- první činitel je počet možností jak vybrat zmrzlinu, druhý jak vybrat crumble a třetí určuje vzájemné pořadí zmrzliny a crumble
    \end{enumerate}
    nakonec pravidlo součtu: $N = N_i + N_{ii} = 161051 + 332750 = 493801$
\end{enumerate}

\section*{Příklad 2. Dokažte}
    kombinatoricky: ${n \choose k}$ -- počet $k$-prvkových podmnožin $n$-prvkové množiny, ${m\choose r-k}$ -- počet $(r-k)$-prvkových podmnožin $m$-prvkové množiny, když je násobíme pro dané $k$ máme počet podmnožin množiny s $(m+n)$ prvky, které mají právě $k$ prvků z $n$-prvkové množiny a právě $(r-k)$ prvků z $m$ prvkové množiny, když tyto hodnoty sečteme přes $k$ uvažujeme všechny poměry vybrání prvků z původní $n$-prvkové a $m$-prvkové množiny, dostaneme tedy počet všech $(r-k+k)$- tedy $r$-prvkových podmnožin množiny s $m+n$ prvky, tedy přesně ${n+m\choose r}$ -- QED
\end{document}
