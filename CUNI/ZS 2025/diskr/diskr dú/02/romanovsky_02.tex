\documentclass{article}

% Language setting
% Replace `english' with e.g. `spanish' to change the document language
\usepackage[czech]{babel}
\usepackage{amsthm,thmtools,xcolor,amsmath,amssymb, mathtools}

% Set page size and margins
% Replace `letterpaper' with `a4paper' for UK/EU standard size
\usepackage[a4paper,top=2cm,bottom=2cm,left=3cm,right=3cm,marginparwidth=1.75cm]{geometry}

% Useful packages
\usepackage{amsmath}
\usepackage{graphicx}
\usepackage[colorlinks=true, allcolors=blue]{hyperref}
\graphicspath{images/}

\usepackage{tikzit}
\input{default.tikzstyles}

\title{DÚ Diskrétní matematika -- Sada 1}
\author{Jan Romanovský}

\begin{document}
\maketitle

\section*{Příklad 1. Šachovnice}
Postupujme od nejmenších $n$ a \uv{silnou indukcí}, BÚNO mějme vždy vykousnutou kostičku vpravo nahoře.
\begin{itemize}
    \item $n=1:$ triválně, to je přesně jedna kostička
    \item $n=2:$ lze např. dle obrázku: 
    \begin{figure}[ht!]
    \centering
    \scalebox{0.6}{\tikzfig{figures/1}}
    \end{figure}
    \item $n = 3:$ přidejme jednu kostičku naproti té vykousnuté, tj. vlevo dole:
    \begin{figure}[ht!]
    \centering
    \scalebox{0.6}{\tikzfig{figures/2}}
    \end{figure}
    
    potom ale vidíme, že nám vznikly tři sekce čtvrtinového obsahu, stejného tvaru a obsahu jako v případě o jedna menším, tj. $n=2$, pouze některé jsou otočené:
    \begin{figure}[ht!]
    \centering
    \scalebox{0.6}{\tikzfig{figures/3}}
    \end{figure}

    a o těch už víme, že jdou vyplnit:
    \begin{figure}[ht!]
    \centering
    \scalebox{0.6}{\tikzfig{figures/4}}
    \end{figure}
\end{itemize}
a vidíme, že dle tohoto postupu umíme pomocí plochy $2^{n-1}\times 2^{n-1}$ vyplnit jakoukoliv další plochu o rozměrech $2^n \times 2^n$ -- QED

\section*{Příklad 2. Suma}
MI: \begin{enumerate}
    \item $n = 1: 1^3 = 1^2$ -- platí
    \item Chceme dk. že $\sum_{i=1}^{n}{i^3} = \left(\sum_{i=1}^{n}{i}\right)^2 \implies \sum_{i=1}^{n+1}{i^3} = \left(\sum_{i=1}^{n+1}{i}\right)^2$, předp. že $\sum_{i=1}^{n}{i^3} = \left(\sum_{i=1}^{n}{i}\right)^2$.
\end{enumerate}
\begin{align*}
    \left(\sum_{i=1}^{n+1}{i}\right)^2&=\left(\sum_{i=1}^{n}{i}+n+1\right)^2\\
    &= \left(\sum_{i=1}^{n}{i}\right)^2+2\sum_{i=1}^{n}{i}\cdot(n+1)+(n+1)^2\\
    &= \sum_{i=1}^{n}{i^3}+2\sum_{i=1}^{n}{i}\cdot(n+1)+(n+1)^2 \text{-- IP}\\
    &= \sum_{i=1}^{n}{i^3}+2\frac{n(n+1)}{2}(n+1)+(n+1)^2\\
    &= \sum_{i=1}^{n}{i^3}+n^3+3n^2+3n+1\\
    &= \sum_{i=1}^{n}{i^3} + (n+1)^3\\
    &= \sum_{i=1}^{n+1}{i^3} \text{-- QED}
\end{align*}
    
\end{document}

