\documentclass[11pt]{template/cauchy}
\usepackage[czech]{babel}

% IMPORTY
\usepackage{enumerate}
\usepackage{mlmodern}
\usepackage{template/mathsphystools}
\usepackage{amsthm,thmtools,xcolor,amsmath,amssymb, mathtools}
\usepackage[scr]{rsfso}
% \usepackage{thmstyles}
\usepackage{tabularx}
\usepackage{graphicx}
\usepackage{multicol}
\usepackage{appendix}
\usepackage{tabularx}
\usepackage{subcaption}
\usepackage{hyperref}
\usepackage{comment}
\usepackage{lscape}
\usepackage{xcolor}
\usepackage{wrapfig}
\usepackage{centernot}
\usepackage{pgfplots}
\usepackage{caption}
\graphicspath{{images/}}
\pgfplotsset{width=10cm,compat=1.9}

% \usepackage{tikzit}
% \input{default.tikzstyles}
% je potřeba přiložit soubor default.tikzstyles


% STYLY
% definice, příklad, cvičení, věta, lemma, řešení, poznámka, důsledek, axiom, motivace

\declaretheoremstyle[
  headfont=\color{red}\normalfont\bfseries,
  bodyfont=\color{black}\normalfont,
]{def}

\declaretheoremstyle[
  headfont=\color{blue}\normalfont\bfseries,
  bodyfont=\color{black}\normalfont,
]{pr}

\declaretheoremstyle[
  headfont=\color{teal}\normalfont\bfseries,
  bodyfont=\color{black}\normalfont\itshape,
]{cv}

\declaretheoremstyle[
  headfont=\color{green}\normalfont\bfseries,
  bodyfont=\color{black}\normalfont\itshape,
]{veta}

\declaretheoremstyle[
  headfont=\color{lime}\normalfont\bfseries,
  bodyfont=\color{black}\normalfont\itshape,
]{lemma}

\declaretheoremstyle[
  headfont=\color{black}\normalfont\itshape,
  bodyfont=\color{black}\normalfont,
  numbered=no
]{res}

\declaretheoremstyle[
  headfont=\color{brown}\normalfont\bfseries,
  bodyfont=\color{black}\normalfont,
]{pozn}

\declaretheoremstyle[
  headfont=\color{brown}\normalfont\bfseries,
  bodyfont=\color{black}\normalfont,
]{dusl}

\declaretheoremstyle[
  headfont=\color{cyan}\normalfont\bfseries,
  bodyfont=\color{black}\normalfont,
]{axiom}

\declaretheoremstyle[
  headfont=\color{black}\normalfont\bfseries,
  bodyfont=\color{black}\normalfont,
]{motivace}

\declaretheorem[
  style=def,
  name=Definice,
]{definice}

\declaretheorem[
  style=pr,
  name=Příklad,
]{priklad}

\declaretheorem[
  style=cv,
  name=Cvičení,
]{cviceni}

\declaretheorem[
  style=res,
  name=Řešení,
]{reseni}

\declaretheorem[
  style=veta,
  name=Věta,
]{veta}

\declaretheorem[
  style=lemma,
  name=Lemma,
]{lemma}

\declaretheorem[
  style=pozn,
  name=Poznámka,
]{poznamka}

\declaretheorem[
  style=dusl,
  name=Důsledek,
]{dusledek}

\declaretheorem[
  style=axiom,
  name=Axiom,
]{axiom}

\declaretheorem[
  style=motivace,
  name=Motivace,
]{motivace}

\DeclareMathOperator{\tg}{tg}
\DeclareMathOperator{\cotg}{cotg}
\DeclareMathOperator{\arctg}{arctg}
\DeclareMathOperator{\arccotg}{arccotg}

\title{Úvod do sem patří název \\ {\large Zápisky z přednášky Jméno Příjmení učitele}}
\author{Jméno Příjmení}
\date{}

\begin{document}

\maketitle

% ÚVODNÍ INFORMACE
% dobré pro poznačení informačních věcí od učitele,
% např. jeho email, jména skript apod.
%!TEX root = ../main.tex
%
\phantomsection
\addcontentsline{toc}{section}{Úvodní informace}
\begin{center}
{\bfseries\Large Úvodní informace}\par\vspace{1em}
\end{center}

% SEM PATŘÍ TEXT


% ZNAČENÍ
% pokud nepotřebujete, zakomentujte řádek níže
% všechny změny provádějte v souboru additional/conventions.tex
%!TEX root = ../main.tex
%
\phantomsection
\addcontentsline{toc}{section}{Značení}
\vspace{2em}
\begin{center}
{\bfseries\Large Značení}\par\vspace{1em}
\end{center}

% SEM PATŘÍ TEXT
látka -- skládá se z částic s klidovou hmotností, nějaký objekt
pole -- neskládá se z částic, ale zprostředkuje silové působení mezi částicemi (grav., el., mag., ...)
popis pole -- klasicky pomocí fyzikální veličiny nebo kvantovš jako výměnu zprostředkujících (intermediálních) polních částic
typy polí
\begin{itemize}
\item skalární pole -- popsáno skalární veličinou v prostoru
\item vektorové pole -- popsáno vektorovou veličinou v prostoru
\item homogenní x heterogenní
\item isotropní x anisotropní
\end{itemize}

Metoda:
Hypotéza
\begin{itemize}
\item předpoklad možného stavu, domněnka, jejíž platnost není ještě plně prokázána
\item je pdoložená řadou faktů vytyčujících salší směr výzkumu
\item hyp. je testovatelná, lze potvrdit nebo vyvrátit
\end{itemize}
Teorie
\begin{itemize}
\item popisuje zákonitosti a souvislosti celé skupiny jevů v určitém oboru
\item co nejlepší přiblížení realitě, vystavěno na objektivních důkazech
\item je vnitřně konsistentní, každá teorie ale má své limity
\item na základě teorie a hypotéz vytvoříme model
\item Model -- kvalitativní, kvantitativní (matem.), nějaké zjednodušení problému
\end{itemize}
Zákon
\begin{itemize}
\item popisuje pozorování v přírodě (zatímco teorie se ho snaží vysvětlit)
\item je vždy pravdivý -- vychází z mnoha pozorování
\end{itemize}
Postulát -- Axiom
\begin{itemize}
\item je výchozí předpoklad (tvrzení), který je v dané teorii obecně přijímán jako pravdivý (a ta teorie na tom stojí)
\end{itemize}

Fyzikální veličiny
\begin{itemize}
\item extenzivní (kvantita -- hmota, náboj), itenzity (kvality -- teplota, napětí)
\item určeny velikostí a rozměrem (jednotkou): veličina = číselný údaj krát jednotka
\item soustavy jednotek jsou věc dohody
\item nejběžnější SI, dále např. absolutní -- Gaussova (cgs -- převodové konstanty (v grav., elstat. síle) položíme rovny 1, pak se dají věci navzájem odvodit)
\end{itemize}

Fyzikální veličiny -- přehled
\begin{itemize}
\item skaláry -- vyjádřeny jedním údajem (velikostí); invariantní vůči volbě souřadnicové soustavy
\item vektory -- vyjádřeny obecně $n$ složkami v $n$-D prostoru, ve 3D si můžeme představit jako orientovanou úsečku; závisí na volbě souřadnicové soustavy
\item ve fyzice se zpravidla použivá pravotočivá pravoúhlá kartézská souřadnicová soustava
\item můžeme využít také např. polární souřadnice, sférické souřadnice
\end{itemize}

Základy vektorového počtu
\begin{itemize}
\item skalární součin -- vektor $\times$ vektor $\rightarrow$ skalár; používá se pro velikosti, úhly, kolmost, ...; značíme $\textbf{a}\cdot\textbf{b}$ nebo $(\textbf{b})$
\item vektorový součin -- vektor $\times$ vektor $\rightarrow$ vektor, funguje jen ve 3D, pozor na správný směr -- pravotočivo; používá se pro určení kolmice k rovině, počítání obsahu, ...
\end{itemize}

Přehled vztahů TODO zkopírovat z prezentace; zjistit co je tenzor


% KAPITOLY
% \input{chapters/01_nazev_kapitoly}
% ...

% \listoffigures

\backmatter

\end{document}
