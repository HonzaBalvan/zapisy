\section{Historie}

Základem ve fyzice je \textbf{experiment}.

Historický přehled -- starověk
\begin{itemize}
\item Pythagoras, Eukleides -- geometrie
\item Demokritos -- atomismus
\item Platón -- Akademia ("nevstupuj, kdo neznáš geometrii)
\item Múseion (Chrám můz, alexandrijská knihovna) (Ptolemaios I.) -- opisovali všechno na co přišli, otevřené "vědcům"
\item Aristoteles -- Kučera hater, spekulativní závěry
\end{itemize}
Historický přehled -- středověk
\begin{itemize}
\item Koperník
\item Bacon
\item Galilejo Galilei -- "první člověk, který studoval přírodu moderním způsobem" -- experiment, závěr, zobecnění; prvotní princip relativity
\item Kepler -- Praha
\item Descartes
\item Hooke
\item Christian Huygens
\item první vědecké akademie -- Londýn (Newton), Paříž, Berlín (Leibniz), (Petrohrad)
\end{itemize}
Historický přehled -- 2. pol. 17. stol.
\begin{itemize}
\item Isaac Newton -- vysypal pohybové zákony prakticky z rukávu
\item Leibniz -- souběžně s Newtonem dif. a int. počet
\end{itemize}
Historický přehled -- 18. až 20. stol.
\begin{itemize}
\item Leonhard Euler -- jmenuje se po něm všechno
\item Karl F. Gauss --
\item Michael Faraday -- propojení elektřiny a magnetismu
\item James C. Maxwell -- elektromagnetismus
\end{itemize}

Moderní fyzika -- od poč. 20. stol.
\begin{itemize}
\item na konci 19. stol. už se zdálo že fyzika je hotová, pak ale našli několik věcí co tohle rozbilo (ZČT -- UV katastrofa)
\item Max Planck -- kvantování
\item Wilhelm C. Roentgen -- RTG
\item Bragg, Laue -- difrakce na krystalech -- první "zobrazení atomů"
\item Albert Einstein -- navazuje na výsledky Lorentze, Poincarého -- STR
\end{itemize}

\subsection{Čím se budeme zabývat}
Mechanika (mechané = stroj) -- popis pohybu těles v čase a jejich chování při působení síly
Newtonova klasická mechanika -- zapomeň na relativitu a kvantování (toto ale má svoje limity)

\section{Kinematika}
Anotace: Kinematika bodu. Parametrický popis pohybu, rychlost, zrychlení, rozklad zrychlení na tečnou a normálovou složku. Základní druhy pohybů.
\begin{itemize}
\item zkoumá pohyb těles v prostoru
\item určení polohy v čase, relativně vzhledem k vztažné soustavě
\item objektem je hmotný bod
\item poloha je určena polohovým vektorem, ten je vázaný
\item trajektorie je geometrická křivka v prostoru po které se bod pohybuje
\item dráha je délka trajektorie (není tady ten údaj o směru, zakřivení, trase)
\end{itemize}
\subsection{Pohyb hmotného bodu}
TODO vložit celý slide č. 4 prezentace FI\_1.kinematika
\begin{itemize}
\item vektor okamžité rychlosti $\textbf{v}$ TODO slide č. 5
\end{itemize}
Ekvivalentní popis: TODO slide č. 6
zrychlení TODO slide č. 7 zrychlení má obecný směr (lze pak rozložit na normálové a tečné)
přičemž $a_t = \frac{dv}{dt}$ -- pozor to tečné ne to celkové (to se hodí)
TODO přepastit poučky ze cvik 02/10, navíc $\frac{v^2}{R}$, $R$ poloměr oskulační kružnice
TODO přepastit složky slide č. 9

\begin{priklad}
Vodorovný vrh -- zrychlení se s časem "přelévá" z normálového do tečnéh
\end{priklad}
