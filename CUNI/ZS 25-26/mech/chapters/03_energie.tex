\section{Energie}
Práce, výkon, kinetická energie. Konzervativní pole, intenzita a potenciál, centrální síla, lineární
harmonický oscilátor, potenciální energie. Nekonzervativní síly, tření. Gravitační zákon. Pohyb v
gravitačním poli, Keplerovy zákony.

\subsection{Práce}
účinek síly:
\begin{itemize}
\item podle dráhy na níž působila síla na h. b. -- dráhový
\item podle doby po kterou síla působila -- časový

Práce -- úhrnné působení síly na h. b., který vykonal pohyb po dané dráze
$$W = \vec F \cdot \vec s \to \sum_{k=1}^{N}{\vec F_k \Delta \vec s_k} \to \int_{\mathscr l}{\vec F \cdot d \vec s}\text{, kde } d\vec s = ds \cdot \vec \tau$$
W -- skalární veličina, závisí na počátečním a koncovém bodu, na tvaru dráhy a působící síle
můžu počítat taky integrál po složkách
Elementární práce: (okamžitá práce?) $dW$
práci koná pouze tečná složka, tj. ve směru pohybu
počítáme křivkový integrál (křivkový integrál 2. druhu)

TODO až do
\subsection{Potenciální energie}
Pokud práce závisí pouze na poloze počátečního a koncového bodu, můžeme zavést odpovídající veličinu závislou na poloze -- potenciální (konfigurační, polohovou) energii $E_P$
pozor na znaménka, moje práce vydaná vs. něčeho práce přijatá
$$-W_{12} = -\int_{1}^{2}{\vec F \cdot d\vec s}= E_{P2}-E_{P1}=\Delta E_P$$
