\section{Dynamika}
newtonovy zákony, síly, hmotnost
příčiny pohybu a zájemné působení těles
pro síly platí tzv. princip superpozice -- jsou to prostě vektory, vektorově to sčítám, vektorově to rozkládám

1. zákon setrvačnosti (Galileo): Nepůsobí-li na těleso vnější fyzikální vlivy (tj. žádné pravé síly, popř. výslednice pravých sil je nulová - tzv. volná částice), pak soustava souřadná, vůči níž je těleso v klidu nebo v rovnoměrném přímočarém pohybu, je soustava inerciální.
2. zákon síly:  časová změna hybnosti tělesa je rovna výslednici vnějších sil,
které na těleso působí
$$\overrightarrow{F} = \frac{d\overrightarrow{p}}{dt}=\frac{d}{dt}(m\overrightarrow{v})$$
3. zákon akce a reakce

setrvačnost = odpor proti změně pohybového stavu, m = setrvačná hmota
1. zákon určuje inerciální soustavy, vůči nimž je volný h. b. v klidu nebo se pohybuje rovnoměrně přímočaře, v těchto soustavách měříme zrychlení
2. zákon umožňuje stanovit hmotnost těles, tímto je definována síla (jednotka síly) na levé straně pohybové rovnice
hybnost: $\overrightarrow{p}=m\overrightarrow{v}$

\subsection{Síly při různých druzích pohybu}
$\overrightarrow{F} = m\overrightarrow{a}$
$prp: \overrightarrow{F} = 0, przp: \overrightarrow{F} = konst. (\propto m)$
Harmonický pohyb:
TODO slide č. 9

\subsection{Newtonovy pohybové rovnice -- inerciální vztažné soustavy}
$$\overrightarrow{F} = m\frac{d^2\overrightarrow{r}}{dt^2}$$
$$F_i = m\frac{d^2x_i}{dt^2}, i \in \{1, 2, 3\}$$
lineární diferenciální rovnice 2. řádu, ale 3 lineární pohybové rovnice v jednotlivých dimenzích -- princip superpozice
Musím ještě dodat počáteční podmínky: $F(t, x_1, x_2, x_3, \frac{dx_1}{dt}, \frac{dx_2}{dt}, \frac{dx_3}{dt})$
Klasická mechanika je deterministická = pro stejné počáteční podmínky se h. b. bude chovat stejně.

\begin{priklad} Vodorovný vrh
$\overrightarrow{r} = (0, h, 0)$
$\overrightarrow{v_0} = (v_{01}, 0, 0)$

$$m\frac{dv_1}{dt} = 0$$
$$m\frac{dv_2}{dt} = -mg$$
$$m\frac{dv_3}{dt} = 0$$

$$dv_1 = \int{dt}$$
$$v_1 = C_1 = v_01$$
$$v_2 = -gt + C_2 = -gt$$
$$v_3 = C_3 = 0$$

$$x_1 = \int{v_{01}dt}= v_{01}t + C_4 = v_{01}t$$
$$x_2 = -\frac{1}{2}gt + C_5 = -\frac{1}{2}gt + h$$
$$x_3 = C_6 = 0$$

\end{priklad}

\begin{priklad} Harmonický pohyb
$$\overrightarrow{v} = (v_{01}, v_{02}, 0)$$
$$\overrightarrow{r} = (r_0 \cos \alpha, r_0 \sin \alpha, O)$$

$$\frac{d^2x}{dt^2} = -\frac{k}{m}x$$
$$x = C \sin \omega t, [\omega]=s^{-1}$$
$$-C\omega^2\sin \omega t = \frac{k}{m}(\sin \omega t)$$
$$\omega ^2 = \frac{k}{m}$$
$$\omega = \sqrt{\frac{k}{m}}$$

$$x = C_1\sin \omega t+ C_2\cos \omega t$$
$$x = A (sin \omega t + \varphi)$$

trošku jinak (exp. fce rovnost siny cosy)
$$x = Ce^{\lambda x}$$
$$C\lambda^2e^{\lambda x} = -\frac{k}{m}e^{\lambda x} C$$
$$\lambda^2 = -\frac{k}{m}$$
$$\lambda = \pm i\sqrt{\frac{k}{m}}=\pm i \omega$$

$$x = C_1e^{i \omega t}+C_2e^{-i \omega t}$$

$\rightarrow$ kdykoliv když vidím dif. rci tvaru $x^{\dot \dot}+ \omega^2x = 0$ je to harm. pohyb a jdu na siny
\end{priklad}

\begin{priklad} Pohyb při odporu prostředí

$$m\frac{d^2x}{dt^2}=-kv$$
$$m\frac{dv}{dt}=-kv$$
$$\int_{v_0}^{v}{\frac{dv}{v}}=\int_{0}^{t}{-\frac{k}{m}dt}$$
$$\ln{v}-\ln{v_0}=-\frac{k}{m}t$$
$$\frac{dx}{dt} = v = v_0e^{-\frac{k}{m}t}$$

$$\int_{0}^{x}{dx} = \int_{0}^{t}{v_0e^{-\frac{k}{m}t}dt}$$
$$x = -v_0\frac{m}{k}[e^{-\frac{k}{m}t}-1]=\frac{v_0m}{k}[1-e^{-\frac{k}{m}t}]$$
\end{priklad}

\begin{priklad} Kladka TODO Obrazek slide č. 12, TODO celý
\begin{itemize}
\item nehmotné kladky, nehmotné závěsy, tělesa hmotnosti $m_1$ a $m_2$, $m_1>m_2$
\item pohyb v jednom směru $\rightarrow$ skalární značení
\item těžší těleso klesá $\rightarrow$ kladný směr zrychlení $a$
\item tahová síla vlákna F
$$m_1g - F = m_1a$$
$$m_2g - F = m_2a$$

$$a = g \frac{m_1-m_2}{m_1+m_2}$$
\end{itemize}
\end{priklad}

\subsubsection{Pohyb soustavy?}
$$\overrightarrow{r} = \overrightarrow{r}^\prime + \overrightarrow{R}$$
$$\overrightarrow{R}^\prime = \overrightarrow{u}t$$
$$\overrightarrow{v}^\prime = \overrightarrow{v}-\overrightarrow{u}$$
$$\overrightarrow{a}^\prime = \overrightarrow{a} + \overrightarrow{a_u}$$
$$\overrightarrow{a_u} = \frac{d^2\overrightarrow{R}}{dt^2}=$$

\subsection{Pohyb v neinerciální soustavě}

pokud $\overrightarrow{a_u}\neq 0$:
$$\overrightarrow{F}^\prime = m\overrightarrow{a}^\prime = m (\overrightarrow{a}-\overrightarrow{a_u})=\overrightarrow{F}-m\overrightarrow{a_u}=\overrightarrow{F}+\overrightarrow{F^*}$$
kde $\overrightarrow{F^*}=-m\overrightarrow{a_u}$ je setrvačná síla

a teďka experimenty

sloupeček magnetů zespoda je odpinkávám pravítkem, jestli pomalu posune se celý sloupeček, pokud rychle odělávám jednotlivé magnety

koule na laně, pod ní na laně držátko, když trhnu rychle utrhnu spodní nit, když pomalu horní i s koulí -- šíření síly

setrvačnost korku v lahvi vody -- korek se při zrychlení pohne ve směru pohybu -- je vytlačen vodou s větší hustotou, ta se chová podle očekávání (lags behind)

foucaultovo kyvadlo ukázka corriolise rotace země

$\overrightarrow{F}$ není pravá síla
\begin{priklad}
    Pohyb v otáčivé neinerciální soustavě

    mějme polární souřadnice, pak
    $\overrightarrow{v}^\prime=\overrightarrow{v}-\overrightarrow{u}=\overrightarrow{v}-\overrightarrow{\omega} \times \overrightarrow{r}$, tedy $\frac{d^\prime\overrightarrow{r}^}{dt}=\frac{d\overrightarrow{r}}{dt}-\overrightarrow{\omega}\times \overrightarrow{r}$ tj. časové derivace téhož vektoru v různých s. s. se líší!
    tyto vztahy platí pro lib. vektor tj. $\frac{d^\prime\overrightarrow{A}}{dt}=\frac{d\overrightarrow{A}}{dt}-\overrightarrow{\omega}\times \overrightarrow{A}$

    $\vec a^\prime = \frac{d^\prime v^\prime}{dt}=\frac{dv^\prime}{dt}-\vec \omega \times \vec v^\prime = \frac{d \vec v}{dt}-\frac{d\vec \omega}{dt}\times \vec r - \vec \omega \times \frac{d \vec r}{dt}-\vec \omega \times \vec v^\prime$
\end{priklad}

TODO odstředivá, eulerova, coriolisova síla slide č. 20
TODO cvika příklad 2.14.

Newtonovy pohybové rce v základním tvaru platí jen v inerciálních soustavách, v neinerciálních je třeba do vztahů doplnit setrvačné síly

\subsubsection{Další veličiny}
hybnost: $\vec p = m \vec v$
moment síly: $\vec M = \vec r \times \vec F = (\vec r_B - \vec r_A)\times \vec F$
moment hybnosti: $\vec L = \vec r \times p = \vec r \times m \vec v = (\vec r_B \vec r_A)\times m \vec v$
impuls síly (časový účinek síly): $\vec I = \vec F \delta t \rightarrow \sum_{i=1}^{n}{\vec F_i \delta t_i} \rightarrow \int_{t_1}^{t_2}{\vec F dt} = \int_{t_1}^{t_2}{\frac{d\vec p}{dt}dt} = \delta \vec p$
změna hybnosti h. b. = impulsu síly vykonaného na h. b.
TODO slide č. 25, 26
3. NZ jako důsledek ZZH, ZZMH

experimenty:
