\section{Taylorův polynom}

\begin{motivace}
  Chci aproximaci $f(x) = e^{-2x}$ v bodě $x=0$, hledáme aproximaci $p(x) = a + bc + cx^2$

  Nikoho nepřekvapí, že lineární aproximace bude tečna, tedy $p(x) = f(0) + f^\prime (0) x = 1- 2x$

  Chceme ji ale nějak vylepšit, zpřesnit, a to tím, že přidáme kvadratický člen. Jak ho ale najdeme?

  IDEA: napasujeme vyšší derivace, tj. $f^{\prime\prime}(0) = p^{\prime \prime} (0)$

  $p^{\prime\prime}(0) = 2c, f^{\prime\prime}(0)= 4 \implies c = 2 \implies p(x) = 1 - 2x + 2x^2$

  Jak toto dělat obecně? Jak určit další vyšší členy? Když aproximaci useknu, jak velká je chyba?
\end{motivace}

\begin{definice}
  Nechť $f(x), g(x)$ jsou definovány na $\mathscr P(x_0)$. Řekneme, že
  \begin{itemize}
    \item $f(x)$ je malé ó od $g(x)$ pro $x\to x_0 $, pokud $\frac{f(x)}{g(x)} \to 0, x \to x_0$
    \item $f(x)$ je velké Ó od $g(x)$ pro $x  \to x_0$, pokud $\exists C, \delta > 0$ t. ž. $|f(x)| \leq C|g(x)|\,\forall x \in \mathscr P(x_0, \delta)$
    \item $f(x)$ je řádově rovno $g(x)$ pro $x \to x_0$, pokud $\exists a \in \mathbb R \smallsetminus \{0\}$ t. ž. $\frac{f(x)}{g(x)} \to a, x \to x_0$
  \end{itemize}
  Značíme $f(x) = o(g(x))$ resp. $f(x) = O(g(x))$ resp. $f(x) \sim g(x), x \to x_0$
\end{definice}

\begin{priklad}
  \begin{enumerate}[1.]
    \item $\ln x = o(\sqrt x), x \to + \infty \dots \frac{\ln x }{\sqrt x} \to 0, x \to + \infty$
    \item $\frac{\sin x + \cos x}{x^2 + 1} = O(\frac{1}{x^2}), x \to + \infty \dots |f(x)| \leq \frac{2}{x^2} = 2 \frac{1}{x^2} = C g(x)$
    \item $\sin x \sim x, 1-\cos x \sim x^2, x \to 0 \dots$ (pouze pro cos) $\frac{1-\cos x}{x^2} \to \frac{1}{2} (= a), x \to 0$
  \end{enumerate}
\end{priklad}

\begin{definice}[Derivace vyšších řádů]
  \begin{enumerate}[i.]
    \item $f^{(0)}(x) = f(x)$
    \item $f^{(k+1)}(x) = (f^{(k)}(x))^\prime$, tj. $f^{(1)}(x) = f^\prime (x), f^{(2)}(x) = f^{\prime\prime}(x)$
  \end{enumerate}
  Pro $I \subseteq \mathbb R$ otevřený interval definuji
  $$C^n(I)=\left\{f(x):I \to \mathbb R; f^{(t)}\text{ existuje a je spojitá v }I\, \forall k = 0, 1, \dots n\right\}$$
  Speciálně $C^0(I) = C(I) = \left\{f(x): I \to  \mathbb R \text{ je spojité v } \mathbb R\right\}$
\end{definice}

\begin{definice}
  Pro $x_0 \in \mathbb R, k \geq 0$ celé označme $Q_{k, x_0}(x) = \frac{1}{k!}(x-x_0)^k$, speciálně $Q_{0, x_0}(x) \equiv 1$, $Q_{1, x_0}(x) = \frac{1}{2}(x-x_0)^2$, atd.
\end{definice}

\begin{lemma}[8.1. Vlastnosti funkcí $Q_{k, x_0}$]
  Platí \begin{enumerate}[i.]
    \item $Q_{k, x_0}$ je polynom stupně $k$
    \item $Q^\prime_{0, x_0} \equiv 0, Q^\prime_{k, x_0}=Q_{k-1, x_0}\,\forall k \geq 1$
    \item $Q^{(l)}_{k, x_0} =
      \begin{cases}
        1, l = k \\
        0, l \neq k
      \end{cases}
      \forall k, l \geq 0$ celé
  \end{enumerate}
\end{lemma}

\begin{proof}
  \begin{enumerate}[i.]
    \item $(x-x_0)^k \stackrel{\text{(binom. v.)}}{=} x^k + \dots$
    \item $1^prime = 0$; $(Q_{k, x_0})^\prime = \frac{1}{k!} ((x-x_0)^k)^\prime = \frac{1}{k!} k (x-X-0)^{k-1} = \frac{1}{(k-1)!}(x-x_0)^{k-1} = Q_{k-1, x_0}$
    \item $l = k: Q^{(k)}_{k, x_0}(x_0) = Q_{0, x_0}(x_0) = 1; l > k: Q^{(l)}_{k, x_0} \equiv 0; l < k$ tj. $k = l+s, s \geq 1: Q^{(l)}_{k, x_0}=Q_{s, x_0} = \frac{1}{s!}(x-x_0)^s, x = x_0 \implies Q_{s, x_0}(x_0)=0$
  \end{enumerate}
\end{proof}

\begin{definice}
  Nechť $f(x) \in C^n(\mathscr U(x_0))$. Potom výraz
  $$\sum_{k = 0}^n {\frac{f^{(k)}(x_0)}{k!}(x-x_0)^k}$$
  nazveme Taylorův polynom funkce $f(x)$ v bodě $x_0$ stupně $n$ (též $n$-tý Taylorův polynom). Značíme $T^{f}_{x_0, n}(x)$. Alternativně $T^f_{x_0, n}(x) = \sum_{k=0}^n f^{(k)}(x_0) \cdot Q_{k, x_0}(x)$.
\end{definice}

\begin{priklad}
  \begin{enumerate}[1.]
    \item $f(x) = e^x, x_0 = 0$; $f^{(k)}(x) = e^x \implies f^{(k)}(0) = 1 \, \forall k \geq 0$.
    $$T^{e^x}_{0, n}(x) = \sum^n_{k= 0} \frac{x^k}{k!} = 1 + x + \frac{x^2}{2} + \dots + \frac{x^n}{n!}$$
    \item $f(x) = \sin x, x_0 = 0$; $f^{(k)}(x): \sin x, \cos x, -\sin x, -\cos x, \dots$; $f^{(k)}(0): 1, 0, -1, 0, \dots$
    $$T_{0, 2n+1}^{\sin x} = x- \frac{1}{3!}x^3 + \frac{1}{5!}x^5+ \dots + (-1)^n \frac{x^{2n+1}}{(2n+1)!}$$
    \item $f(x) = (1+x)^a, x_0 = 0, a \in \mathbb R \smallsetminus \{0, 1\}$; $f^{(k)}(x) = a\cdot(a-1)\cdot \dots \cdot (a-k+1)\cdot(1+x)^{a-k}$; $ f^{(k)}(0) = a(a-1)\dots(a-k+1)$.
    $$T^{(1+x)^a}_{0, n}(x) = \sum_{k=0}^n \frac{a(a-1)\dots(a-k+1)}{k!}x^k = 1 + ax + \frac{a(a-1)}{2}x^2 + \frac{a(a-1)(a-2)}{6}x^3 + \dots$$
  \end{enumerate}
\end{priklad}

\begin{veta}[8.1. Aproximační vlastnost Taylorova polynomu]
  Nechť $f(x) \in C^n(\mathscr U(x_0))$. Potom $f(x) = T^{f}_{x_0, n}(x)+o((x-x_0)^n), x\to x_0$. Navíc $T^{f}_{x_0, n}(x)$ je jediný polynom stupně $\leq n$ s touto vlastností.
\end{veta}

\begin{proof}
  BÚNO $x_0 = 0$, označme $p(x) = T^f_{0, n}(x) = \sum_{k=0}^n f^{(k)}(0)\frac{x^k}{k!}$

  klíčové pozorování: $p^{(l)}(0) = f^{(l)}(0)\, \forall l = 0, \dots, n$

  dk. L. 8. 1. $\implies p^{(l)}(0) = \sum_{k=0}^n(f^{(k)}(0)Q_k(x))^{(l)}|_{x=0} = f^{(l)}(0)$

  \begin{enumerate}[1.]
    \item aprox. vlast.: cíl: $\frac{f(x)-p(x)}{(x)^n}\to 0, x \to 0$ -- metoda L´Hospital $\frac{0}{0}$ $n$-krát

    po $n$-tém kroku: $\frac{f^{(n)}(x)-p^{(n)}(x)}{n!}$, když dosadím $x = 0 \implies \frac{0}{n!} = 0$
    \item jedonznačnost: nechť $\exists g(x) \dots$ polynom st. $\leq n$ t. ž. $\frac{f(x)-g(x)}{(x)^n}\to 0, x \to 0$ \dots cíl: $g(x) \equiv p(x) = T^f_{0, n}(x)$

    pomocná úvaha: $\frac{p(x)-g(x)}{x^n} = \frac{f(x)-g(x)}{x^n}-\frac{f(x)-p(x)}{x^n}\to 0-0 = 0, x \to 0$

    pišme $p(x) = \sum_{k=0}^n {a_k x^k}, g(x) = \sum_{k=0}^n {b_k x^k}$

    sporem: když se $p(x), g(x)$ nerovnají $\exists s \in \{0, \dots, n\}$ t. ž. $a_s \neq b_s$, BÚNO $s$ nejmenší takové $\implies p(x)-g(x) = \sum_{k = s}^n (a_k -b_k)x^k, a_k - b_k \eqcolon c_k \implies p(x) - g(x) = c_s x^s + \sum_{k=s+1}^n c_k x^k$

    potom $\frac{p(x)-g(x)}{x^s}=
    \begin{cases}
      c_s + \sum_{k = s+1}^n c_k x^{k-s}$, když $x \to 0$, výraz $\to c_s \neq 0 \\
      \frac{p(x)-g(x)}{x^n}\cdot x^{n-s}$ (viz pomocná úvaha) $\to 0$ -- SPOR$
    \end{cases}$
  \end{enumerate}
\end{proof}

\begin{poznamka}
  Polynomická funkce je přesně rovna svému Taylorovu polynomu pro lib. $x_0 \in D(f)$.
\end{poznamka}

\begin{veta}[8.2. Integrál a derivace Taylorova polynomu]
  Nechť $F(x) \in C^{n+1}(\mathscr U(x_0))$, nechť $F^\prime (x) = f(x)$, tj. $F(x) = \int f(x) dx$ v $\mathscr U(x_0)$. Potom
  \begin{enumerate}[1.]
    \item $(T^F_{x_0, n+1}(x))^\prime = T^f_{x_0, n}(x)\,\forall x \in \mathscr U (x_0)$,
    \item $\int T^f_{x_0, n}(x) dx = T^F_{x_0, n+1}(x) + c \, \forall x \in \mathscr U(x_0)$ a vhodně zvolené $c$.
  \end{enumerate}
\end{veta}

\begin{proof}
  \begin{enumerate}[1.]
    \item $T^F_{x_0, n+1}(x) = \sum_{k=0}^{n+1} F^{(k)}(x_0)\cdot Q_{k, x_0}(x)$ -- zderivujeme

    $\sum_{k=1}^{n+1}F^{(k)}(x_0)\cdot Q_{k+1, x_0}(x) = \sum_{l=0}^{n}F^{(l+1)}(x_0)\cdot Q_{l, x_0}(x) = T^f_{x_0, n}(x)$
    \item $\int T^f_{x_0, n}(x) dx = \int \sum_{k=0}^{n}f^{(k)}(x_0)\cdot Q_{k, x_0}(x) dx = \sum_{l=1}^{n+1}F^{(l)}(x_0)\cdot Q_{l+1, x_0}(x) + c = T^F_{x_0, n+1}(x)$ pro vhodně zvolené $c = F(x_0)\cdot Q_{NECO}(x) = F(x_0)$
  \end{enumerate}
\end{proof}

\begin{priklad}
  \begin{enumerate}[1.]
    \item $\cos x = (\sin x)^\prime $ TODO
    \item $\ln 1+x = \int \frac{dx}{1+x}$ TODO
  \end{enumerate}
\end{priklad}

\begin{priklad}
  $$\lim_{x \to 0} \frac{x-\sin x}{x^3} = \lim_{x \to 0} \frac{x-(x- \frac{x^3}{6}+o(x^3))}{x^3} = \lim_{x \to 0} \frac{1}{6} - \frac{o(x^3)}{x^3} = \frac{1}{6}$$
\end{priklad}

\begin{veta}[8.3. Početní pravidla pro malé ó]
  Nechť $m, n \in \mathbb Z$. Potom
  \begin{enumerate}[1.]
    \item $f(x) = o(x^n), g(x) = o(x^m), x \to 0,$ kde $m \geq n; a, b \in \mathbb R \implies af(x) + bg(x) = o(x^n), x \to 0,$
    \item $f(x) = o(x^n), x \to 0, a \in \mathbb R \implies ax^m f(x) = o(x^{m+n}), x \to 0,$
    \item $f(x) = o(x^n), g(x) = o(x^m), x \to 0 \implies f(x)g(x) = o(x^{m+n}), x \to 0,$
    \item $f(x) = o(x^n), g(x) \sim x^m, x \to 0$, kde $m \geq 1 \implies f(g(x)) = o(x^n+m), x \to 0.$
  \end{enumerate}
\end{veta}

\begin{proof}
  \begin{itemize}
    \item[1.] víme: $\frac{f(x)}{x^n}, \frac{g(x)}{x^m} \to 0, x \to 0$ \dots cíl $\frac{af(x)+bg(x)}{x^n} \to 0, x \to 0$

    $a \frac{f(x)}{x^n} + b \frac{g(x)}{x^m}\cdot x^{m-n} \to (VoAL) 0, x \to 0$
    \item[4.] víme: $\frac{f(y)}{y^n} \to 0, y \to 0; \frac{g(x)}{x^m} \to c, x \to 0$, kde $c \in \mathbb R \smallsetminus \{0\}$

    cíl: $\frac{f(g(x))}{x^{m+n}}\to 0 x \to 0$

    TRIK $\frac{f(g(x))}{(g(x))^n}\cdot (\frac{g(x)}{x^m})^n = P_1 \cdot P_2$

    $P_1 \to 0$, ale VoLSF (b) \dots $g(x) \to 0, x \to 0; g(x) \neq 0$ na $\mathscr P(x_0, \delta), \delta > 0$ malé

    $g(x) = \frac{g(x)}{x^n} x^m \to c \cdot 0 = 0$,

    $\frac{g(x)}{x^m} \to c \neq 0 \implies \frac{g(x)}{x^m} \neq 0$ na $\mathscr P(0, \delta) \implies g(x) \neq 0$ na $ \mathscr P(, \delta)$
  \end{itemize}
\end{proof}

\begin{poznamka}[Doplněk]
  \begin{itemize}
    \item[$2^\prime$] $f(x) = o(x^n), x \to 0, (m \leq n) \implies \frac{f(x)}{x^m} = o(x^{n-m})$
  \end{itemize}
\end{poznamka}

\begin{priklad}
  \begin{enumerate}[1.]
    \item $x \to 0, \frac{x^3}{\sin x \ln (1+x^2)} = \frac{x^3}{(x+o(x))(x^2+o(x^2))}=\frac{x^3}{x^3+x\cdot o(x^2)+x^2 \cdot o(x) + o(x)o(x^2)} = \frac{x^3}{x^3 + o(x^3) + o(x^3) + o(x^3)} = \frac{x^3}{x^3 + o(x^3)} = 1 + \frac{x^3}{o(x^3)} \to 1$
    \item $T^{tg}_0, 5(x) = Ax + Bx^3 + Cx^5+ o(x^5); \tg (x) = \frac{\sin x}{\cos x}$

    $\tg x \cdot \cos x = \sin x$
    $(Ax + Bx^3 + Cx^5+ o(x^5))(1- \frac{x^2}{2} + \frac{x^4}{24}+o(x^4))=x- \frac{x^3}{6} + \frac{x^5}{120} + o(x^5)$ a dále M. P. K.
    $T = x - \frac{x^3}{3} + \frac{2x^5}{15}$
  \end{enumerate}
\end{priklad}

\begin{poznamka}
  Čím větší řád taylorova polynomu, tím přesnější odhad = lepší řád aproximace, bude ale chyba v nekonečnu nulová?
\end{poznamka}

\begin{definice}
  Zbytek po Taylorově polynomu nazveme $R^f{x_0, n+1}(x) = f(x) - T^f_{x_0, n}(x)$.
\end{definice}

\begin{veta}[8. 4. Odhad zbytku Taylorova polynomu]
  Nechť $f(x) \in C^{n+1}(I)$, kde $I$ je otevřený interval, nechť $x, x_0 \in I, x \neq x_0$. Potom:
  \begin{enumerate}[1.]
    \item $\exists \theta$ mezi $x, x_0$ t. ž. $R^f{x_0, n+1}(x) = \frac{f^{(n+1)}(\theta)}{(n+1)!}(x-x_0)^{n+1}$ (L)
    \item $\exists \theta$ mezi $x, x_0$ t. ž. $R^f{x_0, n+1}(x) = \frac{f^{(n+1)}(\theta)}{n!}(x-\theta)(\theta-x_0)^n$ (C)
  \end{enumerate}
  Výraz napravo se nazývá Lagrangeův, resp. Cauchyho tvar zbytku.
\end{veta}

\begin{proof}
  TRIK pomocná fce $\varphi(t), t \in [x_0, x], x_0, x \dots$ pevné, kde $$\varphi(t) \coloneq f(x) - T^{f}_{t, n}(x) = f(x) - \sum_{k=0}^{n} \frac{f^{(k)}(t)}{k!}(x-t)^k$$
  vidíme: $\varphi(x_0) = R^{f}_{x_0, n+1}(x)$, $\varphi(x) = f(x) - T^{f}_{x, n}(x) = f(x) - f(x) - \frac{f^\prime (x)}{1!}(x-x) - \frac{f^{\prime\prime}(x)}{2!}(x-x) - \dots = f(x)- f(x) + 0 = 0$

  pomocný výpočet: $$\varphi^\prime (t) = \frac{d}{dt}\varphi(t) = -\left[f(t)+\sum_{k = 1}^{n}f^{(k)}(t)\frac{(x-t)^k}{k!}\right] = -f^\prime (t) - \sum_{k=1}^{n} f^{(k+1)}(t)\frac{(x-t)^k}{k!}+\sum_{k=1}^{n} f^{(k)}(t)\frac{(x-t)^{k-1}}{(k-1)!}=$$ $$= -\sum_{l=1}^{n-1} f^{(l)}(t) \frac{(x-t)^{l-1}}{(l-1)!} + \sum_{l=1}^{n} f^{(l)} \frac{(x-t)^{l-1}}{(l-1)!} = \frac{-f^{(n+1)}(t)}{n!}(x-t)^n$$
  použiji V. 6. 8. (C. o stř. h.): $\frac{\varphi(x)-\varphi(x_0)}{\psi(x)-\psi(x_0)} = \frac{\varphi^\prime(\theta)}{\psi^\prime(\theta)}$

  volíme $\psi(t) = (x-t)^{n+1} \implies$ (L), resp. $\psi (t) = t \implies$ (C)

  podrobně ad (L): LS$ = \frac{-R^{f}_{x_0, n+1}(x)}{-(x-x_0)^{n+1}} = \frac{\frac{-f^{(n+1)(\theta)}}{n!}(x-\theta)^n}{-(n+1)(x-\theta)^n} = $PS
\end{proof}

\begin{dusledek}
  $f(x) \in C^{n+1}(\mathscr U(x_0)) \implies R^{f}_{x_0, n+1}(x) = O((x-x_0)^{n+1}), x \to x_0$
\end{dusledek}
\begin{proof}
  $|R^{f}_{x_0, n+1}(x)| \leq |\frac{f^{(n+1)(\theta)}}{(n+1)!}(x-x_0)^n+1|\leq C|x-x_0|^{n+1}$, protože když je $f(x_0)$ spojitá dle L. 6. 1. je omezená na nějakém $\mathscr U(x_0)$
\end{proof}

\begin{priklad}
$f(x) = e^x, x_0 = 0 \dots R^{\exp}_{0, n+1}(x) \to 0, n \to +\infty, x \in [-M, M]$ pevné,

neboť $|f^{n+1}(\theta)| = |e^\theta| \leq e^M$, neboť $\theta \in [-M, M]$

$|R^{\exp}_{0, n+1(x)}|\leq e^M \frac{M^{n+1}}{(n+1)!}\to 0, n \to + \infty$

$\implies e^x = T^{\exp}_{0, n}(x) + R^{\exp}_{0, n+1}(x) \implies e^x = \lim_{n \to +\infty} \sum_{k=0}^{n} \frac{x^k}{k!} = \sum_{k=0}^{+\infty} \frac{x^k}{k!}$

\end{priklad}
