\documentclass{article}

% Language setting
% Replace `english' with e.g. `spanish' to change the document language
\usepackage[czech]{babel}
\usepackage{amsthm,thmtools,xcolor,amsmath,amssymb, mathtools}

% Set page size and margins
% Replace `letterpaper' with `a4paper' for UK/EU standard size
\usepackage[a4paper,top=2cm,bottom=2cm,left=3cm,right=3cm,marginparwidth=1.75cm]{geometry}

% Useful packages
\usepackage{amsmath}
\usepackage{graphicx}
\usepackage[colorlinks=true, allcolors=blue]{hyperref}
\graphicspath{images/}
\usepackage{enumerate}

%\usepackage{tikzit}
%\input{default.tikzstyles}

\title{DÚ Lineární algebra -- Sada 1}
\author{Jan Romanovský}

\begin{document}
\maketitle

\textbf{(1.2)} Označme $O_p : \mathbb R^2 \to \mathbb R^2$ osovou symetrii podle přímky $p$ a $O_q : \mathbb R^2 \to R^2$ osovou symetrii podle přímky $q$, kde

$$p = \{(1, 0) + t(0, 1) : t \in \mathbb R\}, q = {(x, y) \in \mathbb R^2: x + y = 4}.$$

\begin{enumerate}[a)]
    \item Najděte obraz bodu $(x, y)$ při zobrazení $O_p$ a při zobrazení $O_q$.
    \item Najděte obraz bodu $(x, y)$ při složených zobrazení $O_q \circ O_p$ a $O_p \circ O_q$.
\end{enumerate}
\,\newline

\begin{enumerate}[a)]
    \item Vyčteme z grafu a definice osové souměrnosti (vidíme, že zobrazujeme počátek a souřadné osy):
    \begin{itemize}
        \item $O_p: (x,y) \mapsto (-x+2,y)$
        \item $O_q: (x,y) \mapsto (4-y, 4-x)$
    \end{itemize}
    \item Zjistíme tak, že najdeme obraz obrazu, obdobně jak v a), tzn.:
        \begin{itemize}
            \item $O_q \circ O_p: O_p: (x,y)\mapsto (-x+2,y)$, $O_q: (-x+2,y) \mapsto (4-y,2+x)$
            \item $O_p \circ O_q: O_q: (x,y) \mapsto (4-y, 4-x)$, $O_p: (4-y, 4-x) \mapsto (y-2, 4-x)$
        \end{itemize}
\end{enumerate}
\end{document}
