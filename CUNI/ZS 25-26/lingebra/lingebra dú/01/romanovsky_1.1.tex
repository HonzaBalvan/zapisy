\documentclass{article}

% Language setting
% Replace `english' with e.g. `spanish' to change the document language
\usepackage[czech]{babel}
\usepackage{amsthm,thmtools,xcolor,amsmath,amssymb, mathtools}

% Set page size and margins
% Replace `letterpaper' with `a4paper' for UK/EU standard size
\usepackage[a4paper,top=2cm,bottom=2cm,left=3cm,right=3cm,marginparwidth=1.75cm]{geometry}

% Useful packages
\usepackage{amsmath}
\usepackage{enumerate}
\usepackage{graphicx}
\usepackage[colorlinks=true, allcolors=blue]{hyperref}
\graphicspath{images/}

%\usepackage{tikzit}
%\input{default.tikzstyles}

\title{DÚ Lineární algebra -- Sada 1}
\author{Jan Romanovský}

\begin{document}
\maketitle

\textbf{(1.1)} Pro která $a \in \mathbb R$ je zobrazení $f_a : \mathbb R^2 \rightarrow \mathbb R^2$ \uv{na}?
$$f_a(x, y) = \left((a + 2)x + ay, ax + y\right)$$

Převedeme předpis fce na soustavu rovnic, pokud $\exists$ řešení pro každou pravou stranu, tj. \uv{parametry} na sobě budou nezávislé, pokryjí celou množinu do které zobrazujeme, tj. fce bude \uv{na}. Hledáme $a$ pro která toto platí:
\begin{align*}
    (a+2)x+ay &= k\\
    ax + y &= l
\end{align*}
\begin{enumerate}[i.]
    \item $a = 0$:
    \begin{align*}
        x &= \frac{k}{2}\\
        y &= l
    \end{align*}
    \item $a \neq 0$:
    \begin{gather*}
        y = \frac{k-(a+2)x}{a}\\
    ax + \frac{k-(a+2)x}{a} = l\\
    ax-x-\frac{2}{a}x=l-\frac{k}{a}\\
    \left(a-1-\frac{2}{a}\right)x=l-\frac{k}{a}
    \end{gather*}
    \begin{enumerate}[1.]
        \item $a-1-\frac{2}{a}= 0\implies a^2-a-2=0\implies (a-2)(a+1) = 0:$
            \begin{gather*}
                k=al
            \end{gather*}
        \item $a-1-\frac{2}{a}\neq 0:$
        $$x = \frac{l-\frac{k}{a}}{a-1-\frac{2}{a}}$$
    \end{enumerate}

\end{enumerate}
Vidíme, že v případech $a=2, a=-1$ nelze $(k, l)$ vybrat libovolně, fce $f_a$ pro ně tedy není \uv{na}. Jinak vždy jindy lze vybrat $(k, l)$ libovolně, funkce $f_a$ pro $a \in \mathbb R \smallsetminus \{2, -1\}$ je \uv{na}.
\end{document}
