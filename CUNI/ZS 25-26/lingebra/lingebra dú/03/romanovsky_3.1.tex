\documentclass{article}

% Language setting
% Replace `english' with e.g. `spanish' to change the document language
\usepackage[czech]{babel}
\usepackage{amsthm,thmtools,xcolor,amsmath,amssymb, mathtools}

% Set page size and margins
% Replace `letterpaper' with `a4paper' for UK/EU standard size
\usepackage[a4paper,top=2cm,bottom=2cm,left=3cm,right=3cm,marginparwidth=1.75cm]{geometry}

% Useful packages
\usepackage{amsmath}
\usepackage{graphicx}
\usepackage[colorlinks=true, allcolors=blue]{hyperref}
\graphicspath{images/}
\usepackage{enumerate}

%\usepackage{tikzit}
%\input{default.tikzstyles}

\title{DÚ Lineární algebra -- Sada 3}
\author{Jan Romanovský}

\begin{document}
\maketitle

\textbf{(3.1)} Hledáme uspořádané čtveřice $(x_1, x_2, x_3, x_4)$. Nalezneme prvně neutrální prvek vůči sčítání (ozn. $0$) a neutrální prvek vůči násobení (ozn. $1$) dle jejich definice.
$$\forall x \in \textbf{T}: 0 + x = x$$
$$\forall x \in \textbf{T} \smallsetminus \{0\}: 1\cdot x = x$$
Z tabulek operací vidíme, že $0 = \gamma$ (sloupeček pod $\gamma$ v tabulce sčítání je nezměněn) a $1 = \beta$ (sloupeček pod $\beta$ v tabulce násobení je nezměněn). Dále najdeme opačné a inverzní prvky (z definice a tabulek).
$$\alpha: \alpha + (-\alpha) = \gamma \implies \underline{-\alpha = \alpha};\, \alpha \cdot \alpha^{-1} = \beta \implies \underline{\alpha^{-1} = \delta}$$
$$\beta: \beta + (-\beta) = \gamma \implies \underline{-\beta = \beta};\, \beta \cdot \beta^{-1} = \beta \implies \underline{\beta^{-1} = \beta}$$
$$\gamma: \gamma + (-\gamma) = \gamma \implies \underline{-\gamma = \gamma};\, \text{ 0 nemá inverzi}$$
$$\delta: \delta + (-\delta) = \gamma \implies \underline{-\delta = \delta};\, \delta \cdot \delta^{-1} = \beta \implies \underline{\delta^{-1} = \alpha}$$
Nyní převedeme matici do odstupňovaného tvaru.
$$\left(\begin{array}{c c c c | c}
    \alpha & \alpha & \delta & \alpha &\delta\\
    \beta & \delta & \beta & \delta & \delta\\
    \gamma & \beta & \alpha & \beta & \delta
\end{array}\right) \sim
\left(\begin{array}{c c c c | c}
    \alpha & \alpha & \delta & \alpha &\delta\\
    1 & \delta & 1 & \delta & \delta\\
    0 & 1 & \alpha & 1& \delta
\end{array}\right) \sim
\left(\begin{array}{c c c c | c}
    \alpha & \alpha & \delta & \alpha &\delta\\
    0 & \delta & 1 & \delta & \alpha\\
    0 & 1 & \alpha & 1& \delta
\end{array}\right) \sim
\left(\begin{array}{c c c c | c}
    \alpha & \alpha & \delta & \alpha &\delta\\
    0 & \delta & 1 & \delta & \alpha\\
    0 & 0 & 0 & 0& 0
\end{array}\right)$$
Dořešíme zpětným dosazením.
\begin{align*}
x_4 &= t_4\\
x_3 &= t_3\\
x_2 &= (x_3 + \delta x_4 + \alpha) \cdot \delta^{-1} = (t_3 + \delta t_4 + \alpha)\cdot \alpha = \alpha t_3 + t_4 + \delta \\
x_1 &= (\alpha x_2 + \delta x_3 + \alpha x_4 + \delta)\cdot \alpha^{-1}= (\alpha(\alpha t_3 + t_4 + \delta)+\delta t_3 + \alpha t_4 + \delta)\cdot \delta =(\alpha t_3 + t_4 + \delta) + \alpha t_3 +\\
&=t_4 + \alpha = 1
\end{align*}

$$P = \left\{\begin{pmatrix}\beta \\ \delta \\ \gamma \\ \gamma\end{pmatrix}+t_3\begin{pmatrix}\gamma \\ \alpha \\ \beta \\ \gamma\end{pmatrix}+t_4\begin{pmatrix}\gamma \\ \beta \\ \gamma \\ \beta\end{pmatrix};t_3, t_4 \in \textbf{T}\right\}.$$

\noindent Máme dva parametry (pivotů je o dva méně než proměnných), které budou nabývat hodnot prvků tělesa, ty jsou čtyři. Celkem je tedy $4\cdot 4 = 16$ řešení.
\end{document}
