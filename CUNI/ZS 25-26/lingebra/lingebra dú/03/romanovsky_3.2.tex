\documentclass{article}

% Language setting
% Replace `english' with e.g. `spanish' to change the document language
\usepackage[czech]{babel}
\usepackage{amsthm,thmtools,xcolor,amsmath,amssymb, mathtools}

% Set page size and margins
% Replace `letterpaper' with `a4paper' for UK/EU standard size
\usepackage[a4paper,top=2cm,bottom=2cm,left=3cm,right=3cm,marginparwidth=1.75cm]{geometry}

% Useful packages
\usepackage{amsmath}
\usepackage{graphicx}
\usepackage[colorlinks=true, allcolors=blue]{hyperref}
\graphicspath{images/}
\usepackage{enumerate}

%\usepackage{tikzit}
%\input{default.tikzstyles}

\title{DÚ Lineární algebra -- Sada 3}
\author{Jan Romanovský}

\begin{document}
\maketitle

\textbf{(3.2)} Označme
$X = \begin{pmatrix}
    a & b\\
    c & d
\end{pmatrix}.$
Potom rovnost $XA = AX$ můžeme přepsat následujícím způsobem.
$$\begin{pmatrix}
    a & b\\
    c & d
\end{pmatrix}\begin{pmatrix}
    2 & 1\\
    3 & 0
\end{pmatrix}=\begin{pmatrix}
    2 & 1\\
    3 & 0
\end{pmatrix}\begin{pmatrix}
    a & b\\
    c & d
\end{pmatrix}$$
$$\begin{pmatrix}
    2a+3b & a\\
    2c+3d & c
\end{pmatrix}=\begin{pmatrix}
    2a + c & 2b + d\\
    3a & 3b
\end{pmatrix}$$
Matice se rovnají právě tehdy, když se rovnají ve všech prvcích, z čehož dostaneme následující soustavu rovnic.
\begin{gather*}
    2a + 3b = 2a + c\\
    a = 2b+d\\
    2c+3d =3a\\
    c =3b
\end{gather*}\begin{gather*}
    a = 2b + d\\
    c = 3b\\
    b = k\\
    d = l
\end{gather*}

$$P = \left\{\begin{pmatrix}
    2k + l & k\\
    3k & l
\end{pmatrix};k,l \in \mathbb Z_5 \right\}.$$
\end{document}
