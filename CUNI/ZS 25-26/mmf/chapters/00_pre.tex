2025/09/29

borec se představil -- nějakej Krkoušek neboco (mimo něj ještě Švarc)
stejně jak derivace na jarošce, fyzika s tím pracuje ale v matice se to musí teprv udělat zdlouhavě a formálně
taylorův rozvoj -- TODO

popis bodu v prostoru, popis pohybu

prostor bodů \mathbb{E}^3 (tedy horní index), body A, B, počátek P
pracujeme s tzv. afinními prostory, tj. že každé dva body určují vektor
vektory \overrightarrow{r} = A - P -- součet bodů nedává smysl, rozdíl ano, to je vektor
ke každému bodu můžu přičíct vektor, tj. vektor nemusí vést jen z počátku -- to je ten vtip tady, jsou to prostě volné vektory, je to rovnoběžné, protipříklad třeba povrch koule
do euklidovského prostoru nám chybí vzdálenost nebo úhel, my si dáme úhel pomocí skalárního součinu \(\overrightarrow{a}, \overrightarrow{b}\) = \overrightarrow{a} \cdot \overrightarrow{b}, který se definuje pomocí vlastností -- positivně definitní symetrická bilineární forma (nevim)
vektorový prostor samozřejmě není jen rovina, ale obecně, zase def. pomocí vlastností
a potom \cos \gamma |\overrightarrow{a}||\overrightarrow{b}|=\overrightarrow{a} \cdot \overrightarrow{b}, kde \gamma je úhel mezi vektory, pomocí tohoto velikost
|\overrightarrow{a}|= \sqrt{\overrightarrow{a}\cdot\overrightarrow{a}}
\cos \gamma = \frac{\overrightarrow{a}\cdot\overrightarrow{b}}{|\overrightarrow{a}||\overrightarrow{b}|}
a vzdálenost d(A, B) = |B-A|
teď chceme souřadnice, aby se s tím dalo nějak počítat
nejběžnější jsou lineární souřadnice -- jejich souřadné osy jsou přímky
zvolíme si bázi ve vektorovém prostoru \overrightarrow{e_1}, \overrightarrow{e_2}, \overrightarrow{e_3}, ..., \overrightarrow{e_j} a k tomu počátek P, pak už je každý jiný vektor lineární kombinace bázových vektorů A = P + \overrightarrow{r} = P + x^1\overrightarrow{e_1} + x^2\overrightarrow{e_2} + x^3\overrightarrow{e_3} = P + \Sigma_{j}{x^j\overrightarrow{e_j}} = P + x\overrightarrow{e_x}+y\overrightarrow{e_y}+z\overrightarrow{e_z} -- horní index správně, koeficienty (tady tzv. komponenty) nedává smysl mocnit a takhle se s tím počítá přehledněji (údajně)
no a nyní si zaveďme kartézské souřadnice, tedy mějme ortonormální bázi \overrightarrow{e_j} \cdot \overrightarrow{e_i} = \begin{cases}{1, i=j \\ 0, i \neq j} toto je výhodné: \overrightarrow{a}\cdot \overrightarrow{b} = (a^1\overrightarrow{e_1}+a^2\overrightarrow{e_2}+a^3\overrightarrow{e_3})\cdot(b^1\overrightarrow{e_1}+b^2\overrightarrow{e_2}+b^3\overrightarrow{e_3}) = a^1b^1 + a^2b^2 + a^3b^3, a potom d(A, B) = |\Delta \overrightarrow{r}| = \sqrt{\Delta x^2 + \Delta y^2 + \Delta z^2}
Parametrická křivka: A(\chi), tj. A: \mathbb{R} \rightarrow \mathbb{E}^3 OBRAZEKOBRAZEK, tj. funkce, která má jako hodnoty body na nějaké dané křivce
parametry třeba obecný, čas, délka
zmatematizuju to pomocí souřadnic (klasika): x(A(\chi)), y(A(\chi)), z(A(\chi)) (retardi píšou x(\chi), y(\chi), z(\chi)
Př.: přímka -- A(\chi) = P + \chi \overrightarrow{a}; x(A(\chi)) = x_0 + \chi a^1, y(A(\chi)) = y_0 + \chi a^2, z(A(\chi)) = z_0 + \chi a^3, ... x^j(A(\chi)) = x_0^j + \chi a^j
Př.: kružnice ve 2D -- A(\varphi); x(A(\varphi) = x_0 + r_0 \cos \varphi, y(A(\varphi) = y_0 + r_0 \cos \varphi
Tečný vektor: vekotr který je tečný já nevim -- A(\chi), x^j(A(\chi)), teď jak na to: mno tak určitě můžu udělat sečnu A(\chi_0) + \Delta\overrightarrow{A} a s tou se přibližovat, tj. zmenšovat vzdálenost dvou bodů sečny
\overrightarrow{t} = \lim_{\Delta \chi \rightarrow 0}{\frac{\Delta\overrightarrow{A}}{\Delta \chi}} = \lim_{\Delta \chi \rightarrow 0}{\frac{A(\chi)-A(\chi_0)}{\chi - \chi_0}} = \frac{\overrightarrow{dA}{d\chi}(\chi_0)
čili jinak: A(\chi) = A(\chi_0) + \Delta \chi \overrightarrow{t} + ... -- ty tři tečky je něco co je v té limitě, což ale s menším \Delta \chi mizí
A(\chi) = P + \overrightarrow{r}(\chi) = P + x(\chi)\overrightarrow{e_x} + y(\chi)\overrightarrow{e_y} + z(\chi)\overrightarrow{e_z}
statická báze = P, \overrightarrow{e_j} konstantní
\overrightarrow{t}= \frac{\overrightarrow{dA}}{d\chi} = \frac{d}{d\chi} (P+x(\chi)\overrightarrow{e_x}+y(\chi)\overrightarrow{e_y}+z(\chi)\overrightarrow{e_z}) = ...
t^x = \frac{dx}{d\chi}, t^y = \frac{dy}{d\chi}, t^z = \frac{dz}{d\chi} -- \overrightarrow{t} = \frac{d\overrightarrow{r}}{d\chi}
když je parametr čas: \chi \rightarrow t, \overrightarrow{t} \rightarrow \overrightarrow{v}
\overrightarrow{v}=\frac{\overrightarrow{dA}}{dt}, v^j = \frac{dx^j}{dt}
když dt tak notace \overdot{x}
Př.: délka křivky -- s = \Sigma_{k=1}^{N}{\Delta s_k} \rightarrow \int_{\chi_Z}^{\chi_K}{ds} = \int_{\chi_Z}^{\chi_K}{|\overrightarrow{t}|d\chi}
\Delta s = |\Delta \overrightarrow{A}| = \Delta \chi |\overrightarrow{t}|
rychlost \Delta s =  \int_{t_Z}^{t_K}{|\overrightarrow{v}|dt}
parametrizace pomocí vzdálenosti \Delta s = \int{ds} = \int{|\overrightarrow{t}|ds}, kde \overrightarrow{t} = 1 (t je vzdálenost a chci to rovnoměrně)
Př.: \overrightarrow{t}
t^x = \frac{dx}{d\varphi} = -r_0\sin \varphi
t^y = \frac{dy}{d\varphi} = r_0\cos \varphi
\overrightarrow{t} = -r_0\sin \varphi \overrightarrow{e_x} + r_0\cos \varphi \overrightarrow{e_y}
|\overrightarrow{t}| = ... = r_0 -- zajímavé
no a potom o = \int_0^{2\pi}{|\overrightarrow{t}|d\varphi} = r_0 \int_0^{2\pi}{d\varphi} = 2\pi r_0 -- zajímavé
Křivočaré souřadnice -- souřadnicové osy nejsou přímky OBRAZEKOBRAZEK, takže mám \mathbb{E_3}, \chi^j(A) (ta parametrizace teda asi?); x = x(\chi^1, \chi^2, \chi^3) a teď je problém najít tečný vektor a zjištíme že tečný vektor ke stejnému vektoru se liší bod od bodu a ani úhel nemusí být pravý a taky se mění bod od bodu
Cylindrické souřadnice 3D OBRAZEKOBRAZEK \chi^1, \chi^2 \rightarrow \rho, \varphi: x = \rho \cos \varphi, y = \rho \sin \varphi, z=z; polární souřadnice (2D) OBRAZEKOBRAZEK x = \rho \cos \varphi, y = \rho \sin \varphi
a hledáme tečnu \overrightarrow{t}_\rho je tam zase nějaká derivace a skončím logicky tam kde čekám t_\rho^x = cos \varphi t_\rho^y = sin \varphi, povšimněme si ještě že když do toho budeš rýpat dost dlouho tak zjistíš že i polární i cylindrické souřadnice jsou ortogonální
Sférické souřadnice OBRAZEKOBRAZEK x = r \cos \vartheta \cos \varphi, x = r \cos \vartheta \sin \varphi, z = r \sin \varphi
\rho = r \cos \vartheta
\overrightarrow{t_r} = \cos \vartheta \cos \varphi \overrightarrow{e_x} + \cos \vartheta \sin -\varphi \overrightarrow{e_y} + \sin \varphi \overrightarrow{e_z}
\overrightarrow{t_\vartheta} = -r \cos \vartheta \cos \varphi \overrightarrow{e_x} -r \cos \vartheta \sin \varphi \overrightarrow{e_y}
\overrightarrow{t_\varphi}

2025/10/06
Funkce více proměnných

derivování složené funkce

Derivace podél křivky

Gradient a jeho vlastnosti

\nable (f + g) = \nabla f + \nabla g
\nagla(fg) = (\nabla f)g+f(\nabla g)
\nabla(H(f)) = H^\prime(f) \nabla f
\nabla (H(f_1, f_2, ...) = \frac{\parc H}{\parc f_1} \nabla f_1 + \frac{\parc H}{\parc f_2} \nabla f_2 + ...
(\nabla f)^k = \nabla_k f = \frac{\parc f}{\parc x^k} -- kart. souřad.
\nabla bez_x f = \frac{\parc f}{\parc x}, \nabla bez_y f = \frac{\parc f}{\parc y}
v kart. souř \nabla x^k = \overrightarrow{e_k}
křivočaré souř.
\chi_1, \chi_2, \chi_3
tečné t_1, t_2, t_3
gradienty \nabla \chi^1, \nabla \chi^2, \nabla \chi^3
t_k\cdot \nabla \chi^k = \frac{d\chi^l (A_l)}{d\chi^k}=\frac{d\chi^l}{d\chi^k} = \delta^l_k
a z tohoto duální báze \overrightarrow{t_k} \cdot \nabla \chi^l (pak vidíme že pokud \overrightarrow{t_k} = \nabla \chi^k jsme v KSS, pokud \propto ortogonální soustavě)

Cylindrické souřadnice

x = \rho \cos \varphi; \rho = \sqrt{x^2 + y^2}
y = \rho \sin \varphi; \varphi = arctan y/x
z = z; z = z
\overrightarrow{t_\rho} = \cos \varphi \overrightarrow{e_x} + \sin \varphi \overrightarrow{e_y} = \overrightarrow{e_\rho}
\overrightarrow{t_\varphi} = - \rho \sin \varphi \overrightarrow{e_x} + \rho \cos \varphi \overrightarrow{e_y} = \rho \overrightarrow{e_\varphi}
\overrightarrow{t_z} = e_z

\overrightarrow{\nabla}_\rho = \frac{\parc \rho}{\parc x} \overrightarrow{e_x} + \frac{\parc \rho}{\parc y} \overrightarrow{e_y} = 1/\rho (x\overrightarrow{e_x} + y\overrightarrow{e_y}) = \cos \varphi \overrightarrow{e_x} + \sin \varphi \overrightarrow{e_y} = \overrightarrow{e_\rho}
\overrightarrow{\nabla} \varphi = \frac{1}{1+(y/x)^2}(-y/x^2)\overrightarrow{e_x}+ \frac{1}{1+y^2/x^2}\cdot1/x\overrightarrow{e_y} = \frac{1}{\rho^2}(-y\overrightarrow{e_x}+x\overrightarrow{e_y})= \frac{1}{\rho^2}\overrightarrow{t_\varphi} = \frac{1}{\rho}\overrightarrow{e_\varphi}

Sférické souřadnice
x = r \sin \theta \cos \varphi; r = \sqrt{x^2 + y^2 + z^2}
y = r \sin \theta \sin \varphi; \theta = \arctan\frac{\sqrt{x^2+y^2}}{z}
z = r \cos \theta; \rho = \sin \theta; \varphi = \arctan \frac{y}{x}
\overrightarrow{t_r} = \sin \theta \cos \varphi \overrightarrow{e_x} + \sin \theta \sin \varphi\overrightarrow{e_y} + \cos \theta \overrightarrow{e_z} = \overrightarrow{e_z}
\overrightarrow{t_\theta} = r\cos \theta \cos \varphi \overrightarrow{e_x} + r\cos \theta\sin\varphi\overrightarrow{e_y} - r \sin \theta \overrightarrow{e_z} = r\overrightarrow{e_\theta}
\overrightarrow{t_\varphi} = -r \sin \theta \sin \varphi \overrightarrow{e_x} + r \sin \theta \cos \varphi \overrightarrow{e_y} = \rho \overrightarrow{e_\varphi}

\overrightarrow{\nabla}_r = \frac{1}{r}(x\overrightarrow{e_x}+y\overrightarrow{e_y} + z\overrightarrow{e_z}) = \frac{\overrightarrow{r}}{r} = \overrightarrow{t_r} = \overrightarrow{e_r}
\overrightarrow{\nabla}_\theta = \frac{1}{r^2}(\frac{xz}{\rho}\overrightarrow{e_x}+\frac{yz}{\rho}\overrightarrow{e_y} - \rho \overrightarrow{e_z}) = \frac{1}{r^2}\overrightarrow{t_\theta}= \frac{1}{r}\overrightarrow{e_\theta}

2025/10/13
\section{Diferenciální rovnice}
= obsahují derivace fce -- nehledám proměnnou, která splňuji rovnici, ale funkci (předpis rovnici funkce), která splňuje dif. rci
\begin{itemize}
\item parciální -- více proměnných, které na sobě můžou nějak záviset, musím na to dávat pozor ($\parc$)
\item obyčejné -- jedné proměnné
\end{itemize}
a ne bo
\begin{itemize}
\item nelineární -- mají proměnnou (nebo proměnnou v derivaci) uvnitř nějaké fce
\item lineární -- pouze proměnná a její derivace
\end{itemize}
a nakonec
\begin{itemize}
\item homogenní -- abs. člen = 0
\end{itemize}
řád rovnice = nejvyšší derivace
\begin{itemize}
\item separovatelná -- jdou separovat proměnné???
\item neseparovatelná -- nejde separovat TODO, pak to třeba zintegruju a jde to???
\end{itemize}
záleží tedy na tzv. počátečních podmínkách, někdy třeba neexistuje jednoznačné řešení pro dané poč. podmínky
budeme se zabývat především lineárními dif. rcemi, homogenními i nehomogenními, ale s konst. koeficienty:
$$y^{(n)}+a_{n-1}y^{(n-1)}}+...+a_0y=f(x)$$
