\documentclass{article}

% Language setting
% Replace `english' with e.g. `spanish' to change the document language
\usepackage[czech]{babel}
\usepackage{amsthm,thmtools,xcolor,amsmath,amssymb, mathtools}

% Set page size and margins
% Replace `letterpaper' with `a4paper' for UK/EU standard size
\usepackage[a4paper,top=2cm,bottom=2cm,left=3cm,right=3cm,marginparwidth=1.75cm]{geometry}

% Useful packages
\usepackage{amsmath}
\usepackage{enumerate}
\usepackage{graphicx}
\usepackage[colorlinks=true, allcolors=blue]{hyperref}
\graphicspath{images/}

%\usepackage{tikzit}
%\input{default.tikzstyles}

\title{DÚ Diskrétní matematika -- Sada 4}
\author{Jan Romanovský}

\begin{document}
\maketitle

\section*{Příklad 1. Kombinační zmrzlina}
$|\Omega| = {11 \choose 3}$ -- kombinace s opakováním, osm kopečků, čtyři příchutě

\noindent$|A| = 4 \cdot 1 \cdot 1 \cdot 1 \cdot {7 \choose 2}$ -- prvním činitelem vybírám, kterou příchuť nebudu mít, jedničky zajišťují, že máme \textit{právě} tři příchutě, dále kombinace s opakováním, pět kopečků, tři příchutě

\noindent$P = \frac{|A|}{|\Omega|} = \frac{4\cdot {7 \choose 2}}{{11 \choose 3}} = \frac{28}{55}\doteq 50,9 \,\%$

\section*{Příklad 2. Nezávislost -- karty a \uv{změna experimentu}}
\begin{enumerate}[(a)]
    \item $P(A) = \frac{1}{2}$ -- půlka karet je červená, půlka černá

    $P(B) = \frac{12}{52}$ -- v každém znaku jsou obrázkové tři karty, celkem čtyři znaky

    $P(A \cap B) = \frac{6}{52}$ -- v každém znaku jsou obrázkové tři karty, celkem dva červené znaky a vidíme, že $P(A \cap B) = P(A) \cdot P(B)$ -- což je definice nezávislých jevů.
    \item $P(A) = \frac{1}{2}$ -- půlka karet je červená, půlka černá, zajímá nás první karta, pravděpodobnost není ničím ovlivněna

    $P(B) = \frac{12}{52}\cdot\frac{11}{51}+\frac{40}{52}\cdot\frac{12}{51} = \frac{51}{221}$ -- první sčítanec uvažuje, že jako první byla vytažena obrázková karta, druhý, že nebyla

    $P(A \cap B) = \frac{6}{52}\cdot\frac{11}{51}+\frac{20}{52}\cdot\frac{12}{51} = \frac{51}{442}$ -- první sčítanec uvažuje, že jako první byla vytažena červená obrázková karta, druhý, že byla jako první vytažena červená neobrázková karta a vidíme, že $P(A \cap B) = P(A) \cdot P(B)$ -- což je definice nezávislých jevů.

\end{enumerate}
\end{document}
