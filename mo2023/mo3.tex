\documentclass{article}
\usepackage{fullpage}
\usepackage[czech]{babel}
\usepackage{amsfonts}

\title{\vspace{-2cm}\vspace{-1.7cm}}
\date{}
\author{}

\begin{document}
\maketitle
\noindent \textbf{Jan Romanovský}

\noindent \textbf{Gymnázium Brno, tř. Kpt. Jaroše}

\noindent \textbf{3.A}

\noindent \textbf{A-\textrm{I}-3}

\textbf{ }

Trojúhelníky TCK a TBL jsou zjevně podobné, jsou oba pravoúhlé a rovnoramenné (takže např. dle věty uuu). Protože mají společný bod T tak budou i spirálně podobné, a to se středem T. Bod D je středem BC a E středem KL. Toto vše tedy dle N2 znamená, že i trojúhelník TDE je (spirálně) podobný s trojúhelníky TCK a TBL, tedy také bude pravoúhlý a rovnoramenný. TD spolu s AT tvoří těžnici trojúhelníku ABC, AT tedy bude dvakrát delší než TD a TD můžeme vyjářit z trojúhelníku TDE:

  \[|TD|=\sqrt{2*|DE|^2}\]

  \[|AT|=2*|TD|\]

  \[|AT|=2*\sqrt{2}*|DE|\]

  \[\frac{|AT|}{|DE|}=2\sqrt{2}\doteq2,82\]



\end{document}
