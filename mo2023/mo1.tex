\documentclass{article}
\usepackage{fullpage}
\usepackage[czech]{babel}
\usepackage{amsfonts}

\title{\vspace{-2cm}\vspace{-1.7cm}}
\date{}
\author{}

\begin{document}
\maketitle
\noindent \textbf{Jan Romanovský}

\noindent \textbf{Gymnázium Brno, tř. Kpt. Jaroše}

\noindent \textbf{3.A}

\noindent \textbf{A-\textrm{I}-1}

\textbf{ }

Bod a) dokážeme protipříkladem, tedy najdeme situaci, ve které žádný pár nelze vytvořit. Jedna dívka se nemůže líbit 6 chlapcům, protože potom by jakýkoliv její výběr 5 chlapců zahrnoval alespoň jednoho chlapce, kterému se dívka líbí a pár by tedy existoval. Celkový počet \uv{líbení se} ze strany chlapců je počet chlapců (10) krát počet dívek, které se líbí jednomu chlapci (5), tedy 50. Dívek je 10, každá se může líbit maximálně 5 chlapcům a celkových \uv{líbení se} je 50, každá dívka se tedy bude muset líbit právě 5 chlapcům (50/10 = 5) a pokud se každé bude líbit 5 zbylých chlapců, kterým se ona nelíbí, nebude možnost vytvořit žádný pár. Toto tedy také bude jediná možnost, kdy vytvořit žádný pár nepůjde (máme určené přesné počty, jiné podmínky nesplňují).

Bod b) odvodíme od bodu a), pokud si vezmeme jediný případ z bodu a), ve kterém žádný pár vytvořit nelze, a navýšíme počet \uv{líbení se}, nebudeme mít kam tyto další umístit, aniž bychom např. chlapci nedali líbit dívku, které se on líbí -- všem dívkám, které se mu v bodě a) nelíbí, se on líbí. V bodě b) tedy pár vždy vytvořit půjde.

\end{document}
