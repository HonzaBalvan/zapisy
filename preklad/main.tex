\documentclass{article}
\usepackage{fullpage}
\usepackage[czech]{babel}
\usepackage{amsfonts}

\title{\vspace{-2cm}\vspace{-1.7cm}}
\date{}
\author{}

\begin{document}
\maketitle
\textbf{Střemhlavý vývoj AI je rizikem pro lidstvo, tvrdí expert}

V dokumentu o zásadách vývoje AI vydaném tento týden vyzývá 23 expertů v oblasti umělé inteligence k vládní regulaci vývoje vysoce výkonných AI, v případě potřeby dokonce k jeho zastavení.

\uv{V případě budoucího vývoje ještě výkonějších umělých inteligencí, např. takových, které by se dokázaly vymykat lidské kontrole, musí být vlády připraveny regulovat tento vývoj a v reakci na jejich potenciálně znepokojující schopnosti i regulovat kdo k nim má přístup. Zároveň by měly vyžadovat od vývojářů AI takový stupeň zabezpečení citlivých informací, které by bylo na roveň zabezpečení státních informací, alespoň dokud nebudou připraveny příslušné ochrany před jejich zneužitím.}

Neřízený a neregulovaný vývoj obecné umělé intelignce, tedy takové, která dokáže splnit širokou škálu úkonů na úrovni lidské inteligence nebo vyšší, je největší obavou těch, kteří žádají přísnější regulaci.

Jsou firmy, které do 18 měsíců planují vyvinout modely se 100x větším výkonem, než mají dnes ty nejvyspělejší. Amazon se také nechal minulý měsíc slyšet, že investuje 4 miliardy dolarů (asi 93 miliard korun) do start-upu Anthropic, založeného bývalým vedením firmy OpenAI, který plánuje sbírat svá data na cloudových službách Amazonu. Kvůli tomu ale některá hudební vydavatelství žalují Anthropic za údajné využívaní textů písní chráněných autorskými právy. Tato investice je asi největším krokem společnosti Amazon v oblasti vývoje AI; stále ale mají co dělat, aby dohnali Microsoft a Alphabet, kteří jsou na tomto poli největšími hráči.


\textbf{Mona Awad -- Rouge}

Na první pohled vypadá Rouge od Mony Awad malinko strašidelně. Krásou posedlá matka Belle zemřela a své dceři zanechala sešlý dům v jižní Kalifornii, obří dluhy a řadu červených lahviček s rozličnými krémy a séry na pleť. Brzy je Belle přitahována dekadentním a bizarním salónem krásy Rouge, kde se nechávala zkrášlovat její matka -- a zjistí že cena za krásu je vyšší, než by si kdy mohla představit.

Tato propozice by mohla lehce spadnout k povrchnímu pokusu kritiky \uv{obchodu s krásou} a nezdravých standardů, které vytváří. Awad ale tvoří něco temně komického, z čeho běží mráz po zádech. Čím déle čtete Rouge, tím je kniha bizarnější; vyskytují se zde červené pulzující medúzy a zrcadla, která se nechovají, jak mají, vyprávění Mony Awad se plazí jako had a je stejně tak kluzké. Nakonec vyvstane ze znepokojivé atomsféry příběh až pohádkový, jehož hrdinové a záporáci jsou zřejmí.

Nejlepší horrory v sobě mají něco osobního, s čím soucítíme; z Rouge se vyklube kniha s tradičním motivem dcery, která se vypořádává s odkazem své matky. Belle, která je napůl Egypťanka, vyrůstala se svou bledou matkou, která jí odmalička vštěpovala nezdravé ideály krásy, které si především cenily bledosti a bělosti kůže. V zápolení s těmito otázkami Mona Awad dovede příběh do mocného zakončení, které se hluboce dotýká vztahu matky a dcery, v němž přes všechny možné problémy nachází srdceryvnou něhu.     
\end{document}
