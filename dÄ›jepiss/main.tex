\documentclass{article}
\usepackage{fullpage}
\usepackage[czech]{babel}
\usepackage{amsfonts}

\title{\vspace{-2cm}Dějepis 24. 2. 2023\vspace{-1.7cm}}
\date{}
\author{}

\begin{document}
\maketitle

\part{Reformace 2}

\section{Švýcarská reformace (Měšťanská reformace)}
\begin{itemize}
  \item v této době probíhal švýcarský boj o nezávislost na SŘŘ v 20. letech 16. st., válka mezi jednotlivými švýcarskými kantony (katolické x švýcarské)
  \item \textbf{Ulrich Zwingli} (1484-1531) - vůdce švýcarských kantonů, radikálnější než Luther, důraz na morálku, poddanstvo má právo se vzbouřit, když se šlechta nechová správně, odmítají katolické tradice (přijímání, ikony), pak padl v boji
  \item \textbf{Jan Kalvín} (1509-1564) - teolog francouzského původu, útočiště v Ženevě, vlivy Luthera, Zwingliho, učení předdestinace - je už o člověku od začátku určeno, zda bude spasen nebo zatracen, ale člověk to neví, takže by se měl chovat správně, v průběhu života se dovídá zda je ten bohem vybraný
  \item \textbf{kalvinismus} - důraz na pracovitost, skromnost, individualitu, podnikavost - jakýsi prvopočátek (dejme tomu) kapitalistického, podnikavého mindsetu, pevné morální zásady - prostě se v životě nebav, nebo nebudeš spasen
  \item kalvinismus rozšířen v Anglii (významné pro osidlování S Ameriky), Uhersku, ...
\end{itemize}

\section{Protireformace}
\begin{itemize}
  \item Katolíci chtějí zpátky získat ztracené pozice, zabránit šíření nekatolických církví, snaží se o přestavbu katolické církve
  \item se souhlasem papeže byl r. 1540 založen nový řád - \textbf{Jezuité} (Societas Jesu), zakladatel Ignác z Loyoly, přísný řád, odpovídají jen papežovi, účely rekatolizace, misionářství, zřizovali a vedli většinu škol
  \item r. 1542 založení \textbf{Svaté inkvizice} - no, inkvizice, dále zakazují některé knihy, píší jejich seznamy (Index librum prohibitorum)
  \item \textbf{Tridentský koncil} (1545-1563) - přelom ve fungovaní katolické církve, odmítnuta radikální reformace, jediná autorita je papež, ale musí se omezit hromadění církevních úřadů v rukou jednotlivce a že duchovní musí být vzdělaní - biskupské semináře pro výchovu nových kněží
\end{itemize}

\part{Francie a Habsburkové a jejich vztahy}
\section{Francie}
\begin{itemize}
  \item Francie silná, posílená po vítězství ve stoleté válce, centralizovaná, král vládne poměrně absolutisticky, rozbití generálních stavů
  \item \textbf{Ludvík XI.} (1461-1483) - začíná byrokratizaci - nová ,,úřednická šlechta"
  \item rozhodující část daní šla z měst, protože tam podnikali, což bylo pod úroveň šlechticů
  \item po stabilizaci po stoleté válce francouzští králové se pokouší územně expandovat
\end{itemize}
\end{document}
