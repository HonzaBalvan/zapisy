\documentclass{article}
\usepackage{fullpage}
\usepackage[czech]{babel}
\usepackage{amsfonts}

\title{\vspace{-2cm}Dějepis 24. 2. 2023\vspace{-1.7cm}}
\date{}
\author{}

\begin{document}
\maketitle

\part{Reformace 2}

\section{Švýcarská reformace (Měšťanská reformace)}
\begin{itemize}
  \item v této době probíhal švýcarský boj o nezávislost na SŘŘ v 20. letech 16. st., válka mezi jednotlivými švýcarskými kantony (katolické x švýcarské)
  \item \textbf{Ulrich Zwingli} (1484-1531) - vůdce švýcarských kantonů, radikálnější než Luther, důraz na morálku, poddanstvo má právo se vzbouřit, když se šlechta nechová správně, odmítají katolické tradice (přijímání, ikony), pak padl v boji
  \item \textbf{Jan Kalvín} (1509-1564) - teolog francouzského původu, útočiště v Ženevě, vlivy Luthera, Zwingliho, učení předdestinace - je už o člověku od začátku určeno, zda bude spasen nebo zatracen, ale člověk to neví, takže by se měl chovat správně, v průběhu života se dovídá zda je ten bohem vybraný
  \item \textbf{kalvinismus} - důraz na pracovitost, skromnost, individualitu, podnikavost - jakýsi prvopočátek (dejme tomu) kapitalistického, podnikavého mindsetu, pevné morální zásady - prostě se v životě nebav, nebo nebudeš spasen
  \item kalvinismus rozšířen v Anglii (významné pro osidlování S Ameriky), Uhersku, ...
\end{itemize}

\section{Protireformace}
\begin{itemize}
  \item Katolíci chtějí zpátky získat ztracené pozice, zabránit šíření nekatolických církví, snaží se o přestavbu katolické církve
  \item se souhlasem papeže byl r. 1540 založen nový řád - \textbf{Jezuité} (Societas Jesu), zakladatel Ignác z Loyoly, přísný řád, odpovídají jen papežovi, účely rekatolizace, misionářství, zřizovali a vedli většinu škol
  \item r. 1542 založení \textbf{Svaté inkvizice} - no, inkvizice, dále zakazují některé knihy, píší jejich seznamy (Index librum prohibitorum)
  \item \textbf{Tridentský koncil} (1545-1563) - přelom ve fungovaní katolické církve, odmítnuta radikální reformace, jediná autorita je papež, ale musí se omezit hromadění církevních úřadů v rukou jednotlivce a že duchovní musí být vzdělaní - biskupské semináře pro výchovu nových kněží
\end{itemize}

\part{Francie a Habsburkové, jejich vztahy}
\section{Francie}
\begin{itemize}
  \item Francie silná, posílená po vítězství ve stoleté válce, centralizovaná, král vládne poměrně absolutisticky, rozbití generálních stavů
  \item \textbf{Ludvík XI.} (1461-1483) - začíná byrokratizaci - nová ,,úřednická šlechta"
  \item rozhodující část daní šla z měst, protože tam podnikali, což bylo pod úroveň šlechticů
  \item po stabilizaci po stoleté válce francouzští králové se pokouší územně expandovat, ale vždy se jim do toho vrtají Habsburkové $\rightarrow$ války mezi Francií a Habsburky
  \item král \textbf{Karel VIII.} (1483 - 1498) - rozhodl se získat Neapolské království, to mu ale vyfoukl Ferdinand Aragonský, v čem mu pomohla Benátská liga - papež, císař SŘŘ Maxmilián Habsburský, Benátky, Milán
  \item král \textbf{Ludvík XII.} (1498 - 1515) - dočasný zisk Milánského vévodství, pak se zase musel stáhnout
  \item král \textbf{František I.} (1515 - 1547) - vzdělaný král (tzv. kníže renesance, na jeho dvoře přední renesantici - Da Vinci, Michalangelo, Tizian, Rabelais), za jeho vlády vrcholí války mezi Francií a Habsburky, za Habsburky \textbf{Karel V.}, rozhodujíci \textsc{bitva u Pavie} - 1525
  \item potom jsou zde taky náboženské války (1562) - mezi katolíky a hugenoty (kalvinisty)
  \begin{itemize}
    \item začínají masakrem hugenotů, pravděpodobně z vůle vůdce katolíků Jindřicha de Guise, ten zavražděn r. 1588
    \item vůdce hugenotů je navarrský princ Jindřich de Bourbon
    \item král Jindřich III. se to snaží zmírnit tím, že provdá svoji sestru Markétu za Jindřicha Bourbonského, ale
    \item 23./24. srpen 1572 - Bartolomějská noc - katolíci velice organizovaně a systematicky zmasakrovali hugenoty, kteří se sjeli při příležitosti výše svatby
    \item Jindřich III. r. 1589 zavražděn radikálním mnichem - není další Valois, takže nastupuje Jindřich Bourbonský
    \item Jindřich Bourbonský 1593 konvertuje na katolicismus, 1594 korunován francouzským králem jako
  \end{itemize}
  \item král \textbf{Jindřich IV. de Bourbon} - 1598 vydává edikt nantský - zrovnoprávnil hugenoty a katolíky, dočasně snížil daně, aby se země zrekonstruovala, vždy ale v konfliktech nadržoval hugenotům, takže taky zavražděn taky mnichem
  \item Jindřich IV. měl po Markétě ještě Marii Medicejskou, s ní měl syna Ludvíka (XIII.), ten ale v době Jindřichově smrti neplnoletý, takže Marie zatím vládne jako regentka
  \item kardinál Richelieu - šedá eminence, 1. ministr, diehard podporovatel Marie a Ludvíka, zatáhl Francii do třicetileté války na straně protihabsburské (i když vlastně v době Marie poměrně close ties s Habsburky (manželka Ludvíka Anna Rakouská))
  \item pevno La Rochelle - hugenotská pevnost poměrně nezávislá na králi, účastnili se protistátního povstání, takže pevnost byla dobyta a asi 20 tis. hugenotů tam bylo zabito
  \item po Ludvíkovi XIII. nastupuje jeho syn Ludvík XIV. (stay tuned)

\section{Nizozemsko}
\begin{itemize}
  \item tak jak už víme máme spojené Španělsko Ferdinandem II. a Isabelou
  \item proti Ferdinadovi se bouří mocní šlechtici - \textit{grandové} , kteří získali moc při boji s Araby, Ferdinand se proti nim spojuje s městy do tzv. \textit{sv. hermanandy}
  \item Ferdinand a Isabela mají dceru Johanu, její manžel byl Filip Habsburský, tento držel Nizozemí (získal po Burgundii)
  \item Johana s Filipem mají syna Karla - tento jako španělský král Karel I., jako císař SŘŘ \textbf{Karel V.}, tento se opírá o vysokou šlechtu, což naopk způsobilo povstání měst - \textit{povstání komunerů} v čele s Toledem, které Karel tvrdě potlačil a vydobyl si na nich velké daně a ekonomicky tím vlastně zkriplil celé Španělsko (byť mělo obrovské koloniální impérium), do čela dění se dostvává Francie, Anglie
\end{itemize}

\end{itemize}
\end{document}
