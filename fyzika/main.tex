\documentclass{article}
\usepackage{fullpage}
\usepackage[czech]{babel}
\usepackage{amsfonts}

\title{\vspace{-2cm}Fyzika 24. 2. 2023\vspace{-1.7cm}}
\date{}
\author{}

\begin{document}
\maketitle

\section{Deformace}
\begin{itemize}
  \item typy:
  \begin{itemize}
    \item tahem/tlakem OBRAZEKOBRAZEK
    \item kroucením OBRAZEKOBRAZEK
    \item ohybem OBRAZEKOBRAZEK
    \item smykem OBRAZEKOBRAZEK
  \end{itemize}
\end{itemize}

\subsection{Deformace tahem/tlakem}
\begin{itemize}
\item Normálové nápětí: $sigma=F/S$ $[N/m2] = [Pa]$ OBRAZEKOBRAZEK
\item Změna délky: $delta l = l - l0$ $m$, užitečnější většinou relativní prodloužení: $epsilon = (delta l)/l0$ $bezrozměrné$
\end{itemize}
Deformační křivka
OBRAZEKOBRAZEK
\begin{itemize}
  \item lineární úsek (0 - A)
  \begin{itemize}
    \item pružná deformace
    \item vratná
    \item platí Hookův zákon: $sigma /rybička/ epsilon$ (slovy rel. prodloužení je přímo úměrné napětí) $sigma = E*epsilon$, E - Youndův modul pružnosti (ocel = 220 GPa, cín = 55 GPa)
  \end{itemize}
  \item 
\end{itemize}
\end{document}
