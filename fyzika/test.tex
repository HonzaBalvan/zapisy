\documentclass{article}
\usepackage{fullpage}
\usepackage[czech]{babel}
\usepackage{amsfonts}
\usepackage{parskip}

\title{\vspace{-2cm}Fyzika DÚ\vspace{-1.7cm}}
\date{}
\author{}

\begin{document}
\maketitle
\part*{Kapitola \textrm{I}}

\section*{Proxima Centauri}
\textbf{Zadání:} Najděte paralaxu Proximy Centauri, která je vzdálená asi 4.3 světelného roku.

\textbf{Řešení:} $p = \arctan \frac{1au}{4.3ly} = \arctan \frac{1,5 \cdot 10^{11}}{4.3\cdot 3 \cdot 10^{8} \cdot \pi \cdot 10^{7}} \doteq 3.7 \cdot 10^{-6}$ rad

\section*{Pogsonova rovnice}
\textbf{Zadání:} Odvoďte vztah mezi absolutní magnitudou a relativní magnitudou v parsecích (tzv. Pogsonovu rovnici).

\textbf{Řešení:}


\section*{Měrný výkon Rigelu}
\textbf{Zadání:} Hvězda Rigel ze souhvězdí Orionu je od Slunce vzdálena 240 pc a její relativní
magnituda je 0,18\textsuperscript{m}. Hmotnost Rigelu je 17 hmotností Slunce. Určete výkon hvězdy na jednotku
hmotnosti tak, že její parametry porovnáte se Sluncem.

\textbf{Řešení:}

\part*{Kapitola \textrm{I\hspace{-.1em}I}}

\section*{Teplota Slunce z vlnové délky světla}
\textbf{Zadání:} Určete povrchovou teplotu Slunce, víte-li, že maximum vyzařování je na vlnové délce
500 nm

\textbf{Řešení:} $\lambda_{max} = \frac{b}{T} \implies T = \frac{b}{\lambda_{max}} \doteq \frac{2.9 \cdot 10^{-3}}{5 \cdot 10^{-7}} = 5800$ K

\section*{Zářivý výkon Slunce}
\textbf{Zadání:} Nalezněte celkový zářivý výkon Slunce, znáte-li jeho povrchovou teplotu $T$ = 5800 K.

\textbf{Řešení:} $r = 6.96 \cdot 10^8 m \implies S = 4 \pi r^2 \doteq 8.7 \cdot 10^{17}$ m\textsuperscript{2}, $I = \sigma T^4 = 5.67 \cdot 10^{-8} \cdot 5800^4 \doteq 6.42 \cdot 10^7$ W/m\textusperscript{2} $ \implies P = I \cdot S = 6.42 \cdot 10^7 \cdot 8.7 \times 10^{17} \doteq 5.59 \cdot 10^{25}$ W

\section*{Měrný výkon Slunce}
\textbf{Zadání:} Jaký výkon se průměrně uvolňuje v jednom kilogramu sluneční hmoty?

\textbf{Řešení:} $p_m = \frac{P}{m_S} = \frac{}{} \doteq \cdot 10^{}$

\section*{Sluneční konstanta}
\textbf{Zadání:} Určete intenzitu slunečního záření v okolí Země.

\textbf{Řešení:}

\section*{Určení poloměru hvězdy}
\textbf{Zadání:} Hvězda s paralaxou $0,03^{\prime\prime}$ a vizuální magnitudou 3,9\textsuperscript{m} má maximum vyzařování na
vlnové délce 500 nm. Určete poloměr této hvězdy.

\textbf{Řešení:}

\part*{Kapitola \textrm{I\hspace{-.1em}I\hspace{-.1em}I}}

\section*{Úbytek sluneční hmoty}
\textbf{Zadání:} Kolik procent sluneční hmoty se přemění v energii za jedno tisíciletí?

\textbf{Řešení:} $E = mc^2  \implies m = \frac{E}{c^2}$, $E = W \cdot t = \cdot \pi \cdot 10^10 = $ J

\part*{Kapitola \textrm{I\hspace{-.1em}V}}

\section*{Kometární dráha}
\textbf{Zadání:} Kometa se v okamžiku opozice se Sluncem nacházela v odsluní. Její vzdálenost od
Země tehdy činila 5 au. Velká poloosa její dráhy je 4 au. Za jak dlouho projde kometa přísluním?
V jaké vzdálenosti bude v tom okamžiku od Slunce a od Země?

\textbf{Řešení:}

\section*{Pád Země do Slunce}
\textbf{Zadání:} Jak dlouho by padala Země do Slunce, kdybychom ji zastavili?

\textbf{Řešení:} F = G\frac{m_Zm_S}{r^2}$, $r = \frac{1}{2}at^2$, $a = \frac{F}{m_Z} \implies r = \frac{Gm_St^2}{2r^2} \implies t = \sqrt{\frac{2r^3}{Gm_S}} \doteq 7.11 \cdot 10^6$ s $\doteq 22.53$ y

\section*{Gravitační působení Slunce a Země na Měsíc}
\textbf{Zadání:}  Nalezněte poměr gravitačních sil, kterými působí na Měsíc Země a Slunce. Která síla je
větší?

\textbf{Řešení:}

\part*{Kapitola \textrm{V}}

\section*{Hvězda měnící rozměry}
\textbf{Zadání:} Spočtěte rotační periodu a magnetické pole našeho Slunce, pokud by změnilo rozměry
podle následující tabulky (stalo se obrem, bílým trpaslíkem nebo neutronovou hvězdou).
Předpokládejte, že se při hvězdném vývoji zachovává moment hybnosti a magnetický indukční
tok.

\textbf{Řešení:}

\section*{Rozměr neutronové hvězdy}
\textbf{Zadání:}  Stanovte horní hranici poloměru neutronové hvězdy o hmotnosti 1,7 $m_S$, která má
periodu rotace 2,1 ms. Použijte klasický výraz pro odstředivou sílu.

\textbf{Řešení:}

\part*{Kapitola \textrm{V\hspace{-.1em}I}}

\section*{Mion}
\textbf{Zadání:}  Doba života mionu (těžký elektron) je $\Delta \tau$ = 2.2×10-6 s. Mion vznikl ve výšce h = 30 km
nad povrchem Země z kosmického záření a dopadl na Zem. Jakou musel mít minimální rychlost
při vzniku?

\textbf{Řešení:}

\part*{Kapitola \textrm{V\hspace{-.1em}I\hspace{-.1em}I}}

\section*{Laplaceův výpočet Schwarzschildova poloměru}
\textbf{Zadání:}  Zjistěte, jak malý poloměr by musel mít objekt o hmotnosti $M$, aby úniková rychlost
dosáhla rychlosti světla.

\textbf{Řešení:}

\part*{Kapitola \textrm{V\hspace{-.1em}I\hspace{-.1em}I\hspace{-.1em}I}}

\section*{Maximální stáří vesmíru pro Fridmanovu expanzi}
\textbf{Zadání:}  Odhadněte z hodnoty Hubblovy konstanty maximální stáří vesmíru.

\textbf{Řešení:}
\end{document}
