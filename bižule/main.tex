\documentclass{article}
\usepackage{fullpage}
\usepackage[czech]{babel}
\usepackage{amsfonts}

\title{\vspace{-2cm}kmen Strunatci\vspace{-1.7cm}}
\date{}
\author{}

\begin{document}

\maketitle

aaaaa věci minulý úterý 14. 2.
\begin{itemize}
  \item obsahuje nadtřídy Bezčelistnatci a Čelistnatci
\end{itemize}

\section{nadtřída Bezčelistnatci}

\subsection{třída Sliznatky}
\begin{itemize}
  \item nejprimitivnější obratlovci, podobní rybám, až 1 m velcí
  \item mají zachovanou chordu, která je místy prostoupena základy obratlů
  \item obývají dna chladnějších mělkých moří v oblasti mírného pásu, jsou slepé
\end{itemize}

\subsection{třída Mihule}
\begin{itemize}
  \item až 1 m dlouhé hadovité tělo, v zadní polovině těla mají souvislý ploutevní lem
  \item zachovaná chorda, chrupavčitá, málo vyvinutá kostra, mají vyvinuté oči
  \item larva minoha, žije 2 až 5 let, připomíná kopinatce, dlouho si mysleli, že je samostatný druh
  \item mihule mořská, říční, potoční (u nás, teď málo častá)
\end{itemize}

\section{nadtřída Čelistnatci}

\subsection{třída Paryby}
\begin{itemize}
  \item starý druh, již od ordoviku, a to nezměněni (evolučně tedy poměrně dokonalí)
  \item končetiny ve tvaru ploutví - párové prsní, párové břišní, ne- i párová hřbetní, řitní, nesouměrná (u žraloků heterocerkní - nestejně rozdvojená) ocasní
  \item mají žaberní štěrbiny a neumí polykat, nedýchají ,,aktivně" a musí tedy (třeba i ve spánku) plavat, aby se neudusili
  \item třetí víčko - mžurka, chrání oči
  \item zahrnuje Chiméry (nějaký sea monsters z hlubin) a Příčnoústé (žraloci, rejnoci)
\end{itemize}

\subsubsection{podtřída Chiméry}
\begin{itemize}
  \item hlubokomořské, např. chiméra podivná
\end{itemize}

\subsubsection{podtřída Příčnoústí}
\begin{itemize}
  \item tvar těla torpédovitý, dorsoventrálně (zádo-břichově) zploštělý
  \item dlouhý rypec (rypák), ústa na spodní straně hlavy
  \item plakoidní šupiny - podobné zubům (tvořené dentinem (naše zuby) a kryta emailem), orientované - živočich hydrodynamičtější, ochrana před parasity
  \item chorda dorsalis zatlačena/potlačena obratly
  \item 5 žaberních štěrbin
  \item dominantní smysl čich a v rypci mají Lorenziho ampule - velké množství volných nervových zakončení, které vnímá elektrické impulsy - vnímá např. el. impulsy srdcí kořisti
  \item trvale se vyměňující zuby, zahnuté dovnitř
  \item lebka je široká, s pouzdry smyslových orgánů, dolní čelist uchycena na delším segmentu než např. naše - mnohem silnější stisk
  \item soustava rozmnožovací a močová u samců propojeny, vejcorodí i nepravě živorodí (ve vejci, ale vejce v matce)
  \item zřídka i sladkovodní žraloci, žraloci buď predátoři, filtrátoři, nebo plují při dnu a loví korýše ap.
  \item patří sem žraloci - žralok obrovský, žralok bílý, máčka skvrnitá; taky rejnoci - rejnok ostnatý, parejnok elektrický, manta obrovská
\end{itemize}

\subsection{třída Ryby}
\begin{itemize}
  \item mají kosti, mají žábra, gonochoristé (vnější rozmnožování - jikry, mlíčí)
\end{itemize}

\subsubsection{podtřída Svaloploutví}
\begin{itemize}
  \item ploutve jako \uv{nohy}, pohybují se po dně, předchůdci nás
  \item např. latimérie podivná
\end{itemize}

\subsubsection{podtřída Paprskoploutví}
\begin{itemize}
  \item ryby podle \uv{tradičního} chápání
  \item nejpočetnější skupina obratlovců (moře je větší než souš)v
  \item vodní obratlovci, prvotně vznikli ve sladkovodním prostředí
  \item převažují kosti nad chrupavkami (až na chrupavčité)
  \item ve škáře šupiny - derivát pokožky (rozdílné od paryb)
  \item žaberní přepážky redukovány, žaberní dutina kryta skřelemi - můžou aktivně dýchat, tj. se neutopí, když se zastaví
  \item tělo má hydrodynamciký tvar (proč asi)
  \item kůže obsahuje slizotvorné žlázy, pigmenty a šupiny, ty jsou:
  \begin{itemize}
    \item ganoidní - jeseter OBRAZEKOBRAZEK
    \item cykloidní - kapr OBRAZEKOBRAZEK
    \item ktenodiní - okoun OBRAZEKOBRAZEK
  \end{itemize}
  \item na těle ploutve - psní, břišní, hřbetní, čitní a ocasní
  \item ocasní ploutev buď:
  \begin{itemize}
    \item homocerkní - souměrně laločná
    \item heterocerkní - nesouměrně laločná
    \item difycerkní - není laločná
  \end{itemize}
  \item kosti jsou dobře osifikované, osou je páteř (chorda je potlačena obratli)
  \item kosti vznikají osifikací chrupavky
  \item ve svalovině převládá podélný boční sval
  \item mozog, mozog v lebce, 5 částí mozogu
  \item smyslový orgán postranní čára - rybou proudí voda, jsou zde nervová zakončení - vnímá proud, teplotu vody, chemoreceptory (čich)
  \item další smysly např. sluch a rovnováha - vnitřní oko se třemi otolity (balanc šutry), oči bez víček, čočka se neroztahuje a smršťuje, ale posouvá se (podle toho zaostření)
  \item trávící soustava trubicovitá, hltan, žaludek, střeva, játra, žlučník, slezina, slinivka, zvláštností jsou požerákové zuby - posouvá potravu hlouběji do úst, ústa buď:
  \begin{itemize}
    \item svrchní - ústa směrem nahoru, chytá věci na  OBRAZEKOBRAZEK
    \item koncová - ústa směrem dopředu, predátoři OBRAZEKOBRAZEK
    \item spodní - ústa směrem dolů, vyhrabává živočichy u dna OBRAZEKOBRAZEK
  \end{itemize}
\end{itemize}

\end{document}
