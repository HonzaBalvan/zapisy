\chapter*{Úvod}
\addcontentsline{toc}{chapter}{Úvod}

Chemie je fascinujícím oborem s širokým využitím v dnešní společnosti. Pro udržení a rozvoj současné úrovně chemie je pak nezbytně důležita dobrá a zajímavá výuka chemie. Jedním z nezbytných prvků výuky chemie jsou chemické pokusy. Tyto pokusy nejen, že obohacují výuku o praktické aspekty, ale také hrají klíčovou roli při osvojování teoretických konceptů. Tato práce je zaměřena na popsání důležitosti chemických experimentů v procesu výuky chemie a jejich značné výhody.

Chemické pokusy představují pro studenty mimořádně atraktivní způsob, jak si osvojit chemické pojmy a principy. Prostřednictvím přímé interakce s materiály a pozorováním jejich chování mají studenti možnost vnímat abstraktní teorie jako konkrétní jevy. Tímto způsobem se upevňuje jejich chápání chemických principů a zároveň se rozvíjí jejich zájem o daný obor.

Tato práce se dále zaměřuje na vytvoření katalogu chemických pokusů, který může sloužit jako nástroj pro pedagogy při plánování výuky. Tento katalog zahrnuje pokusy vhodné pro výuku chemie na druhém stupni základní školy a na střední škole. Cílem tohoto katalogu je poskytnout pedagogům nástroj pro efektivní a interaktivní výuku chemie, který podněcuje zvídavost, experimentování a objevování.
\newpage
