% soctepmlate unofficial - SOČ = Středoškolská odborná činnost - Czech competition
% Author: Vojtěch Boček
% Edit by: Jaroslav Páral
% Version: 2018-02-12
% Source code: https://github.com/RoboticsBrno/soctemplate/
% Base on: http://www.jcmm.cz/cz/sablona-soc.html
% License: CC BY 4.0

\documentclass{template/socthesis}

\usepackage{subcaption}
\usepackage{amsmath}
\usepackage{enumitem}

\addbibresource{text.bib}

\titlecz{Chemické pokusy}
\titleen{Chemical expermients}
\author{Jan Romanovský}
\field{12}
\school{Gymnázium Brno, třída kpt.~Jaroše}
\mentor{Mgr. Zdeněk Moravec, Ph.D.}
\mentorstatement{Mgr. Zdeňka Moravce, Ph.D.}

% Změňte, pokud se liší
%\region{Jihomoravský}
%\placefooter{Brno 2023}

\begin{document}

	\maketitle

	\makecopyrightstatement{V~Brně}

	\makethanks{Děkuji}

	\pagestyle{empty}

	\section*{Anotace}
	Cílem práce je pojednání o důležitosti chemických pokusů v moderní výuce chemie, dále lehce přístupný katalog efektních chemických pokusů, který by sloužil jako pomůcka při výuce chemie na ZŠ a SŠ. Práce je členěna do teoretické a praktické části, dále do podcelků podle logické návaznosti k tématu textu.
	Cílem teoretické části je podat shrnutou historii výuky chemie v ČR, vymezení chemických pokusů jako pomůcek při výuce a význam chemických pokusů ve výuce.
	Cílem praktické části je provedení několika pokusů a jejich následné zpracování do katalogu pro potřeby výuky na ZŠ a SŠ, dále ověření funkčnosti tohoto katalogu.

	\subsection*{Klíčová slova}
	chemický pokus, výuka chemie

	\vspace{20mm}

	\section*{Annotation}
	Insert annotation

	\subsection*{Keywords}
	chemical experiment, chemistry teaching

	\newpage
	\pagestyle{plain}

    \tableofcontents % vysází obsah

	%%% Začátek práce
	\setcounter{figure}{0}
	\setcounter{table}{0}
	\newpage

	%%% Vložení jednotlivých částí
	\chapter*{Úvod}
\addcontentsline{toc}{chapter}{Úvod}

Chemie je fascinujícím oborem s širokým využitím v dnešní společnosti. Pro udržení a rozvoj současné úrovně chemie je pak nezbytně důležitá dobrá a zajímavá výuka chemie. Jedním z neopomenutelných prvků výuky chemie jsou chemické pokusy. Tyto pokusy nejenže obohacují výuku o praktické aspekty, ale také hrají klíčovou roli při osvojování teoretických konceptů. Tato práce je zaměřena na popsání důležitosti chemických experimentů v procesu výuky chemie a jejich značné výhody.

Chemické pokusy představují pro studenty mimořádně atraktivní způsob, jak si osvojit chemické pojmy a principy. Prostřednictvím přímé interakce s materiály a pozorováním jejich chování mají studenti možnost vnímat abstraktní teorie jako konkrétní jevy. Tímto způsobem se upevňuje jejich chápání chemických principů a zároveň se rozvíjí jejich zájem o tento obor.

Tato práce se dále zaměřuje na vytvoření katalogu chemických pokusů, který může sloužit jako nástroj pro pedagogy při plánování výuky. Tento katalog zahrnuje pokusy vhodné pro výuku chemie na druhém stupni základní školy a na střední škole. Cílem tohoto katalogu je poskytnout pedagogům nástroj pro efektivní a interaktivní výuku chemie, který podněcuje zvídavost, experimentování a objevování.
\newpage


	\chapter{Teoretická část}

\section{Historie a vývoj výuky chemie}

Všeobecná býuka chemie na našem území začíná tzv. \uv{Hasnerovým zákonem} z roku 1869, který zavádí mimo povinné osmileté školní docházky na obecné nebo měšťanské škole na školách měšťanských výuku předmětu zvaného přírodozpyt. Přírodozpyt měl za úkol učit žáky přírodním zákonitostem, \uv{rozebírat} svět kolem nich. Z dnešních předmětů zahrnoval kromě chemie i fyziku. V prvních osnovách přírodozpytu měla chemie asi tříkrát méně místa než fyzika. Byl kladen velký důraz na výuku praktickou, jak ve smyslu výuky pomocí pozorování a pokusu, tak ve smyslu své budoucí využitelnosti na trhu práce, například \uv{o výrobě potravin (cukru, mouky, piva), o výrobních materiálech (oceli, porcelánu, skla, papíru)}, na vesnicích tedy využití v hospodářství a v domácnosti, ve městě využití v průmyslu. Teoretická výuka byla nevyvinutá a považována za zbytečně složitou, teoretické poznatky se žákům dostávaly jen okrajově, a to především těm, kteří pokračovali ve studiu na gymnáziích a reálkách. Zprvu měl přírodozpyt časovou dotaci dvě hodiny týdně v šestém, sedmém a osmém ročníku, toto bylo navýšeno na tři hodiny týdně v roce 1932.\cite{prirodozpyt} Předměty chemie a fyzika byly odděleny až na dalším stupni vzdělání, tedy gymnáziích a reálkách. Zde se také začaly používat při výuce chemické pokusy a v roce 1930 zde byla zavedena povinná praktická chemická cvičení.\cite{historie_vyuky}

Větší změny přineslo období po druhé světové válce a celková restrukturalizace školství zákonem z roku 1948. Byla zavedena jednotná základní škola a výuka chemie nyní byla povinná pro všechny obecně vzdělávací školy, tedy pro devítileté základní školy a čtyřletá gymnázia. Větší část učiva nyní tvořilo mimo anorganické a organické chemie využití chemie v dobových technologiích. Zákon z roku 1953 přinesl další změny, základní škola byla o rok zkrácena a místo gymnázií vznikly tříleté střední školy. To znamenalo redukci učiva, další změnou byly přísně závazné osnovy, ve kterých se na úkor organické a anorganické chemie navíc probírala mineralogie a geologie. Další změnou byl zákon z roku 1960, který znovu zavedl devítiletou základní školu. Chemie se vyučovala v osmém a devátém ročníku základní a ve všech třech ročnících středních škol. Obsah vyučované chemie se zvětšil a na středních školách byly budovány první odborné učebny a laboratoře. V učivu byly potlačeny zbytečné výchovné složky a průmyslových poznatků, vyzdvihuje se vzdělávací funkce učiva. Základní škola byla opět zkrácena na osm let zákonem z roku 1976. Ten také znamenal další změnu v osnovách pro chemii, které nyní dávaly příliš velký důraz na teoretické znalosti a upozaďovaly empirické a praktické části chemie. Toto z chemie dělalo předmět obtížnější a mezi žáky méně oblíbený, čemuž rozhodně nepomohlo, že se k němu časté změny ve školství chovaly jako k nadstavbovému předmětu, dost pravděpodobně nepotřebnému pro běžného žáka.\cite{historie_vyuky}\cite{u_nas_v_zahranici}

Po roce 1989 došlo k dalším změnám školského systému. Závazné osnovy byly po roce 2000 přeměněny na rámcové vzdělávací programy (RVP), které dávají školám stupeň volnosti v s sestavování školních vzdělávacích programů (ŠVP), dávají tedy školám podíl na určení způsobu a obsahu výuky. Takto systém funguje dodnes.\cite{u_nas_v_zahranici}

\section{Současnost výuky chemie}
Dnes je v RVP chemie spolu s fyzikou, zeměpisem a biologií součástí celku \uv{Člověk a příroda}, který má za úkol \uv{odkrývat metodami vědeckého výzkumu zákonitosti, jimiž se řídí přírodní procesy}\cite{rvp_g} a \uv{tím si uvědomovat i užitečnost přírodovědných poznatků a jejich aplikací v praktickém životě}\cite{rvp_zv}. V rámci chemie si žák má osvojit obecnou chemii, chemii organickou i anorganickou, základy bezpečnosti práce a praktické využití chemie v dnešním světě.\cite{rvp_g}\cite{rvp_zv}

%priklady skol pokud bude malo textu

\section{Co je to chemický pokus?}
Chemický pokus nebo experiment je záměrně vyvolaný proces prováděn cíleným ovlivňováním chemických podmínek, a to za účelem objevení, ověření nebo demonstrace chemického jevu. Chemické pokusy mohou být především dvojího typu, a to demonstrační nebo žákovské.\cite{badani}
\begin{itemize}
  \item demonstrační -- předváděné učitelem před třídou, jejich výhodou je snížené riziko bezpečnostních problémů s chemikáliemi, protože s chemikáliemi pracuje jen učitel, učitel si tedy třeba může dovolit pokusy s chemikáliemi, které by do rukou žákům nesvěřil, nevýhodou může být neúplné nebo nedostatečné zapojení žáků
  \item žákovské -- prováděné samotnými žáky, jejich výhodou je možnost žáka pronikout hlouběji do dané problematiky, jejich nevýhodou může být značná potřeba času a vhodného prostoru pro jejich provedení
\end{itemize}
Další dělení je na experimenty induktivní a deduktivní.\cite{badani}
\begin{itemize}
  \item induktivní -- podle výsledků pokusu vyvozujeme obecná pravidla, zákonitosti
  \item deduktivní -- konkrétním experimentem ověříme pravdivost sdělených pravidel
\end{itemize}
Důležitá je také struktura a stavba experimentu, v tomto ohledu rozlišujeme tři fáze pokusu.\cite{badani}
\begin{itemize}
  \item příprava -- to, co probíhá před samotným provedením pokusu, takže např. příprava látek, prostředí, postupu pokusu, očekávaného výsledku pokusu, seznámení žáků s podstatou experimentu, seznámení žáku s bezpečostí práce, ap.
  \item realizace -- samotné provedení pokusu, dle postupu, se kterým se učitel/žáci seznámili v přípravné fáze se dostávají k vytyčenému výsledku pokusu
  \item hodnocení -- učitel/žák popíše, co se v rámci experimentu stalo a taky to zhodnotí, např. jestli byl výsledek pokusu v souladu s tím očekávaným, ap.
\end{itemize}

\section{Proč chemický pokus?}
Cílem výuky je předat informace žákovi tak, aby je pochopil a zapamatoval si je. Chemie jako přírodní věda může být předmětem silně teoretickým, o poučkách a pravidlech. Pro některé žáky je tak obtížným a neoblíbeným předmětem.\cite{oblibenost} Učitel by se tedy měl snažit využít dostupných možností, aby látku udělal pro žáka přístupnější, v případě chemie je velmi efektivní pomůckou chemický pokus. Je znamé, že některé metody učení jsou efektivnější než jiné; chemické pokusy vedou k tzv. aktivizaci žáka -- žák jenom pasivně nesedí v lavici, ale sám se do výuky zapojuje. \uv{Při provádění chemického pokusu probíhá smyslové vnímání, přímé pozorování, abstraktní
myšlení.}\cite{ostrava} Látku takto předvedenou si žák zapamatuje spíš než prostý text nebo výklad.\cite{badani} Při demonstračním pokusu nemusí jen žák věřit učiteli, že pravidlo opravdu platí; sám to uvidí. V tomto je pak ještě lepší pokus žákovský, který dovolí žákovi se vlastním tempem seznámit s každým krokem pokusu do té míry, do které potřebuje, a při jeho provedení si pokus nejen \uv{okoukat} ale i \uv{ohmatat} -- zároveň se při něm žák prakticky naučí bezpečnost práce v chemické laboratoři a pravidla manipulace s chemickými látkami, které jsou součastí RVP.

K plnému využití potenciálu chemických pokusů ve výuce jsou potřeba především moderní pomůcky a učitel, který ví, jak správně pokusy do výuky zařadit, dále samozřejmě zdroj, ze kterého učitel postupy chemických pokusů čerpá.

Chemický pokus je zároveň způsob, kterým přiblížit praktické využití chemie, např. v průmyslu.

Chemické pokusy také podněcují samostatné myšlení žáka, ty žákovské pak i jeho tvořivost.


	\chapter{Praktická část}

\section{Metodika}
Z dostupných zdrojů byla nejdříve zjištěna podstatu pokusu a~postup jeho provedení. Pokus byl proveden, přičemž byl zdokumentován (fotoaparátem či kamerou) každý krok za účelem možného doplnění nebo opravení nedostatků v~původním zdroji. V~případě potřeby byl pokus zopakován. Zároveň byly poznamenány všechny bezpečnostní požadavky, které byly potřeba pro provedení pokusů.

Fotky a~videa byly následně zpracovány dle potřeby do shrnujícího doprovodného videa, nebo jen do souboru doprovodných fotografií. Spolu s~doplněným popisem a~postupem pokusu bylo vše nahráno na stránku \uv{Chemické pokusy} na portálu \uv{WikiKnihy} (\url{https://cs.wikibooks.org/wiki/Chemick%C3%A9_pokusy}) do článku příslušného pokusu. Tato možnost byla zvolena kvůli tomu, že je na internetu a~tedy jednoduše přístupná, dále pro svou otevřenost jako u~jiných \uv{wiki projektů} -- v~případě chyby nebo nepřesnosti může kdokoliv texty jednoduše bez dlouhého kontaktování správců stránky opravit nebo doplnit, jednoduše také může katalog rozšířit. WikiKnihy je projekt celosvětový, pomocí překladů do jiných jazyků se katalog může dostat k~ještě většímu počtu lidí. Licencování obsahu podle Creative Commons též umožňuje používaní obsahu bez problémů s~autorským právem.


\section{Zpracované pokusy}
\subsection{Žíhání skalice}

\B{Zařazení do výuky}

Tento pokus by mohl být zařazen při výuce hydrátů a jejich názvosloví nebo zkoušky plamenem jako analytické metody. Pokus může být take použit jako podklad pro příklad na výpočet reálnéo látkového množství nebo hmotnosti odpařené vody nebo zbylého bezvodého síranu v porovnání s jejich teoretickými hodnotami.

\hspace{-21pt} \B{Bezpečnost}

Při tomto pokusu se manipuluje s~otevřeným ohněm.

\hspace{-21pt} \B{Popis}

Řada solí s~krystalicky vázanou vodou, tzv. hydráty, jsou barevné (zpravidla díky přítomným aquakomplexům kationtů). Barva modré skalice je způsobena přítomností koordinačního kationtu. Při žíhání se modrá skalice zbavuje vázaných molekul vody a~přechází na bílý bezvodý síran měďnatý.
	%: <chem>CuSO<sub>4</sub> · 5H<sub>2</sub>O ->[\overset{}\ce{H2O}] CuSO<sub>4</sub> + 5H<sub>2</sub>O</chem> (nefunguje)

\hspace{-21pt} \B{Postup}

\begin{enumerate}
  \item Sestavíme žíhací aparaturu: na trojnožku umístíme triangl a~pod něj plynový kahan.
  \item v~suché třecí misce rozetřeme asi 1,5 g pentahydrátu síranu měďnatého.
  \item Zvážíme čistý a~suchý žíhací kelímek a~poté do něj nasypeme rozetřený pentahydrát síranu měďnatého (přesné navážky zaznamenáme).
  \item Žíhací kelímek umístíme pomocí laboratorních kleští do trianglu a~žíháme, dokud se zbarvení žíhané látky nezmění z~modré na bílou.
  \item Kelímek necháme zchladnout, zvážíme jej a~z rozdílů hmotnosti před a~po žíhání vypočítáme obsah krystalové vody.
  \item Bezvodý síran měďnatý můžeme pozorovat vlivem vzdušné vlhkosti měnit zabarvení zpátky na modrou.
\end{enumerate}

\subsection{Zlatý déšť}

\B{Zařazení do výuky}

Tento pokus by mohl být zařazen při výuce srážení jako separační metody nebo podvojné záměny jako typu chemické reakce. Hodnota pokusu je pak především v jeho visuální přitažlivosti pro žáka.

\hspace{-21pt} \B{Bezpečnost}

Dusičnan olovnatý je poměrně dobře rozpustný ve vodě a~obsahuje olovo -- je nezbytné dbát zvýšené opatrnosti při jeho manipulaci.

\hspace{-21pt} \B{Popis}

Dusičnan olovnatý a~jodid draselný v~roztoku zreagují na jodid olovnatý. Při snížení teploty se snižuje jeho rozpustnost a~z roztoku se vysráží zlaté krystalky jodidu olovnatého tvořící zlatý déšť.
%:Pb(NO<sub>3</sub>)<sub>2</sub> + 2 KI → PbI<sub>2</sub> + 2 KNO<sub>3</sub>

\hspace{-21pt} \B{Postup}

\begin{enumerate}
  \item v~kádince rozpustíme asi 0,3 g dusičnanu olovnatého ve 100 ml vody.
  \item v~druhé kádince rozpustíme asi 0,3 g jodidu draselného ve 100 ml vody.
  \item Oba roztoky zahřejeme blízko k~varu. Zahřívání potrvá pár minut.
  \item Horké roztoky slijeme do baňky a~necháme volně chladnout, nebo chladíme pod proudem studené vody nebo vhozením několika kostek ledu.
  \item Při chladnutí pozorujeme vznik žlutých krystalků -- různých velikostí podle rychlosti chlazení.
\end{enumerate}

\subsection{Chromatografie na papíře}

\B{Zařazení do výuky}

Tento pokus by mohl být zařazen při výuce chromatografie. Chromatografie je jednou ze separačních metod, a to z těch běžně vyučovaných asi nejneintuitivnější. Pokus názorně vysvětluje její princip, má malé časové i materiální nároky a může být proveden každým žákem samostatně.

\hspace{-21pt} \B{Bezpečnost}

Žádné zvláštní bezpečnostní požadavky.

\hspace{-21pt} \B{Popis}

Chromatografie je souhrnné označení pro skupinu separačních technik spočívajících v~rozdělování látek mezi dvě nemísitelné fáze - nepohyblivou (stacionární) a~pohyblivou (mobilní). Spolu s~pohybující se mobilní fází je soustavou unášen také vzorek. Dělené složky vzorku (analyty) interagují v~různé míře se stacionární a~mobilní fází. Analyty, které se poutají více ke stacionární fázi, se pohybují pomaleji a~jsou zadržovány déle, než analyty, které se ke stacionární fázi poutají méně. Na základě tohoto principu dochází k~rozdělení složek směsi.

V tomto experimentu je provedena chromatografie v~plošném uspořádání. Jako stacionární fáze je použit filtrační papír, jako mobilní voda nebo ethanol. \newline

\hspace{-21pt} \B{Postup}

\begin{enumerate}
\item Do kádinky nalijeme vrstvu asi 5 mm mobilní fáze (volíme podle typu fixů či obecně látek, které chceme dělit, např. voda, ethanol) a~přikryjeme hodinovým sklem.
\item Vystřihneme obdélník z~filtračního papíru (velký tak, aby se vešel do kádinky) a~tužkou označíme asi 1-2 cm od dolního okraje startovací čáru, na kterou uděláme puntíky fixami asi 1 cm od sebe (nebo naneseme vzorky kapilárou či kapátkem).
\item Vložíme papír s~nanesenými vzorky do kádinky tak, aby se nedotýkal stěn. Abychom zamezili kontaktu, můžeme horní okraj papíru navléct na špejli nebo drátek (případně papír můžeme přehnout do tvaru obráceného ""V"" a~nanést vzorky na obě strany). Po vložení papíru opět přikryjeme kádinku hodinovým sklem.
\item Necháme mobilní fázi vzlínat až do vzdálenosti 1 cm pod okraj papíru. Poté papír vyjmeme, označíme tužkou čelo mobilní fáze (= místo, kam vystoupala), papír usušíme a~vyhodnotíme rozdělení barviv.
\item v~případě zájmu můžeme vypočítat retenční faktor Rf pro každou látku. k~tomu potřebujeme určit vzdálenost, kterou urazila látka (střed skvrny) od startovní linie (a), a~vzdálenost, kterou urazila mobilní fáze (b). Rf získáme jako podíl (a)/(b)."
\end{enumerate}


	\newpage
\chapter*{Závěr}
\addcontentsline{toc}{chapter}{Závěr}

závěr


	\newpage
	\printbibliography[title=Literatura]
	\addcontentsline{toc}{chapter}{Literatura}

	%%%\listoffigures
	%%%\addcontentsline{toc}{section}{Seznam obrázků}

	%%%\listoftables
	%%%\addcontentsline{toc}{section}{Seznam tabulek}

	%%%\listoflistedequation
	%%%\addcontentsline{toc}{section}{Seznam rovnic}

\end{document}
