% soctepmlate unofficial - SOČ = Středoškolská odborná činnost - Czech competition
% Author: Vojtěch Boček
% Edit by: Jaroslav Páral
% Version: 2018-02-12
% Source code: https://github.com/RoboticsBrno/soctemplate/
% Base on: http://www.jcmm.cz/cz/sablona-soc.html
% License: CC BY 4.0

\documentclass{template/socthesis}

\usepackage{subcaption}
\usepackage{amsmath}
\usepackage{enumitem}

\addbibresource{text.bib}

\titlecz{Chemické pokusy}
\titleen{Chemical expermients}
\author{Jan Romanovský}
\field{12}
\school{Gymnázium Brno, třída Kpt.~Jaroše}
\mentor{Mgr. Zdeněk Moravec, Ph.D.}
\mentorstatement{Mgr. Zdeňka Moravce, Ph.D.}

% Změňte, pokud se liší
%\region{Jihomoravský}
%\placefooter{Brno 2023}

\begin{document}

	\maketitle

	\makecopyrightstatement{V~Brně}

	\pagestyle{empty}

	\section*{Anotace}
	Cílem práce je pojednání o důležitosti chemických pokusů v moderní výuce chemie, dále lehce přístupný katalog efektních chemických pokusů, který by sloužil jako pomůcka při výuce chemie na ZŠ a SŠ. Práce je členěna do teoretické a praktické části, dále do podcelků podle logické návaznosti k tématu textu.
	Cílem teoretické části je podat shrnutou historii výuky chemie v ČR, vymezení chemických pokusů jako pomůcek při výuce a význam chemických pokusů ve výuce.
	Cílem praktické části je provedení několika pokusů a jejich následné zpracování do katalogu pro potřeby výuky na ZŠ a SŠ, dále ověření funkčnosti tohoto katalogu.

	\subsection*{Klíčová slova}
	chemický pokus, výuka chemie,

	\vspace{20mm}

	\section*{Annotation}
	Insert annotation

	\subsection*{Keywords}
	chemical experiment, chemistry teaching

	\newpage
	\pagestyle{plain}

    \tableofcontents % vysází obsah

	%%% Začátek práce
	\setcounter{figure}{0}
	\setcounter{table}{0}
	\newpage

	%%% Vložení jednotlivých částí
	\chapter*{Úvod}
\addcontentsline{toc}{chapter}{Úvod}

Chemie je fascinujícím oborem s širokým využitím v dnešní společnosti. Pro udržení a rozvoj současné úrovně chemie je pak nezbytně důležita dobrá a zajímavá výuka chemie. Jedním z nezbytných prvků výuky chemie jsou chemické pokusy. Tyto pokusy nejen, že obohacují výuku o praktické aspekty, ale také hrají klíčovou roli při osvojování teoretických konceptů. Tato práce je zaměřena na popsání důležitosti chemických experimentů v procesu výuky chemie a jejich značné výhody.

Chemické pokusy představují pro studenty mimořádně atraktivní způsob, jak si osvojit chemické pojmy a principy. Prostřednictvím přímé interakce s materiály a pozorováním jejich chování mají studenti možnost vnímat abstraktní teorie jako konkrétní jevy. Tímto způsobem se upevňuje jejich chápání chemických principů a zároveň se rozvíjí jejich zájem o daný obor.

Tato práce se dále zaměřuje na vytvoření katalogu chemických pokusů, který může sloužit jako nástroj pro pedagogy při plánování výuky. Tento katalog zahrnuje pokusy vhodné pro výuku chemie na druhém stupni základní školy a na střední škole. Cílem tohoto katalogu je poskytnout pedagogům nástroj pro efektivní a interaktivní výuku chemie, který podněcuje zvídavost, experimentování a objevování.
\newpage


	\chapter{Teoretická část}

\section{Historie a vývoj výuky chemie}
Všeobecná výuka chemie na našem území začíná v polovině 19. století, kdy byl ve školách zaváděn předmět \uv{přírodozpyt}, a to nejprve na měšťanských školách r. 1869. Tento předmět měl za účel zkoumání přírodních zákonitostí a zahrnuta v něm byla z dnešních předmětů kromě chemie i fyzika. Byl kladen velký důraz na výuku praktickou, ve smyslu své budoucí využitelnosti na trhu práce, například \uv{o výrobě potravin (cukru, mouky, piva), o výrobních materiálech (oceli, porcelánu, skla, papíru)}\cite{prirodozpyt} a teoretické poznatky se žákům dostávaly jen v případě pokračování studia na gymnáziích nebo reálkách. Tyto teoretické poznatky byly navíc neúplné a jejich výuka nesystematická. Na těchto školách též postupně došlo k oddělení chemie a fyziky. Na reálkách rovněž byla v poslední třetině 19. století vyučována praktická chemická cvičení, která se stala povinná pro gymnázia i reálky ve školním roce 1930/31.\cite{historie_vyuky}

Všeobecná býuka chemie na našem území začíná tzv. \uv{Hasnerovým zákonem} z roku 1869, který zavádí mimo povinné osmileté školní docházky na obecné nebo měšťanské škole na školách měšťanských výuku předmětu zvaného přírodozpyt. Přírodozpyt měl za úkol učit žáky přírodním zákonitostem, \uv{rozebírat} svět kolem nich, z dnešních předmětů zahrnoval kromě chemie i fyziku. V prvních osnovách přírodozpytu měla chemie asi tříkrát méně místa než fyzika. Byl kladen velký důraz na výuku praktickou, ve smyslu své budoucí využitelnosti na trhu práce, například \uv{o výrobě potravin (cukru, mouky, piva), o výrobních materiálech (oceli, porcelánu, skla, papíru)}, na vesnicích tedy využití v hospodářství a v domácnosti, ve městě využití v průmyslu.\cite{prirodozpyt} Teoretická výuka byla nevyvinutá a považována za zbytečně složitou, teoretické poznatky se žákům dostávaly jen okrajově, a to především těm, kteří pokračovali ve studiu na gymnáziích a reálkách. 

vysvetlit skolni system tehdy, taky jazyky, ucebnice

Tento systém se udržel až do období po druhé světové válce a celkové restrukturalizace školství zákonem z r. 1948. Výuka chemie nyní byla povinná pro všechny obecně vzdělávací školy. V padesátých letech byla chemie vyučována po sověteském vzoru velmi polytechnicky, to znamenalo nižší úrovně teoretické chemie na úkor využití vzdělaných žáků v průmyslu. Dalším významným milníkem jsou šedesátá léta, kdy se zvětšil obsah vyučované chemie a na středních školách byly budovány první odborné učebny a laboratoře. Sedmdesátá a osmdesátá léta znamenala další změny ve výuce chemie. Osnovy pro chemii dávaly příliš velký důraz na teoretické znalosti a upozaďovaly praktické části chemie, což z ní dělalo předmět obtížnější a mezi žáky méně oblíbený.\cite{historie_vyuky}\cite{u_nas_v_zahranici}

tohle nejak zaramcovat

Sametová revoluce přinesla změny i ve školství. Závazné osnovy byly přeměněny na rámcové vzdělávací programy (RVP), které dávají školám určitou volnost ve způsobu a obsahu výuky; takto systém funguje dodnes.

rozvest??

\section{Současnost výuky chemie}
Dnes je v RVP chemie spolu s fyzikou, zeměpisem a biologií součástí celku \uv{Člověk a příroda}, který má za úkol \uv{odkrývat metodami vědeckého výzkumu zákonitosti, jimiž se řídí přírodní procesy}\cite{rvp_g} a \uv{tím si uvědomovat i užitečnost přírodovědných poznatků a jejich aplikací v praktickém životě}\cite{rvp_zv}. V rámci chemie se tedy žák má osvojit obecnou chemii, chemii organickou i anorganickou, základy bezpečnosti práce a praktické využití chemie v dnešním světě.\cite{rvp_g}\cite{rvp_zv}

priklady skol?

\section{Co je to chemický pokus?}
Chemický pokus je záměrně vyvolaný proces prováděn cíleným ovlivňováním chemických podmínek, a to za účelem objevení, ověření nebo demonstrace chemického jevu.\cite{badani} Chemické pokusy mohou být především dvojího typu, demonstrační, které jsou předváděné učitelem před celou třídou nebo žákovské, které každý žák zpracovává samostatně.

typy pokusu?

\section{Proč chemický pokus?}
Cílem výuky je předat informace žákovi tak, aby je pochopil a zapamatoval si je. Chemie jako přírodní věda může být předmětem silně teoretickým, o poučkách a pravidlech. Pro některé žáky je tak obtížným a neoblíbeným předmětem.\cite{oblibenost} Učitel by se tedy měl snažit využít dostupných možností, aby látku udělal pro žáka přístupnější, v případě chemie je velmi efektivní pomůckou chemický pokus. Je znamé, že některé metody učení jsou efektivnější než jiné; chemické pokusy vedou k tzv. aktivizaci žáka -- žák jenom pasivně nesedí v lavici, ale sám se do výuky zapojuje. Látku takto předvedenou si žák zapamatuje spíš než prostý text nebo výklad.\cite{badani}

vice do hloubky pls


	\chapter{Praktická část}

\section{Metodika}
Z dostupných zdrojů byla nejdříve zjištěna podstatu pokusu a~postup jeho provedení. Pokus byl proveden, přičemž byl zdokumentován (fotoaparátem či kamerou) každý krok za účelem možného doplnění nebo opravení nedostatků v~původním zdroji. V~případě potřeby byl pokus zopakován. Zároveň byly poznamenány všechny bezpečnostní požadavky, které byly potřeba pro provedení pokusů.

Fotky a~videa byly následně zpracovány dle potřeby do shrnujícího doprovodného videa, nebo jen do souboru doprovodných fotografií. Spolu s~doplněným popisem a~postupem pokusu bylo vše nahráno na stránku \uv{Chemické pokusy} na portálu \uv{WikiKnihy} (\url{https://cs.wikibooks.org/wiki/Chemick%C3%A9_pokusy}) do článku příslušného pokusu. Tato možnost byla zvolena kvůli tomu, že je na internetu a~tedy jednoduše přístupná, dále pro svou otevřenost jako u~jiných \uv{wiki projektů} -- v~případě chyby nebo nepřesnosti může kdokoliv texty jednoduše bez dlouhého kontaktování správců stránky opravit nebo doplnit, jednoduše také může katalog rozšířit. WikiKnihy je projekt celosvětový, pomocí překladů do jiných jazyků se katalog může dostat k~ještě většímu počtu lidí. Licencování obsahu podle Creative Commons též umožňuje používaní obsahu bez problémů s~autorským právem.


\section{Zpracované pokusy}
\subsection{Žíhání skalice}

\B{Bezpečnost}

Při tomto pokusu pracujeme s~otevřeným ohněm.

\hspace{-21pt} \B{Popis}

Řada solí s~krystalicky vázanou vodou, tzv. hydráty, jsou barevné (zpravidla díky přítomným aquakomplexům kationtů). Barva modré skalice je způsobena přítomností koordinačního kationtu. Při žíhání se modrá skalice zbavuje vázaných molekul vody a~přechází na bílý bezvodý síran měďnatý.
	%: <chem>CuSO<sub>4</sub> · 5H<sub>2</sub>O ->[\overset{}\ce{H2O}] CuSO<sub>4</sub> + 5H<sub>2</sub>O</chem> (nefunguje)

\hspace{-21pt} \B{Postup}

\begin{enumerate}
  \item Sestavíme žíhací aparaturu: na trojnožku umístíme triangl a~pod něj plynový kahan.
  \item v~suché třecí misce rozetřeme asi 1,5 g pentahydrátu síranu měďnatého.
  \item Zvážíme čistý a~suchý žíhací kelímek a~poté do něj nasypeme rozetřený pentahydrát síranu měďnatého (přesné navážky zaznamenáme).
  \item Žíhací kelímek umístíme pomocí laboratorních kleští do trianglu a~žíháme, dokud se zbarvení žíhané látky nezmění z~modré na bílou.
  \item Kelímek necháme zchladnout, zvážíme jej a~z rozdílů hmotnosti před a~po žíhání vypočítáme obsah krystalové vody.
  \item Bezvodý síran měďnatý můžeme pozorovat vlivem vzdušné vlhkosti měnit zabarvení zpátky na modrou.
\end{enumerate}

\subsection{Zlatý déšť}

\B{Bezpečnost}

Dusičnan olovnatý je poměrně dobře rozpustný ve vodě a~obsahuje olovo -- dbejte zvýšené opatrnosti při jeho manipulaci.

\hspace{-21pt} \B{Popis}

Dusičnan olovnatý a~jodid draselný v~roztoku zreagují na jodid olovnatý. Při snížení teploty se snižuje jeho rozpustnost a~z roztoku se vysráží zlaté krystalky jodidu olovnatého tvořící zlatý déšť.
%:Pb(NO<sub>3</sub>)<sub>2</sub> + 2 KI → PbI<sub>2</sub> + 2 KNO<sub>3</sub>

\hspace{-21pt} \B{Postup}

\begin{enumerate}
  \item v~kádince rozpustíme asi 0,3 g dusičnanu olovnatého ve 100 ml vody.
  \item v~druhé kádince rozpustíme asi 0,3 g jodidu draselného ve 100 ml vody.
  \item Oba roztoky zahřejeme blízko k~varu. Zahřívání potrvá pár minut.
  \item Horké roztoky slijeme do baňky a~necháme volně chladnout, nebo chladíme pod proudem studené vody nebo vhozením několika kostek ledu.
  \item Při chladnutí pozorujeme vznik žlutých krystalků -- různých velikostí podle rychlosti chlazení.
\end{enumerate}

\subsection{Chromatografie na papíře}

\B{Bezpečnost}

Žádné zvláštní bezpečnostní požadavky.

\hspace{-21pt} \B{Popis}

Chromatografie je souhrnné označení pro skupinu separačních technik spočívajících v~rozdělování látek mezi dvě nemísitelné fáze - nepohyblivou (stacionární) a~pohyblivou (mobilní). Spolu s~pohybující se mobilní fází je soustavou unášen také vzorek. Dělené složky vzorku (analyty) interagují v~různé míře se stacionární a~mobilní fází. Analyty, které se poutají více ke stacionární fázi, se pohybují pomaleji a~jsou zadržovány déle, než analyty, které se ke stacionární fázi poutají méně. Na základě tohoto principu dochází k~rozdělení složek směsi.

V tomto experimentu provedeme chromatografii v~plošném uspořádání. Použijeme filtrační papír jako stacionární fázi a~vodu nebo ethanol jako mobilní fázi. \newline

\hspace{-21pt} \B{Postup}

\begin{enumerate}
\item Do kádinky nalijeme vrstvu asi 5 mm mobilní fáze (volíme podle typu fixů či obecně látek, které chceme dělit, např. voda, ethanol) a~přikryjeme hodinovým sklem.
\item Vystřihneme obdélník z~filtračního papíru (velký tak, aby se vešel do kádinky) a~tužkou označíme asi 1-2 cm od dolního okraje startovací čáru, na kterou uděláme puntíky fixami asi 1 cm od sebe (nebo naneseme vzorky kapilárou či kapátkem).
\item Vložíme papír s~nanesenými vzorky do kádinky tak, aby se nedotýkal stěn. Abychom zamezili kontaktu, můžeme horní okraj papíru navléct na špejli nebo drátek (případně papír můžeme přehnout do tvaru obráceného ""V"" a~nanést vzorky na obě strany). Po vložení papíru opět přikryjeme kádinku hodinovým sklem.
\item Necháme mobilní fázi vzlínat až do vzdálenosti 1 cm pod okraj papíru. Poté papír vyjmeme, označíme tužkou čelo mobilní fáze (= místo, kam vystoupala), papír usušíme a~vyhodnotíme rozdělení barviv.
\item v~případě zájmu můžeme vypočítat retenční faktor Rf pro každou látku. k~tomu potřebujeme určit vzdálenost, kterou urazila látka (střed skvrny) od startovní linie (a), a~vzdálenost, kterou urazila mobilní fáze (b). Rf získáme jako podíl (a)/(b)."
\end{enumerate}


	\newpage
\chapter*{Závěr}
\addcontentsline{toc}{chapter}{Závěr}

Výuka chemie prostřednictvím chemických pokusů představuje nejenom efektivní metodu, jak přiblížit abstraktní teorie konkrétním jevům, ale také prostředek k rozvoji klíčových dovedností u studentů.

Katalog chemických pokusů, na kterém jsem se podílel, představuje zdroj inspirace pro pedagogy a studenty. Obsahuje škálu experimentů, které lze přizpůsobit různým úrovním studia a zájmům studentů. Věřím, že tento katalog přispěje k obohacení výuky chemie a poskytne pedagogům užitečný nástroj pro interaktivní a efektivní vzdělávání.

Nicméně je důležité si uvědomit, že úspěšná výuka chemie není pouze o provádění chemických experimentů, ale také o podpoře diskuzí, porozumění konceptům a podněcování zvídavosti studentů. Chemické pokusy by měly být vnímány jako součást širšího pedagogického přístupu, který klade důraz na aktivní zapojení studentů a podporuje jejich zájem o chemii jako obor.

Věřím, že moje práce přispěje k debatě nad významem chemických pokusů ve výuce chemie a třeba i poskytne inspiraci pro další práce v této oblasti.


	\newpage
	\printbibliography[title=Literatura]
	\addcontentsline{toc}{chapter}{Literatura}

	%%%\listoffigures
	%%%\addcontentsline{toc}{section}{Seznam obrázků}

	%%%\listoftables
	%%%\addcontentsline{toc}{section}{Seznam tabulek}

	%%%\listoflistedequation
	%%%\addcontentsline{toc}{section}{Seznam rovnic}

\end{document}
