\chapter{Teoretická část}

\section{Historie a vývoj výuky chemie}

Všeobecná býuka chemie na našem území začíná tzv. \uv{Hasnerovým zákonem} z roku 1869, který zavádí mimo povinné osmileté školní docházky na obecné nebo měšťanské škole na školách měšťanských výuku předmětu zvaného přírodozpyt. Přírodozpyt měl za úkol učit žáky přírodním zákonitostem, \uv{rozebírat} svět kolem nich. Z dnešních předmětů zahrnoval kromě chemie i fyziku. V prvních osnovách přírodozpytu měla chemie asi tříkrát méně místa než fyzika. Byl kladen velký důraz na výuku praktickou, jak ve smyslu výuky pomocí pozorování a pokusu, tak ve smyslu své budoucí využitelnosti na trhu práce, například \uv{o výrobě potravin (cukru, mouky, piva), o výrobních materiálech (oceli, porcelánu, skla, papíru)}, na vesnicích tedy využití v hospodářství a v domácnosti, ve městě využití v průmyslu. Teoretická výuka byla nevyvinutá a považována za zbytečně složitou, teoretické poznatky se žákům dostávaly jen okrajově, a to především těm, kteří pokračovali ve studiu na gymnáziích a reálkách. Zprvu měl přírodozpyt časovou dotaci dvě hodiny týdně v šestém, sedmém a osmém ročníku, toto bylo navýšeno na tři hodiny týdně v roce 1932.\cite{prirodozpyt} Předměty chemie a fyzika byly odděleny až na dalším stupni vzdělání, tedy gymnáziích a reálkách. Zde se také začaly používat při výuce chemické pokusy a v roce 1930 zde byla zavedena povinná praktická chemická cvičení.\cite{historie_vyuky}

Větší změny přineslo období po druhé světové válce a celková restrukturalizace školství zákonem z roku 1948. Byla zavedena jednotná základní škola a výuka chemie nyní byla povinná pro všechny obecně vzdělávací školy, tedy pro devítileté základní školy a čtyřletá gymnázia. Větší část učiva nyní tvořilo mimo anorganické a organické chemie využití chemie v dobových technologiích. Zákon z roku 1953 přinesl další změny, základní škola byla o rok zkrácena a místo gymnázií vznikly tříleté střední školy. To znamenalo redukci učiva, další změnou byly přísně závazné osnovy, ve kterých se na úkor organické a anorganické chemie navíc probírala mineralogie a geologie. Další změnou byl zákon z roku 1960, který znovu zavedl devítiletou základní školu. Chemie se vyučovala v osmém a devátém ročníku základní a ve všech třech ročnících středních škol. Obsah vyučované chemie se zvětšil a na středních školách byly budovány první odborné učebny a laboratoře. V učivu byly potlačeny zbytečné výchovné složky a průmyslových poznatků, vyzdvihuje se vzdělávací funkce učiva. Základní škola byla opět zkrácena na osm let zákonem z roku 1976. Ten také znamenal další změnu v osnovách pro chemii, které nyní dávaly příliš velký důraz na teoretické znalosti a upozaďovaly empirické a praktické části chemie. Toto z chemie dělalo předmět obtížnější a mezi žáky méně oblíbený, čemuž rozhodně nepomohlo, že se k němu časté změny ve školství chovaly jako k nadstavbovému předmětu, dost pravděpodobně nepotřebnému pro běžného žáka.\cite{historie_vyuky}\cite{u_nas_v_zahranici}

Po roce 1989 došlo k dalším změnám školského systému. Závazné osnovy byly po roce 2000 přeměněny na rámcové vzdělávací programy (RVP), které dávají školám stupeň volnosti v s sestavování školních vzdělávacích programů (ŠVP), dávají tedy školám podíl na určení způsobu a obsahu výuky. Takto systém funguje dodnes.\cite{u_nas_v_zahranici}

\section{Současnost výuky chemie}
Dnes je v RVP chemie spolu s fyzikou, zeměpisem a biologií součástí celku \uv{Člověk a příroda}, který má za úkol \uv{odkrývat metodami vědeckého výzkumu zákonitosti, jimiž se řídí přírodní procesy}\cite{rvp_g} a \uv{tím si uvědomovat i užitečnost přírodovědných poznatků a jejich aplikací v praktickém životě}\cite{rvp_zv}. V rámci chemie si žák má osvojit obecnou chemii, chemii organickou i anorganickou, základy bezpečnosti práce a praktické využití chemie v dnešním světě.\cite{rvp_g}\cite{rvp_zv}

%priklady skol pokud bude malo textu

\section{Co je to chemický pokus?}
Chemický pokus je záměrně vyvolaný proces prováděn cíleným ovlivňováním chemických podmínek, a to za účelem objevení, ověření nebo demonstrace chemického jevu. \cite{badani} Chemické pokusy mohou být především dvojího typu, demonstrační, které jsou předváděné učitelem před celou třídou nebo žákovské, které každý žák zpracovává samostatně.

\section{Proč chemický pokus?}
Cílem výuky je předat informace žákovi tak, aby je pochopil a zapamatoval si je. Chemie jako přírodní věda může být předmětem silně teoretickým, o poučkách a pravidlech. Pro některé žáky je tak obtížným a neoblíbeným předmětem.\cite{oblibenost} Učitel by se tedy měl snažit využít dostupných možností, aby látku udělal pro žáka přístupnější, v případě chemie je velmi efektivní pomůckou chemický pokus. Je znamé, že některé metody učení jsou efektivnější než jiné; chemické pokusy vedou k tzv. aktivizaci žáka -- žák jenom pasivně nesedí v lavici, ale sám se do výuky zapojuje. Látku takto předvedenou si žák zapamatuje spíš než prostý text nebo výklad.\cite{badani} Při demonstračním pokusu nemusí jen žák věřit učiteli, že pravidlo opravdu platí; sám to uvidí. V tomto je pak ještě lepší pokus žákovský, který dovolí žákovi se vlastním tempem seznámit s každým krokem pokusu do té míry, do které potřebuje, a při jeho provedení si pokus nejen \uv{okoukat} ale i \uv{ohmatat} -- zároveň se při něm žák prakticky naučí bezpečnost práce v chemické laboratoři a pravidla manipulace s chemickými látkami, které jsou součastí RVP.

K plnému využití potenciálu chemických pokusů ve výuce jsou potřeba především moderní pomůcky a učitel, který ví, jak správně pokusy do výuky zařadit, dále samozřejmě zdroj, ze kterého učitel postupy chemických pokusů čerpá.
