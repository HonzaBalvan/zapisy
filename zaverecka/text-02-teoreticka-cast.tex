\chapter{Teoretická část}

\section{Historie a vývoj výuky chemie}
Všeobecná výuka chemie na našem území začíná v polovině 19. století, kdy byl ve školách zaváděn předmět \uv{přírodozpyt}, a to nejprve na měšťanských školách r. 1869. Tento předmět měl za účel zkoumání přírodních zákonitostí a zahrnuta v něm byla z dnešních předmětů kromě chemie i fyzika. Byl kladen velký důraz na výuku praktickou, ve smyslu své budoucí využitelnosti na trhu práce, například \uv{o výrobě potravin (cukru, mouky, piva), o výrobních materiálech (oceli, porcelánu, skla, papíru)}\cite{prirodozpyt} a teoretické poznatky se žákům dostávaly jen v případě pokračování studia na gymnáziích nebo reálkách. Tyto teoretické poznatky byly navíc neúplné a jejich výuka nesystematická. Na těchto školách též postupně došlo k oddělení chemie a fyziky. Na reálkách rovněž byla v poslední třetině 19. století vyučována praktická chemická cvičení, která se stala povinná pro gymnázia i reálky ve školním roce 1930/31.\cite{historie_vyuky}

Všeobecná býuka chemie na našem území začíná tzv. \uv{Hasnerovým zákonem} z roku 1869, který zavádí mimo povinné osmileté školní docházky na obecné nebo měšťanské škole na školách měšťanských výuku předmětu zvaného přírodozpyt. Přírodozpyt měl za úkol učit žáky přírodním zákonitostem, \uv{rozebírat} svět kolem nich, z dnešních předmětů zahrnoval kromě chemie i fyziku. V prvních osnovách přírodozpytu měla chemie asi tříkrát méně místa než fyzika. Byl kladen velký důraz na výuku praktickou, ve smyslu své budoucí využitelnosti na trhu práce, například \uv{o výrobě potravin (cukru, mouky, piva), o výrobních materiálech (oceli, porcelánu, skla, papíru)}, na vesnicích tedy využití v hospodářství a v domácnosti, ve městě využití v průmyslu.\cite{prirodozpyt} Teoretická výuka byla nevyvinutá a považována za zbytečně složitou, teoretické poznatky se žákům dostávaly jen okrajově, a to především těm, kteří pokračovali ve studiu na gymnáziích a reálkách. 

vysvetlit skolni system tehdy, taky jazyky, ucebnice

Tento systém se udržel až do období po druhé světové válce a celkové restrukturalizace školství zákonem z r. 1948. Výuka chemie nyní byla povinná pro všechny obecně vzdělávací školy. V padesátých letech byla chemie vyučována po sověteském vzoru velmi polytechnicky, to znamenalo nižší úrovně teoretické chemie na úkor využití vzdělaných žáků v průmyslu. Dalším významným milníkem jsou šedesátá léta, kdy se zvětšil obsah vyučované chemie a na středních školách byly budovány první odborné učebny a laboratoře. Sedmdesátá a osmdesátá léta znamenala další změny ve výuce chemie. Osnovy pro chemii dávaly příliš velký důraz na teoretické znalosti a upozaďovaly praktické části chemie, což z ní dělalo předmět obtížnější a mezi žáky méně oblíbený.\cite{historie_vyuky}\cite{u_nas_v_zahranici}

tohle nejak zaramcovat

Sametová revoluce přinesla změny i ve školství. Závazné osnovy byly přeměněny na rámcové vzdělávací programy (RVP), které dávají školám určitou volnost ve způsobu a obsahu výuky; takto systém funguje dodnes.

rozvest??

\section{Současnost výuky chemie}
Dnes je v RVP chemie spolu s fyzikou, zeměpisem a biologií součástí celku \uv{Člověk a příroda}, který má za úkol \uv{odkrývat metodami vědeckého výzkumu zákonitosti, jimiž se řídí přírodní procesy}\cite{rvp_g} a \uv{tím si uvědomovat i užitečnost přírodovědných poznatků a jejich aplikací v praktickém životě}\cite{rvp_zv}. V rámci chemie se tedy žák má osvojit obecnou chemii, chemii organickou i anorganickou, základy bezpečnosti práce a praktické využití chemie v dnešním světě.\cite{rvp_g}\cite{rvp_zv}

priklady skol?

\section{Co je to chemický pokus?}
Chemický pokus je záměrně vyvolaný proces prováděn cíleným ovlivňováním chemických podmínek, a to za účelem objevení, ověření nebo demonstrace chemického jevu.\cite{badani} Chemické pokusy mohou být především dvojího typu, demonstrační, které jsou předváděné učitelem před celou třídou nebo žákovské, které každý žák zpracovává samostatně.

typy pokusu?

\section{Proč chemický pokus?}
Cílem výuky je předat informace žákovi tak, aby je pochopil a zapamatoval si je. Chemie jako přírodní věda může být předmětem silně teoretickým, o poučkách a pravidlech. Pro některé žáky je tak obtížným a neoblíbeným předmětem.\cite{oblibenost} Učitel by se tedy měl snažit využít dostupných možností, aby látku udělal pro žáka přístupnější, v případě chemie je velmi efektivní pomůckou chemický pokus. Je znamé, že některé metody učení jsou efektivnější než jiné; chemické pokusy vedou k tzv. aktivizaci žáka -- žák jenom pasivně nesedí v lavici, ale sám se do výuky zapojuje. Látku takto předvedenou si žák zapamatuje spíš než prostý text nebo výklad.\cite{badani}

vice do hloubky pls
