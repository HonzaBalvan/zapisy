\newpage
\chapter*{Závěr}
\addcontentsline{toc}{chapter}{Závěr}

Výuka chemie prostřednictvím chemických pokusů představuje nejenom efektivní metodu, jak přiblížit abstraktní teorie konkrétním jevům, ale také prostředek k rozvoji klíčových dovedností u studentů.

V první části jsem podal shrnutou časovou osu výuky chemie u nás od 19. století až po současnost, včetně jejích rozličných změn po druhé světové válce. Dále jsem vymezil, co je to chemický pokus, jak probíhá, a jak se pokusy dají klasifikovat. Rozebral jsem i důležitost a hodnotu ve využívání chemických pokusů při výuce chemie.

V druhé části jsem přispěl ke tvorbě internetového katalogu chemických pokusů, který by obsahoval pokusy přímo připravené k použití učitelem chemie. V laboratoři jsem provedl pokusy \uv{Žíhání skalice}, \uv{Zlatý déšť} a \uv{Chromatografie na papíře}, jejich provádění pečlivě zdokumentoval, toto následně shrnul do popisného odstavce a návodu k provedení.

Katalog chemických pokusů, na kterém jsem se podílel, představuje zdroj inspirace pro pedagogy a studenty. Obsahuje škálu experimentů, které lze přizpůsobit různým úrovním studia a zájmům studentů. Věřím, že tento katalog přispěje k obohacení výuky chemie a poskytne pedagogům užitečný nástroj pro interaktivní a efektivní vzdělávání.

Nicméně je důležité si uvědomit, že úspěšná výuka chemie není pouze o provádění chemických experimentů, ale také o podpoře diskuzí, porozumění konceptům a podněcování zvídavosti studentů. Chemické pokusy by měly být vnímány jako součást širšího pedagogického přístupu, který klade důraz na aktivní zapojení studentů a podporuje jejich zájem o chemii jako obor.

Věřím, že moje práce přispěje k debatě nad významem chemických pokusů ve výuce chemie a třeba i poskytne inspiraci pro další práce v této oblasti.
