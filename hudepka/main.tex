\documentclass{article}
\usepackage{fullpage}
\usepackage[czech]{babel}
\usepackage{amsfonts}

\title{\vspace{-2cm}\textbf{Hudební výchova}\vspace{-1.7cm}}
\date{}
\author{}

\begin{document}
\maketitle

\section{Leoš Janáček}
\subsection{Dílo}
\begin{itemize}
  \item Janáček napsal 10 oper, hrají se po celém světě, libreta si psal sám, většinou podle nějakého románu, dělal to podle lidských nápěvků, které notově zapisoval
  \begin{itemize}
    \item opera Její pastorkyňa -- pastorkyňa je schovanka, podle románu Gabriely Preissové, příběh Jenúfy, která se zamiluje do Štefa, ten ale chce jinou; má ji rád Laco, kostelnička která se o ni stará zabije chlapečka, aby se mohla vdát, po zimě až roztají ledy tak se to ukáže, špatně to končí
    \item opera Příhody lišky Bystroušky -- podle romána Rudolfa Těsnohlídka, vypráví o příběhu lišky Bystroušky, která přijde o matku, domů si ji přivede hajný, pak uteče, najde krásného lišáka a mají spolu liščátka, revírník ale Bystroušku zastřelí, kolem ní pobíhají malá liščátka, koukají na mrtvou mámu a teď co budeme dělat, lišák to samé, nakonec odbíhají pryč a o liščátka se stará lišák, dostávají se přes to -- smrt je součástí života
    \item opera Věc Makropulos -- podle románu Karla Čapka, Emilie Makropulos, v době Rudolfa II., její otec je šarlatán, jeden lektvar vyzkouší na ni a ona žije dlouze, pak chce najít recept na elixír věčného života, nakonec pergamen hází do ohně -- \uv{život už mně všechno dal, všechno jsem zažila, život už mi nemá co dát, chci přijmout to, co je součástí života -- smrt}
    \item opera Výlety pana Broučka -- podle románu Svatopluka Čecha, příběh ve snu, pan Brouček se hrozně napil v hodpodě, spadl do nějakého sudu a usnul, zdál se mu sen, že je v 15. st., v době Jana Žižky a Jan Žižka se ho ptá kdo je, on mu říká, že je z jiného věku, jde do bitvy, je málem zabit a to ho probudí
    \item opera Káťa Kabanová, opera Šárka, opera Osud
  \end{itemize}
  \item symfonická díla
  \begin{itemize}
    \item Lašské tance
    \item Taras bulba -- z ruských dějin
    \item Sinfonietta pro Brno
  \end{itemize}
  \item skladby pro mužský sbor Marička Magdonová, Kantor Halbar -- zhudebněný Bezruč
  \item pro děti Říkadla, dechový sextet Mládí, ze souboru klavírních skladeb skladby Po travnatém chodníčku, V mlhách
  \item smyčcový kvartet Listy důverné (věnován Kamile, s ní se pak soudila Janáčkova žena o peníze z hraní skladby)
  \item Glagolská mše -- náměty z 10., 11. st., staroslověnština; Janáček žil v době kdy byla velká slovanská vzájemnost, jako Slovani měli držet pospolu, společný jazyk, původ -- měl bratra v Rusku, miloval Rusko, jeho dcera jela do Ruska -- vrací se ke všemu co je slovanské, tedy i staroslověnskou kulturu
\end{itemize}

\section{Bohuslav Martinů}
\subsection{Život}
\begin{itemize}
  \item narodil se v Poličce, v kostelní věži
  \item odjíždí studovat na konzervatoř do Prahy, kromě hudby měl rád divadlo a malířství, na kozervatoři studoval housle, věnoval se i kompozici
  \item je přijat do filharmonie a r. 1919 s filharmonií odjiždí koncertovat do Paříže
  \item vrací se domů a r. 1923 odjíždí studovat do Paříže, tam má svého učitele kompozice, potkává se s Igorem Stravinský, žije v Paříži, pak se vrácí domů a jede do Paříže zpátky, po r. 1939 prvně emigruje na jih Francie, potom lodí do USA, tam žije, dostává se mu uznání a domů se už nikdy nevrátí
  \item na konci života v 50. letech hodně času trávil v Evropě -- především ve Švýcarsku, r. 1958 tam prodělává operaci rakoviny žaludku a zůstává tam v sanatoriu, tam o rok později umírá
  \item svým dílem tesknil po domově, po Vysočině, ikdyž mu tam bylo velice dobře, nikdy v cizině nebyl úplně doma
\end{itemize}
\subsection{Dílo}
\begin{itemize}
  \item šest symfonií, kantáty Otevírání studánek, Dým bramborové natě, skladby pro klavír Louky, oratorium O Gilgamešovi, opery Mirandolina a Řecké pašije, skladby s prvky jazzu Poločas, Vřava
\end{itemize}

\end{document}
