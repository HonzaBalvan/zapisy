\documentclass{article}
\usepackage{fullpage}
\usepackage[czech]{babel}
\usepackage{amsfonts}

\title{\vspace{-2cm}\textbf{Hudební výchova}\vspace{-1.7cm}}
\date{}
\author{}

\begin{document}
\maketitle

\part{Český romatismus -- Národní divadlo}
\begin{itemize}
  \item konzervatoř 1808, národní divadlo 1868 začátek stavby, hned vedle divadla stojiícího, Prozatimního dovezeny základní kameny z posvátných míst, nutnost divadla v jazyce českém, architekt Josef Zítek, vybrali si moc malý pozemek, takže divadlo je komicky vysoké a na těch nejvyšších sedadlech je to pěkně na nic, divadlo 1881 vyhořelo, rekonstruuje Zítkův žák Josef Schulz, udělal úpravy interiéru a přidal místa k sezení, divadlo otevřeno Smetanovou operou Libuší, buditelský charakter, ,,český národ neskoná, on všechny … (strasti) … překoná”, trigy - trojspřeží sochy od Schnircha, oponu vytvořil František Ženíšek, ta zhořela a novou oponu vytvořil Vojtěch Hynais, tato schována v muzeu, dále se podíleli např. Mikoláš Aleš (stropní malby), Josef Václav Myslbek (sochy), Václav Brožík, Julius Mařák, nad oponou nápis Národ sobě, divadlo postaveno ze sbírky (do které velkou část přispěl císař a jeho rodina mimochodem)
\end{itemize}

\section{Bedřich Smetana}

\subsection{Život}
\begin{itemize}
  \item narodil se v Litomyšli na zámku jako syn sládka v zámeckém pivovaru roku 1824, umřel 1884, jeden z našich nejvýznamnějších skladatelů, jeden ze zakladatelů české národní hudby
  \item v dětství se přestěhovali do Jindřichova Hradce kde studoval na gymnasiu, to mu ale nešlo takže tady nedostudoval, až pak dostudoval v Plzni díky svému bratranci profesorovi Františkovi, po studiu se chtěl věnovat kariéře pianisty a proto se učil u vynikajícího hudebního pedagoga Proksche, chtěl se stát virtuosem, to se ale nestalo, r. 1848 otvírá u Staromáku učitelský ústav, žení se s pianistkou Kateřinou, měli spolu 4 dcery, dospělosti se dožila jen Žofie, Kateřina umírá na tuberkulózu, bere si pak Bethy Ferdinandovou, s ní měl další 2 dcery, Boženu a Zdeňku (obě přežily), mezi tímto dvě angažmá v Götteborgu, pak se stal kapelníkem Prozatimního divadla, kde dirigoval i Antonína Dvořáka (ten tam hraje na violu), pak onemocňuje, přestává slyšet, Mou Vlast stvořil celou jako hluchý, na skolnku života pobýval u Žofie na myslivně, ta ho pak odvezla do Prahy do ústavu choromyslných
\end{itemize}

\subsection{Dílo}
\begin{itemize}
  \item oper 9 -- Libuše, Prodaná nevěsta, Čertova stěna, Dalibor, Hubička, dále polky -- Bettina polka (ukázka), Našim děvám, Venkovanka, Ze studenstkého života, Luisina, taky 14 českých tanců -- Furiant, Hulán, Obkročák, Skočná (ukázka), Dupák (ukázka), skladby pro klavír -- Nevinnost, Láska, Přívětivá krajina (ukázka), polky ve stylizaci -- Polka G mol, Polka E mol, duo pro housle a klavír Z domoviny, smyčcový kvartet Z mého života (ukázka), skladby pro sbor Má hvězda, Západ, Vlaštovičky, nejznámější je však soubor 6 symfonických básní Má vlast, mezi částmi se netleská (btw), v přesném pořadí, symf. básně tedy hudba programní
  \begin{enumerate}
    \item Vyšehrad -- začíná motivem harf (což je zaímavé, to se často nedělo), chce zde vyjádřit slávu Vyšehradu (tedy ne toho současného, ale toho minulého, kde sídlila česká knížata), snaha vylíčit lesk a krásu tehdejší funkce Vyšehradu a jeho teskný a smutný konec
    \item Vltava -- znázorňuje tok řeky Vltavy, a to od malinkého praménku na Šumavě až po obrovskou mohutnost řeky Vltavy, zpočátku malá flétnička, ja rychle teče, potom velké mohutné nástroje, velká mohutná řeka, parafráze na Kočka leze dírou pes oknem, jsou tady zastávky -- předěly, teče vesnicí, zde zaznívá v pozadí polka, pak protéká lesem, je tady rej rusalek, pak majestátnost -- má znázorňovat české hrady a zámky, konec je smířlivý, klidný, vléváme se do Labe
    \item Šárka -- velice bojovná část, hudebně popisuje Dívčí válku, začíná imitací dustotu koní, příjezdem Ctiradovy družiny, ti se potom veselí radují, usínají a pak přichází bojovnice a Vlasta a muže pobijí
    \item Z českých luhů a hajů -- znázorňuje nám kraj kolem Mladé Boleslavi, jeho přírodu,kterou Smetana miloval - oslava krásy přírody Čech
    \item Tábor, Blaník -- obě používají píseň Ktož jsú boží bojovníci -- parafrázuje ji, upravuje ji, je základní myšlenkou a stavebním kamenem obou
  \end{enumerate}
\end{itemize}

\section{Antonín Dvořák}

\subsection{Život}
\begin{itemize}
  \item narodil se v Nelahozevsi u Kralup r. 1841 do rodiny řezníka, on měl být taky řezník, ale byl takový slabounký, odpuozvala ho krev atd., hrál na housle, violu, začal hrát na klavír, od mládí velice talentovaný, ale tatínek ho pořád nutil, než přišel pan učitel, že hoch má veliký talent, že ho má pustit kam chce, tatínek pana učitele poslechl a udělal dobře, nejprve má soukromého učitele, od 16 nastupuje na varhanickou školu v Praze, tam se učily i jiné nástroje, dejme tomu dnešní konzevatoř - zde studoval violu, housle a varhany, jakmile se zdokonalil zak se dostavuje na konkurs do Prozatimního divadla na hráče na violu, kam se dostává (zrovna tam diriguje Smetana btw), po nějaké době odchází a působí jako varhaník v různých krajských kostelích a začíná skládat, chce skládat, rozhoduje se kvůli tomu, že do Anglie (tam je nějaká soutěž) pošle svoji skladbu Moravské dvojzpěvy - vyrhrává a stává se znamým, pak do Anglie jede a tam se zpopularizuje, posléze tam získává čestný doktorát Cambridgeské university, přátelí se s Johannesem Brahmsem (máme jejich dopisy) - jeho přítel je hudební nakladatel v Berlině (Simrock), který vlastní velké nakladatelství a Brahms mu řeku o Dvořákovi - Simrock vydává všechno co Dvořak napsal - Dvořák za chvíli všude velice známý dokonce i doma, doma mu nabídli ředitelování na pražské konzervatoři, v r. 1891 nastupuje na místo ředitele na pražské konzervatoři, přijíždějí z New Yorku a dostává nabídku, aby dělal ředitele New Yorské konzervatoře, v Americe se stal slavný, odvděčil se jí symfonií Novosvětskou (lol), zde pobýval tři roky pak se vrací domů, zde znovu ředitelem pražské konzervatoře, kde je až ,,do důchodu”, umírá r. 1904 v Praze, v Praze často učil své žáky i doma, ty nejtalentovanějším se věnoval, byli dva - Josef Suk a Vítězslav Novák, Suka měl nejradši, zval ho k sobě domů kde potkal Dvořákovu dceru Otylku, která ale brzo umírá na srdce, zůstane jim syn Josef (taky lol), Suk dobrý houslista, po odchodu Dvořáka je ředitelem pražské konzervatoře Vítězslav Novák
\end{itemize}

\subsection{Dílo}
\begin{itemize}
  \item Moravské dvojzpěvy - pro 2 hlasy a 4-ruký klavír, Slovanské tance (ukázka), Biblické písně, koncertantní opera - oratorium Svatá Ludmila, skladba pro sbor a orchestr Stabat mater, skladbička Humoreska (ukázka), pro smyčcový kvartet Americký kvartet, ,,Ambrozové oblíbený” violoncellový koncert h-moll, píše symfonie, celkem devět, poslední - skladba inspirovaná Amerikou, má čtyři části (1. příjezd, nejistota, 2. (jméno Largo) nachomýtl se k pohřbu malé indiánky kde indiáni zpívali nějakou smutnou píseň jejíž melodii zde použil, 3. mohutná, majestátní, motivy velkoleposti, oslavy, 4. \uv{ta známá}
  itemtaky opery, nejvíce hrané Rusalka (příběh Malé mořské víly) a Čert a Káča
\end{itemize}

\part{Pozdní český romantismus (sentimentalismus)}
\begin{itemize}
  \item doba např. Debussyho, táhlé dlouhé melodie, rovné držené plochy, velice tklivá lyrika
  \item Josef Suk (1874 - 1935) -- ,,náš největší romantik a lyrik”, narodil se u Benešova a od malička byl hudebník, hrál na housle, klavír a varhany, už v 11 letech přichází na pražskou konzervatoř kde si ho všímá Antonín Dvořák, stane se jeho nejoblíbenějším žákem, už v 19 skládá své první opravdové klavírní a houslové skladby, Dvořák ho měl velice rád, zval ho v neděli na bábovku a čaj, tady poznal Dvořákovu dceru Otylku, tu si bere za ženu, tedy Dvořákovým zetěm, Otylka umírá brzy, kolem 30 let a Josef Suk zůstává sám se synem Josífkem, celý život (profesní) - přes 40 let, byl 2. houslistou v Českém kvartetu, psal symfonické básně, nejvíce se hraje Braga, potom Zrání, popř. Azrael, hodně se taky hraje skladba Pohádka - scénická hudba, předehra, ke div. hře Radúz a Mahulena, taky se hrají dvě skladby pro klavír, a to Píseň lásky a O matince, pak taky pochod V nový život a jeho symf. skladba Serenáda
  \item Vítězslav Novák (1870 - 1949) -- ,,náš největší impresionista”, narodil s v Kamenici nad Lipou, od mládí hrál na housle a klavír odchází studovat do Jindřichova Hradce na gymnázium, pak v Praze na právníka, pak se přihlašuje na pražskou konzervatoř, jeho učitelem taktéž Antonín Dvořák, po Dvořákovi je ředitelem konzervatoře, na konzervatoři vychovává slovenské skladatele 20. st. např Eugena Suchoňe, Jána Cikkera, avidní horolezec a turista, jezdil do Tater a do Alp, při jedné takové cestě do Tater ho zastihla z ničeho nic obrovská bouře a tady tento dojem, vjem, okamžik se snažil hudebně přenést do díla, symf. básně, V Tatrách, taky symf. báseň Májová, taky opery, ale ty se nehrají -- Zvíkovský rarášek, Dědův odkaz, taky se hrají jeho dvě suity (moderní, nikoliv barokní) - popisný obrázek něčeho, např. Slovácká suita nám hudebně popisuje neděli na Slovácku, má části tak, co lidé v tu neděli dělali (v kostele, mezi dětmi, zamilovaní, u muziky, v noci), taky Jihočeská suita - hudebně popisuje krásy jihočeské krajiny (4 části -- pastorale, snění, kdysi, epilog)

\end{itemize}

\section{Leoš Janáček}

\subsection{Život}
\begin{itemize}
  \item oblíbený a hraný ve velké míře ve světě dodnes (narozdíl od např. smetany, dvořáka), životně romantik, boučlivák, říkal co si myslel ale hudebně to posouvá dál, přes romantismus
  \item narozen 1854 na Hukvaldech (u Ostravy), tam jeho otec řiditelem školy, jeho otec tam pomáhá nějakému Pavlu Křížkovskému, který v Brně vystudoval a stal se knězem na Starém Brně, pak je navštívil a řekl, že kdyby něco potřebovali tak ať řeknou, byli to prostě chábři, nedlouho poté otec zemřel, vzala malého Leoše (10 let) a jela s ním do Brna, dala ho do církevní školy při Augustýnském klášteře, tam se mu moc nedařilo, tedy až na zpěv ve sboru Modráčci a začala ho zajímat hudba, Pavel Křížkovský se stal prvním učitelem hudby Leoše Janáčka, začal ho učit na klavír, na varhany, Janáček začal hrát i na jiné nástorje, Křižkovský na něm pracoval, viděl, že má talent, jak dokončil školu tak šel studovat na učitelský ústav (dnes fakulta architektury na Poříčí), tam byl řiditelem nějaký pan Schulz, brněnský němec, Schulz ho poprosil, aby učil na klavír jeho dceru Zdeničku, která se Janáčkovi zalíbila, ten s ,,začal jaksi s ní chodit”, po učitelském ústavu odjel do prahy a že začne studovat varhanickou školu v Praze, (sidenote -- Ambrozová chce školné, student loans, aby si to studenti museli zasloužit) kde napadl místního řiditele, že špatně skládá a tedy ho vyhodili, pak jede do Lipska a zkusí studovat varhanickou školu v Lipsku, kde se mu nelíbilo, zase se tam pohádal a řekl, že ho to tam ,,absolutním způsobem” nebaví, že mu ta škola nic nedává, vrátil se do Brna a řekl, že se z Brna, už nehne, že chce být v Brně, celou tuto dobu si psal se Zdeničkou a přes tatínkův ,,ne moc souhlas” si ji vzal v 28 letech, jí bylo asi 17, bydleli na Mendelově náměstí  řekl si, že když tak neuspěl na těch varhanických školách, že založí v Brně další, dobrou, tehdy se mu povedlo získat místo pro tu školu, je to roh ulice Smetanovy a Kounicovy, toto místo dostal od města, to teda šlo, žáků přibývalo a město se rozhodlo, že Mistrovi, který se snaží o kulturu v Brně, rozvíjí brněnskou kulturu, dělal pro brno hodně, že mu postaví domeček na ulici Smetanova (za ním je dnešní fakulta elektrotechnická), tam tedy stojí malý domoček, na svou dobu ale moderní -- byla tam voda, splachovací záchod, Janáček v této době prožil dvě velké tragédie -- měl dvě děti, syna Vladimíra a dceru Olgu, Vladimír umřel ve 2,5 letech na spálu, Olinka dostala tu spálu taky a přežila, byla ale celý zbytek života velice nemocná, i když byla nemocná rozhodla se pro cestu do Ruska (bydlel tam Janáčkův bratr) -- chtěla se zdokonalit v ruštině, ,,jenomže se tam roznemohla tak, že domů přijela už celá nemocná, ležela, umírala” a zachytil slova své umírající dcery do not, věta ,,Já nechci umřít, já chci žít.”, umírá tedy ve 22 letech, pak své ženě vyčítal, že se Vladimír nakazil, že mu nerozumí, atd., ,,umělci potřebují inspiraci” takže si šel najít jinou, tak byly různé zpěvačky, milenky, které procházely a odcházely z Janáčkova života (které mezitím žili u Zdeničky protože žádný rozvod), Janáček přišel do let (65) a myslel si, že život je za ním a jezdil strašně rád do Luhačovic za manžely Veselými -- Veselý zakladatel lázní, Calma Veselá -- operní zpěvačka, tam potkal Kamilu Stösslovou, té bylo 35, jezdila do Luhačovic a měla manžela a dva syny, bydlela v Písku, jezdila se léčit do Luhačovic, když nebyli spolu tak jí Janáček psal asi tři dopisy denně, pro Janáčka obrovskou múzou ve smyslu platonické lásky -- měla manžela kterého obdivovala, Janáček ji miloval a psal ji, manžel se tomu smál, ovšem Janáček jezdil do Písku ke Stösslovým jako skladatel znamý po celé zemi, je tedy přivedl do vyšší společnosti, Kamila mu někdy ani nedopisovala, nechápala hudbu, byla ,,free” (lol) nikdy nepsala o nějakém vztahu, nějaké vazbě -- dělalo jí dobře, že se o ni zajímá slavný skladatel, dělalo jí dobře že jezdili do Brna, Luhačovic, všude byli jako přátelé Janáčka jinak bráni a přijati, no, Janáček díky Kamile vytvořil svá nejlepší díla na konci svého života, jakožto zamilovaný byl obdařen a vytváří krásné skladby, Kamila se pro něj stala ideálem toho, po čem touží, Janáček udělal v 75 machra, pozval Kamilu aby přijela přijela se synem a manželem a pozval je všechny na prázdniny a chtěl jim všem udělat výlet, byla bouřka, on zmokl, a dostal zápal plic, horečku, odvezli ho ještě do nemocnice a tam umřel
\end{itemize}

\subsection{Dílo}
\begin{itemize}
  \item Janáček napsal 10 oper, hrají se po celém světě, libreta si psal sám, většinou podle nějakého románu, dělal to podle lidských nápěvků, které notově zapisoval
  \begin{itemize}
    \item opera Její pastorkyňa -- pastorkyňa je schovanka, podle románu Gabriely Preissové, příběh Jenúfy, která se zamiluje do Štefa, ten ale chce jinou; má ji rád Laco, kostelnička která se o ni stará zabije chlapečka, aby se mohla vdát, po zimě až roztají ledy tak se to ukáže, špatně to končí
    \item opera Příhody lišky Bystroušky -- podle romána Rudolfa Těsnohlídka, vypráví o příběhu lišky Bystroušky, která přijde o matku, domů si ji přivede hajný, pak uteče, najde krásného lišáka a mají spolu liščátka, revírník ale Bystroušku zastřelí, kolem ní pobíhají malá liščátka, koukají na mrtvou mámu a teď co budeme dělat, lišák to samé, nakonec odbíhají pryč a o liščátka se stará lišák, dostávají se přes to -- smrt je součástí života
    \item opera Věc Makropulos -- podle románu Karla Čapka, Emilie Makropulos, v době Rudolfa II., její otec je šarlatán, jeden lektvar vyzkouší na ni a ona žije dlouze, pak chce najít recept na elixír věčného života, nakonec pergamen hází do ohně -- \uv{život už mně všechno dal, všechno jsem zažila, život už mi nemá co dát, chci přijmout to, co je součástí života -- smrt}
    \item opera Výlety pana Broučka -- podle románu Svatopluka Čecha, příběh ve snu, pan Brouček se hrozně napil v hodpodě, spadl do nějakého sudu a usnul, zdál se mu sen, že je v 15. st., v době Jana Žižky a Jan Žižka se ho ptá kdo je, on mu říká, že je z jiného věku, jde do bitvy, je málem zabit a to ho probudí
    \item opera Káťa Kabanová, opera Šárka, opera Osud
  \end{itemize}
  \item symfonická díla
  \begin{itemize}
    \item Lašské tance
    \item Taras bulba -- z ruských dějin
    \item Sinfonietta pro Brno
  \end{itemize}
  \item skladby pro mužský sbor Marička Magdonová, Kantor Halbar -- zhudebněný Bezruč
  \item pro děti Říkadla, dechový sextet Mládí, ze souboru klavírních skladeb skladby Po travnatém chodníčku, V mlhách
  \item smyčcový kvartet Listy důverné (věnován Kamile, s ní se pak soudila Janáčkova žena o peníze z hraní skladby)
  \item Glagolská mše -- náměty z 10., 11. st., staroslověnština; Janáček žil v době kdy byla velká slovanská vzájemnost, jako Slovani měli držet pospolu, společný jazyk, původ -- měl bratra v Rusku, miloval Rusko, jeho dcera jela do Ruska -- vrací se ke všemu co je slovanské, tedy i staroslověnskou kulturu
\end{itemize}

\section{Bohuslav Martinů}
\subsection{Život}
\begin{itemize}
  \item narodil se v Poličce, v kostelní věži
  \item odjíždí studovat na konzervatoř do Prahy, kromě hudby měl rád divadlo a malířství, na kozervatoři studoval housle, věnoval se i kompozici
  \item je přijat do filharmonie a r. 1919 s filharmonií odjiždí koncertovat do Paříže
  \item vrací se domů a r. 1923 odjíždí studovat do Paříže, tam má svého učitele kompozice, potkává se s Igorem Stravinský, žije v Paříži, pak se vrácí domů a jede do Paříže zpátky, po r. 1939 prvně emigruje na jih Francie, potom lodí do USA, tam žije, dostává se mu uznání a domů se už nikdy nevrátí
  \item na konci života v 50. letech hodně času trávil v Evropě -- především ve Švýcarsku, r. 1958 tam prodělává operaci rakoviny žaludku a zůstává tam v sanatoriu, tam o rok později umírá
  \item svým dílem tesknil po domově, po Vysočině, ikdyž mu tam bylo velice dobře, nikdy v cizině nebyl úplně doma
\end{itemize}
\subsection{Dílo}
\begin{itemize}
  \item šest symfonií, kantáty Otevírání studánek, Dým bramborové natě, skladby pro klavír Louky, oratorium O Gilgamešovi, opery Mirandolina a Řecké pašije, skladby s prvky jazzu Poločas, Vřava
\end{itemize}

\part{Moderní klasická hudba}
\begin{itemize}
  \item romantismus byl melodický, tonální, tklivý
  \item najednou disharmonie, nemelodičnost, atonální
  \item 20. st. = bourá tonální systém
  \item 19. st. = dominovaly smyčce, málem jsme se rozplakali, melodika, dominovaly nástroje, které dávaly melancholii, sentimentální; 20. st. = bicí, kravál, dechy, žestě, sonické -- zvukové
  \item Igor Stravinskij -- žil celý život ve Francii, nebo v USA, začal tvořit jinak, balet Svěcení jara, pořád je to ale poměrně normální a poslouchatelné
  \item v 20. století je řada směrů, proudů, např. dodekafonie -- řada skladatelů vůbec nebyla inspirovaná, ale hudba se stává matematickou záležitostí, autor si zvolí řadu tzv. dodekafonickou řadu dvanácti tónů, jakoukoliv a s těmito tóny pracuji, různě je spojují, vytvářím melodii, prodlužuji, zkracuji, hudba neosobní, bez citu, nejznámější z nich Arnold Schönberg
  \item podobně modální hudba -- autor si vytváří své vlastní stupnice (oproti durové, mollové, atd.) a teď si řeknu, aha, já to budu různě spojovat, rytmicky upravovat, nějakým způsobem vytvořím skladbu, taky jakýsi necitlivý, matematický charakter, např. Olivier Messiaen
  \item proti těmto novátorům Stravinskij vystupuje
  \item dále taky neofolklorismus, neoklasicismus, neobaroko -- směry, které se snaží přinést zpět minulé umělecké slohy, skládám v jejich duchu, ale novobdobým pohledem z toho původního vycházím, je to něco i trochu nového, tedy neofolklorismus vycházím z folkloru, např. i Janáček, Stravinskij neoklasik např. i Bohuslav Martinů
  \item modernější je témbrová hudba -- hudba pracuje s barvou zvuku a dynamikou, s hudebními nástroji se pracuje zcela netradičně, autoři požadují hru na hranici tonového rozsahu nástrojů, používá se glissando, flažolety (co to je?), hraní dřevěnou částí smyčce, do klavíru se hazí tyče a živé ryby, z orchestru vycházejí zvuky které jsme dosud ještě neslyšeli, např. klastr -- nahuštění víc tonů do sebe, zní jako šum, např. Kryštof Penderecki -- skladby Sedm bran jeruzalémských, De natura sonoris
  \item potom taky elektronická klasická hudba -- např. E. Varése
  \item téže minimalismus -- např. S. Reich
\end{itemize}

\end{document}
