\documentclass{article}
\usepackage{fullpage}
\usepackage[czech]{babel}
\usepackage{amsfonts}

\title{\vspace{-2cm}Literatura\vspace{-1.7cm}}
\date{}
\author{}

\begin{document}
\maketitle

\part{3. fáze Národního obrození}
\begin{itemize}
  \item 1830 - 1848
  \item převládá romantismus; čeština ustálena, i odborná a národní literatura
  \item ve 30. letech umí 90 \% obyvatelstva číst a psát, myšlenky NO se rozšířily všude
  \item 1830 povstání Poláků, pomoc i z Česka -- starší stojí za slovanským Ruskem, mladší nesouhlasí s absolutistickým carem
  \item rozvíjení společenského života, hrdost na české hrady, přírodu
  \item kancléř Metternich, \uv{Metternichův absolutismus} $\rightarrow$ Češi nechtějí vybočovat a provokovat, obrozenci tvoří kulturu a o politiku či práva se nezabývají
  \item biedermeier -- životní styl měšťanstva, které se nebouří, pohodlně žije, organizují spol. akce
  \item žánry
  \begin{itemize}
    \item povídky, kalendářové povídky, básnické povídky, novely, balady
    \item dramatické báchorky -- divadelní hra se soudobým dějem, nadpřirozeno, hudba / zpěvy, polepšení hl. hrdiny
    \item obrazy = až publicisticky popisují dění a krajinu třeba na venkově
  \end{itemize}
  \item český romantismus jiný než světový (až na Máchu) –- oživování ÚLS, české kultury celkově, vymaňování se němčině, německé kultuře
\end{itemize}

\section{Josef Kajetán Tyl}
\begin{itemize}
  \item vzor Klicpera -- divadlo, 1848 členem říšského sněmu
  \item činnost žurnalistická -- časopis Květy (výchovná funkce, hodně lidí se tímto naučilo číst)
  \item činnost literární -- povídky
  \begin{itemize}
    \item vlastenecké ze současnosti -- \textbf{Poslední Čech} (Borovský kritizuje jako příliš vlastenecké), \textbf{Rozervanec} (obraz K. H. Máchy)
    \item historické -- Dekret Kutnohorský
  \end{itemize}
  \item činnost dramatická: \textbf{Kajetánské divadlo}, Stavovské divadlo
  \begin{itemize}
    \item soudobé hry -- \textbf{Fidlovačka}, aneb žádný hněv a žádná rvačka (jako Romeo a Julie, ale končí dobře) -- píseň Kde domov můj? (státní hymna, hudba František Škroup), \textbf{Paličova dcera} (zapálí chalupu, Rozárka vezme vinu na sebe)
    \item historické hry -- aktualizované = snažil se poukázat na soudobý problém; \textbf{Kutnohorští kováři} (stěžují si na nízké mzdy -- zatčeni -- poprava, 1 uteče a informuje krále, pozdě); \textbf{Jan Hus} (Hus s myšlenkami 1848)
    \item dramatické báchorky -- \textbf{Strakonický dudák} (chce si vzít Dorotku, vydělává hrou na dudy, válka -- cizina, poražen, Dorotka ho jde hledat; má možnost velkého bohatství a zapomene, proč se vlastně vydal do ciziny)
  \end{itemize}
\end{itemize}

\section{Karel Hynek Mácha}
\begin{itemize}
  \item 1810 -- 1836
  \item v tomto období u nás jediný romantik, neorientuje se na tradice nebo lidové vlastenectví
  \item narozen v Praze v chudé rodině, zemřel v Litoměřicích, na gymnáziu žák Jungmanna
  \item poté VŠ v Praze (práva), do advokátní kanceláře v Litoměřicích; v divadle J. K. Tyla
  \item miloval české hrady a zříceniny
  \item milenka Marinka Štichová, po smrti tragická tvorba
  \item druhá milenka Eleonora Šomková, zakazuje jí hrát v divadle, žárlí; otěhotní, syn Ludvík
  \item pomáhal hasit požár, napil se infikované vody, zemřel na dehydrataci kvůli průjmu
  \item pohřeb v den, kdy měl mít svatbu. pomník na Vyšehradě
  \item poezie -- první básně v němčině, později česky, \textbf{Máj} (jediná kniha, který vyšla za jeho života, špatně přijat jako málo vlastenecký, později však přehodnocena jako přelomová)
  \item próza -- bývá epická, ale propojena s lyrickými prvky; lyrizovaná próza \textbf{Obrazy ze života mého} (2 části: večer na Bezdězu, Marinka – příběh o lásce k dívce, ideál krásy, vykoná pouť do Krkonoš, při návratu je Márinka mrtvá kvůli nemoci); básnická próza \textbf{Pouť krkonošská} (až horror, na vrcholu Sněžky je chrám s mrtvými mnichy, kteří vstávají 1 za rok, ale nemohou vyjít z kláštera); povídka \textbf{Cikáni} (částečně připomíná Máj, postava z okraje spol., starý a mladý cikán mají milenky, jsou opuštěni, cikán trestá toho, co je svedl, je to však otec mladého cikána, starý cikán je za jeho vraždu popraven); pokus o román \textbf{Kat} (jednotlivé díly podle hradů, jen 1. díl místo 4, propojeno s historickou tematikou, děj za Václava IV., kat je levobočkem posledního Přemyslovce, inteligentní, Milada ho miluje, ale musel popravit jejího otce, poté umírá i Milada)
\end{itemize}

\section{Karel Jaromír Erben}
\begin{itemize}
  \item 1811 -- 1870
  \item hudební talent, učil hudbu, r. 1850 sekretář a archivář Národního muzea, poté archivář města Prahy -– dostal se k RKZ, ÚLS
  \item inspiruje se a sbírá ÚLS -- balady, povídky, říkadla, písně, zapisuje je
  \item myšlenky Herdera -- v ÚLS je národní duch, na základě toho psal autorské pohádky -- např. \textbf{Tři zlaté vlasy děda Vševěda} a \textbf{Dlouhý, Široký a Bystrozraký} -- zlidověly
  \item sbírka básní \textbf{Kytice}
  \begin{itemize}
    \item oblíbená již ve své době
    \item inspirace lidovovou epikou -- prvky báje, pověsti, pohádky, proroctví, legendy
    \item vyzdvihuje tradiční křesťanské hodnoty (čestnost, morálku), ale čerpá z pohanských bájí a pověr (vodník, polednice)
    \item silný motiv mateřství, mateřské lásky
    \item až horrorové motivy
    \item inspirace dodnes
  \end{itemize}
\end{itemize}

\part{4. fáze Národního obrození}
\begin{itemize}
  \item 1850 -- 1860
  \item literatura více propojená se životem
  \item převládá realismus, stále ale výrazný vliv romantismu
  \item Božena Němcová, Karel Havlíček Borovský
  \item období \uv{Bachova absolutismu}
\end{itemize}

\section{Božena Němcová}
\begin{itemize}
  \item její narození je záhada, možná dcerou šlechtičny, rodné jméno Barbora Panklová
  \item východní Čechy (Ratibořice), prostředí Babičky, vyrůstala tam
  \item průkopnice ženské prózy českého jazyka
  \item v 17 provdána za Josefa Němce (úředník, o dost starší, sňatek z rozumu), ten ji přivedl do vlasteneckých kruhů
  \item propojuje realismus a romantismus, romantické postavy, v duchu českého romantismu konec, idyla (pod vlivem ÚLS), realismus -- soc. vrstvy, venkov
  \item píše \textbf{obrazy} -- propojení dokumentaristiky s příběhy a situacemi ze života lidí, hodně reportážních prvků, důrazná pojmenování
  \item \textbf{Divá Bára} (povídka) -- venkovská dívka, neustále u přírody, nevěří pověrám, vše si vyzkouší, je jiná než ostatní děti, nemá moc kamarádů, jen Elišku
  \item \textbf{Pan učitel} (povídka) -- částečně autobiografický hold jejímu učiteli, dovedl ji ke knihám
  \item \textbf{Karla (povídka)} -- Markyta se stane chudou vdovou, má syna, mnoho problémů; měla strach, že syna Karla odvedou na vojnu, vychovávalo ho jeho děvče Karla
  \item \textbf{Babička} (obsáhlá povídka) -- Ratibořické údolí na Náchodsku v 30. letech 19. st., Staré bělidlo; začala psát, když jí zemřel syn -- vzpomínky na dětství; ne vše je pravda, realistické v rozvrstvení společnosti, mnoho postav má jen dobré rysy, babička – Magdaléna Novotná, Proškovi – Panklovi, babička -- vlastenkyně, výchovná fce, vlastenecké příhody, idealizovaná; děti Barunka (BN, nejučenější, naslouchá), Vilém, Jan, Adélka; pak Viktorka (čistě romantická postava z okraje společnosti, zamiluje se do vojáka, jde s ním do světa, vrátí se zblázněná, žije v jeskyni, porodí dítě, hodí ho do splavu); panské postavy -- sluhové negativní, kněžna pozitivní, oblíbí si babičku; příběh končí úmrtím babičky, chudé postavy mají dobré vlastnosti (až na maminku, která se chová pansky); kompozice = jako jeden cyklus roku, ve skutečnosti uběhne 12 let + vložený epizodní příběh o Viktorce
\end{itemize}

\section{Karel Havlíček Borovský}
\begin{itemize}
  \item 1821 -- 1856
  \item narodil se v Borové u Německého (dnes Havlíčkova) brodu, příjmení Havlíček, Borovský přídomek
  \item kritik, např. Tyla (povídku Poslední Čech), básník, hodně satirický, taky překládal
  \item v jeho rodině se mluvilo německy, studoval ne německém gymnáziu, ale čeština se mu líbila, takže používla češtinu častěji než němčinu
  \item určitou dobu působil jako vychovatel v Moskvě, toto inspirace pro dílo Obrazy z Rus, kde je velice kritický k podmínká carského Ruska
  \item napsal zejména tři velké satirické skladby, které zůstaly částečně nedokončené
  \begin{itemize}
    \item Král Lávra -- vychází ze staré pohádky, báje o irském králi s oslíma
    ušima, stejný motiv už se objevuje v řeckých pověstech o králi Midasovi -- říká se, že existuje dobrý, hodný král, ale má jeden zvláštnost -- jednou za rok se nechává stříhat a svého kadeřníka nechá vždy popravit, jednou měl jít stříhat syn nějaké chudé vdovy, ona šla prosit krále, aby jejího syna nezabíjel, on ho tedy ušetřil, ale musel mu slíbit, že po stříhání o tom nikomu nebude povídat, on to nevydržel a svěřil se vrbě, někdy později nějaký muzikant si z té vrby udělal kolíček do basy a ta basa začala hrát o tom, co jí kadeřník svěřil, že král má oslí uši, které schovává pod dlouhými vlasy a korunou, ale že se jednou za rok nechává stříhat a proto nechává popravovat ty kadeřníky, aby to nikomu nevyzradili -- kritika cenzury, absolutismu a lidí, co si to nechají líbit
    \item Křest svatého Vladimíra -- vychází z Nestorova letopisu z 12. st., ten pojednával o přijetí křesťanství na Kyjevské Rusi, dílo je vtipné a velmi satirické, ale nedokončené, jde o pohanského boha Peruna a o tom, jak odmítl požadavek vládce a ten ho tedy nechal popravit, Kyjevská Rus se ocitne bez boha a vyhlásí konkurz na boha kam přicházejí různí bohové se o tuto pozici ucházet, vyhrává bůh křesťanů
    \item Tyrolské elegie -- zachycuje okamžik svého zatčení (pro své názory, vyjadřování nesouhlasu s cenzurou, s národnostním útlakem zatčen a deportován) a o tom co se odehrávalo cestou do Brixenu -- po zatčení měl asi jen dvě hodiny se sbalit, rozloučit se s dcerou, zajímavá událost, že koně se splašily, všichni kromě Havlíčka vyskočili, on koně zastavil a potom počkal až potlučení policisté a kočí doběhnou a pokračují v cestě -- snaží se poukázat asi i na neschopnost policie v Rakousku
  \end{itemize}
  \item dále také psal celý život epigramy (lyrický útvar zachycující různé situace lidského života s výraznou satirou, kritika nešvaru ve společnosti, s výraznou pointou), uspořádal je do několika souborů v pěti oddílech: církvi, králi, vlasti, múzám, světu
\end{itemize}

\part{Realismus ve světové literatuře}
\begin{itemize}
  \item 2. pol. 19. st.
  \item zachycuje věrně skutečnost, jde o skutečné zachycení prostředí postav, zachycení nečeho, co by se mohlo opravdu stát
  \item autoři hodně zkoumali příběhy které se opravdu staly, na základě toho vymýšleli příběhy vlastní, autoři nejdříve spoustu času studovali prostředí, ta místa o kterých chtějí psát, dobře promýšleli svět okolo postav a dopodrobna promýšleli jednotlivé postavy, tak aby to všechno bylo reálné, a to včetně např. zachycení jazyku postav uměrně jejim soc. postavení -- dialekty, argot, apod.
  \item vypravěč většinou nad příběhem a jazyk vypravěče bývá spisovný
  \item postavy bývají typizované, objevuje se typizace postav -- postava je typickým představitelem své dané soc. vrstvy, svého prostředí
  \item ale nechceme mít dvě stejné postavy, postavy mají samozřejmě i nějakou individuální složku, i přesto bývají typizované
  \item autoři se nevyhýbají (až je vyhledávají) prostředím, která nejsou příjemná -- o prostituci, o soc. slabých vrstvách, ale i o korupci, apod.
  \item svými díly často vyzdvihují různé společenské problémy a otevřeně kritizují jejich viníka, ať je jakýkoliv -- kritický realismus
  \item odnoží realismu je i naturalismus -- vyhrocený realismus, který říká, že podstata jednotlivce je víceméně dána a je neměnná, člověka tedy ovlivňují jen dědičnost a výchova, resp. vliv prostředí, úplně tedy pomíjejí, že by lidská vůle mohla být natolik silná, aby toto dokázala změnit (dnes, když se řekne, že je něco naturalistické tak je to přehnaně detailní a tyto detaily jsou až nechutné)
\end{itemize}

\section{Anglie}
AAAAAAA Dickens

\section{Francie}
AAAAAAA Balzac a Flaubert, Zola

\subsection{Émile Zola}
\begin{itemize}
  \item soubor románů Rougor-Macqartové -- nestihl dodělat, podobné Honoré de Balzacovi, některé postavy se taky objevují ve více příbězích
  \begin{itemize}
    \item součástí román Zabiják, román Nana
  \end{itemize}
\end{itemize}

\subsection{Guy de Maupassant}
\begin{itemize}
  \item narozen do zajištěné rodiny, matka šlechtična, takže taky šlechtic
  \item byl na nějaké církevní škole, tam se mu nelíbilo a němel proto rád církev, pak studoval na lyceu a pak práva
  \item prvně nechtěl psát pod svým pravým jménem
  \item měl syfilis, ke konci života měl deprese, halucinace a pokusil se o sebevraždu, dožil v léčebně
  \item romány a povídky, významný román Miláček (o šarmantím muži šplhajícím mezi spol. vrstvami), kvanta povídek, např. Kulička, Muška, Dítě
\end{itemize}

\section{Rusko}
AAAAAAAAA

\section{Henry/ik}

\subsection{Henryk Sienkiewicz}
\begin{itemize}
  \item 1846-1916
  \item Polák, šlechtic, ale chudý
  \item začal studovat medicínu, pak přešel na filosofii
  \item věnoval se literatuře, především historické próze -- studoval, poté i psal
  \item cestoval, 3 roky strávil v USA což pro něj bylo velmi inspirativní
  \item psal především pro dospělé, napsal ale i jeden dobrodružný román pro mladší Pouští a pralesem -- dobrodružný příběh dvou dětí, v době stavby Suezského průplavu, 14letý Polák Staš, 8letá Angličanka Nela, jejich otcové pracují na té stavbě, byli uneseni -- pokus za ně získat výměnou nějakého vězně, oni se ale díky Stašově důvtipu osvobodit, ale ocitnou se v divočině (poušť, prales) a snaží se nějak přežít, nakonec se zachrání; i zfilmováno
  \item historická próza Křižáci -- období bojů Poláku proti Německým rytířům, motiv národního uvědomování
  \item románová trilogie Ohněm a mečem, Potopa, Pan Volodyjovski -- historie 17. st., boje mezi Poláky a ukrajinskými kozáky
  \item stěžejní román Quo vadis? -- děj ve starověkém Římě za vlády Nerona, zpracovává téma nástupu křesťanství
\end{itemize}

\subsection{Henrik Ibsen}
\begin{itemize}
  \item 1828-1906
  \item Nor, dramatik
  \item většinou dodržoval jednotu času, místa a děje, nicméně konfliktry mezi postavami odhaluje retrospektivně a minulost postupně vyplývá na povrch (skrz dialogy) a divákovi se tak skládá mozaika toho, jako minulost ovlivňovala životy jednotlivých postav
  \item jeho psychologie postav je docela výrazná, proniká hluboko do nitra postav, často ukazuje, že realita bývá jiná, než vypadá na první pohled
  \item neváhá ve svých dílech zničit to, co vypadá jako pozitivní, dobré -- až smrt, nebo postavy dojdou k poznání toho, že co si mysleli co je ideální opravdu není
  \item drama Divoká kachna -- hra je protknuta symboly, rodina chová na půdě divokou kachnu, kterou zachránili, přenesli ji ale do umělého prostředí, ta kachna tam žije, ale nepřirozený život -- jakýsi symbol spoutanosti, nesvobody člověka, symbolem přetvářky (dělá se, že je to ideální prostředí, ale není)
  úkoly: přečti to, pochop to, zapamatuj si to
  \item drama Nora (popř. Domov loutek, Domeček pro panenky) -- žena žije v eštastném manželství, mají děti, její manžel má nějakou chronickou nemoc, měl by se léčit, ale oddalují to, ona mu přispívá a výrábí tedy vánoční ozdoby, ale on by měl jet teď, takže žena získává peníze podfukem (nějaký bankovní podvod), věří je získá a brzo vrátí, dřív než k tomu dojde se na to ale zjistí, manžel ji označí za nesvéprávnou, nedovolí jí vidět děti atd., \uv{ona si uvědomí, že není součást jeho života, ale je jen nějaký doplněk}, rozhodne se, že takhle ne a odchází od manžela, tedy i definitivně od dětí
\end{itemize}

\part{Česká literatura 2. pol. 19. st.}
\begin{itemize}
  \item prosazuje se realismus, stále ale i trošku romantické, liší se autor od autora
  \item revoluční rok 1848, pořážka revoluce, potom období 50. let., bachovského absolutismu, pro kulturu a literaturu to znamenalo docela přísnou cenzuru, bylo třeba si hlídat aby v textech nebylo něco protirakouského
  \item 60. léta určité uvolnění politiky a společnosti, méně přísná cenzura, Bach končí, 1861 vydaná ústava, 1867 rakousko-uherský dualismus, tedy Uhersko má daleko větší samostatnost, Češi ostrouhali
\end{itemize}

\section{Májovci}
\begin{itemize}
  \item v roce 1858 vychází almanach (= sborník děl různých autorů) Máj, o něj se zasloužili především Jan Neruda, Vítězslav Hálek a Josef Václav Frič, přispěli do něj ale i další, např. Karolína Světlá -- Máj, protože se inspirovali Máchou, vzdávali mu tímto hold
  \item májovci stavěli na Máchovi, ale i Erbenovi, překládali i zahraniční práce, např. Poea, Heineho, Huga
  \item máme realismus, takže se zabývali i různými soudobými sociálními problíémy -- postavení žen, podmínky dělníků
  \item necítili potřebu se odkazovat na historii, ale hledět do budoucnosti, celkově poměrně novátorští
\end{itemize}

\subsection{Vítězslav Hálek}
\begin{itemize}
  \item 1835 - 1874
  \item především básník, ale i prozaik, dramatik
  \item ve své době známější než Neruda, dnes pro nás důležitější spíš Neruda
  \item ve své poezii psal témata, která byla lehčí, jednodušší, líbivější, jeho lyrika byla milostná a přírodní (zamilovaný, dlouho se snažil o dívku), líbivé čtivé verše, byl oblíbený
  \item sbírky básní Večerní písně, V přírodě
  \item dále psal povídky, např. Muzikanstká Liduška -- dívka, která se má vdát proti své vůli, což se stane a ona se z toho zblázní
  \item jeho hra Král Vukašín -- ze srbských dějin, hrál se jako premierové představení pči otevření Prozatimního divadla r. 1862
  \item AAAAAAA
\end{itemize}

\subsection{Jan Neruda}
\begin{itemize}
  \item formuluj hlavní myšlenku, do jakého žánru přiřadíš a proč -- chytrý, co bojuje s hloupostí se zblázní, epigram
  \item na jaké kompozici je báseŇ založena, jakou syboliku báseň obsahuje -- paralelismus, kytky rostou podle emocí, povaha
  \item AAAAAAA
\end{itemize}

\subsection{Karolína Světlá}
\begin{itemize}
  \item 1830 - 1899
  \item rodným jménem Johanna Rottová, do vlasteneckých kruhů ji přivedl manžel, až už vdaná zamilovaná s Nerudou
  \item nejdříve psala povídky z pražského prostředí, např. Černý petříček, pak psala romány (už pod pseudonymem -- Světlá pod Ještědem) -- říká se jim ještědské romány (odehrávají se v té oblasti)
  \item jeden z těchto románů je Frantina -- žena se zamiluje, muž ji miluje, vezmou se, ona je starostka a řeší mezitím loupeže, přijde na to že za nimi je její snoubenec, takže ho zabije (protože by ho nemohla vydat, ale nemohla by s ním žít)
  \item další je Kříž u potoka -- Eva se provdá za muže, který je z rodu, který má být prokletý, že žádné manželství není šťastné a přestože jejich manželství začíná dobře on začne pít, ji bít, začne se sbližovat s jeho bratrem, on jí nabízí aby s ní odešel a začali jinde, ona to neudělá, protože manželství je posvátné a tím zlomí tu kletbu, její manžel se vzpamatuje
  \item tyto dva romány podobné -- moralita ženských hrdinek, jednodušší psychologie postav, prvky tajmena, dobrodružství, ženské hrdinky mají dilema -- aby byly šťastné, musí porušit morální pravidla, ale ony to nikdy neudělají (reference na její život -- Neruda)
\end{itemize}

\subsection{Jakub Arbes}
\begin{itemize}
  \item 1840 - 1914
  \item především známý pro svá romaneta -- novely, které se točí okolo nějaké záhady, která je nakonec vyřešena rozumem a vědou
  \item např. Etiopská lilie -- udělá matematický objev, pak zalisuje kytku, kterou mu poslal kamarád, další den nemůže nikde najít, po dlouhé době ho najde jako papír, do kterého zalisoval kytku
  \item nebo Newtonův mozek -- příběh dvou kamarádů, kteří od dětství milovali eskamotérství (hokus pokus), jeden se dá na vojáka, druhý dělá něco jiného, ten první se zúčastní bitvy u Hradce Králové, druhému přijde zpráva, že padl, má ho dojet identifikovat, ale nachází ho živého, ten první kamarád mu vypráví, že upadl do bezvědomí, že na něm udělali pokus a vyměnili mu mozek za Newtonův mozek, vymyslí stroj času a zbytek spoilovat nebudeme, celé vlastně doopravdy společnost moment
  \item dále Ďábel na skřipci, Svatý Xaverius
\end{itemize}

\section{Ruchovci}
\begin{itemize}
  \item r. 1868 skupina spisovatelů vydává almanach Ruch ku příležitosti položení základních kamenů Národního divadla
  \item přirodní lyrika, milostná lyrika, ale i milostné příběhy, dále žalozpěvy/elegie s vlasteneckým podtextem
  \item tento almanach inicioval Josef Václav Sládek (my ho ale považujeme za lumírovce)
  \item ti, kteří s tímto almanachem souzněli založili tzv. školu národní -- důraz na vlastenectví, umění má sloužit aktuální společenské potřebě (v tomto případě nějčastějí vlastenectví), jestliže má umění takto sloužit, podle toho mají být tvořeny i postavy, které by měly být pro čtenáře určitým vzorem postavy typizované
  \item sem patří Svatopluk Čech, dále Elišká Krásnohorská -- novinářka, spisovatelka, psala libreta pro opery, taky Karel Václav Rais, Alois Jirásek
\end{itemize}

\subsection{Svatopluk Čech}
\begin{itemize}
  \item - 1908
  \item próza i poezie, lyrika i epika
  \item hodně a rád cestoval, procestoval velkou část Evropy
  \item z poezie a lyriky
  \begin{itemize}
    \item sbírka Písně otroka -- rozsáhlá skladba, ve které převažuje lyrika s drobnou epikou, vychází z Afriky, děj je v Africe, je to na plantážích, které patří bělochům a otroci zpívají písně, ve kterých vyjadřují smutek nad svou situací a touhu po svobodě, patrná alegorie pro českou touhu po svobodě
  \end{itemize}
  \item z epiky (prózy i poezie)
  \begin{itemize}
    \item satirický epos Hanuman -- v asijské mytologii opičí král, tady tedy opičí národ, který se snaží napodobit lidi, vlastnosti, které z toho vyházejí a celkový národ je poměrně ztřeštěný, autor karikuje českého měšťáka, začínají se pak oblékat, přestávají skákat po stromech, atd., celkově kritika českého člověka, který se \uv{rád} opičí po zahraničních, což se mu nelíbí
    \item cyklus povídek Ve stínu lípy -- epika s drobnými lyrickými pasážemi, příběhy, které si vykládají vesničané ve stínu lípy večer, sejdouce se na lavičce (připomína trochu Dekameron), příběhy satirické, s nadsázkou, často humorné
  \end{itemize}
  \item z prózy a epiky
  \begin{itemize}
    \item tzv. broučkiády -- romány Pravý výlet pana Broučka do Měsíce a Nový epochální výlet pana Broučka, tentokrát do XV. století (viz Janáček) -- Matěj Brouček typický český měšťák, takový \uv{obraz českého pivního vlastence}, rád si posedí s kamarády u piva a to je také vysvětlení jeho příběhu -- Brouček se \uv{zkáruje} do němoty, tyto příběhy jsou jeho sny, vize a představy, jeden výlet je mezi Měsíčnany, potkává je zde, mluví velmi uhkazeně, jemně, Čech si tady trochu děla srandu z lumírovců, druhý výlet je do dob Husitů, mezi těmi Husity mýá plnou pusu řečí o své udatnosti, jak bojovat atd., ale když se má opravdu bojovat tak se jako zbabělec schovává
  \end{itemize}
\end{itemize}

\section{Lumírovci}
\begin{itemize}
  \item podle časopisu Lumír
  \item časově asi stejně jako Ruch
  \item tvořili tzv. školu kosmpolitní -- hlavně estetika, autor má právo na vyjadřování, které si sám zvolí, má právo na tvorbu takových postav, které si zvolí, ideje v díle nemusí mít nějaký vyšší cíl
  \item představiteli Josef Václav Sládek, Julius Zeyer a Jaroslav Vrchlický
  \item umění pro umění (parnasismus -- Parnas, kde sídlí múzy) -- nesmírně důležitá forma při překladu
  \item překládali, snažili si převést i formu jazyka
  \item na přelomu 70./80. let se skupiny čím dál víc dostávaly do sporů
  \item obě dvě skupiny patetické, sentimentální, novoromantické, psaly např. balady, romance
\end{itemize}

\subsection{Josef Václav Sládek}
\begin{itemize}
  \item 1845 - 1912
  \item především básník, někdy se mu říkalo \uv{básník vesnice} -- volil venkovských motivů
  \item studoval přírodní vědy, pak se rozhodl vydat na cestu do USA asi na dva roky, tady se naučil výborně anglicky -- překládal (např. Shakespearea), získal spoustu zkušeností, obdivoval jejich fungování společnosti, formu vlády, ale odsuzoval zacházení s původními obyvateli
  \item Sládkova metoda -- snažil se překládat tak, aby seděla i forma, když to ale nešlo dával přednost obsahu
  \item sbírka Jiskry na moři -- sbírku psal pod vlivem cesty do USA, daleko více jsou v ní smutné tóny, protože mu umřela manželka při porodu dítěte, o dítě taky přišel (pak se oženil znovu, ale později)
  \item sbírky Selské písně a České znělky (většinou vydávány spolu) -- Selské písně ukázka venkovské poezie, oslavy venkova, básně inspirované lidovou tvorbou a podle Sládka je drobný sedlák nositelem dobrých vlastností českého národa, idealizuje venkov, České znělky příležitostné básně, které reagují na nějakou aktuální situaci, víra v dobrou budoucnost, demokracii, svobodu
  \item taky autor poezie pro děti
\end{itemize}

\subsection{Julius Zeyer}
\begin{itemize}
  \item finančně dobře zajištěn, mohl si dovolit psát a cestovat, cestoval hodně
  \item zájem o mytologii, staré mýty a pověsty se ve své literatuře snažil oživovat, mytologie různé ze světa
  \item v jeho tvorbě ideál, ke kterému je snaha směřovat -- novoromantismus
  \item čerpá motivy ze středověku, které vnímá jako dobré období, obdivoval sílu přátelství, sílu rytířských ctností
  \item Román o věrném přátelství Amise a Amila -- dlouhé, složité, pro Literákovou kontroverzní -- jeden z nich zachrání toho druhého před valkýrou, která byla jeho ženou, musí ho zachránit tím, že obětuje krev své děti a on to udělá, pak ale i děti žijí
  \item Radúz a Mahulena -- scénická pohádka, Mahulena byla prokletá ženou, která chtěla Radúze, on na ni zapomněl, ona se promění ve strom, ale Radúz je ke stromu přitahován, jeho matka chce strom porazit, ...
\end{itemize}

\subsection{Jaroslav Vrchlický}
\begin{itemize}
  \item ve tvorbě se projevují snad veškeré lidské city a vášně
  \item náměty čerpal i z mytologie i z cizích zemí, dokonce i z fantazie
  \item velmi důležitá forma, často překládal, jeho metoda byla, že vždy musí zachovat formu, ne nutně obsah
  \item doteď živé jeho básně milostné, používají je oddávající při oddávání
  \item poezie o významu velkých dějinných okamžiků Zlomky epopeje -- víc basnických sbírek dohromady, není to úplně chronologické, i mytologie, cizí i naše historie, básně lyrické, lyricko-epické až epické
  \item Noc na Karlštejně
\end{itemize}

\part{Realismus a naturalismus v české literatuře}
\begin{itemize}
  \item neco pomoc
\end{itemize}

\section{Historická tématika}
\begin{itemize}
  \item populární v dobách NO, u ruchovců i lumírovců (Zeyer), v romantismu poměrně přikrášlovaná
  \item v době 80. let 19. st. je historická tématika podrobena většímu výzkumu, ověřování faktů a díla s historickou tématikou se dají považovat za realistická
\end{itemize}

\subsection{Alois Jirásek}
\begin{itemize}
  \item 1851 - 1930
  \item vystudoval dějiny na FFUK a byl i učitelem dějepisu
  \item zajímala ho období tři -- husitské, pobělohorské a období NO
  \item zdatně popisoval velké kolektivní scény např. bitvy, velice živě reálně až naze a surově
  \item v dílech se snažil zachytit národního ducha a někdy tomu děj podléhal, měl jasno, která strana je \uv{ta správná}
  \item postavy zploštělé, stojí na svých stranách, zachovává většinou typizaci postav
  \item tvorba
  \begin{itemize}
    \item Filosofská historie -- kratší prósa, děj kolem roku 1848, o nadšení studentů filosofie pro ideály revoluce
    \item Mezi proudy, Proti všem, Bratrstvo -- trilogie románů s husitskou tématikou, jednotlivě počátek, vrchol a doznívání husitství, platí tady obecné charakteristiky jeho děl
    \item Temno -- doba pobělohorská, děj se odehrává zhruba 100 let po Bílé hoře, po rekatolisaci, příběh lásky prostestantky a katolíka, \uv{je to tak nějak dané}
    \item Psohlavci -- menší prósa (kratší román (lol)), o Chodech, kteří strážili česko-bavorské hranice, o Janu Sladkém Kozinovi, dostane se do sporu se šlechtou, která Chodům chtěla odebrat historická práva které měli, v čele šlechticů Lomikar, Jan Sladký nakonec odsouzen k trestu smrti, těsně před provedením je ta známá věta
    \item F. L. Věk -- pětidílný román z období NO, život a osudy Františka Ladislava, potkává se i s historickými postavami
    \item Staré pověsti české -- pověsti od praotce Čecha až...
    \item dramatické báchorky Jan Hus, Jan Žižka, Jan Roháč, Lucerna
  \end{itemize}
\end{itemize}

\subsection{Zikmund Winter}
\begin{itemize}
  \item též učitel dějepisu
  \item jiný přístup než Jirásek, zasádními dějovými kulisami ne historické události, ale tyto události zásadně ovlivňují osudy a rozhodnutí jednotlivých postav, které těmito událostmi procházejí
  \item snažil se studovat jak lidé ve svých dobách opravdu žili -- jak vypadaly jejich domácnosti, jejich oblečení, apod.
  \item dílo
  \begin{itemize}
    \item Mistr Campanus -- (asi) román, boj Campana (Campanuse?) o universitu, aby se nedostala do spárů jesuitů, v průběhu děje se setkáváme i s popravou 27 českých pánů
    \item sbírka povídek Pražské obrázky, jedna z nich Rozina sebranec -- nalezena na prahu kostela, umístil ji na vychování do jedné rodiny DÚ Č302 přečíst ukázku a zjistit jak to je, soustřeď se na vypravěčskou rovinu a jak pracuje s formami řeči (přímá, nepřímá, nevlastní, polopřímá)
  \end{itemize}
\end{itemize}

\section{Venkovská tématika}
\begin{itemize}
  \item \uv{teď už ten venkov nebude taková idylka}
  \item řeší se společenské poměry na vesnici, např. vnímání a dodržování tradic, víry a když je něco jiné, nové tak je to špatně; nebo majetkové poměry -- velké sociální rozdíly na vesnici ať už majetkové (velkostatkáři, sedláci x bezzemci) nebo hierarchální (postavení žen), nebo např. domluvené sňatky, nemanželské děti
\end{itemize}

\subsection{Karel Václav Rais}
\begin{itemize}
  \item 1859 - 1926
  \item velmi populární ve své době
  \item profesně učitel, řada postav v jeho dílech byli učitelé
  \item nejčastěji psal o Podkrkonoší a o Vysočině
  \item zaměřuje se na výše popsané problémy, především pak mezigenerační soužití, domluvené sňatky
  \item výrazně se zaměřuje n apsychologii postav
  \item knihy povídek Výminkáři -- mezigenerační soužití, Potměchuť -- nitro zlého člověka, často rozvráceno sobectvím, vášněmi, oni si s tím neumí poradit; často se to dědí v rodině (velmi naturalistické), naproti těmto ale i dobří lidé
  \item humoristický román Pantáta Bezoušek -- založen na komice mezi městem a venkovem, Bezoušek je sedlák, který přijiždí do Prahy za svým synem a město nezvládá, je to vtipné, humorné, komické; dílo ne tolik kritické jako ostatní
  \item román Kalibův zločin -- \uv{venkovského balíka} -- dobráka Vojtu všichni šidí a pomlouvají (dokonce ho dostali i do vězení), on si stále myslí, že je to jen náhoda, všem odpouští, vzal se s nejkrásnější dívkou ve vsi, mají dítě a doufá, že bude mít pokoj, najednou ta jeho manželka dělá brikule,
\end{itemize}

\subsection{Josef Karel Šlejhar}
\begin{itemize}
  \item povídka Kuře melancholik -- chlapci umře matka, jemu to úplně nedochází, dojde mu to až nad matčinou rakví, on z toho dostane depresi (?), protože matka byla jediná kdo o něj pečovala (ještě stará služebná), kdo ho skutečně milovala; otec si najde macechu (a ona si přivede vlastní služebnictvo, takže byebye) a té na chlapci vůbec nezáleží; jméno protože chlapec se ujme kuřete mrzáka, zvyknou si na sebe jsou rádi, že mají jeden druhého
\end{itemize}

\subsection{Karel Matěj Čapek-Chod}
\begin{itemize}
  \item novinář
  \item výrazná ironie, až sarkasmus, jeho pohled na skutečnosti přechází až v grotesku -- vyhrocené až přehnané až úsměvné
  \item román Jindrové -- otec a syn Jidřich, syn se narodil otci ve velmi mladém věku (17), otec syna nevychovával -- matka (i se synem odjeli pryč), pak se ale najdou a sblíží se; syn musí do války a tam umře, jeho milá je smutná, pak žije s otcem, protože jí připomíná syna, syn se ale vrátí, tohle všechno zjišťuje (otec a dívka mají dítě, dají se dohromady ale syn a dívka a spolu chtějí vychovávat to dítě, dívka ale při porodu umře)
  \item román Kašpar Lén mstitel -- příbeh mladého muže co žije v podnájmu, zamiluje se navzájem s dcerou nájemců, Kašpar ale musí na vojnu a neví co se děje; jak se vrátí zjistí, že se nemá kam vrátit protože pan nájemce umřel, nakonec najdou děvče jako \uv{zaměstnankyni bordelu} (protože byli chudí, nájemce kradl jídlo, okradený chtěl po dívce \uv{službu} za to, že to neřekne, ale řekl do a tatínek se šel zabít)
\end{itemize}

\subsection{Vilém Mrštík}
\begin{itemize}
  \item bratr Alois, chudá rodina, ale dbají na vzdělání
  \item žil na Moravě, pět let v Praze, několik cest do zahraničí
  \item bouřlivák, bohém; spáchal sebevraždu
  \item román Pohádka máje -- toto v knižním vydání velice lyrické; student z města Ríša se zamiluje do Helenky někde na prázdninách, prvně to bere lehce, nakonec ho to chytne, vezmou se -- v časopisovém vydání to končí scénou po letech, kdy je z páru protivný maloměšťácký pár, to se lidem nelíbilo a v knize toto není, román je až impresionistický
  \item román Santa Lucia -- hrdina má obrovsky idealisované představy o Praze, celé dílo je retrospektivně -- Jiří Jordán vypráví jako nemocný o svých studentských letech, prvně v Brně (na Jarošce) kde se mu nelíbí, pak jde do Prahy -- jako talentovaný ale chudý student se neprobije nemá na topení, jídlo; onemocní TBC a umře -- kritika městského prostředí -- měšťáky nezajímá, že nemá na jídlo, na topení, Jiří postupně přichází o iluse
  \item jako pomocník Aloisovi románová kronika Rok na vsi --
  \item s pomocníkem Aloisem drama Maryša -- námětem případ v Těšanech, nikdo ale neumírá; viz rozbor
\end{itemize}

\part{Národní divadlo}
\begin{itemize}
  \item historický vývoj divadelnictví u nás
  \begin{itemize}
    \item počátek - konec 14. st. -- divadlo je spjaté s církví, hraje se v kostelech, hrají se výjevy z bible, života INRI, např. o Velikonocích, vánocích, postupně sem začínají pronikat světské prvky, nedrží se to té bible, přestalo se to líbit církevníkům takže je vyhodili z kostela, jsou před kostelem, na náměstí, na tržišti, snaží se, aby to bylo zajímavé zábavné, např. Mastičkář
    \item 15. st. - 18. st. -- Matěj Kopecký (cestovní loutky), Jan Ámos Komenský (škola hrou), 3. třetina 18. st. - divadla Bouda (Václav Thám -- historie), Stavovské (v neděli odpoledne česky, též Nosticovo, premiéra Dona Giovanniho, premiéra Fidlovačky), V Kotcích (první hra \uv{v češtině} -- německý překlad)
    \item 1. pol. 19. st. -- Václav Kliment Klicpera, Josef Kajetán Tyl
    \item 2. pol. 19. st. -- máme myšlenku Národního divadla, máme u něj Prozatimní divadlo
  \end{itemize}
\end{itemize}

\end{document}
