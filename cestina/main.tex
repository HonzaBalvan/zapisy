\documentclass{article}
\usepackage{fullpage}
\usepackage[czech]{babel}
\usepackage{amsfonts}

\title{\vspace{-2cm}3. a 4. fáze Národního obrození\vspace{-1.7cm}}
\date{}
\author{}

\begin{document}
\maketitle

\part{3. fáze Národního obrození}
\begin{itemize}
  \item 1830 - 1848
  \item převládá romantismus; čeština ustálena, i odborná a národní literatura
  \item ve 30. letech umí 90 \% obyvatelstva číst a psát, myšlenky NO se rozšířily všude
  \item 1830 povstání Poláků, pomoc i z Česka -- starší stojí za slovanským Ruskem, mladší nesouhlasí s absolutistickým carem
  \item rozvíjení společenského života, hrdost na české hrady, přírodu
  \item kancléř Metternich, \uv{Metternichův absolutismus} $\rightarrow$ Češi nechtějí vybočovat a provokovat, obrozenci tvoří kulturu a o politiku či práva se nezabývají
  \item biedermeier -- životní styl měšťanstva, které se nebouří, pohodlně žije, organizují spol. akce
  \item žánry
  \begin{itemize}
    \item povídky, kalendářové povídky, básnické povídky, novely, balady
    \item dramatické báchorky -- divadelní hra se soudobým dějem, nadpřirozeno, hudba / zpěvy, polepšení hl. hrdiny
    \item obrazy = až publicisticky popisují dění a krajinu třeba na venkově
  \end{itemize}
  \item český romantismus jiný než světový (až na Máchu) –- oživování ÚLS, české kultury celkově, vymaňování se němčině, německé kultuře
\end{itemize}

\section{Josef Kajetán Tyl}
\begin{itemize}
  \item vzor Klicpera -- divadlo, 1848 členem říšského sněmu
  \item činnost žurnalistická -- časopis Květy (výchovná funkce, hodně lidí se tímto naučilo číst)
  \item činnost literární -- povídky
  \begin{itemize}
    \item vlastenecké ze současnosti -- \textbf{Poslední Čech} (Borovský kritizuje jako příliš vlastenecké), \textbf{Rozervanec} (obraz K. H. Máchy)
    \item historické -- Dekret Kutnohorský
  \end{itemize}
  \item činnost dramatická: \textbf{Kajetánské divadlo}, Stavovské divadlo
  \begin{itemize}
    \item soudobé hry -- \textbf{Fidlovačka}, aneb žádný hněv a žádná rvačka (jako Romeo a Julie, ale končí dobře) -- píseň Kde domov můj? (státní hymna, hudba František Škroup), \textbf{Paličova dcera} (zapálí chalupu, Rozárka vezme vinu na sebe)
    \item historické hry -- aktualizované = snažil se poukázat na soudobý problém; \textbf{Kutnohorští kováři} (stěžují si na nízké mzdy -- zatčeni -- poprava, 1 uteče a informuje krále, pozdě); \textbf{Jan Hus} (Hus s myšlenkami 1848)
    \item dramatické báchorky -- \textbf{Strakonický dudák} (chce si vzít Dorotku, vydělává hrou na dudy, válka -- cizina, poražen, Dorotka ho jde hledat; má možnost velkého bohatství a zapomene, proč se vlastně vydal do ciziny)
  \end{itemize}
\end{itemize}

\section{Karel Hynek Mácha}
\begin{itemize}
  \item 1810 -- 1836
  \item v tomto období u nás jediný romantik, neorientuje se na tradice nebo lidové vlastenectví
  \item narozen v Praze v chudé rodině, zemřel v Litoměřicích, na gymnáziu žák Jungmanna
  \item poté VŠ v Praze (práva), do advokátní kanceláře v Litoměřicích; v divadle J. K. Tyla
  \item miloval české hrady a zříceniny
  \item milenka Marinka Štichová, po smrti tragická tvorba
  \item druhá milenka Eleonora Šomková, zakazuje jí hrát v divadle, žárlí; otěhotní, syn Ludvík
  \item pomáhal hasit požár, napil se infikované vody, zemřel na dehydrataci kvůli průjmu
  \item pohřeb v den, kdy měl mít svatbu. pomník na Vyšehradě
  \item poezie -- první básně v němčině, později česky, \textbf{Máj} (jediná kniha, který vyšla za jeho života, špatně přijat jako málo vlastenecký, později však přehodnocena jako přelomová)
  \item próza -- bývá epická, ale propojena s lyrickými prvky; lyrizovaná próza \textbf{Obrazy ze života mého} (2 části: večer na Bezdězu, Marinka – příběh o lásce k dívce, ideál krásy, vykoná pouť do Krkonoš, při návratu je Márinka mrtvá kvůli nemoci); básnická próza \textbf{Pouť krkonošská} (až horror, na vrcholu Sněžky je chrám s mrtvými mnichy, kteří vstávají 1 za rok, ale nemohou vyjít z kláštera); povídka \textbf{Cikáni} (částečně připomíná Máj, postava z okraje spol., starý a mladý cikán mají milenky, jsou opuštěni, cikán trestá toho, co je svedl, je to však otec mladého cikána, starý cikán je za jeho vraždu popraven); pokus o román \textbf{Kat} (jednotlivé díly podle hradů, jen 1. díl místo 4, propojeno s historickou tematikou, děj za Václava IV., kat je levobočkem posledního Přemyslovce, inteligentní, Milada ho miluje, ale musel popravit jejího otce, poté umírá i Milada)
\end{itemize}

\section{Karel Jaromír Erben}
\begin{itemize}
  \item 1811 -- 1870
  \item hudební talent, učil hudbu, r. 1850 sekretář a archivář Národního muzea, poté archivář města Prahy -– dostal se k RKZ, ÚLS
  \item inspiruje se a sbírá ÚLS -- balady, povídky, říkadla, písně, zapisuje je
  \item myšlenky Herdera -- v ÚLS je národní duch, na základě toho psal autorské pohádky -- např. \textbf{Tři zlaté vlasy děda Vševěda} a \textbf{Dlouhý, Široký a Bystrozraký} -- zlidověly
  \item sbírka básní \textbf{Kytice}
  \begin{itemize}
    \item oblíbená již ve své době
    \item inspirace lidovovou epikou -- prvky báje, pověsti, pohádky, proroctví, legendy
    \item vyzdvihuje tradiční křesťanské hodnoty (čestnost, morálku), ale čerpá z pohanských bájí a pověr (vodník, polednice)
    \item silný motiv mateřství, mateřské lásky
    \item až horrorové motivy
    \item inspirace dodnes
  \end{itemize}
\end{itemize}

\part{4. fáze Národního obrození}
\begin{itemize}
  \item 1850 -- 1860
  \item literatura více propojená se životem
  \item převládá realismus, stále ale výrazný vliv romantismu
  \item Božena Němcová, Karel Havlíček Borovský
  \item období \uv{Bachova absolutismu}
\end{itemize}

\section{Božena Němcová}
\begin{itemize}
  \item její narození je záhada, možná dcerou šlechtičny, rodné jméno Barbora Panklová
  \item východní Čechy (Ratibořice), prostředí Babičky, vyrůstala tam
  \item průkopnice ženské prózy českého jazyka
  \item v 17 provdána za Josefa Němce (úředník, o dost starší, sňatek z rozumu), ten ji přivedl do vlasteneckých kruhů
  \item propojuje realismus a romantismus, romantické postavy, v duchu českého romantismu konec, idyla (pod vlivem ÚLS), realismus -- soc. vrstvy, venkov
  \item píše \textbf{obrazy} -- propojení dokumentaristiky s příběhy a situacemi ze života lidí, hodně reportážních prvků, důrazná pojmenování
  \item \textbf{Divá Bára} (povídka) -- venkovská dívka, neustále u přírody, nevěří pověrám, vše si vyzkouší, je jiná než ostatní děti, nemá moc kamarádů, jen Elišku
  \item \textbf{Pan učitel} (povídka) -- částečně autobiografický hold jejímu učiteli, dovedl ji ke knihám
  \item \textbf{Karla (povídka)} -- Markyta se stane chudou vdovou, má syna, mnoho problémů; měla strach, že syna Karla odvedou na vojnu, vychovávalo ho jeho děvče Karla
  \item \textbf{Babička} (obsáhlá povídka) -- Ratibořické údolí na Náchodsku v 30. letech 19. st., Staré bělidlo; začala psát, když jí zemřel syn -- vzpomínky na dětství; ne vše je pravda, realistické v rozvrstvení společnosti, mnoho postav má jen dobré rysy, babička – Magdaléna Novotná, Proškovi – Panklovi, babička -- vlastenkyně, výchovná fce, vlastenecké příhody, idealizovaná; děti Barunka (BN, nejučenější, naslouchá), Vilém, Jan, Adélka; pak Viktorka (čistě romantická postava z okraje společnosti, zamiluje se do vojáka, jde s ním do světa, vrátí se zblázněná, žije v jeskyni, porodí dítě, hodí ho do splavu); panské postavy -- sluhové negativní, kněžna pozitivní, oblíbí si babičku; příběh končí úmrtím babičky, chudé postavy mají dobré vlastnosti (až na maminku, která se chová pansky); kompozice = jako jeden cyklus roku, ve skutečnosti uběhne 12 let + vložený epizodní příběh o Viktorce
\end{itemize}

\section{Karel Havlíček Borovský}
\begin{itemize}
  \item 1821 -- 1856
  \item narodil se v Borové u Německého (dnes Havlíčkova) brodu, příjmení Havlíček, Borovský přídomek
  \item kritik, např. Tyla (povídku Poslední Čech), básník, hodně satirický, taky překládal
  \item v jeho rodině se mluvilo německy, studoval ne německém gymnáziu, ale čeština se mu líbila, takže používla češtinu častěji než němčinu
  \item určitou dobu působil jako vychovatel v Moskvě, toto inspirace pro dílo Obrazy z Rus, kde je velice kritický k podmínká carského Ruska
  \item napsal zejména tři velké satirické skladby, které zůstaly částečně nedokončené
  \begin{itemize}
    \item Král Lávra -- vychází ze staré pohádky, báje o irském králi s oslíma
    ušima, stejný motiv už se objevuje v řeckých pověstech o králi Midasovi -- říká se, že existuje dobrý, hodný král, ale má jeden zvláštnost -- jednou za rok se nechává stříhat a svého kadeřníka nechá vždy popravit, jednou měl jít stříhat syn nějaké chudé vdovy, ona šla prosit krále, aby jejího syna nezabíjel, on ho tedy ušetřil, ale musel mu slíbit, že po stříhání o tom nikomu nebude povídat, on to nevydržel a svěřil se vrbě, někdy později nějaký muzikant si z té vrby udělal kolíček do basy a ta basa začala hrát o tom, co jí kadeřník svěřil, že král má oslí uši, které schovává pod dlouhými vlasy a korunou, ale že se jednou za rok nechává stříhat a proto nechává popravovat ty kadeřníky, aby to nikomu nevyzradili -- kritika cenzury, absolutismu a lidí, co si to nechají líbit
    \item Křest svatého Vladimíra -- vychází z Nestorova letopisu z 12. st., ten pojednával o přijetí křesťanství na Kyjevské Rusi, dílo je vtipné a velmi satirické, ale nedokončené, jde o pohanského boha Peruna a o tom, jak odmítl požadavek vládce a ten ho tedy nechal popravit, Kyjevská Rus se ocitne bez boha a vyhlásí konkurz na boha kam přicházejí různí bohové se o tuto pozici ucházet, vyhrává bůh křesťanů
    \item Tyrolské elegie -- zachycuje okamžik svého zatčení (pro své názory, vyjadřování nesouhlasu s cenzurou, s národnostním útlakem zatčen a deportován) a o tom co se odehrávalo cestou do Brixenu -- po zatčení měl asi jen dvě hodiny se sbalit, rozloučit se s dcerou, zajímavá událost, že koně se splašily, všichni kromě Havlíčka vyskočili, on koně zastavil a potom počkal až potlučení policisté a kočí doběhnou a pokračují v cestě -- snaží se poukázat asi i na neschopnost policie v Rakousku
  \end{itemize}
  \item dále také psal celý život epigramy (lyrický útvar zachycující různé situace lidského života s výraznou satirou, kritika nešvaru ve společnosti, s výraznou pointou), uspořádal je do několika souborů v pěti oddílech: církvi, králi, vlasti, múzám, světu
\end{itemize}

\part{Realismus ve světové literatuře}
\begin{itemize}
  \item 2. pol. 19. st.
  \item zachycuje věrně skutečnost, jde o skutečné zachycení prostředí postav, zachycení nečeho, co by se mohlo opravdu stát
  \item autoři hodně zkoumali příběhy které se opravdu staly, na základě toho vymýšleli příběhy vlastní, autoři nejdříve spoustu času studovali prostředí, ta místa o kterých chtějí psát, dobře promýšleli svět okolo postav a dopodrobna promýšleli jednotlivé postavy, tak aby to všechno bylo reálné, a to včetně např. zachycení jazyku postav uměrně jejim soc. postavení -- dialekty, argot, apod.
  \item vypravěč většinou nad příběhem a jazyk vypravěče bývá spisovný
  \item postavy bývají typizované, objevuje se typizace postav -- postava je typickým představitelem své dané soc. vrstvy, svého prostředí
  \item ale nechceme mít dvě stejné postavy, postavy mají samozřejmě i nějakou individuální složku, i přesto bývají typizované
  \item autoři se nevyhýbají (až je vyhledávají) prostředím, která nejsou příjemná -- o prostituci, o soc. slabých vrstvách, ale i o korupci, apod.
  \item svými díly často vyzdvihují různé společenské problémy a otevřeně kritizují jejich viníka, ať je jakýkoliv -- kritický realismus
  \item odnoží realismu je i naturalismus -- vyhrocený realismus, který říká, že podstata jednotlivce je víceméně dána a je neměnná, člověka tedy ovlivňují jen dědičnost a výchova, resp. vliv prostředí, úplně tedy pomíjejí, že by lidská vůle mohla být natolik silná, aby toto dokázala změnit (dnes, když se řekne, že je něco naturalistické tak je to přehnaně detailní a tyto detaily jsou až nechutné)
\end{itemize}
\section{Anglie}
AAAAAAA Dickens
\section{Francie}
AAAAAAA Balzac a Flaubert, Zola
\subsection{Émile Zola}
\begin{itemize}
  \item soubor románů Rougor-Macqartové -- nestihl dodělat, podobné Honoré de Balzacovi, některé postavy se taky objevují ve více příbězích
  \begin{itemize}
    \item součástí román Zabiják, román Nana
  \end{itemize}
\end{itemize}

\subsection{Guy de Maupassant}
\begin{itemize}
  \item narozen do zajištěné rodiny, matka šlechtična, takže taky šlechtic
  \item byl na nějaké církevní škole, tam se mu nelíbilo a němel proto rád církev, pak studoval na lyceu a pak práva
  \item prvně nechtěl psát pod svým pravým jménem
  \item měl syfilis, ke konci života měl deprese, halucinace a pokusil se o sebevraždu, dožil v léčebně
  \item romány a povídky, významný román Miláček (o šarmantím muži šplhajícím mezi spol. vrstvami), kvanta povídek, např. Kulička, Muška, Dítě
\end{itemize}

\subsection{Henryk Sienkiewicz}
\begin{itemize}
  \item 1846-1916
  \item Polák, šlechtic, ale chudý
  \item začal studovat medicínu, pak přešel na filosofii
  \item věnoval se literatuře, především historické próze -- studoval, poté i psal
  \item cestoval, 3 roky strávil v USA což pro něj bylo velmi inspirativní
  \item psal především pro dospělé, napsal ale i jeden dobrodružný román pro mladší Pouští a pralesem -- dobrodružný příběh dvou dětí, v době stavby Suezského průplavu, 14letý Polák Staš, 8letá Angličanka Nela, jejich otcové pracují na té stavbě, byli uneseni -- pokus za ně získat výměnou nějakého vězně, oni se ale díky Stašově důvtipu osvobodit, ale ocitnou se v divočině (poušť, prales) a snaží se nějak přežít, nakonec se zachrání; i zfilmováno
  \item historická próza Křižáci -- období bojů Poláku proti Německým rytířům, motiv národního uvědomování
  \item románová trilogie Ohněm a mečem, Potopa, Pan Volodyjovski -- historie 17. st., boje mezi Poláky a ukrajinskými kozáky
  \item stěžejní román Quo vadis? -- děj ve starověkém Římě za vlády Nerona, zpracovává téma nástupu křesťanství
\end{itemize}

\subsection{Henrik Ibsen}
\begin{itemize}
  \item
  \item Nor, dramatik
  \item většinou dodržoval jednotu času, místa a děje, nicméně konfliktry mezi postavami odhaluje retrospektivně a minulost postupně vyplývá na povrch (skrz dialogy) a divákovi se tak skládá mozaika toho, jako minulost ovlivňovala životy jednotlivých postav
  \item jeho psychologie postav je docela výrazná, proniká hluboko do nitra postav, často ukazuje, že realita bývá jiná, než vypadá na první pohled
  \item neváhá ve svých dílech zničit to, co vypadá jako pozitivní, dobré -- až smrt, nebo postavy dojdou k poznání toho, že co si mysleli co je ideální opravdu není
  \item drama Divoká kachna -- hra je protknuta symboly, rodina chová na půdě divokou kachnu, kterou zachránili, přenesli ji ale do umělého prostředí, ta kachna tam žije, ale nepřirozený život -- jakýsi symbol spoutanosti, nesvobody člověka, symbolem přetvářky (dělá se, že je to ideální prostředí, ale není)
  úkoly: přečti to, pochop to, zapamatuj si to
  \item drama Nora (popř. Domov loutek, Domeček pro panenky) -- žena žije v eštastném manželství, mají děti, její manžel má nějakou chronickou nemoc, měl by se léčit, ale oddalují to, ona mu přispívá a výrábí tedy vánoční ozdoby, ale on by měl jet teď, takže žena získává peníze podfukem (nějaký bankovní podvod), věří je získá a brzo vrátí, dřív než k tomu dojde se na to ale zjistí, manžel ji označí za nesvéprávnou, nedovolí jí vidět děti atd., \uv{ona si uvědomí, že není součást jeho života, ale je jen nějaký doplněk}, rozhodne se, že takhle ne a odchází od manžela, tedy i definitivně od dětí n
\end{itemize}

\end{document}
