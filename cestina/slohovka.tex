\documentclass{article}
\usepackage{fullpage}
\usepackage[czech]{babel}
\usepackage{amsfonts}

\title{Krysař}
\date{}
\author{}

\begin{document}
\maketitle

\large
Jednou z knih, které jsem četl v rámci povinné četby, byla novela Krysař od Viktora Dyka. Zpracovává středověký příběh tajemné postavy, která svou flétnou zbaví město jeho problému s krysami. V této verzi je však příběh lehce pozměněn a doplněn o osobní rovinu Krysařova života ve městě Hammeln. Dává nám tedy možnost Krysaře poznat a tím pádem ho i popsat.

Krysařovo vzezření je tak záhadné, jako jeho původ či příběh. Je vysoké štíhlé postavy, kterou zdůrazňuje i jeho oblek. Představuji si, že nosí kabát, barvou šedočerný, podobný srstem myší, které hubí. Ten je doplněný úzkými přiléhavými kalhotami, dále černým pláštěm s velkou kapucí, kterou si nasazuje vždy na cestách, aby se chránil před nepřízní počasí. Jediné části těla, které nejsou tímto šatem zakryty, jsou Krysařovy ruce a obličej. Ruce má~jemné, jeho prací je přeci jen pískání na flétnu. Prsty má jako kostlivec, na dlouhé píšťale tak bez~problémů dosáhne na všechny otvory a může svou hrou vábit nejrůznější tvory. V obličeji vypadá trochu pohuble. Má výrazný rovný špičatý nos a nad ním dvě temné, ale plamenné oči. Jak se říká oko je do duše okno, Krysařova mysl nám tedy není nikterak skryta.

Krysařovo povolání vzbuzuje v běžných lidech podobně jako například hrobařina nebo řemeslo kata nechuť až strach. Krysař se už proti těmto náladám naučil obalovat vrstvou pesimismu, ponurosti a celkové odtažitosti od společnosti. Vede samotářský způsob života. Když se však Agnes podaří přes tuto vrstvu dostat, zjišťujeme jak dokáže být Krysař laskavý a schopný milovat. Zde se také ukáže jeho nerozhodnost, nebo možná spíš chvilková úzkost ze~situace, která se mu předtím nikdy nestala, když prchne z města před Agnes. Když si~utřídí svoje priority, vrací se za Agnes do Hammeln.

Ve chvíli kdy se ale rozhodne s Agnes zůstat se mu celý jeho sen zbortí. Krysař je tragická postava, jejíž konec je smutný zbytečně, jen kvůli předsudkům zatuchlé společnosti.

\end{document}
