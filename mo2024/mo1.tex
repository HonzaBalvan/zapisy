\documentclass{article}
\usepackage{fullpage}
\usepackage[czech]{babel}
\usepackage{amsfonts}
\usepackage{amsmath}
\usepackage{graphicx}
\usepackage{caption}
\usepackage{enumerate}
\graphicspath{{images/}}

\title{\vspace{-2cm}\vspace{-1.7cm}}
\date{}
\author{}

\begin{document}
\maketitle
\noindent \textbf{Jan Romanovský}

\noindent \textbf{Gymnázium Brno, tř. Kpt. Jaroše}

\noindent \textbf{4.A}

\noindent \textbf{A-\textrm{I}-1}

\textbf{ }

\textit{Předpokládejme, že pro reálná čísla $a, b$ mají výrazy $a^2 + b$ a $a + b^2$ stejnou hodnotu. Jaká nejmenší může tato hodnota být?}

\textbf{ }

\begin{align*}
  a^2 + b &= a + b^2\\
  a^2 - b^2 &= a-b\\
  (a+b)(a-b) &= a-b
\end{align*}

\textbf{ }

{\begin{minipage}[t]{0.49\textwidth}
    i. $a-b=0 \implies a =b$:
  \begin{align*}
    &\implies \text{hledáme min. fce } f: y = x^2 + x\\
    &f^\prime :y = 2x + 1 \text{, extrém pro } y = 0 \implies x = \frac{-1}{2}\\
    &f^{\prime \prime}: y = 2 \implies \text{minimum}\\
    &\implies a = b = \frac{-1}{2} \text{, } a^2 + b = a + b^2 = \frac{-1}{4}
  \end{align*}
\end{minipage}
\hfill
\noindent\begin{minipage}[t]{0.49\textwidth}
    ii. $a-b \neq 0 \implies a \neq b$
  \begin{align*}
    &(a+b)(a-b)=(a-b)\\
    &a+b = 1\\
  &b = 1-a\\
    &\implies \text{hledáme min. fce } F: y = x^2 -x +1\\
    &f^\prime: y = 2x -1 \text{, extrém pro } y = 0 \implies x = \frac{1}{2}\\
    &f^{\prime \prime}:y =2 \implies \text{minimum}\\
    &\implies a = \frac{1}{2}, a^2 - a + 1 = \frac{3}{4} > \frac{-1}{4}
  \end{align*}
\end{minipage}}

\,\\

Tato hodnota bude nejmenší pro $a = b = \frac{-1}{2}$, $a^2 + b = a + b^2 = \frac{-1}{4}$.
\end{document}
