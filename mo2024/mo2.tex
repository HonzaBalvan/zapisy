\documentclass{article}
\usepackage{fullpage}
\usepackage[czech]{babel}
\usepackage{amsfonts}
\usepackage{amsmath}
\usepackage{graphicx}
\usepackage{caption}
\usepackage{enumerate}
\graphicspath{{images/}}

\title{\normalsize{\vspace{-2cm}\textsc{74. ročník Matematické olympiády (2024/2025)}\vspace{-1.7cm}}}
\date{}
\author{}

\begin{document}
\maketitle

\textbf{ }

\noindent \textbf{Jan Romanovský}

\noindent \textbf{Gymnázium Brno, tř. Kpt. Jaroše}

\noindent \textbf{4.A}

\noindent \textbf{A-\textrm{I}-2}

\textbf{ }

\textit{Martin k sobě přikládá hrací kostky (stejné velikostí i rozmístěním čísel) tak, aby byly vyskládané do tvaru čtverce libovolné velikosti a aby vždy na dvou přiléhajících bočních stěnách byla táž čísla. Kolik nejvíce různých čísel se může vyskytnout na horních stěnách kostek?}

\textbf{ }

Když mám 1x1xn řadu kostek, tak jsou jako korálky na provázku, ty dvě kontaktní číslice jsou set, navrchu je teda max 4 picků. Pak z jedné strany přidám lib. jednu kostku, navrchu můžou být zase 4, přičemž overlap s předch. 4 jsou jen 2, když vyberu číslo mimo overlap, nemůžu dosadit čtvrtou do čtverce, tj. zase si musím vybrat z prvních 4
\end{document}
