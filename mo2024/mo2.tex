\documentclass{article}
\usepackage{fullpage}
\usepackage[czech]{babel}
\usepackage{amsfonts}
\usepackage{amsmath}
\usepackage{graphicx}
\usepackage{caption}
\usepackage{enumerate}
\graphicspath{{images/}}
\setlength\parindent{0pt}

\title{\normalsize{\vspace{-2cm}\textsc{74. ročník Matematické olympiády (2024/2025)}\vspace{-1.7cm}}}
\date{}
\author{}

\begin{document}
\maketitle

\textbf{ }

\noindent \textbf{Jan Romanovský}

\noindent \textbf{Gymnázium Brno, tř. Kpt. Jaroše}

\noindent \textbf{4.A}

\noindent \textbf{A-\textrm{I}-2}

\textbf{ }

\textit{Martin k sobě přikládá hrací kostky (stejné velikostí i rozmístěním čísel) tak, aby byly vyskládané do tvaru čtverce libovolné velikosti a aby vždy na dvou přiléhajících bočních stěnách byla táž čísla. Kolik nejvíce různých čísel se může vyskytnout na horních stěnách kostek?}

\textbf{TODO }

Když mám 1x1xn řadu kostek, tak jsou jako korálky na provázku, ty dvě kontaktní číslice jsou set, navrchu je teda max 4 picků. Pak z jedné strany přidám lib. jednu kostku, navrchu můžou být zase 4, přičemž overlap s předch. 4 jsou jen 2, když vyberu číslo mimo overlap, nemůžu dosadit čtvrtou do čtverce, tj. zase si musím vybrat z prvních 4

Začneme řadou kostek $2 \times 1$. Mezi jednotlivými kostkami jsou dotykovými číslicemi jen určité dvě, které jsou na kostce naproti sobě, BÚNO vyberme $1, 6$. Potom nahoře můžou být ostatní čtyři číslice, podle toho které si vybereme, tedy z $2, 3, 4, 5$. Když doplním ze strany jednu kostku budeme mít jinou dotykovou číslici než $1, 6$, vybereme si BÚNO $2$, naproti tomuto dotyku je tedy číslice $5$. Na této kostce potom můžu mít nahoře zase čtyři číslice $1, 3, 4, 6$. Tento stejný výběr mám na dotykovou číslici s kostkou, kterou bych chtěl doplnit tak, abych získal čtverec $2 \times 2$. Z jedné strany tedy mám výběr $1,3,4,6$ a z druhé $2,3,4,5$. Když budeme řešit kombinace těchto dvou výběrů vidíme, že nemůžeme vybrat dvě stejná čísla, ani dvě čísla, která jsou na kostce naproti sobě. Zbývají výběry: $1,2$, nahoře bude $4$; $1,3$, nahoře bude $2$; $1,4$, nahoře bude $5$; $1,5$, nahoře bude $3$; $3,2$, nahoře bude $1$; $3,5$, nahoře $6$; $4,2$, nahoře bude $6$; $4 ,5$, nahoře bude $1$; $6,2$, nahoře bude $3$; $6,3$, nahoře bude $5$; $6,4$, nahoře bude $2$; $6,5$, nahoře bude $4$ -- nahoře je vždy jedna z číslic $2, 3, 4, 5$, výběr se nám od řádku $1 \times n$ nerozšířil, podle tohoto postupu můžeme rozšířit do plochy $n \times n$ a vždy budeme mít na výběr jen ze čtyř číslic.

Na horních stěnách kostek se můžou objevit nejvíce $4$ číslice.
\end{document}
