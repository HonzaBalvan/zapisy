\documentclass{article}
\usepackage{fullpage}
\usepackage[czech]{babel}
\usepackage{amsfonts}
\usepackage{amsmath}
\usepackage{graphicx}
\usepackage{caption}
\usepackage{enumerate}
\graphicspath{{images/}}
\setlength\parindent{0pt}

\title{\normalsize{\vspace{-2cm}\textsc{74. ročník Matematické olympiády (2024/2025)}\vspace{-1.7cm}}}
\date{}
\author{}

\begin{document}
\maketitle

\textbf{ }

\noindent \textbf{Jan Romanovský}

\noindent \textbf{Gymnázium Brno, tř. Kpt. Jaroše}

\noindent \textbf{4.A}

\noindent \textbf{A-\textrm{I}-3}

\textbf{ }

\textit{Na tabuli jsou napsána navzájem různá přirozená čísla se součtem $2024$. Každé z nich kromě nejmenšího je násobkem součtu všech menších napsaných čísel. Kolik nejvíce čísel může na tabuli být?}

\textbf{ }

Čísla napsaná na tabuli můžeme brát jako členy rostoucí posloupnosti z přirozených čísel, kdy pro všechny členy kromě prvního platí, že jsou násobkem součtu všech předchozích členů. Součet této posloupnosti je 2024.
\begin{gather*}
  a_1,\, a_2,\, a_3,\, ...,\, a_n\\
  a_n = k_n\cdot\sum_{i=1}^{n-1} a_i,\, k_n \in \mathbb N\\
  S_{a_n} = 2024
\end{gather*}

%To znamená, že n-tý člen dělí ten (n+1)tý, a že součet (n)tého a (n+1)tého dělí ten (n+2)tý. Ten druhý je ale také dělitelný tím prvním, takže i třetí musí být dělitelný prvním, atd. $\implies$ n-tý člen dělí všechny další členy, takže dělí i jejich součet, tedy 2024.
Rozepišme si každý člen:
\begin{flalign*}
  &a_1 = a_1\\
  &a_2 = k_2 \cdot a_1\\
  &a_3 = k_3 \cdot (a_1 + a_2) = k_3 \cdot (k_2 + 1) \cdot a_1 \\
  &a_4 = k_4 \cdot (a_1 + a_2 + a_3) = k_4 \cdot (a_1 + k_2 \cdot a_1 + k_3 \cdot (k_2 + 1) \cdot a_1) = k_4 \cdot (k_3 + 1) \cdot (k_2 + 1) \cdot a_1\\
  &...\\
  &a_n = k_n \cdot (k_{n-1} + 1) \cdot (k_{n-2} + 1) \cdot ... \cdot (k_2 + 1) \cdot a_1
\end{flalign*}

\begin{flalign*}
  S_{a_n} &= a_n + (k_{n-1} + 1) \cdot (k_{n-2} + 1) \cdot ... \cdot (k_2 + 1) \cdot a_1 = \\
  &= k_n \cdot (k_{n-1} + 1) \cdot (k_{n-2} + 1) \cdot ... \cdot (k_2 + 1) \cdot a_1 + (k_{n-1} + 1) \cdot (k_{n-2} + 1) \cdot ... \cdot (k_2 + 1) \cdot a_1 = \\
  &= (k_n + 1) \cdot (k_{n-1} + 1) \cdot (k_{n-2} + 1) \cdot ... \cdot (k_2 + 1) \cdot a_1 = 2024
\end{flalign*}

Zaměřme se na poslední rovnost. Všechny $k$ jsou přirozená, $a_1$ také, takže všichni činitelé jsou přirození a jejich součin je $2024$. Chceme, aby jich bylo co nejvíc -- takový součin ale známe, je to rozklad na prvočinitele, protože žádný prvočinitel už nejde v přirozeném oboru dále rozložit. Každá závorka tedy reprezentuje jednoho prvočinitele čísla $2024 = 2 \cdot 2 \cdot 2 \cdot 11 \cdot 23$, za první člen $a_1$ nám zbývá vybrat $1$, aby se $S_{a_n}$ opravdu rovnal $2024$. Tento součin tedy může mít maximálně $6$ činitelů.

\textbf{ }

Na tabuli může být maximálně $6$ čísel.

\end{document}
