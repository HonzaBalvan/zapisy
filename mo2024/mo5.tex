\documentclass{article}
\usepackage{fullpage}
\usepackage[czech]{babel}
\usepackage{amsfonts}
\usepackage{amsmath}
\usepackage{graphicx}
\usepackage{caption}
\usepackage{enumerate}
\graphicspath{{images/}}
\setlength\parindent{0pt}

\title{\normalsize{\vspace{-2cm}\textsc{74. ročník Matematické olympiády (2024/2025)}\vspace{-1.7cm}}}
\date{}
\author{}

\begin{document}
\maketitle

\textbf{ }

\noindent \textbf{Jan Romanovský}

\noindent \textbf{Gymnázium Brno, tř. Kpt. Jaroše}

\noindent \textbf{4.A}

\noindent \textbf{A-\textrm{I}-5}

\textbf{ }

\textit{Saba se snaží z přízemí nekonečně vysokého mrakodrapu dostat do n-tého patra pomocí zvláštního výtahu. Ve výtahu jsou tlačítka $0$, $1$, $2$, ... Po prvním stisknutí tlačítka pojede výtah nahoru a po každém dalším jede vždy opačným směrem, než posledně, přičemž po stisknutí tlačítka k popojede vždy o $2^k$ pater. Navíc každé další stisknuté tlačítko musí mít menší číslo než to předešlé. Dokažte, že Saba se do každého patra $n \geq 1$ může dostat právě dvěma různými postupy.}
\end{document}
