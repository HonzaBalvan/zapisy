\documentclass{article}
\usepackage{fullpage}
\usepackage[czech]{babel}
\usepackage{amsfonts}
\usepackage{amsmath}
\usepackage{graphicx}
\usepackage{caption}
\usepackage{enumerate}
\graphicspath{{images/}}
\setlength\parindent{0pt}

\title{\normalsize{\vspace{-2cm}\textsc{74. ročník Matematické olympiády (2024/2025)}\vspace{-1.7cm}}}
\date{}
\author{}

\begin{document}
\maketitle

\textbf{ }

\noindent \textbf{Jan Romanovský}

\noindent \textbf{Gymnázium Brno, tř. Kpt. Jaroše}

\noindent \textbf{4.A}

\noindent \textbf{A-\textrm{I}-5}

\textbf{ }

\textit{Saba se snaží z přízemí nekonečně vysokého mrakodrapu dostat do n-tého patra pomocí zvláštního výtahu. Ve výtahu jsou tlačítka $0$, $1$, $2$, ... Po prvním stisknutí tlačítka pojede výtah nahoru a po každém dalším jede vždy opačným směrem, než posledně, přičemž po stisknutí tlačítka k popojede vždy o $2^k$ pater. Navíc každé další stisknuté tlačítko musí mít menší číslo než to předešlé. Dokažte, že Saba se do každého patra $n \geq 1$ může dostat právě dvěma různými postupy.}

\textbf{ }

Zřejmě platí $\pm 2^n = \pm 2^{n+1} \mp 2^{n}$ -- tzn. že zmáčknutí (n)tého tlačítka lze zaměnit za zmáčknutí (n+1)tého tlačítka a potom stisknutí (n)tého tlačítka (střídají se znaménka), pokud už (n+1)té tlačítko bylo zmáčknuto tak použijeme ekvivalenci na druhou stranu $\implies$ pokud existuje nějaká možnost dostat se na dané číslo, jsou nutně dvě.

\textbf{ }

Každá jízda výtahem půjde zapsat jako rozdíl dvou binárních čísel (jednotlivé zmáčknutí tlačítek jsou posuny přímo o mocniny dvojky), kladně jsou všechna patra vyjetá nahoru, záporně všechna sjetá dolů. První zmáčknutí je vždy kladné a každé další vede k menšímu posunu, tento rozdíl tak bude vždycky kladný. Z N3 víme, že každý kladný rozdíl mocnin dvojky se dá zapsat jako součet nějakých mocnin dvojky a zřejmě víme, že každé přirozené číslo lze zapsat v binární soustavě (dk. MI: 1. $1 = [1]_2$; 2. chceme dk., že lze pro $n \implies$lze pro $n+1$, předp. že lze pro $n$: $n+1 = [n]_2 + [1]_2$; např. $2 = [1]_2 + [1]_2 = [10]_2$), ke každému číslu tedy existuje možnost zápisu, z prvního odstavce tak vyplývá, že existují vždy dvě. $\square$
\end{document}
