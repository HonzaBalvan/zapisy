\documentclass{article}
\usepackage{fullpage}
\usepackage[czech]{babel}
\usepackage{amsfonts}

\title{\vspace{-2cm}Politologie\vspace{-1.7cm}}
\date{}
\author{}

\begin{document}
\maketitle

AAAAAA konfucianismus, taoismus

\begin{itemize}
  \item věda o vládnutí, orgnaizaci a stavbě státu
  \item již od starověku, ve středověku např. Niccoló Machiavelli (a jeho kniha Vladař), od r. 1880 katedra politologie na Kolumbijské universitě, jako separátní specializovaná věda po druhé sv. v.
  \item politologie se zabývá politikou (umění řídit stát), politickými stranami, zájmovými skupinami, politickou mocí (a jak by měla fungovat), státem (jak má vypadat, kdo ho má řídit ap.), politickými systémy, politickým chováním lidí, jejich míněním o politice, politickými ideologiemi a teoriemi, mezinárodními vztahy a organizacemi
  \item základní okruhy politologie:
  \begin{itemize}
    \item dějiny politických teorií
    \item politické instituce
    \item politická sociologie
    \item mezinárodní vztahy
  \end{itemize}
\end{itemize}

\section{Dějiny politického myšlení}
\begin{itemize}
  \item začínají již v antice, ve starověkém Řecku
  \begin{itemize}
    \item \textit{sofisté} - nezáleží na morálce, k moci se má dostat nejsilnější, cílem debaty není dojít k pravdě, ale obhájit si názor
    \item \textbf{Sokrates} - je zapotřebí, aby v čele státu byli moudří jedinci, kteří poznali dobro, stoupenec aristokracie
    \item \textbf{Platón} - sofokracie - vládce má vědět, má být moudrý král filosof, pak vrstva strážců - udržují pořádek v státě, pořádek státu, tito nemají mít sokroumé vlastnictví (epické řešení střetu zájmů?), dílo Ústava
    \item \textbf{Aristoteles} - dílo Politika - je důležité, aby vláda byla v zájmu celku, v zájmu státu, prošel se studenty všechny řecké ústavy a našel šest typů státu, a to tři dobré (monarchie, aristokracie, politeia - vládne většina schopných a vzdělaných státu) a tři špatné
  \end{itemize}
  \item ve středověku do toho těžce zasahuje křesťanství, vládne jeden či skupina (rozhodně ne normální člověk)
  \begin{itemize}
    \item \textbf{sv. Augustýn} (Augustinus Aurelius) - patristika, jsou dobří lidi - od Boha a zlí lidi - od Ďábla, dějiny jsou o boji dobra se zlem
    \item \textbf{Tomáš Akvinský} - scholastika, přejímá Aristotelovo učení, racionalismus (věda dává smysl), tomismus (víra i racionalismus jsou validní cesty k pravdě)
  \end{itemize}
  \item a konečně novověk, prvně samozřejmě v renesanci, 15. st., klesá vliv církve, klesá vliv jedince
  \begin{itemize}
    \item \textbf{Niccoló Machiavelli} - dílo Vladař - popisuje vlády, jak se dostat k moci, jak vládnou atd.; vedla ho k tomu myšlenka sjednocení Itálie
    \item Jean Bodin - teorie suverenity - stát má být suverenní organizace
  \end{itemize}
  \item po renesanci, 17. st.
  \begin{itemize}
    \item \textbf{Thomas Hobbes} - žil v době anglické občanské války, jeden ze zakladatelů tzv. smluvní teorie státu - říká, že před vznikem státu, když byl tzv. přirozený stav, tak platilo právo silnějšího, navzájem jsme si lidé byli nebezpeční (\uv{Homo homini lupus.}), v určité fázi se ale lidé domluvili, že se vzdají svých absolutních svobod vůči státu, který si založí, a stát jim za to poskytne nějaké výdobytky (bezpečí, nastaví pravidla chování), když už tedy stát vznikne tak má obrovskou sílu
    \item \textbf{John Locke} - představitel smluvní teorie, říká ale, že pokud stát vznikne, tak musí být jeho panovník nějak limitován a nemůže svévolně vytvářet zákony, nejlepší formou vlády tedy bude konstituční monarchie, požaduje dělbu státní moci - moc výkonnou oddělit od moci zákonodárné, pokud panovník nebude dodržovat smlouvu, tak ho lidé mají právo sesadit, jedním ze zakladatelů liberlismu, zdůrazňuje, že když vznikne stát tak nám nemůže sebrat naše přirozená práva - na život, na svobodu, na majetek
  \end{itemize}
  \item a pak osvícenství, 18. st.
  \begin{itemize}
    \item \textbf{Charles Louis Monstesquieu} - k Lockově teorii přidává oddělení třetí moci - soudní, tím chce zajistit neuzurpovatelnost moci (lol)
    \item \textbf{Jean Jacques Rousseau} - oponent ke smluvní teorii - před státem bylo všechno skvělé, že všechno se kazí se vznikem soukromého vlastnictví, se vznikem státu, nejde to ale spravit tím, že zrušíme stát, ale tím, že se budeme na tvorbě státu a především jeho zákonů podílet všichni a všichni budeme pak tyto zákony všichni dodržovat a aby všichni byli před zákonem rovni, taky zastánce přímě demokracie - máme mít malé státečky, každý má mít možnost (a má se) podílet na státě
    \item \textbf{Immanuel Kant} - říká, že je potřeba do státu vnést morálku, chovat se správně a jít příkladem
  \end{itemize}
  \item jedeme do 19. st.
  \begin{itemize}
    \item \textbf{Georg F. W. Hegel} - \uv{pruský státní ideolog}, stát je důležitější než jednotlivec
    \item \textbf{Karl Marx} - odstranit soc. třídy, ať jsou si všichni rovni, ideolog socialismu, komunismu
    \item \textbf{John Stuart Mill} - ideologie sociálního liberalismu - stát má mít i sociální funkci, ne všichni mají v kapitalistickém státě rovné možnosti a tohle bychom měli spravit
    \item \textbf{Max Weber} - kapitalismus se zrodil z protestantských náboženství
  \end{itemize}
  \item 20. st. uuu
  \begin{itemize}
    \item \textbf{Karl R. Popper} - velký kritik totalitních režimů, totalitní režimy jsou nezdravé, uzavřené a lidstvu neprospívají
    \item \textbf{Hannas Arendtová} - kritizuje totalitní režimy, psala o tom, že vlastně někteří náckové nejsou vinní, že je poslouchali příkazy
  \end{itemize}
  \item naši filosofové máčející se v politologii
  \begin{itemize}
    \item \textbf{Mistr Jan Hus} - volnost víry
    \item \textbf{Petr Chelčický} - pacifista
    \item \textbf{Jiří z Poděbrad} - idea evropské jednoty
    \item \textbf{Jan A. Komenský} - domníval se, že když budou vládnout vzdělanci tak nebudou války, že by měl být jeden světový jazyk
    \item \textbf{Bernard Bolzano} - dílo O nejlepším státě (další utopie)
    \item \textbf{František Palacký} - prosazoval ideu austroslavismu - aby vznikla na habsburském území federace rovnocenných národů
    \item \textbf{Tomáš G. Masaryk} - prostě Masaryk, idk
    \item \textbf{Edvard Beneš} (jak taky jinak) - no tipni si, dělal to co Masaryk
  \end{itemize}
\end{itemize}

\section{Stát}
\textbf{Definice:} Sdružení AAAAAAAAAA
\begin{itemize}
  \item zabývá se jím státověda, tedy jako součást politologie
  \item první státy vznikají ve starověku, na území Mesopotámie, Egypta, Číny, Indie
  \item řec. polis, lat. civitas, od renesance lat. stato - řád, status
  \item co musí vykazovat každý stát, \textbf{znaky státu}:
  \begin{itemize}
    \item obyvatelstvo - alespoň částečně stále
    \item státní hranice - vymezené území nad kterým stát udržuje kontrolu
    \item svrchovanost a suverenita - svobodně se mohou rozhodovat o vnitřních a vnějších záležitostech
    \item státní správa, státní moc - ústavní instituce, státní aparát včetně ozbrojené moci, policie, ale i ostatních institucí, např. školy, zdravotnictví
    \item právo - stát tvoří právní normy a garantuje, že zákony budou dodržovány or else
    \item (mezinárodně-)právní subjektivita - stát může samostatně vstupovat do vztahů s ostatními státy
  \end{itemize}
  \item \textbf{funkce} (správně fungujícího lol) \textbf{státu}:
  \begin{itemize}
    \item vnitřní
    \begin{itemize}
      \item bezpečnostní - zajištuje bezpečnost občanů, bezpečnost jejich majetku apod.
      \item právní - tvoření a upravování zákonů a norem
      \item ekonomická - nastavení podmínek fungování ekonomiky, garance těchto podmínek
      \item sociální - sociální zabezpečení starých, nemocných, apod.
      \item kulturní - školství, kultura, věda a výzkum, atd.
    \end{itemize}
    \item vnější
    \begin{itemize}
      \item diplomatická - zajišťení a udržování vztahů s ostatními zeměmi
      \item bezpečnostní - ochrana před případným napadením státu
      \item regulace zahraničního obchodu - stanovení podmínek pro zahraniční obchodní vztahy
    \end{itemize}
  \end{itemize}
  \item \textbf{teorie vzniku státu}:
  \begin{itemize}
    \item prvotní - jak vznikají státy z ničeho
    \begin{itemize}
      \item smluvní = konsensuální - lidé se domluvili, že si udělají stát, že se vzdají určitých svobod za účelem nějakých věcí, které jim poskytne stát
      \item náboženská - stát má božský původ, je, protože to tak bůh chce
      \item theokratická - nejen stát, ale i panovník je božského původu
      \item patriarchální - srovnávají stát a rodinu - otec jako hlava rodiny a panovník jako hlava státu - jak se rozrůsta rodina, tak se rozrůstá stát
      \item mocenská - lidé jsou náchylní buď k tomu někoho ovládat, nebo být ovládáni, stát je tedy tohoto produkt - silný vládne nad slabším
      \item násilí - stát chápán jako produkt války
    \end{itemize}
    \item druhotná - státy vznikají spojováním nebo rozdělováním již existujících států
  \end{itemize}
  \item \textbf{formy státu}
  \begin{itemize}
    \item podle způsobu vykonávání státní moci
    \begin{itemize}
      \item demokracie -- vláda lidu
      \begin{itemize}
        \item znaky demokracie
        \begin{enumerate}
          \item pravidelné, rovné, svobodné volby
          \item svoboda slova, projevu, svoboda shromažďování, nezávislá média
          \item právní stát -- vztah občana a státu je vymezen zákonem, nikdo si nemůže vymýšlet pičoviny
          \item dodržování lidských práv a svobod
          \item dělba státní moci -- zmenožnění strhnutí státní moci
          \item menšina respektující rozhodnutí většiny
          \item fungování politické kultury -- konstruktivní kritika, respektování práv, atd.
        \end{enumerate}
        \item přímá -- všichni občané se přímo podílejí na vládě, např. referendum, plebiscit, např. řecký městský stát (nejblíže Švýcarsko -- polopřímá demokracie)
        \item nepřímá (representativní)-- občané se podílejí nepřímo, pomocí volených zástupců
      \end{itemize}
      \item autokracie -- samovláda, vláda jednotlivce nebo úzké skupiny
      \begin{itemize}
        \item znaky autokracie
        \begin{enumerate}
          \item nedodržování lidských práv a svobod
          \item represe opozice
          \item vládce si dělá co chce
        \end{enumerate}
        \item absolutismus -- autokracie, která tak má fungovat
        \item diktatura -- autokracie, která vznikla nějakým zvrtnutím ne-autokracie
        \item despocie -- AAAAAA
        \item totalitarismus -- snaha o absolutní kontrolu občanů, nejen v politickém a veřejném životě
        \item dále např. aristokracie (vláda urozených), timokracie/plutokracie (vláda bohatých), ochlokracie (vláda lůzy), sofokracie (vláda moudrých), oligarchie (vláda úzké skupiny (ne nutně se společnou atributou))
      \end{itemize}
    \end{itemize}
  \end{itemize}
  \item idk
\end{itemize}

\section{Národ}
\begin{itemize}
  \item společenství lidí se stejnou národností -- patříci k jednomu etnickému celku (společný původ, tradice, zvyky, jazyk, popř. náboženství)
  \item často žije na určitém území, zde si může i vytyčit svůj vlastní (národní) stát
  \item národní příslušnost - národ (národnost) ke které se člověk hlásí
  \item nacionalismus - hrdost na národ (positivní x negativní)
  \item kosmopolitanismus (idk) - \uv{světový národ}
\end{itemize}

AAAAAAAAAA 22.3.2023
AAAAA třeba tak 200 hodin reálně, viz Dominik

\begin{itemize}
  \item věda o hospodářství -- ekonomie -- zabývá se hospodářstvím a chováním lidí v podmínkách omezených zdrojů (omezené surové zdroje, omezené množství práce, omezený kapitál)
  \item ekonomika -- to hospodářství
  \item střet mezi neomezenými lidskými potřebami a omezenou dostupností statků, služeb
  \item dle díla Oikonomikos od Xenofona -- zabývá se zde jak nejlépe hospodařit na půdě
  \item dělení ekonomie mikro x makro
  \begin{itemize}
    \item mikroekonomie -- sleduje chování jednotlivých ekonomických subjektů -- domácnosti, firmy, spotřebitelé -- ekonomie peněženky, otázky poptávky, nabídky, vlivu daní, cen
    \item makroekonomie -- sleduje fungování celkového systému ekonomie, sleduje ekonomiku jako celek -- státy, svět -- otázky HDP, inflace, nezaměstnanosti, určování daní, politiky státu
  \end{itemize}
  \item dělení ekonomie pozitivní x normativní
  \begin{itemize}
    \item pozitivní -- popisuje to, co je
    \item normativní -- popisuje jak by to správně mělo být
  \end{itemize}
  \item statky
  \begin{itemize}
    \item volné -- volně dostupné -- např. vzduch
    \item vzácné/ekonomické -- někdo je musí vyrobit, musíme za ně poskytnout protihodnotu -- věci, co si půjdu koupit, služby co si můžu koupit
  \end{itemize}
\end{itemize}

\end{document}
