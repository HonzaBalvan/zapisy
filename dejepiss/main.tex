\documentclass{article}
\usepackage{fullpage}
\usepackage[czech]{babel}
\usepackage{amsfonts}

\usepackage[utf8]{inputenc}
\usepackage[T1]{fontenc}
\usepackage{enumitem}
\usepackage{color}

\definecolor{highlight}{RGB}{0, 102, 204}

\title{\vspace{-2cm}Dějepis 24. 2. 2023\vspace{-1.7cm}}
\date{}
\author{}

\begin{document}
\maketitle

\part{Reformace 2}

\section{Švýcarská reformace (Měšťanská reformace)}
\begin{itemize}
  \item v této době probíhal švýcarský boj o nezávislost na SŘŘ v 20. letech 16. st., válka mezi jednotlivými švýcarskými kantony (katolické x švýcarské)
  \item \textbf{Ulrich Zwingli} (1484-1531) - vůdce švýcarských kantonů, radikálnější než Luther, důraz na morálku, poddanstvo má právo se vzbouřit, když se šlechta nechová správně, odmítají katolické tradice (přijímání, ikony), pak padl v boji
  \item \textbf{Jan Kalvín} (1509-1564) - teolog francouzského původu, útočiště v Ženevě, vlivy Luthera, Zwingliho, učení předdestinace - je už o člověku od začátku určeno, zda bude spasen nebo zatracen, ale člověk to neví, takže by se měl chovat správně, v průběhu života se dovídá zda je ten bohem vybraný
  \item \textbf{kalvinismus} - důraz na pracovitost, skromnost, individualitu, podnikavost - jakýsi prvopočátek (dejme tomu) kapitalistického, podnikavého mindsetu, pevné morální zásady - prostě se v životě nebav, nebo nebudeš spasen
  \item kalvinismus rozšířen v Anglii (významné pro osidlování S Ameriky), Uhersku, ...
\end{itemize}

\section{Protireformace}
\begin{itemize}
  \item Katolíci chtějí zpátky získat ztracené pozice, zabránit šíření nekatolických církví, snaží se o přestavbu katolické církve
  \item se souhlasem papeže byl r. 1540 založen nový řád - \textbf{Jezuité} (Societas Jesu), zakladatel Ignác z Loyoly, přísný řád, odpovídají jen papežovi, účely rekatolizace, misionářství, zřizovali a vedli většinu škol
  \item r. 1542 založení \textbf{Svaté inkvizice} - no, inkvizice, dále zakazují některé knihy, píší jejich seznamy (Index librum prohibitorum)
  \item \textbf{Tridentský koncil} (1545-1563) - přelom ve fungovaní katolické církve, odmítnuta radikální reformace, jediná autorita je papež, ale musí se omezit hromadění církevních úřadů v rukou jednotlivce a že duchovní musí být vzdělaní - biskupské semináře pro výchovu nových kněží
\end{itemize}

\part{Francie a Habsburkové, jejich vztahy}
\section{Francie}
\begin{itemize}
  \item Francie silná, posílená po vítězství ve stoleté válce, centralizovaná, král vládne poměrně absolutisticky, rozbití generálních stavů
  \item \textbf{Ludvík XI.} (1461-1483) - začíná byrokratizaci - nová ,,úřednická šlechta"
  \item rozhodující část daní šla z měst, protože tam podnikali, což bylo pod úroveň šlechticů
  \item po stabilizaci po stoleté válce francouzští králové se pokouší územně expandovat, ale vždy se jim do toho vrtají Habsburkové $\rightarrow$ války mezi Francií a Habsburky
  \item král \textbf{Karel VIII.} (1483 - 1498) - rozhodl se získat Neapolské království, to mu ale vyfoukl Ferdinand Aragonský, v čem mu pomohla Benátská liga - papež, císař SŘŘ Maxmilián Habsburský, Benátky, Milán
  \item král \textbf{Ludvík XII.} (1498 - 1515) - dočasný zisk Milánského vévodství, pak se zase musel stáhnout
  \item král \textbf{František I.} (1515 - 1547) - vzdělaný král (tzv. kníže renesance, na jeho dvoře přední renesantici - Da Vinci, Michalangelo, Tizian, Rabelais), za jeho vlády vrcholí války mezi Francií a Habsburky, za Habsburky \textbf{Karel V.}, rozhodujíci \textsc{bitva u Pavie} - 1525
  \item potom jsou zde taky náboženské války (1562) - mezi katolíky a hugenoty (kalvinisty)
  \begin{itemize}
    \item začínají masakrem hugenotů, pravděpodobně z vůle vůdce katolíků Jindřicha de Guise, ten zavražděn r. 1588
    \item vůdce hugenotů je navarrský princ Jindřich de Bourbon
    \item král Jindřich III. se to snaží zmírnit tím, že provdá svoji sestru Markétu za Jindřicha Bourbonského, ale
    \item 23./24. srpen 1572 - Bartolomějská noc - katolíci velice organizovaně a systematicky zmasakrovali hugenoty, kteří se sjeli při příležitosti výše svatby
    \item Jindřich III. r. 1589 zavražděn radikálním mnichem - není další Valois, takže nastupuje Jindřich Bourbonský
    \item Jindřich Bourbonský 1593 konvertuje na katolicismus, 1594 korunován francouzským králem jako
  \end{itemize}
  \item král \textbf{Jindřich IV. de Bourbon} - 1598 vydává edikt nantský - zrovnoprávnil hugenoty a katolíky, dočasně snížil daně, aby se země zrekonstruovala, vždy ale v konfliktech nadržoval hugenotům, takže taky zavražděn taky mnichem
  \item Jindřich IV. měl po Markétě ještě Marii Medicejskou, s ní měl syna Ludvíka (XIII.), ten ale v době Jindřichově smrti neplnoletý, takže Marie zatím vládne jako regentka
  \item kardinál Richelieu - šedá eminence, 1. ministr, diehard podporovatel Marie a Ludvíka, zatáhl Francii do třicetileté války na straně protihabsburské (i když vlastně v době Marie poměrně close ties s Habsburky (manželka Ludvíka Anna Rakouská))
  \item pevno La Rochelle - hugenotská pevnost poměrně nezávislá na králi, účastnili se protistátního povstání, takže pevnost byla dobyta a asi 20 tis. hugenotů tam bylo zabito
  \item po Ludvíkovi XIII. nastupuje jeho syn Ludvík XIV. (stay tuned)
\end{itemize}

\section{Španělsko}
\begin{itemize}
  \item tak jak už víme máme spojené Španělsko Ferdinandem II. a Isabelou
  \item proti Ferdinadovi se bouří mocní šlechtici - \textit{grandové} , kteří získali moc při boji s Araby, Ferdinand se proti nim spojuje s městy do tzv. \textit{sv. hermanandy}, společně grandy porazili
  \item Ferdinand a Isabela mají dceru Johanu, její manžel byl Filip Habsburský, tento držel Nizozemí (získal od Burgundie)
  \item Johana s Filipem mají syna Karla - tento jako španělský král Karel I., jako císař SŘŘ \textbf{Karel V.}, tento se opírá o vysokou šlechtu, což naopak způsobilo povstání měst - \textit{povstání komunerů} (1520/21) v čele s Toledem, které Karel tvrdě potlačil a vydobyl si na nich obrovské daně, ekonomicky tím vlastně zkriplil do budoucna celé Španělsko (byť mělo obrovské koloniální impérium), do čela dění se dostvává Francie, Anglie
  \item Karlův syn byl \textbf{Filip II.} - fanatický katolík, oporou mu byla katolická církev a střední šlechta, stejně jako otec to pěkně kazil, dovedl Španělsko až k vyhlášení státního bankrotu (1557), taky tento král ten, kterému Angličani rozbili flotilu (armadu) (ve válce s nimi kvůli Filipovu claimu na anglický trůn, jeho žena byla Marie I. Tudorovna)
\end{itemize}

\section{Nizozemsko}
\begin{itemize}
  \item ze začátku ke Španělsku volně připojeno, téměř nezávislí
  \item velice bohaté, vyvinutá řemesla, např. krajky, střelné zbraně, na severu svobodní zemědělci, zprostředkovávali obchod s celou Evropou - významné přístavy, bohatá města
  \item skládá se ze 17 provincií, každá má svůj sněm
  \item za Filipa je zde místodržitelkou levoboček jeho otce Markéta Parmská
  \item nábožensky rozdělené - luteráni, kalvíni, novokřtěnci i katolíci, hugenoti
  \item my tedy máme španělský státní bankrot, Filip ho řeší tak, že zvyšuje daně, kde lépe svýšit daně než v nejbohatší provincii, zároveň se snaží Nizozemsko víc svázat se Španělskem a vládnout pevněji, toto vyvolalo v Nizozemsku nepokoje
  \item prvně obrazoborectví (1566/67) - lid ničí výzdoby katolických kostelů, jejich malby apod., šlechtici se to snaží diplomaticky vyřešit, Filip to ale využil jako záminku poslat sem armádu v čele s \textbf{vévodou z Alby}, který zde zavedl krutovládu (1567) - stíhal nekatolíky, posilovali habsburskou moc, popravovali nejen lid, ale i šlechtice atd. - šlechtici před tímto prchají, např. \textbf{Vilém Oranžský} (foreshadowing)
\end{itemize}

\subsection{Nizozemská revoluce}
\begin{itemize}
  \item 1572 - 1579
  \item velitel Vilém Oranžský, povstání se zúčastní lid i šlechta i nejchudší lidé - posměšně gézové (překl. žebráci), severní jsou mořští, jižní jsou lesní (lol) a začala tedy válka
  \item o Markétě je zde místodržícím Juan d'Austria - další Karlův levoboček
  \item válčíme válčíme, 1576 Španělé plení Antverpy - jedno z nejbohatších měst
  \item r. 1579 se povstávající stavové rozdělují na
  \begin{itemize}
    \item Arraská unie - provincie na jihu, které se chtějí domluvit a zůstat pod Španělskem pod podmínkou, že Španělé stáhnou svoje armády
    \item Utrechtská unie - provincie na severu, budou bojovat tak dlouho \uv{dokud Španěle nevytlačí za Rýn}, což se jim daří - 1581 vyhlašují Spojené nizozemské provincie (nezávislé na Španělsku) v čele s Vilémem I. Oranžským, tento nedlouho poté nějakým katolíkem zavražděn, po něm jeho syn Mořic Oranžský
    \item 1609 Nizozemí a Španělé uzavírají příměří, které trvá do připojení Nizozemska do třicetileté války (1621), po třicetileté válce Nizozemsko definitivně uznáno jako nezávislý stát
    \item Nizemsko tedy silný nezávislý stát s velice dobrou ekonomikou - začátky kapitalismu - Nizozemsko začíná kolonizovat Asii, atd.
    \item toto tedy vše severní Nizozemí, jižní Nizozemí (dnes tedy Belgie) zůstávají nadále pod habsburskou nadvládou
  \end{itemize}
\end{itemize}

\part{Anglie za Tudorovců}
\begin{itemize}
  \item 1485 - 1603
  \item spojení rodu Lancasterů a rodu Yorku po válce růží (viz dříve) - Jindřich VII. Lancaster a Alžběta z Yorku
  \item po válce růží se stát konečně stabilizuje, postupně se ze zemědělské země stává řemeslnická, obchodní, a to kvůli ohrazování - zabírají zemědělcům půdu, dávají do velkých pozemků a na těchto ve velkém začne nová šlechta (stará vybita při válce, nová necití, že by podnikání bylo pod jejich úroveň) pěstovat ovce, ale nevyvážejí přímo rouno, ale až výrobky z něho - big profit, začínají se stavět manufaktury (soustředěné, rozptýlené), což vede k zefektivnění, zkvalitnění a zlevnění výroby - big big profit, rozvíjí se tedy i námořní obchod (Východoindická, Moskevská společnost), obchod s otroky, postupně se tedy zde nastavily podmínky kapitalistického podnikání
  \item \textbf{Jindřich VII. Tudor} - vyhrál válku růží, král Anglie, protože ve válce vymřela šlechta tak měl malou opozici, opíral se o šlechtu novou kterou i sám částečně jmenoval, významnou oporou taky arcibiskup/kancléř John Morton, dále \textit{Yeomani} - svobodní vlastníci půdy, kteří měli právo nosit zbraň, rekrutovala se z nich pěchota, taky tzv. \textit{gentry} - nižší, převážně venkovská šlechta (popř. obchodníci), nevalnou pověst měla tzv. \textit{hvězdná komora} - soud pro souzení těch, co jsou proti králi, nejvyšší územně správní jednotkou jsou hrabství, základní jednotkou jsou farnosti, zavedl daně, které sloužily k obnově státu, buduje a rozšiřuje flotily, buduje doky, měl po něm nastoupit syn Arthur, ale ten zemřel
  \item \textbf{Jindřich VIII.} (1509 - 1547) - mladší syn Jindřicha VII., vzdělaný (mluví několika jazyky), hudebník, sportovec (renesanční člověk? megamind-no-bitches), oporou kardinál/kancléř Thomas Wolsey, jako kancléře ho pak vyměnil za Thomase Moorea (oba pak nechal uvěznit/popravit), za něj základy kapitalismu, ohrazování, manufaktury (viz výše), první manželka Kateřina Aragonská (dcera zakl. Španělska), která mu nedala syna (jen dceru), Jindřich tedy chtěl u papeže toto manželství zrušit, ale papež mu to nepovolil, takže se Jindřich obrátil na anglické duchovenstvo, odřekl poslušnost papeži, udělal si svoji církev a canterburský arcibiskup toto manželstí pak tedy prohlásil za neplatné $\Rightarrow$ anglikánská církev (1533), pak si bere Annu Boleynovou, s ní má jen dceru Alžbětu, takže ji popraví (obviní ji z cizoložství, incestu), pak si bere Janu Seymourovou, která mu dá syna, ale umře, pak si bere Annu Klévskou, s tou jen šest měsíců, jakmile dojela tak se ní rozvedl, protože po neštovicích se mu nelíbila, pak si bere Kateřinu Howardovou, tu popravil, pak si bere Kateřinu Parrovou, která ho přežívá
  \item \textbf{Jindřich IX.} - umírá mladý, po čtyřech letech vlády
  \item \textbf{Marie I.} (Krvavá) -
\end{itemize}
AAAAAA 22.3. DODĚLÁNÍ TUDOROVCŮ, ZAČÁTEK RUSKA

\part{Rusko}
\begin{itemize}
  \item Kyjevská Rus se rozpadla, pak to ovládli Mongolové, východní Slované tedy zaostávají
  \item \textbf{Ivan I. Kalita} -
  \item \textbf{Dmitrij Donský} -
  \item \textbf{Ivan III.} - jako prvnímu se mu povedlo zbavit nadvlády Mongolů (1480), je to moskevský kníže, Moskva dobývá ostatní ruské knížectví, buduje Moskvu jako \uv{třetí Řím} -- nasledníka padlé Byzance, buduje se administrativa, základní právní normy, jednotné velení armády, národním jazykem je ruština, buduje se sídlo ruských vládců (předtím ze dřeva lol) -- Kreml, oženil se s neteří posledního Byzantského císaře, \textit{votčina} -- dědičná půda \textit{bojarů} -- šlechticů, \textit{poměstí} - nedědičná půda, kterou dostávala nižší a střední šlechta za zásluhy
  \item \textbf{Ivan IV. Hrozný} (Výhružný lol) - prohlásil se za prvního ruského cara, \textit{samoděržaví} -- ruský způsob absolutismu, snažil se oslabit bojary, reformuje soudy, stát, armádu, dobijí AAAAAAAIU
\end{itemize}

AAAAAA rusko zbytek, celá anglie, ludvík xiv.

\part{Ludvík XIV. (via robot)}

\section{Úvod}
\begin{itemize}[label=$\bullet$,itemsep=2pt]
  \item Ludvík XIV., známý také jako \textcolor{highlight}{Král Slunce}, byl jedním z nejvlivnějších a nejdéle panujících francouzských králů v historii.
  \item Narodil se 5. září 1638 ve Versailles a zemřel 1. září 1715 ve stejném paláci.
  \item Jeho dlouhé panování trvalo 72 let, od roku 1643 až do jeho smrti.
  \item Ludvík XIV. byl symbolem \textcolor{highlight}{absolutistické monarchie} a jeho vláda měla hluboký vliv na politiku, kulturu a umění v Evropě.
\end{itemize}

\section{Rané roky}
\begin{itemize}[label=$\bullet$,itemsep=2pt]
  \item Ludvík XIV. se stal králem ve věku čtyř let po smrti svého otce Ludvíka XIII.
  \item Regentství se ujala jeho matka \textcolor{highlight}{Anna Rakouská}, která vládla jako regentka až do roku 1651.
  \item Byl vychováván kardinálem \textcolor{highlight}{Mazarinem}, který měl na něj velký vliv.
\end{itemize}

\section{Absolutistická vláda}
\begin{itemize}[label=$\bullet$,itemsep=2pt]
  \item Po smrti kardinála Mazarina v roce 1661 Ludvík XIV. převzal plnou moc a rozhodl se prosadit \textcolor{highlight}{absolutistický režim}.
  \item Jeho slavné prohlášení "Stát jsem já" (\textit{L'État, c'est moi}) zdůrazňovalo jeho roli jako jediného vládce Francie.
  \item Během své vlády se Ludvík XIV. snažil posílit královskou moc tím, že omezil vliv šlechty a centralizoval správu.
\end{itemize}

\section{Versailles}
\begin{itemize}[label=$\bullet$,itemsep=2pt]
\item Jedním z nejvýznamnějších dědictví Ludvíka XIV. je výstavba paláce ve \textcolor{highlight}{Versailles}.
\item Původně lovecký zámeček se pod jeho vedením rozrostl na monumentální komplex, který se stal centrem politického a kulturního života ve Francii.
\item Versailles bylo symbolem \textcolor{highlight}{luxusu} a \textcolor{highlight}{moci}, kde se konaly velké dvorské slavnosti a události.
\end{itemize}

\section{Válečné tažení}
\begin{itemize}[label=$\bullet$,itemsep=2pt]
\item Ludvík XIV. byl také \textcolor{highlight}{válečnickým králem} a vedl několik významných válečných tažení během svého panování.
\item Jeho cílem bylo rozšířit francouzskou moc a získat územní zisky.
\item Mezi jeho nejznámější války patří \textcolor{highlight}{Devítiletá válka} (1688-1697) a \textcolor{highlight}{Válka o španělské dědictví} (1701-1714).
\end{itemize}

\section{Podpora kultury a umění}
\begin{itemize}[label=$\bullet$,itemsep=2pt]
\item Ludvík XIV. byl známý také pro svou podporu \textcolor{highlight}{kultury} a \textcolor{highlight}{umění}.
\item Během jeho vlády se Francie stala kulturním centrem Evropy.
\item Ludvík XIV. osobně podporoval literaturu, hudbu, divadlo a výtvarné umění.
\item Na jeho dvoře působili významní umělci, jako byli spisovatelé \textcolor{highlight}{Molière} a \textcolor{highlight}{Jean Racine}, hudební skladatel \textcolor{highlight}{Jean-Baptiste Lully} a malíř \textcolor{highlight}{Charles Le Brun}.
\end{itemize}

\section{Náboženská politika}
\begin{itemize}[label=$\bullet$,itemsep=2pt]
\item Ludvík XIV. se také angažoval v \textcolor{highlight}{náboženské politice}.
\item Za jeho vlády došlo k revokaci \textcolor{highlight}{Ediktu nantského}, který zaručoval náboženskou svobodu hugenotům.
\item To vedlo k pronásledování a útlaku protestantů ve Francii.
\item Ludvík XIV. prosazoval \textcolor{highlight}{katolickou ortodoxii} a usiloval o sjednocení víry ve své říši.
\end{itemize}

\section{Závěr}
\begin{itemize}[label=$\bullet$,itemsep=2pt]
\item Ludvík XIV. zanechal nezmazatelné stopy ve francouzské historii.
\item Jeho dlouhé panování, politické reformy, válečné tažení a podpora kultury ho učinily jedním z nejvýznamnějších panovníků své doby.
\item I když byl považován za absolutistického vládce, jeho odkaz a vliv na Francii a Evropu přetrvaly i po jeho smrti.
\end{itemize}

\part{Habsburská monarchie po třicetileté válce}

\section{Ferdinand III.}
\begin{itemize}
  \item 1637 - 1657
  \item přebírá dokončuje třicetiletou válku
  \item dochází k soupisu poddaných po třicetileté válce, dochází k soupisu všech obcí -- berní rula (katastr) -- měl přesnější podklady k vybírání daní
  \item syn nástupce mu umírá nastupuje po něm další syn Leopold I.
\end{itemize}

\section{Leopold I.}
\begin{itemize}
  \item 1657 - 1705
  \item současník, protivník a obdivovatel Ludvíka XIV.
  \item měl být původně církevník, ale succession
  \item dlouhá vláda, upevňuje absolutismus po třicetileté válce
  \item upevňuje robotní povinnosti (protože po válce je málo pracovníků) -- robotní patent (1680) -- poddaní mohou robotovat max. 3 dny v týdnu, kvůli tomuto jsou povstání
  \item povstání Chodů (1692-95) -- Jan Sladký Kozina -- Chodům jsou postupně odebírána privilegia
  \item povstání Uhrů
  \item dále válčí s Turky, od r. 1683 se jim začíná dařit, ale neměli na válku peníze takže uzavírají 1599 karlovický mír, dále dobývá Turky Evžen Savojský
  \item válka o španělské dědictví
\end{itemize}

\section{Josef I.}
\begin{itemize}
  \item 1705 - 1711
  \item další uherské povstání které končí Szátmárským mírem (1711) -- uherští stavové si ponechají privilegia, stavové uznají dědičné právo Habsburků n uherský trůn
  \item pokračuje válku o dědictví španělské
  \item umírá mladý, nastupuje jeho bratr
\end{itemize}

\section{Karel VI.}
\begin{itemize}
  \item 1711 - 1740
  \item dotahuje válku o španělské dědictví
  \item neúspěšně válčí s Turky
  \item Pragmatická sankce (1713) -- dědici i ženy
\end{itemize}

\section{Marie Terezie}
\begin{itemize}
  \item války o rakouské dědictví (je to žena)
  \item manžel František Štěpán Lotrinský
  \item AAAAAAAAA spánek omlouvám se, ukradnu od Dominika (sike)
  \item podívej se na seriál od ČT, stojí to za to
  \item hodně reforem např.
  \begin{itemize}
    \item katastry, aby se mohly dobře vybírat daně
    \item úřední sčítání lidu, aby se mohly dobře vybírat daně
    \item číslování domů, tak jak jej máme dnes
    \item povinné přijmení
    \item společné úřady (to jsem asi nepochopil AAAAAAAA)
    \item třístupňový státní aparát -- ústřední (Vídeň), gubernia (země), magistráty (kraje, města)
    \item Nejvyšší soudní dvůr ve Vídni -- humánnější (jak se to píše), tereziánský trestní zákoník -- často zmírněné nejvyšší tresty
  \end{itemize}
  \item naše země byly středem zájmu Marie Terezie
  \begin{itemize}
    \item zakládání manufaktur u nás
    \item konec cla mezi českými zeměmi a Rakouskem
    \item omezování cechů, podpora podnikání, protekcionalismus -- děláme to u nás, nedovážeje odjinud a podporujme si ekonomiku
    \item budování silnic, poštovního spojení, sjednocení měn, mír a vah, první papírové bankovky (bankocetle)
    \item tzv. jetelová revoluce -- už pěstujeme dost, abysme měli jídlo pro dobytek i na zimu
    \item pěstování brambor (miluju)
  \end{itemize}
\end{itemize}

\section{Josef II.}
\begin{itemize}
  \item od smrti Štěpána císař SŘŘ, spoluvládce MT
  \item reformy armády, stavba pevností
  \item občanský zákoník -- zrušení trestu smrti, zločinu čarodějnictví
  \item celkově osvícenská, reformní vláda, snaží se zemi \uv{zesvětštit}
  \item zavedl povinnost vedení matrik
  \item 1781 -- Toleranční patent (legalizujeme lutherány, kalvinisty a pravoslaví) a Patent o zrušení nevolnictví (zrušení nevolnictví)
\end{itemize}

\section{Leopold II.}
\begin{itemize}
  \item ruší část Josefových reforem, více konservativní
  \item jeho syn František v SŘŘ II., ale Napoleon, čili jako rakouský císař František I.
\end{itemize}

\part{Rusko v 17. a 18. st.}
\begin{itemize}
  \item musíme navázat kde jsem skončili (zapomněr), čili Ivanem IV., prvním carem
  \item po Ivanovi IV. nastupuje slaboyslný Fjodor, vládne za něj švagr Boris
  \item nastává období krize, tzv, smuty, r. 1613 je zvolen carem Michail Romanov $\rightarrow$ dynastie Romanovců
\end{itemize}

\section{Michail Romanov}
\begin{itemize}
  \item 1618-
  \item absolutistická vláda (samoděržaví), význam má pravoslavná církev, drtivá většina společnosti jsou nevolníci, zemědělský zaostalý stát, nízká úroveň vzdělanosti
  \item je patrné, že jsou potřeba reformy napříč celou společností -- ekonomické, správní, kulturní, atd.
  \item dál probíhá kolonizace Sibiře, zakládají se pevnůstky, těží se suroviny, pronikají až k Tichému oceánu
  \item kozáci -- svobodní lidé, kteří bránili jihoruské stepi (Ukrajina) od Tatarů, za doby Polsko-litevského soustátí (dále kozáci i v Rusku na Donu, na Sibiři), měli několik povstání proti Polsku, v tom pro nás důležitém, Bohdana Chmelčickho, území levého břehu Dněpru (vč. Kyjiva) připojena k Rusku
\end{itemize}

\section{Petr Veliký}
\begin{itemize}
  \item 1689-1725
  \item přeskočili jsme dva cary btw
  \item energický, reformátor, ale velmi krutý (popravil si syna, tvrdě potlačil několik povstání), obdivuje západní Evropu
  \item na začátku vlády regentka sestra Sofie, tu r. 1689 sesazuje, pár let potom umírá bratr, tedy je jediným vládcem
  \item boje s Turky vyhrává, pak zase prohrává
  \item vyráží tajně do Evropy, dělá normální práce -- zjišťuje jak to tam funguje, za doby jeho nepřítomnosti vypukne povstání dělostřelců, ale potlačeno
  \item dosáhl Baltského moře díky tzv. severní válce (1700-1721) -- všichni (Dánsko, Polsko, Rusko, Sasko) se spojili proti Švédsku, prvně Švédové vyhrávají, pak se to obrací, končí Nystadským mírem -- Švédsko musí postoupit velká území Rusku,
  \item r. 1703 zakládá Petrohrad (1712 hlavní město)
  \item zavádí stálou armádu, buduje se válečné námořnictvo (Balt), zavádí se školství, buduje se infrastruktura, zakládají se manufaktury, sjednocena měna, zve do Ruska odborníky, nový samosprávný systém, reforma písma,
\end{itemize}

\section{Kateřina}
\begin{itemize}
  \item 1762-1796
  \item zase jsme někoho přeskočili, idk
  \item AAAAAAAAA viz Dominik
\end{itemize}

\part{Polsko-litevská unie}
\begin{itemize}
  \item šlechta měla obrovské pravomoce, král volený
  \item AAAAAA viz Domink /zase
\end{itemize}

\part{Prusko}
\begin{itemize}
  \item AAAAAAAAA
\end{itemize}

\part{Severní Amerika}
\begin{itemize}
  \item pro Španěly a Portugalce neatraktivní, dominují Nizozemsko, Británie a Francie
  \item na východním pobřeží anglické kolonie (13 kolonií), už od dob Alžběty první osady, ve Virginii na ostrově Roanoke osada, neúspěšná, dále Jamestown
  \item připluli jsem tzv. otcové poutníci -- puritáni, kteří opouštěli Anglii v době vlády Stuartovců, založili Boston
  \item z této doby i Den díkůvzdání -- indiáni jim pomohli při první zimě
  \item na východním pobřeží jsou i Nizozemci (Nový Amsterdam), to propadlo britům
  \item při sedmileté válce propadají britům i francouzské kolonie dnešního Quebecu, Francouzům zůstává Louisiana
  \item 13 kolonií -- každá kolonie má samosprávu, v čele volení zástupci -- guvernéři, kolonie neměly zastoupení v parlamentu, kolonie se rozšiřovaly, zakládaly školy, což se nelíbilo britům
  \item po sedmileté válce, kterou Angličané zvítězili, měli ale málo peněz, zvyšují americké daně, zavádějí cla, atd., osady se chtěly víc rozvíjet, a to samostatně
  \item rozdíly mezi severními a jižními koloniemi -- na severu menší farmáři, spíš pěstují věci na jídlo, rybaří, vyváží rudu; na jihu velké farmy, plantáže s otroky černochy, pěstovaly se plodiny spíš na profit
  \item v Anglii vládne Jiří III., nemají moc peněz kvůli válce takže
  \begin{itemize}
    \item 1765 -- tzv. zákon o kolkovném -- listiny musí být opatřeny kolkem, za který se platí
    \item 1767 -- cla -- na věci, které se sem dováží z Anglie, např. papír, čaj, sklo, atd.
    \item 1770 -- \uv{Bostonský masakr} -- střelba vojáků na kolonisty (po nějaké provokaci asi)
    \item 16.12.1773 -- ,,Bostonské pití čaje" -- vyhození krabic čaje do moře na odpor britům, briti uzavřeli bostonský přístav, do Ameriky poslali vojsko a zástupci z osad se sešli
  \end{itemize}
  \item 1774 -- 1. kontinentální kongres -- osady chtějí mít zastoupení, píší Jiřímu petice, ale on neodpovídá, takže se kolonisté pomalu začali připravovat na válku
  \item 1775 -- vypukl ozbrojený konflikt mezi vojáky a kolonisty (lexington a Concorde), vojáci chtěli kolonisty odzbrojit, oni se nenechali, osadnící staví improvizovanou armádu, v čele s Georgem Washingtonem
  \item 1776 -- osadnící dobývají Boston
  \item 1775-1781 -- II. kontinentální kongres -- už je to nebaví a vyhlašují nezávislost
  \item 4.7.1776 -- Prohlášení nezávislosti, tedy začátek americké války za nezávislost -- 1776-1783
\end{itemize}


\end{document}
