\documentclass{article}
\usepackage{fullpage}
\usepackage[czech]{babel}
\usepackage{amsfonts}

\title{\vspace{-2cm}3. a 4. fáze Národního obrození\vspace{-1.7cm}}
\date{}
\author{}

\begin{document}
\maketitle



Josef Kajetán Tyl
vzor Klicpera -- divadlo, 1848 členem říšského sněmu
činnost žurnalistická: časopis Květy (výchovná funkce, hodně lidí se tímto naučilo číst)
činnost literární: povídky
vlastenecké ze současnosti: Poslední Čech (Borovský kritizuje jako příliš vlastenecké), Rozervanec (obraz K. H. Máchy)
historické: Dekret Kutnohorský
činnost dramatická: Kajetánské divadlo, Stavovské divadlo
soudobé hry: Fidlovačka, aneb žádný hněv a žádná rvačka (jako Romeo a Julie, ale končí dobře) -- píseň Kde domov můj? (státní hymna, hudba: František Škroup);
Paličova dcera (zapálí chalupu, Rozárka vezme vinu na sebe)
historické hry: aktualizované = snažil se poukázat na soudobý problém; Kutnohorští kováři (stěžují si na nízké mzdy -- zatčeni -- poprava, 1 uteče a informuje krále, pozdě); Jan Hus (Hus s myšlenkami 1848)
dramatické báchorky: Strakonický dudák (chce si vzít Dorotku, vydělává hrou na dudy, válka -- cizina, poražen, Dorotka ho jde hledat; má možnost velkého bohatství a zapomene, proč se vlastně vydal do ciziny)

Karel Hynek Mácha (1810-1836)
v tomto období u nás jediný romantik, neorientuje se na tradice nebo lidové vlastenectví
narozen v Praze v chudé rodině, zemřel v Litoměřicích, na gymnáziu žák Jungmanna
poté VŠ v Praze (práva), do advokátní kanceláře v Litoměřicích; v divadle J. K. Tyla
miloval české hrady a zříceniny, milenka Marinka Štichová, po smrti tragická tvorba
druhá milenka Eleonora Šomková, zakazuje jí hrát v divadle, žárlí; otěhotní, syn Ludvík
pomáhal hasit požár, napil se infikované vody, zemřel na dehydrataci kvůli průjmu
pohřeb v den, kdy měl mít svatbu. pomník na Vyšehradě
poezie: první básně v němčině, později česky, Máj (jediná kniha, který vyšla za jeho života, špatně přijat jako málo vlastenecký, později však přehodnocena jako přelomová)
próza: bývá epická, ale propojena s lyrickými prvky; lyrizovaná próza Obrazy ze života mého (2 části: večer na Bezdězu, Marinka – příběh o lásce k dívce, ideál krásy, vykoná pouť do Krkonoš, při návratu je Márinka mrtvá kvůli nemoci); básnická próza Pouť krkonošská (až horor, na vrcholu Sněžky je chrám s mrtvými mnichy, kteří vstávají 1 za rok, ale nemohou vyjít z kláštera); povídka Cikáni (částečně připomíná Máj, postava z okraje spol., starý a mladý cikán mají milenky, jsou opuštěni, cikán trestá toho, co je svedl, je to však otec mladého cikána, starý cikán je za jeho vraždu popraven); pokus o román Kat (jednotlivé díly podle hradů, jen 1. díl místo 4, propojeno s historickou tematikou, děj za Václava IV., kat je levobočkem posledního Přemyslovce, inteligentní, Milada ho miluje, ale musel popravit jejího otce, poté umírá i Milada)

Karel Jaromír Erben (1811-1870)
hudební talent, učil hudbu, problémy: TBC, finance
žena, syn zemřel, tři dcery; žena brzy umírá -- ožení se podruhé, další dvě děti
1850 sekretář a archivář Národního muzea, poté arch. města Prahy – dostal se k RKZ, ÚLS
sběratelská činnost: balady, povídky, říkadla, písně, zapisuje je
myšlenky Herdera: v ÚLS je národní duch, na zákl. toho psal autorské pohádky: Tři zlaté vlasy děda vševěda, Dlouhý, široký a bystrozraký -- zlidověly
Kytice – nejdůležitější životní dílo, nejoblíbenější bás. sbírka 19. století (sláva za života), tresty (křesťanská morálka), balady, ale i prvky bájí nebo pověstí, odpuštění Boha jakožto jediná cesta od zkázy, motivy mateřské lásky, inspirace pro mnoho tvůrců

Čtvrtá fáze
Alexandr Bach -- bachovský absolutismus

Božena Němcová (1820-1862)
její narození je záhada, možná dcerou šlechtičny, rodné jméno Panklová
východní Čechy (Ratibořice), prostředí Babičky, vyrůstala tam
průkopnice ženské prózy českého jazyka
v 17 provdána za Josefa Němce (úředník, o dost starší, sňatek z rozumu, ne lásky)
po smrti nejstaršího syna Hynka krize v manželství
propojuje realismus a romantismus, romantické postavy, v duchu českého rom. konec, idyla (pod vlivem ÚLS), realismus - soc. vrstvy, venkov
Obrazy: propojení dokumentaristiky s příběhy a situacemi ze života lidí, hodně reportážních prvků, důrazná pojmenování
Divá Bára (povídka): venkovská dívka, neustále u přírody, nevěří pověrám, vše si vyzkouší, je jiná než ostatní děti, nemá moc kamarádů, jen Elišku
Pan učitel (povídka): částečně autobiografický hold jejímu učiteli, dovedl ji ke knihám
Karla (povídka): Markyta se stane chudou vdovou, má syna, mnoho problémů; měla strach, že syna Karla odvedou na vojnu, vychovávalo ho jeho děvče Karla
Babička (obsáhlá povídka): Ratibořické údolí na Náchodsku v 30. letech 19. st., Staré bělidlo; začal psát, když jí zemřel syn -- vzpomínky na dětství; ne vše je pravda, real. v rozvrstvení spol., mnoho postav má jen dobré rysy, babička – Magdalena Novotná, Proškovi – Panklovi, babička: vlastenkyně, výchovná fce, vlastenecké příhody, idealizovaná; děti Barunka (BN, nejučenější, naslouchá), Vilém, Jan, Adélka, Viktorka (čistě rom. postava z okraje spol., zamiluje se do vojáka, jde s ním do světa, vrátí se zblázněná, žije v jeskyni, porodí dítě, hodí ho do splavu); panské postavy: sluhové – negativní, kněžna pozitivní, oblíbí si babičku; příběh končí úmrtím babičky, chudé postavy mají dobré vlastnosti (až na maminku, která se chová pansky); kompozice = jako jeden cyklus roku, ve skutečnosti uběhne 12 let + vložený epizodní příběh o Viktorce

Karel Havlíček Borovský (1821-1856)
Borovský je přídomek, protože se narodil v Borové u Havlíčkova Brodu
básník, velmi satirický, překladatel, vychovatel v Moskvě -- námět na dílo Obrazy z Rus
tři satirické skladby, částečně nedokončené:
Král Lávra: vychází ze staré pohádky o irském králi s oslíma ušima; nechá se ostříhat a popraví holiče, jednou vylosují syna chudé ženy, slíbí, že ho nepopraví, on ale vyžvaní vrbě, že má oslí uši, hudebníci si z ní udělají kolíček na basu a poté na královském hradě basa zpívá: “král Lávra má oslí uši”
Křest sv. Vladimíra: vychází z Nestorova rukopisu, je o přijetí křesťanství v KR, o bohu Perunovi, co se mu stalo, KR nemá boha -- konkurz, nedokončeno -- Č 150
Tyrolské elegie: zachycuje své zatčení, deportaci do Brixenu, co se odehrálo cestou (splašili se koně)
epigramy (= krátké satirické básně, původně na náhrobcích, kritika jevu ve spol., pointa), psal je celoživotně, uspořádané do souborů, které jsou věnovány: církvi, králi, vlasti, múzám, světu









\part{3. fáze Národního obrození}
\begin{itemize}
  \item 1830 - 1848
  \item převládá romantismus; čeština ustálena, i odborná a národní literatura
  \item ve 30. letech umí 90 \% obyvatelstva číst a psát, myšlenky NO se rozšířily všude
  \item 1830 povstání Poláků, pomoc i z Česka -- starší stojí za slovanským Ruskem, mladší nesouhlasí s absolutistickým carem
  \item rozvíjení společenského života, hrdost na české hrady, přírodu
  \item kancléř Metternich, \uv{Metternichův absolutismus} $\rightarrow$ Češi nechtějí vybočovat a provokovat, obrozenci tvoří kulturu a o politiku či práva se nezabývají
  \item biedermeier -- životní styl měšťanstva, které se nebouří, pohodlně žije, organizují spol. akce
  \item žánry
  \begin{itemize}
    \item povídky, kalendářové povídky, básnické povídky, novely, balady
    \item dramatické báchorky -- divadelní hra se soudobým dějem, nadpřirozeno, hudba / zpěvy, polepšení hl. hrdiny
    \item obrazy = až publicisticky popisují dění a krajinu třeba na venkově
  \end{itemize}
  \item český romantismus jiný než světový (až na Máchu) –- oživování ÚLS, české kultury celkově, vymaňování se němčině, německé kultuře
\end{itemize}

\section{Josef Kajetán Tyl AAAAAAA}

\section{Karel Hynek Mácha AAAAAAA}
\begin{itemize}
  \item 1810 -- 1836
  \item item Jungmannův žák na německém gymnáziu, vystudoval práva
  \item působil chvíli v Tylově divadle (zde poznal Lori -- viz níže)
  \item \uv{světový} romantik -- tedy nejen NO
  \begin{itemize}
    \item postavy z okraje společnosti, vnitřní rozpory, líčení pocitů
    \item inspirace zahraničními romantiky (Byron)
  \end{itemize}
  \item psal poezii i prózu, pokoušel se i o drama, stěžejní dílo Máj
  \begin{itemize}
    \item poezie
    \begin{itemize}
      \item první básně v němčině, pak v češtině
    \end{itemize}
    \item próza -- velmi často propojena s lyrickými prvky, povídky
    \begin{itemize}
      \item Obrazy ze života mého
      \begin{itemize}
        \item dvojice próz s autobiografickými prvky, asi chtěl napsat další, ale nestihl
        \item Večer na Bezdězu, Márinka -- AAAAAAA
      \end{itemize}
      \item Pouť krkonošská -- \uv{první český horor}, AAAAAAA
      \item Cikáni -- AAAAAAA
      \item Kat - měla být tetralogie o hradech, stihl jen první díl -- AAAAAAA
    \end{itemize}
    \item Máj -- AAAAAAA
  \end{itemize}
\end{itemize}

\section{Karel Jaromír Erben AAAAAAA}
\begin{itemize}
  \item 1811 -- 1870
  \item život AAAAAAA
  \item inspiruje se a sbírá ÚLS
  \item jeho pohádky, které zlidověly Tři zlaté vlasy Děda Vševěda, Dlouhý, Široký a Bystrozraký
  \item sbírka básní Kytice
  \begin{itemize}
    \item oblíbená již ve své době
    \item inspirace lidovovou epikou -- prvky báje, pověsti, pohádky, proroctví, legendy
    \item vyzdvihuje tradiční křesťanské hodnoty (čestnost, morálku), ale čerpá z pohanských bájí a pověr (vodník, polednice)
    \item silný motiv mateřství, mateřské lásky
    \item až horrorové motivy
    \item inspirace dodnes
  \end{itemize}
\end{itemize}

\part{4. fáze Národního obrození}
\begin{itemize}
  \item 1850 -- 1860
  \item literatura více propojená se životem
  \item převládá realismus, stále ale výrazný vliv romantismu
  \item Božena Němcová, Karel Havlíček Borovský
  \item období \uv{Bachova absolutismu}
\end{itemize}

\section{Božena Němcová AAAAAAA}


\section{Karel Havlíček Borovský}
\begin{itemize}
  \item 1821 -- 1856
  \item narodil se v Borové u Německého (dnes Havlíčkova) brodu, příjmení Havlíček, Borovský přídomek
  \item kritik, např. Tyla (povídku Poslední Čech), básník, hodně satirický, taky překládal
  \item v jeho rodině se mluvilo německy, studoval ne německém gymnáziu, ale čeština se mu líbila, takže používla češtinu častěji než němčinu
  \item určitou dobu působil jako vychovatel v Moskvě, toto inspirace pro dílo Obrazy z Rus, kde je velice kritický k podmínká carského Ruska
  \item napsal zejména tři velké satirické skladby, které zůstaly částečně nedokončené
  \begin{itemize}
    \item Král Lávra -- vychází ze staré pohádky, báje o irském králi s oslíma
    ušima, stejný motiv už se objevuje v řeckých pověstech o králi Midasovi -- říká se, že existuje dobrý, hodný král, ale má jeden zvláštnost -- jednou za rok se nechává stříhat a svého kadeřníka nechá vždy popravit, jednou měl jít stříhat syn nějaké chudé vdovy, ona šla prosit krále, aby jejího syna nezabíjel, on ho tedy ušetřil, ale musel mu slíbit, že po stříhání o tom nikomu nebude povídat, on to nevydržel a svěřil se vrbě, někdy později nějaký muzikant si z té vrby udělal kolíček do basy a ta basa začala hrát o tom, co jí kadeřník svěřil, že král má oslí uši, které schovává pod dlouhými vlasy a korunou, ale že se jednou za rok nechává stříhat a proto nechává popravovat ty kadeřníky, aby to nikomu nevyzradili -- kritika cenzury, absolutismu a lidí, co si to nechají líbit
    \item Křest svatého Vladimíra -- vychází z Nestorova letopisu z 12. st., ten pojednával o přijetí křesťanství na Kyjevské Rusi, dílo je vtipné a velmi satirické, ale nedokončené, jde o pohanského boha Peruna a o tom, jak odmítl požadavek vládce a ten ho tedy nechal popravit, Kyjevská Rus se ocitne bez boha a vyhlásí konkurz na boha kam přicházejí různí bohové se o tuto pozici ucházet, vyhrává bůh křesťanů
    \item Tyrolské elegie -- zachycuje okamžik svého zatčení (pro své názory, vyjadřování nesouhlasu s cenzurou, s národnostním útlakem zatčen a deportován) a o tom co se odehrávalo cestou do Brixenu -- po zatčení měl asi jen dvě hodiny se sbalit, rozloučit se s dcerou, zajímavá událost, že koně se splašily, všichni kromě Havlíčka vyskočili, on koně zastavil a potom počkal až potlučení policisté a kočí doběhnou a pokračují v cestě -- snaží se poukázat asi i na neschopnost policie v Rakousku
  \end{itemize}
  \item dále také psal celý život epigramy (lyrický útvar zachycující různé situace lidského života s výraznou satirou, kritika nešvaru ve společnosti, s výraznou pointou), uspořádal je do několika souborů v pěti oddílech: církvi, králi, vlasti, múzám, světu
\end{itemize}

\part{Realismus ve světové literatuře}
\begin{itemize}
  \item 2. pol. 19. st.
  \item zachycuje věrně skutečnost, jde o skutečné zachycení prostředí postav, zachycení nečeho, co by se mohlo opravdu stát
  \item autoři hodně zkoumali příběhy které se opravdu staly, na základě toho vymýšleli příběhy vlastní, autoři nejdříve spoustu času studovali prostředí, ta místa o kterých chtějí psát, dobře promýšleli svět okolo postav a dopodrobna promýšleli jednotlivé postavy, tak aby to všechno bylo reálné, a to včetně např. zachycení jazyku postav uměrně jejim soc. postavení -- dialekty, argot, apod.
  \item vypravěč většinou nad příběhem a jazyk vypravěče bývá spisovný
  \item postavy bývají typizované, objevuje se typizace postav -- postava je typickým představitelem své dané soc. vrstvy, svého prostředí
  \item ale nechceme mít dvě stejné postavy, postavy mají samozřejmě i nějakou individuální složku, i přesto bývají typizované
  \item autoři se nevyhýbají (až je vyhledávají) prostředím, která nejsou příjemná -- o prostituci, o soc. slabých vrstvách, ale i o korupci, apod.
  \item svými díly často vyzdvihují různé společenské problémy a otevřeně kritizují jejich viníka, ať je jakýkoliv -- kritický realismus
  \item odnoží realismu je i naturalismus -- vyhrocený realismus, který říká, že podstata jednotlivce je víceméně dána a je neměnná, člověka tedy ovlivňují jen dědičnost a výchova, resp. vliv prostředí, úplně tedy pomíjejí, že by lidská vůle mohla být natolik silná, aby toto dokázala změnit (dnes, když se řekne, že je něco naturalistické tak je to přehnaně detailní a tyto detaily jsou až nechutné)
\end{itemize}

\end{document}
