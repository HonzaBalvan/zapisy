\documentclass{article}
\usepackage{fullpage}
\usepackage[czech]{babel}
\usepackage{amsfonts}

\title{\vspace{-2cm}3. a 4. fáze Národního obrození\vspace{-1.7cm}}
\date{}
\author{}

\begin{document}
\maketitle

\part{3. fáze Národního obrození}

\section{Josef Kajetán Tyl}

\section{Karel Hynek Mácha}

\section{Karel Jaromír Erben}

\part{4. fáze Národního obrození}

\section{Božena Němcová}

\section{Karel Havlíček Borovský}
\begin{itemize}
  \item 1821 -- 1856
  \item narodil se v Borové u Německého (dnes Havlíčkova) brodu, příjmení Havlíček, Borovský přídomek
  \item kritik, např. Tyla (povídku Poslední Čech), básník, hodně satirický, taky překládal
  \item v jeho rodině se mluvilo německy, studoval ne německém gymnáziu, ale čeština se mu líbila, takže používla češtinu častěji než němčinu
  \item určitou dobu působil jako vychovatel v Moskvě, toto inspirace pro dílo Obrazy z Rus, kde je velice kritický k podmínká carského Ruska
  \item napsal zejména tři velké satirické skladby, které zůstaly částečně nedokončené
  \begin{itemize}
    \item Král Lávra -- vychází ze staré pohádky, báje o irském králi s oslíma
     ušima, stejný motiv už se objevuje v řeckých pověstech o králi Midasovi -- říká se, že existuje dobrý, hodný král, ale má jeden zvláštnost -- jednou za rok se nechává stříhat a svého kadeřníka nechá vždy popravit, jednou měl jít stříhat syn nějaké chudé vdovy, ona šla prosit krále, aby jejího syna nezabíjel, on ho tedy ušetřil, ale musel mu slíbit, že po stříhání o tom nikomu nebude povídat, on to nevydržel a svěřil se vrbě, někdy později nějaký muzikant si z té vrby udělal kolíček do basy a ta basa začala hrát o tom, co jí kadeřník svěřil, že král má oslí uši, které schovává pod dlouhými vlasy a korunou, ale že se jednou za rok nechává stříhat a proto nechává popravovat ty kadeřníky, aby to nikomu nevyzradili -- kritika cenzury, absolutismu a lidí, co si to nechají líbit
    \item Křest svatého Vladimíra -- vychází z Nestorova letopisu z 12. st., ten pojednával o přijetí křesťanství na Kyjevské Rusi, dílo je vtipné a velmi satirické, ale nedokončené, jde o pohanského boha Peruna a o tom, jak odmítl požadavek vládce a ten ho tedy nechal popravit, Kyjevská Rus se ocitne bez boha a vyhlásí konkurz na boha kam přicházejí různí bohové se o tuto pozici ucházet, vyhrává bůh křesťanů
    \item Tyrolské elegie -- zachycuje okamžik svého zatčení (pro své názory, vyjadřování nesouhlasu s cenzurou, s národnostním útlakem zatčen a deportován) a o tom co se odehrávalo cestou do Brixenu -- po zatčení měl asi jen dvě hodiny se sbalit, rozloučit se s dcerou, zajímavá událost, že koně se splašily, všichni kromě Havlíčka vyskočili, on koně zastavil a potom počkal až potlučení policisté a kočí doběhnou a pokračují v cestě -- snaží se poukázat asi i na neschopnost policie v Rakousku
  \end{itemize}
  \item dále také psal celý život epigramy (lyrický útvar zachycující různé situace lidského života s výraznou satirou, kritika nešvaru ve společnosti, s výraznou pointou), uspořádal je do několika souborů v pěti oddílech: církvi, králi, vlasti, múzám, světu
\end{itemize}

\end{document}
