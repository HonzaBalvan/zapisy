\documentclass{article}
\usepackage{fullpage}
\usepackage[czech]{babel}
\usepackage{amsfonts}

\title{\vspace{-2cm}Mluvnice, sloh\vspace{-1.7cm}}
\date{}
\author{}

\begin{document}
\maketitle
styly: prostěsdělovací, esejistický, odborný, umělecký, řečnický, administrativní, publicistický
postupy: informační, vyprávěcí, popisný, výkladový, úvahový

\part{Odborný styl}
\begin{itemize}
  \item[a)] vědecký -- pro vědeckou veřejnost, vysoce odborný, hodně termínů
  \item[b)] učební -- pro žáky, studenty, snažíme se o něčem vzdělat někoho nevzdělaného (idk)
  \item[c)] populárně naučný -- snaží se vzdělat širší veřejnost, minimum termínů nebo jsou vysvětleny
  \item[d)] prakticky odborný -- např. návody, příručky, nějaké praktické poznatky, které můžeme využít
\end{itemize}
\begin{itemize}
  \item komposice logická, strukturovaná, bude uvedena problematika, mělo by to být přehledné, ne na přeskáčku, nějak souvisle za sebou, na konci shrnutí (resumé)
  \item anotace, abstrakt, resumé
\end{itemize}
\end{document}
