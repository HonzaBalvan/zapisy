\documentclass{article}
\usepackage{fullpage}
\usepackage[czech]{babel}
\usepackage{amsfonts}

\title{\vspace{-2cm}Mluvnice, sloh\vspace{-1.7cm}}
\date{}
\author{}

\begin{document}
\maketitle
styly: prostěsdělovací, esejistický, odborný, umělecký, řečnický, administrativní, publicistický
postupy: informační, vyprávěcí, popisný, výkladový, úvahový

\part{Odborný styl}
\begin{itemize}
  \item[a)] vědecký -- pro vědeckou veřejnost, vysoce odborný, hodně termínů
  \item[b)] učební -- pro žáky, studenty, snažíme se o něčem vzdělat někoho nevzdělaného (idk)
  \item[c)] populárně naučný -- snaží se vzdělat širší veřejnost, minimum termínů nebo jsou vysvětleny
  \item[d)] prakticky odborný -- např. návody, příručky, nějaké praktické poznatky, které můžeme využít
\end{itemize}
\begin{itemize}
  \item komposice logická, strukturovaná, bude uvedena problematika, mělo by to být přehledné, ne na přeskáčku, nějak souvisle za sebou, na konci shrnutí (resumé)
  \item anotace, abstrakt, resumé
\end{itemize}

Slohovka bude na líčení
\begin{itemize}
  \item líčení je subjektivní popis něčeho
  \item je zde velké množství obrazných pojmenování, a to především přirovnání, metafor, personifikací
  \item velmi často tím, kdo tvoří ličení je vypravěč v první osobě -- jde o vastní pocity vypravěče -- líčení v ich-formě
  \item snaha zachytit atmosféru a náš vztah k popisovanému
  \item musíme ale pořád popisovat smyslově vnímatelné okolnosti
  \item musí z toho být cítit ta emoce, to subjektivno, ten popis toho něčeho
  \item může být zabarveno i negativně, ale je to těžší
  \item cvičné líčení: oblíbené místo, hluboká noc, bouře v horách
\end{itemize}

Oblíbené místo

Když mám přemýšlet o tom, co je moje oblíbené místo, odpověď nachazím brzy. Je to můj domov. Můj rodný dům, můj pokoj, jistota místa, kam se můžu navrátit kdykoliv, po jakkoliv dlouhé cestě.

Je to starý dům se silnými zdmi, s hnědou střechou a béžovou fasádou, který vždy rád vidím, a když odemykám jeho bílé dveře a vítá mě můj pes, když vejdu přes síň do teplého a útulného obývacího pokoje můžu si oddechnout a hodit veškeré starosti dne za hlavu.
\end{document}
