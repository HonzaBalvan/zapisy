\documentclass{article}
\usepackage{fullpage}
\usepackage[czech]{babel}
\usepackage{amsfonts}

\title{\vspace{-2cm}Politologie\vspace{-1.7cm}}
\date{}
\author{}

\begin{document}
\maketitle

\begin{itemize}
  \item věda o vládnutí, orgnaizaci a stavbě státu
  \item již od starověku, ve středověku např. Niccoló Machiavelli (a jeho kniha Vladař), od r. 1880 katedra politologie na Kolumbijské universitě, jako separátní specializovaná věda po druhé sv. v.
  \item politologie se zabývá politikou (umění řídit stát), politickými stranami, zájmovými skupinami, politickou mocí (a jak by měla fungovat), státem (jak má vypadat, kdo ho má řídit ap.), politickými systémy, politickým chováním lidí, jejich míněním o politice, politickými ideologiemi a teoriemi, mezinárodními vztahy a organizacemi
  \item základní okruhy politologie:
  \begin{itemize}
    \item dějiny politických teorií
    \item politické instituce
    \item politická sociologie
    \item mezinárodní vztahy
  \end{itemize}
\end{itemize}

\section{Dějiny politického myšlení}
\begin{itemize}
  \item začínají již v antice, ve starověkém Řecku
  \begin{itemize}
    \item \textit{sofisté} - nezáleží na morálce, k moci se má dostat nejsilnější, cílem debaty není dojít k pravdě, ale obhájit si názor
    \item \textbf{Sokrates} - je zapotřebí, aby v čele státu byli moudří jedinci, kteří poznali dobro, stoupenec aristokracie
    \item \textbf{Platón} - sofokracie - vládce má vědět, má být moudrý král filosof, pak vrstva strážců - udržují pořádek v státě, pořádek státu, tito nemají mít sokroumé vlastnictví (epické řešení střetu zájmů?), dílo Ústava
    \item \textbf{Aristoteles} - dílo Politika - je důležité, aby vláda byla v zájmu celku, v zájmu státu, prošel se studenty všechny řecké ústavy a našel šest typů státu, a to tři dobré (monarchie, aristokracie, politeia - vládne většina schopných a vzdělaných státu) a tři špatné
  \end{itemize}
  \item ve středověku do toho těžce zasahuje křesťanství, vládne jeden či skupina (rozhodně ne normální člověk)
  \begin{itemize}
    \item Augustinus Aurelius - patristika, jsou dobří lidi - od Boha a zlí lidi - od Ďábla, dějiny jsou o boji dobra se zlem
    \item Tomáš Akvinský - scholastika, přejímá Aristotelovo učení, racionalismus, tomismus (víra i racionalismus)
  \end{itemize}
  \item a konečně novověk
  \begin{itemize}
    \item
  \end{itemize}
\end{itemize}

\end{document}
