\documentclass{article}
\usepackage{fullpage}
\usepackage[czech]{babel}
\usepackage{amsfonts}

\title{\vspace{-2cm}Divná geometrie\vspace{-1.7cm}}
\date{}
\author{}

\begin{document}
\maketitle

\section{Projektivní geometrie}

\subsection{Axiomy}
\begin{itemize}
  \item každé 2 body zadávají právě 1 přímku
  \item každé 2 přímky se protínají
  \item existují 3 body neležíí na jedné přímce
\end{itemize}
Z tohoto plyne:
\begin{itemize}
  \item každý bod má stejně přímek
  \item každá přímka má stejně bodů
  \item je stejně přímek jako bodů - $n^2+n+1$
  \end{equation}
\end{itemize}

\section{Afinní rovina}
\subsection{Axiomy}
\begin{itemize}
  \item stejné jako Projektivní geometrie, ale ne každé 2 přímky se musí potkat - existují ,,rovnoběžky" (právě jedna)
\end{itemize}
Takže:
\begin{itemize}
  \item každá přímka má stejný počet bodů - $n$
  \item každým bodem prochází stejně přímek - $n+1$
  \item celkem $n^2$ bodů, $n^2+n$ přímek
\end{itemize}

\end{document}
