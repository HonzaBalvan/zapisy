\documentclass{article}
\usepackage{fullpage}
\usepackage[czech]{babel}
\usepackage{amsfonts}

\title{\vspace{-2cm}Divná geometrie\vspace{-1.7cm}}
\date{}
\author{}

\begin{document}
\maketitle

\section{Projektivní geometrie}

\subsection{Axiomy}
\begin{itemize}
  \item každé 2 body zadávají právě 1 přímku
  \item každé 2 přímky se protínají
  \item existují 3 body neležíí na jedné přímce
\end{itemize}
Z tohoto plyne:
\begin{itemize}
  \item každý bod má stejně přímek
  \item každá přímka má stejně bodů
  \item je stejně přímek jako bodů - $n^2+n+1$
  \end{equation}
\end{itemize}

\section{Afinní rovina}
\subsection{Axiomy}
\begin{itemize}
  \item stejné jako Projektivní geometrie, ale ne každé 2 přímky se musí potkat - existují ,,rovnoběžky" (právě jedna)
\end{itemize}
Takže:
\begin{itemize}
  \item každá přímka má stejný počet bodů - $n$
  \item každým bodem prochází stejně přímek - $n+1$
  \item celkem $n^2$ bodů, $n^2+n$ přímek
\end{itemize}
\subsection{Příklady}
VLOŽTE OBRÁZKY AFFINNÍCH ROVIN PRO N=1, 2, 3, 4
\begin{itemize}
  \item vždycky můžeme přímky afinní roviny rozdělit do $n$ kategorií rovnoběžnosti - že vždycky přímky z jedné kategorie jsou navzájem rovnoběžné (ekvivalentní relace)
\end{itemize}

\section{Latinské čtverce}
\subsection{Motivační úkol od Eulera}
\begin{itemize}
  \item Postavte do čtverce 36 důstojníků z 6 pluků o 6 hodnostech tak, aby v každém řádku i sloupci nebyl dvakrát stejný pluk ani hodnost.
\end{itemize}
VLOŽTE OBRÁZEK AAAA
\begin{itemize}
  \item nejde to lol
\end{itemize}
\subsection{Definice}
\begin{itemize}
  \item Je to $n*n$ čtverec, který musíme zaplnit prvky z $n$ kategorií tak, aby v žádném řádku ani sloupci nebyly dva prvky ze stejné kategorie.
\end{itemize}
\subsection{Počet možností}
\begin{itemize}
  \item Kolik je možností utvořit latinský čtverec pro dané $n$? Je jich $n!*(n-1)!*(n-2)!*...*1!$
  \item můžeme to spočítat pomocí perfektních párování bipartitních grafů
\end{itemize}
AAAAAA VYSVĚTLI TO MAGORE A OBRÁZEK NEJLÉPE
\subsection{Vraťme se}
\begin{itemize}
  \item Eulerův úkol můžeme vyřešit tak, že zkombinujeme 2 latinské čtverce
  \item což má nějakou souvislost s afinní rovinou
\end{itemize}
\end{document}
