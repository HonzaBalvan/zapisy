\documentclass{article}
\usepackage{fullpage}
\usepackage[czech]{babel}
\usepackage{amsfonts}

\title{\vspace{-2cm}Organická chemie\vspace{-1.7cm}}
\date{}
\author{}

\begin{document}
\maketitle

\begin{itemize}
  \item věda, který se zabývá studiem struktury, vlasností, přípravou a použitím organických sloučenin
  \item je to chemie sloučenin uhlíku s biogenními prvky - vodíkem, kyslíkem, dusíkem, sírou, fosforem, hoříček, vápníkem
  \item základní vlastnost uhíku v organických sloučeninách = čtyřvaznost, důvodem je hybridizace - sjednocení energeticky různých orbitalů daného atomu, přičemž vznikají nové orbitaly, tzv. orbitaly hybrididní
  \item prvně středověk - \textbf{Paracelsus} - doktoři jsou na to, aby míchali léky - iatrochemie (16. st.)
  \item pak novověk - \textbf{Berzelius} - živočišné a rostlinné sloučeniny vznikají jenom díky ,,životní síle" (vis vitalis) - tzv. \textit{vitalistická teorie} (18. st.), tato brzo vyvrácena r. 1828 \textbf{F. Wöhlerem}, který v laboratoři z anorg. látky vytvořil org. látku močovinu
  \item na konci 19. st. průkopník \textbf{F. A. Kekulé} - definoval, že chemické vlasnosti organických sloučenin souvisí s jejich vnitřní stavbou, pak definoval několik základních postulátu, které platí dodnes
  \begin{itemize}
    \item uhlík je vždy čtyřvazný - z uhlíku vždycky vychází čtyři vazby
    \item všechny čtyři vazby atomu uhlíku jsou rovnocenné - důvodem je \textit{hybridizace} - viz dále
    AAAAAAAAA OBRÁZEK ZÁKLADNÍ STAV, EXCITOVANÝ STAV
    \item uhlíkové atomy mají schopnost vytvářet řetězce otevřené i uzavřené (tedy i cyklické)
    \item atomy jsou v nich vázány jednoduchými, dvojnými, nebo trojnými vazbami (vždy tedy tak, aby z jednoho uhlíku vycházely dohromady čtyři vazby)
  \end{itemize}
  \item hybridizace
  \begin{itemize}
    \item proces sjednocení energeticky různých orbitalů daného atomu, přičemž vznikají nové orbitaly, tzv. \textit{hybridní orbitaly}
    \item typy hybridizace:
    \begin{itemize}
      \item úplná - sp3 - čtyři stejné vazby, každá sigma - do tetraedru
      \item částečná trigonaální - sp2 trojúhelníková 120 stupnů - jedna dvojna (sigma, pi), dve jednoduche
      \item částečná lineární - sp lineární 180 stupnu, jedna trojna (sigma, dve pi), jedna jednoducha
    \end{itemize}

  \end{itemize}
  \item klasifikace uhlovodíků
  \begin{itemize}
    \item acyklické
    \begin{itemize}
      \item nasycené - všechny vazby jednoduché
      \item nenasycené - ne všechny vazby jednoduché
    \end{itemize}
    \item cyklické
    nedavam AAAAAAAAAAAA DODĚLAT
    stereochemie - zabývá se strukturou látek, nauka o prostorovém uspořádání atomů v molekule
    konstituce a konfigurace
      konstituce - řazení atomů za sebou
      konfigurace - umístění atomů v prostoru
    názvosloví - NAPROSTO DEBILNÍ
  \end{itemize}
  \item reakce organických látek
  \begin{itemize}
    \item bývají pomalejší než anorganické, mají složitjší průběh
    \item nevyrovnáváme, protože bysme se nedopočítali
    \item základní typy
    \begin{itemize}
      \item podle způsobu štěpení vazby
      \begin{itemize}
        \item homolýza - rovnoměrné, symetrické štěpení, vnikají radikály - částice s jedním volným nepárovným elektronem
        \item heterolýza - nerovnoměrné, nesymetrické štěpení, vznik nových, el. nabitých částic - jedna část si odtáhne elektrony, jedna ne
      \end{itemize}
      \item podle charakteru částic v reakci
      \begin{itemize}
        \item elektrofilní - vzniklé částice vyhledávají záporný náboj (vyhledávají přebytek elektronů), tedy jsou kladně nabité, např $H^+$
        \item nukleofilní - částice vyhledávají kladný náboj (mají přebytek elektronů), tedy jsou záporně nabité, např. $OH^-$
        \item radikálové - částice nesoucí nepárový elektron, velice reaktivní
      \end{itemize}
      \item podle celkové změny na substrátu - na tom, co vchází do reakce
      \begin{itemize}
        \item substituce = nahrazování = zaměňování - dochází k náhradě jednoho nebo více atomů (jedné nebo více atom. skupin) substrátu jiným atomem nebo skupinou atomů
        ite
        \begin{enumerate}
        \item radikálová - dochází k homolytickému štěpení pomocí radikálů, má tři fáze
        \begin{enumerate}
          \item iniciace - jde nám o vznik radikálů
          \item propagace - jde nám o samotnou reakci
          \item terminace - jde nám o zánik radikálů a o izolaci produktů
        \end{enumerate}
        \item elektrofilní - reakce s elektrofilním činidlem, které vzniká v průběhu reakce, např. nitrace benzenu
        \item nukleofilní - nukleofilní činidlo reaguje s uhlíkovým atomem s částečně kladným nábojem
        \end{enumerate}
        \item eliminace = odštěpení = odejmutí - děj, při kterém se uvolňuje molekula jednoduché, většinou anorganiceké látky, kvůli čemuž vzniká v molekule substrátu násobná vazba, nebo se zvyšuje její násobnost
        \begin{enumerate}
          \item dehydratace - osštěpují se molekuly vody
          \item dehdrogenace - odštěpují se atomy vodíku
          \item dehydrohalogenace - odštěpují se molekuly halogenovodíků
        \end{enumerate}
        \item adice = připojení = opak eliminace - vecpeme tam molekulu, snížíme násobnost vazby
        \begin{enumerate}
          \item elektrofilní - elektrofilní činidlo reaguje s pi-elektrony násobných vazeb uhlíku
          \item nukleofilní - nukleofilní činidlo se aduje na uhlík ve vazbě nesoucí částečný kladný náboj, probíhají na dvoujnou vazbu C=O
        \end{enumerate}
        \item molekulový přesmyk = isomerace, reakce v jejímž průběhu dochází k přesunu (přeskupení) určitých atomů z jednoho místa v molekule na místo jiné, aniž se měni chemické složení (souhrnný vzorec) v dané sloučenině
      \end{itemize}
    \end{itemize}
    \item v organice najdeme i běžné redoxní a acidobazické reakce
  \end{itemize}
  \item zápis reakce
  \begin{itemize}
    \item reakční schéma - zjednodušený zápis reakce: suroviny $\rightarrow$ produkty OBRAZEKOBRAZEK
    \item reakční mechanismus - podrobný popis přeměny výchozích látek na produkty včetně popisu všech meziproduktů OBRAZEKOBRAZEK
  \end{itemize}
  \item pravidla pojmenovávání org. sloučenin
  \begin{itemize}
    \item viz obrázek OBRAZEKOBRAZEK na discordu
    \item začínáme nasycenými uhlovodíky - alkany
    \begin{itemize}
      \item musíme najít tzv. základní hybrid - nejdelší řetězec, nejvíce násobných vazeb a nejvíce vedlejších řetězců - substituentů a pojmenuju ho podle počtu uhlíků (níže) s příponou -an
      \item tento pak očísluju ze směru, kde mám dříve substituent (když jsou substituenty stejně, tak porovnávám druhé substituenty, třetí substituenty, ...) a substituenty taky pojmenuju podle počtu uhlíků, ale přípona je -yl, před ně ještě přidám číslovku uhlíku základního hydridu - lokantu, na který je připojený, když je jich více stejných tak čísla dávám za sebe, odděluju čárkou a píšu di-,tri-,...
      \item pak to spojuju pomlčkama, substituenty seřazuju podle abecedy (příčemž ignoruju předpony di-, tri-, ..., cyklo-, takže např. 1-ethyl-1,2-dimethyl), mezi posledním substituentem a názvem základního hydridu pomlčka není, takže např. 4-ethyl-3-methylhexan
      \item když mám někde cyklickou část, tak je to vždy základní hydrid, před její název dám cyklo- a číslování bude zase takové, aby součet čísel lokantů se substituenty byl co nejmenší (ekvivalentní s pravidlem pro necyklické uhlovodíky), takže např. 4-ethyl-1,1,2-trimethylcyklohexan
      \item můžeme mít uhlovodík s nějakou volnou vazbou, že se od něj něco odtrhlo (on se od něčeho odtrhl), pak názvosloví funguje pořád stejně, ale představím si tu volnou vazbu jako substituent beze jména, tedy jen číslo lokantu a -yl a toto napíšu až nakonec, za jméno hlavního řetězce, takže např. propan-2-yl; popř. si můžu vybrat základní řetězec tak, aby na té volné vazbě začínal, pak i hlavní řetězec nazvu jako -yl (ne jako -an) a číslo vynechám (protože tedy vždy 1), takže např. 1-methylethyl, btw vždycky dávám v číslování přednost volné vazbě před ostatníma srandárnama
      \item když to všecho poskládám - když mám větvený substituent tak tu vazbu, kterou se připojuje na hlavní hydrid beru jako volnou, podle toho taky pojmenovávám, pak to poskládám dohromady, názvy substituentů ve větveném substituentu se řadí absolutně podle abecedy (včetně di-, tri-, ..., cyklo-), toto všechno dám do závorky a už se k tomu chovám jako k normálnímu názvu (před to číslo lokantu hl. hydridu, pomlčku, po něm pomlčku, popř. jestli je to jako poslední substituent tak po něm nenapíšu pomlčku)
      \item názvy podle počtu uhlíků
      \begin{itemize}
        \item meth -
        \item eth  -
        \item prop -
        \item but -
        \item pent -
        \item hex
        \item hept
        \item okt
        \item non
        \item dek
        \item undek
        \item dodek
        ...
        \item ikos
        ...
        \item triakont
      \end{itemize}
    \end{itemize}
  \end{itemize}
\end{itemize}

\end{document}
