\documentclass{article}
\usepackage{fullpage}
\usepackage[czech]{babel}
\usepackage{amsfonts}

\title{\vspace{-2cm}Organická chemie\vspace{-1.7cm}}
\date{}
\author{}

\begin{document}
\maketitle

\begin{itemize}
  \item věda, který se zabývá studiem struktury, vlasností, přípravou a použitím organických sloučenin
  \item je to chemie sloučenin uhlíku s biogenními prvky - vodíkem, kyslíkem, dusíkem, sírou, fosforem, hoříček, vápníkem
  \item základní vlastnost uhíku v organických sloučeninách = čtyřvaznost, důvodem je hybridizace - sjednocení energeticky různých orbitalů daného atomu, přičemž vznikají nové orbitaly, tzv. orbitaly hybrididní
  \item prvně středověk - Paracelsus - iatrochemie (16. st.)
  \item pak novověk - Berzelius - živočišné a rostlinné sloučeniny vznikají jenom díky ,,životní síle" (vis vitalis) - tzv. vitalistická teorie (18. st.), tato brzo vyvrácena r. 1828 F. Wöhlerem, který v laboratoři uměle vytvořil org. látku močovinu
  \item na konci 19. st. průkopník F. A. Kekulé - definoval, že chemické vlasnosti organických sloučenin souvisí s jejich vnitřní stavbou, pak definoval několik základních postulátu, které platí dodnes
  \begin{itemize}
    \item uhlík je vždy čtyřvazný
  \end{itemize}
\end{itemize}

\end{document}
