\documentclass{article}
\usepackage{fullpage}
\usepackage[czech]{babel}
\usepackage{amsfonts}

\title{\vspace{-2cm}Organická chemie\vspace{-1.7cm}}
\date{}
\author{}

\begin{document}
\maketitle

\begin{itemize}
  \item věda, který se zabývá studiem struktury, vlasností, přípravou a použitím organických sloučenin
  \item je to chemie sloučenin uhlíku s biogenními prvky - vodíkem, kyslíkem, dusíkem, sírou, fosforem, hoříček, vápníkem
  \item základní vlastnost uhíku v organických sloučeninách = čtyřvaznost, důvodem je hybridizace - sjednocení energeticky různých orbitalů daného atomu, přičemž vznikají nové orbitaly, tzv. orbitaly hybrididní
  \item prvně středověk - \textbf{Paracelsus} - doktoři jsou na to, aby míchali léky - iatrochemie (16. st.)
  \item pak novověk - \textbf{Berzelius} - živočišné a rostlinné sloučeniny vznikají jenom díky ,,životní síle" (vis vitalis) - tzv. \textit{vitalistická teorie} (18. st.), tato brzo vyvrácena r. 1828 \textbf{F. Wöhlerem}, který v laboratoři z anorg. látky vytvořil org. látku močovinu
  \item na konci 19. st. průkopník \textbf{F. A. Kekulé} - definoval, že chemické vlasnosti organických sloučenin souvisí s jejich vnitřní stavbou, pak definoval několik základních postulátu, které platí dodnes
  \begin{itemize}
    \item uhlík je vždy čtyřvazný - z uhlíku vždycky vychází čtyři vazby
    \item všechny čtyři vazby atomu uhlíku jsou rovnocenné - důvodem je \textit{hybridizace} - viz dále
    AAAAAAAAA OBRÁZEK ZÁKLADNÍ STAV, EXCITOVANÝ STAV
    \item uhlíkové atomy mají schopnost vytvářet řetězce otevřené i uzavřené (tedy i cyklické)
    \item atomy jsou v nich vázány jednoduchými, dvojnými, nebo trojnými vazbami (vždy tedy tak, aby z jednoho uhlíku vycházely dohromady čtyři vazby)
  \end{itemize}
  \item hybridizace
  \begin{itemize}
    \item proces sjednocení energeticky různých orbitalů daného atomu, přičemž vznikají nové orbitaly, tzv. \textit{hybridní orbitaly}
    \item typy hybridizace:
    \begin{itemize}
      \item úplná - sp3 - čtyři stejné vazby, každá sigma - do tetraedru
      \item částečná trigonaální - sp2 trojúhelníková 120 stupnů - jedna dvojna (sigma, pi), dve jednoduche
      \item částečná lineární - sp lineární 180 stupnu, jedna trojna (sigma, dve pi), jedna jednoducha
    \end{itemize}

  \end{itemize}
  \item klasifikace uhlovodíků
  \begin{itemize}
    \item acyklické
    \begin{itemize}
      \item nasycené - všechny vazby jednoduché
      \item nenasycené - ne všechny vazby jednoduché
    \end{itemize}
    \item cyklické
    nedavam AAAAAAAAAAAA DODĚLAT
    stereochemie - zabývá se strukturou látek, nauka o prostorovém uspořádání atomů v molekule
    konstituce a konfigurace
      konstituce - řazení atomů za sebou
      konfigurace - umístění atomů v prostoru
    názvosloví - NAPROSTO DEBILNÍ
  \end{itemize}
\end{itemize}

\end{document}
